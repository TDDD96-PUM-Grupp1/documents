
\documentclass[10pt]{article}
\usepackage[utf8]{inputenc}
\usepackage[swedish]{babel}
\usepackage[margin=2cm]{geometry}
\usepackage{calc}
\usepackage{graphicx}
\usepackage{filecontents}
\usepackage{etoolbox}
\usepackage[backend=bibtex,style=authoryear,natbib=true,style=numeric-comp]{biblatex}
\graphicspath{ {images/}}

\selectlanguage{swedish}

\usepackage{xifthen}
\usepackage{enumitem}
\usepackage{scrextend}
\newcounter{switchcase}

\newcommand{\ifequals}[3]{\ifthenelse{\equal{#1}{#2}}{\stepcounter{switchcase} #3}{}}
\newcommand{\case}[2]{#1 #2} % Dummy, so \renewcommand has something to overwrite...
\newenvironment{switch}[1]{
  %Executed at \begin{switch}
  \setcounter{switchcase}{0}
  \renewcommand{\case}{\ifequals{#1}}
}{
 % Executed at \end{switch}
\ifthenelse{\equal{\value{switchcase}}{0}}{
  \PackageError{ProjectDefinitions}{Could not find given definition}{}}{}
}

\newcommand{\definition}[1]
{
  \begin{switch}{#1}
    \case{Cachning}{\item [\textbf{#1}]
      Temporär lagning av data för snabb åtkomst.}
    \case{Instans}{\item [\textbf{#1}]
      En spelsession som startas från UI-applikationen och spelare kan gå med i för att spela spelet tillsammans.}
    \case{IoT-backend}{\item [\textbf{#1}]
      Existerande system som kan dirigera data mellan många uppkopplade enheter.}
    \case{Kontroll-applikation}{\item [\textbf{#1}]
      Applikation som körs på en mobil eller surfplatta och tar input från användare.}
    \case{Progressive Web Apps}{\item [\textbf{#1}]
      Förkortat PWA, är ett mellanting mellan en hemsida och en applikation.
      Med en PWA behöver man inte ladda ner en app, men den ger viss funktionalitet som appar har. \cite{bib-pwa}}
    \case{Resurs}{\item [\textbf{#1}]
      Media som används i spelet, t.ex. bilder och ljud.}
    \case{Sensor}{\item [\textbf{#1}]
      En sensor som sitter på kontroll-applikationen och inte är en pekskärm, t.ex. en accelerometer.}
    \case{Server-klient-modell}{\item [\textbf{#1}]
      Struktur på ett system där någon enhet tillhandahåller resurser, information eller tjänster och flera andra enheter interagerar med denna.}
    \case{Spelläge}{\item [\textbf{#1}]
      En utökning av grundspelet som definierar speciella regler och spelmekanik.}
    \case{Spelmekanik}{\item [\textbf{#1}]
      Regler och möjligheter som definierar ett spel.}
    \case{Tunn klient}{\item [\textbf{#1}]
      Specialfall av server-klient-modell där mycket få beräkningar sker på klienten.}
    \case{UI-applikation}{\item [\textbf{#1}]
      Applikationen som kör spelet och visar spelplanen.}
    \case{Use Case Map}{\item [\textbf{#1}]
      Diagram som illustrerar hur olika händelser interagerar med arkitekturen. \cite[p.~30--33]{bib-architecture-primer}}
    \case{Scrum-board}{\item [\textbf{#1}]
      En tavla med post-it lappar som innehåller aktiviteter som ska göras under
      projektet. Detta komplementeras med olika kolumner i tavlan såsom planerad, pågående,
      testning och utgåva. Dessa bestämmer i vilket stadie lapparna befinner sig i.}
    \case{Burndown-chart}{\item [\textbf{#1}]
      En graf som visar hur många timmar medlemmarna har lagt ner i förhållande till vad som krävs för att hinna med projektet.}
    \case{Acceptanstest}{\item [\textbf{#1}]
      Slutgiltiga testet som kund utför för att se att produkten lever upp till förväntningarna.}
    \case{Enhetstest}{\item [\textbf{#1}]
      Testa varje enhet så den fungerar när den är färdig.}
    \case{Integrationstest}{\item [\textbf{#1}]
      Testa att en ny enhet som läggs till i projektet fungerar som den ska tillsammans med de andra enheterna.}
    \case{Kund}{\item [\textbf{#1}]
      Cybercom Sweden.}
    \case{Regressionstest}{\item [\textbf{#1}]
      Testa ny kod enligt gamla parametrar för att säkerställa att ingen funktionalitet försvunnit.}
    \case{Systemtest}{\item [\textbf{#1}]
      Test för att säkerställa att enheten uppfyller kraven för projektet.}
    \case{Cybercom}{\item [\textbf{#1}]
      Kortare variant av Cybercom Sweden, företaget produkten utvecklas åt.}
    \case{Enkäten}{\item [\textbf{#1}]
      Den enkät som ska användas för att utvärdera användarupplevelsen, se avsnitt  3.3 Demo och enkät.}
    \case{Kvalitet}{\item [\textbf{#1}]
        I likhet med IEEE 730 definierar denna rapport kvalitet som konformitet till projektets krav. \cite{ieee730}}
    \case{Projektet}{\item [\textbf{#1}]
        Processen att framställa en produkt åt Cybercom Sweden.}
    \case{Software Quality Asssurance}{\item [\textbf{#1}]
    	Förkortat SQA, är en samling aktiviteter som bedömmer lämpligheten och inger förtroende
    	för utvecklingsmetodiken som används.}
    \case{SQA-process}{\item [\textbf{#1}]
      I likhet med IEEE 730 definieras en SQA-process som aktiviteten att samla underlag för att med säkerhet ta
      beslutet av produkten uppnår sina kvalitetskrav}
    \case{Teamet}{\item [\textbf{#1}]
      Det team av åtta studenter som tillsammans ska utföra projektet}
    \case{Trello}{\item [\textbf{#1}]
      En hemsida för att lägga till och fördela uppgifter bland flera personer, kan liknas till en whiteboard som
      postit lappar fästs på.}
    \case{Speldata}{\item [\textbf{#1}]
      Information om handlingar och status i spelet samt nödvändig teknisk data för
      att upprätthålla kommunikation.}
    \case{Realtidsmultiplayerspel}{\item [\textbf{#1}]
      Spel där flera användares handlingar har en direkt inverkan på spelets tillstånd.}
    \case{Gamemode}{\item [\textbf{#1}]
      En variant av basspelet med eventuellt andra funktioner och regler.}
    \case{Vanliga nätverksförhållanden}{\item [\textbf{#1}]
      En enhet med en stabil internetuppkoppling utan yttre störningar.}
    \case{React}{\item [\textbf{#1}]
      Javascript-bibliotek för att bygga hemsidor och mer avancerade webbsystem.\cite{bib-react}}
    \case{Deep Stream}{\item [\textbf{#1}]
    Kommunikationssystem som tillåter synkronisering av data mellan många enheter i realtid. Tillgängligt i många olika programmeringsspråk, bland annat javascript.\cite{bib-deepstream}}
    \case{Impact Map}{\item [\textbf{#1}]
    Diagram som visar inverkan av händelser under ett mjukvarusystems livstid. Kan visa på effekterna av implementation av ny funktionalitet, fel i systemet eller säkerhetsintrång.\cite[p.~91--93]{bib-architecture-primer}}
    \case{IoT, Internet of things}{\item [\textbf{#1}]
    Internet of things -- Ett begrepp som beskriver den tekniska och samhälleliga utveckling då fler och fler saker blir uppkopplade mot internet.}
    \case{Gitrepo}{\item [\textbf{#1}]
    En datastruktur för att lagra och hantera olika versioner av kod i git.}
    \case{Master-branch}{\item [\textbf{#1}]
    Standardgrenen till ett gitrepo som vanligtvis reflekterar repot i ett fungerande tillstånd.}
	\case{Kursen}{\item [\textbf{#1}]
    Den kurs som detta projekt utförs inom, det vill säga LiTHs kurs ''Kandidatprojekt i programvaruutveckling'' med kurskod TDDD96}
  \case{npm}{\item [\textbf{#1}]
  Node Package Manager -- En pakethanterare för Javascripts ekosystem}
  \case{npm-paket}{\item [\textbf{#1}]
  Ett paket med Javascript-kod som finns tillängligt i npm}


  \end{switch}
}


\tolerance=1
\emergencystretch=\maxdimen
\hyphenpenalty=10000
\hbadness=10000

\begin{filecontents*}{\jobname.bib}
@techreport{bib-name,
    title={title},
    author={author},
    year={XXXX},
    institution={institition},
    month={month}
}
\end{filecontents*}

\addbibresource{\jobname.bib}

\title{Kvalitetsplan}

\author{
    Joel Almqvist\\
    \texttt{joeal360@student.liu.se}
    \and
    Björn Detterfelt\\
    \texttt{bjode786@student.liu.se}
    \and
    Tim Håkansson\\
    \texttt{timha404@student.liu.se}
    \and
    David Kjellström\\
    \texttt{davkj168@student.liu.se}
    \and
    Axel Löjdquist\\
    \texttt{axelo225@student.liu.se}
    \and
    Joel Oskarsson\\
    \texttt{joeos014@student.liu.se}
    \and
    Lieth Wahid\\
    \texttt{liewa893@student.liu.se}
    \and
    Alexander Wilkens\\
    \texttt{alewi684@student.liu.se}
}

\begin{document}

\maketitle
\pagebreak
\tableofcontents
\pagebreak
\section{Inledning}
	I detta kapitel definieras en kvalitetsplan, dess syfte samt tillhörande terminologi.

	\subsection{Definitioner}
  \begin{itemize}[leftmargin=5cm]
    \definition{Cybercom}
    \definition{Enkäten}
    \definition{Kvalitet}
    \definition{Projektet}
    \definition{Progressive Web Apps}
    \definition{SQA-process}
    \definition{Teamet}
    \definition{Trello}
	\end{itemize}	
	
	\subsection{Syfte}
		Syftet med denna kvalitetsplan är att definiera de arbetsprocesser teamet ska efterfölja, med målet att det ska få produkten att hålla hög kvalitet. Kvalitetsplanen ska också definiera de standarder teamet ska efterfölja med målet att de ska ge produkten en högre minimumnivå gällande kvalitet och samtidigt få produktens olika delar mer konsekventa.

		
\pagebreak
\section{Krav på systemet}
	Detta kapitel ger en överskådlig blick över projektet, hur det är tänkt att utföras och vilka krav som satts på det.

	\subsection{Grov beskrivning av Projektet}
	Projektet går ut på att teamet ska skapa ett realtidsmultiplayerspel åt företaget Cybercom. Produkten ska vara en tvådelad PWA som enbart kommunicerar med varandra genom Cybercoms backend. Målet med produkten är att visa upp Cybercoms backend, framförallt dess hastighet.
	Projektet ska utföras av studenter som saknar god erfarenhet av att arbeta i team och de roller som finns i teamet. Projektet i sig beräknas ta elva veckor att genomföra. 
	
	\subsection{Systemets behov}
	Syftet med produkten är att demonstrera responsitiviteten hos Cybercoms backend, för att uppfylla detta ställs flera krav på produkten. Den måste vara både responsiv och stabil för att ge ett bra intryck av backenden den ska demonstrera. Själva användningen av produkten är tänkt att utföras av Cybercom på mässor, vid sådana tillfällen är det också viktigt att produkten är användarvänlig. Om så inte är fallet kan produkten ge en negativ bild av backenden den är tänkt att uppvisa. Dessa tre är de mest basala behov systemet har. Efter dessa finns fler egenskaper som systemet önskas uppvisa, bland annat så vill kunden ha möjligheten att bygga ut den färdiga produkten.
	
	
	\subsection{Kvalitetskrav}
	Utifrån dessa behov har teamet sammanställt kvalitetskrav som ser till att kundens behov blir uppfyllt när produkten färdigställts.
	\begin{enumerate}
		\item UI-applikationen ska kunna köras i 4 timmar utan avbrott.
		\item Koden ska följa en enhetlig standard för att underlätta vidareutveckling av projektet.
		\item Tiden att ansluta sig till en spelsession får inte överskrida 10 sekunder under standard nätverksförhållanden
		\item Enkätens genomsnittliga betyg för responsivitet ska åtminstone vara 6 av 10 vid en undersökning 20 deltagare.
		\item Enkätens genomsnittliga betyg för användbarhet ska åtminstone vara 6 av 10 vid en undersökning 20 deltagare.
		\item Enkätens genomsnittliga betyg för den alternativa styrningen ska åtminstone vara 6 av 10 vid en undersökning 20 deltagare.
		\item Enkätens genomsnittliga betyg för ''lätt att förstå vad som händer'' ska åtminstone vara 6 av 10 vid en undersökning 20 deltagare.
		
	\end{enumerate}

\pagebreak
\section{Processer}
	Detta avsnitt går igenom de aktiviteter teamet ska genomföra för att uppfylla de satta kvalitetskraven och på så sätt säkra att produkten håller hög kvalitet.
	\subsection{Kommenteringsstandard}
	\textbf{Syfte:} För att Cybercom med lätthet ska kunna utöka produkten ska koden vara lätt för en utomstående att sätta sig in i.
	\\\\
	\textbf{Metod:} Koden ska följa en tydlig och enhetlig kommenteringsstandard för att göra det lättare att sätta sig in i samt för att få kommentarerna mer konsekventa. För mer detaljer gällande denna standard se avsnitt 4.1.
	\\\\
	\textbf{Relaterat kvalitetskrav:} 3
	\\

	\subsection{Kommenteringsstandard V2***}
	\textbf{Syfte:}	För att Cybercom med lätthet ska kunna utöka produkten ska koden vara lätt för en utomstående att sätta sig in i.
	\\\\
	\textbf{Relaterat kvalitetskrav:} 3
	\\\\
	\textbf{Utförande:} All kod som gruppen skriver ska ha förklarande kommentarer där skribenten anser att en förklaring skulle underlätta för läsaren. Ovanför varje funktion ska doc-strings skrivas som förklarar parametrarna funktionen tar in samt returvärden för funktionen.
	Kommentaren i sig ska följa Airbnbs stil-guide som finns tillgänglig på https://github.com/airbnb/javascript.
	****TODO FIX CORRECT REFERENCE
	\\\\
	\textbf{Utvärdering:} Denna process saknar en egen utvärderingsprocess utan kommer istället att utvärderas under kodinspektionerna som måste göras innan en git-branch accepteras in till huvudprojektet. För mer detaljer gällande denna process se sektion gällande git 3.X*********TODO REFERENS.
	\\\\
	\textbf{Problem hantering:} Ifall det upptäcks att kod inte följer denna kommenteringsstandard så kommer personen som skrev den få uppgiften att lägga till saknade kommentarer alternativt skriva om de kommentarer som finns. Efter omskrivning ska koden inspekteras igen för att undersöka ifall den uppfyller de standarder som finns.
	\\\\
	\textbf{Roller:} Personen som skriver koden är ansvarig för att kommentarer skrivs på korrekt sätt. Det är personen som skriver kod som kommer att initialisera denna process och personen som granskar koden som avslutar den. Alla team medlemmar kommer anta båda rollerna och det spelar ingen roll vem som granskar den andres kod.
	

	\subsection{Kodstandard}
	\textbf{Syfte:} För att Cybercom med lätthet ska kunna utöka produkten ska koden vara lätt för en utomstående att sätta sig in i.
	\\\\
	\textbf{Metod:} För att göra koden med lättläslig och konsekvent ska alla teamets medlemmar följa kodstandarden definierad i avsnitt 4.2.
	\\\\
	\textbf{Relaterat kvalitetskrav:} 3
	\\
	
	\subsection{Demo och enkät}
	\textbf{Syfte:} De viktigaste kvalitetsaspekterna hos produkten är att den är responsiv, användarvänlig och stabil. Syftet med produkten är att demonstrera Cybercoms backend för en användbas som testar produkten för första gången. Den måste alltså vara användarvänlig, men också stabil och responsiv för att visa att backenden erbjuder dessa egenskaper.
	
	Dessa aspekter lär förändras under projektets gång då nya delar läggs till och förändras i enlighet med projektets agila arbetsmetod. Kontinuerlig verifiering av dessa kvalitetskrav kommer därav behövas göras så att projektet alltid håller rätt riktning.
	\\\\
	\textbf{Metod:} För att kontinuerligt utvärdera dessa centrala kvalitetskrav så ska teamet vid slutet av varje sprint demonstrera den senaste versionen av produkten och låta 20 personer testa att spela spelet. Dessa personer ska sedan fylla i enkät där de svarar på frågor gällande: responsitivitet, stabilitet, enkelhet att följa vad som händer på skärmen och enkelhet att använda kontroller (se appendix för en mock-enkät). Dessa fyra kriterier representerar kvalitetskrav och ges ett betyg mellan ett och tio, där tio är positivt. Vid den sista sprinten måste det genomsnittliga betyget på den sista enkäten vara åtminstone sex för frågorna gällande responsitivitet och stabilitet. De andra två kriterierna måste också få ett genomsnittligt betyg av åtminstone sex vid den sista demonstrationen. Kraven på dessa betyg kan anses låga, men de är inte vad teamet eftersträvar att nå utan ska ses som en minimum gräns för vad som måste klaras av för att produkten kan anses hålla hög kvalitet.
	
	Själva arbetsprocessen blir alltså ett demo och en enkät för att utvärdera hur väl kvalitetskraven uppfylls för tillfället. Den sista sprinten utförs demot på samma sätt, men nu så måste de ovannämnda betygen uppfyllas för att produkten ska sägas uppnå sina kvalitetskrav. Denna kontinuerliga process visar ifall projektet rör sig i rätt riktning, hur nära målet vi är, samt som ett smoke test ifall en förändring skapar en stark negativ upplevelse hos användarna.
	\\\\
	\textbf{Relaterat kvalitetskrav:} 4, 5, 6, 7
	\\
	
	
	\subsection{Trello}
	\textbf{Syfte:} En viktig del av kvalitet är att ge processer den tid som krävs, för att uppnå detta behöver teamet uttnytja sin tid på ett effektivt sätt.
	\\\\
	\textbf{Metod:} Teamet ska inför varje sprint, och under en sprint vid behov, dela upp de uppgifter som behöver genomföras till mindre uppgifter och lägga upp detta på Trello. Sedan är det varje medlems uppgift att ta den uppgift personen ska arbeta med och inte arbeta på någon annans del.
	\\\\
	\textbf{Relaterat kvalitetskrav:} Alla
	\\
	
	\subsection{Prettier}
	\textbf{Syfte:} För att koden ska gå att vidareutveckla är det viktigt att det är enkelt att läsa den.
	\\\\
	\textbf{Metod:} Teamet ska använda sig av verktyget Prettier för att automatiskt indentera koden, vilket gör indenteringen konsekvent och koden mer lättläslig. Prettier ska användas innan kod trycks in i huvudgrenen av produkten.
	\\\\
	\textbf{Relaterat kvalitetskrav:} 3
	\\
	
	
\pagebreak
\section{Standarder	och	Arbetsflöden}
	För att garantera att produkten håller hög kvalitet ska teamet inte enbart genomföra de SQA-processerna utan även följa dessa standarder.
	
	\subsection{Cybercoms standard}
	Standarden Cybercom använder sig av ska även detta projekt följa för att göra det lättare att utöka. Projektet ska använda sig av JavaScript-biblioteket React och följa den standard som används där. Dessutom så ska koden i enlighet med Cybercoms standard indenteras med två blanksteg och sparas i formatet UTF-8.
	
	
	\subsection{Kommenteringsstandard}
	Teamet ska följa en tydlig kommenteringsstandard som utvärderas vid slutet av varje sprint och som kan utökas och minskas vid behov.
	
	\begin{itemize}
		\item 	Ovan varje funktion ska en kommentar skrivas som beskriver vad funktionen gör, vad dess parametrar är och vad vad dess returvärde är.
		\item Alla kommentarer ska vara på engelska med amerikansk stavning
		\item Kommentarer ska skrivas ovan komplicerade stycken med målet att underlätta för läsaren.
		\item Teamet ska skriva kommentarer i enlighet med Airbnbs JavaScript standard.
	\end{itemize}

	
	\subsection{Kodstandard}
	För att se till att koden blir lätt att sätta sig in i och utöka så har teamet en standard att följa. Vid slutet av sprints ska denna standard utvärderas efter ifall punkterna upplevs som onödigt begränsade, överflödiga eller inte applicerbara. Dessutom kan nya punkter att läggas till då ett behov upptäcks.

	\paragraph{Nuvarande kodningsprinciper} \mbox{}
	\\
	Teamet ska följa Airbnbs JavaScripts standard med följande utökningar:
	\begin{itemize}
	\item All kod ska vara på engelska med amerikansk stavning
	\item Koden ska skrivas med målet att vara lättläst, inte efter hur lätt den är att skriva
	\item Undvik ovanliga och komplicerade metoder/lösningar då de gör det svårare att hitta buggar samt sätta sig in i koden efter projektet är över.
	\item Undvik att fylla upp den globala namnrymd om möjligt.
	\item Inga  ''and'' i funktionsnamn ifall de inte kan motiveras väl
	
	\end{itemize}
	
	\subsection{Arbetsflödet för Git}
	Teamets arbetsflöde för git är att varje självstående funktionalitet ska arbetas på en egen gren om möjligt. Commits ska ske oftare än dagligen, och med korta men beskrivande kommentarer skriva på engelska i nutid (ex ''Add this new cool feature''). När en gren ska ''pushas'' till master-grenen måste åtminstone en annan teammedlem gå igenom grenen som ska läggas till och godkänna denna. Innan ett godkännandes ges måste inspektören ha en helhetsbild av vad som läggs till. När en gren sedan läggs till ska automatiska tester köras, som sedan kan neka en gren.
	
	En mer utförlig beskrivning av arbetsflödet finns tillgänglig på git-repot. Där finns även en mer detaljerad beskrivning av de automatiska testerna som kommer att köras av testprogrammet Travis.
	
\pagebreak
\section*{Appendix}
	\subsection*{Mock enkätfrågor}
	\begin{enumerate}
		\item Jag tycker att spelet var lätt att lära sig. 
		\item Jag tycker att det var svårt att förstå vad som skulle göras under spelets gång.
		\item Jag tycker att spelet kändes responsivt.
		\item Jag tycker inte att responsen för mina handlingar visades snabbt.
		\item Jag tycker att den alternativa styrningen var lätt att använda.
		\item Jag tycker att den alternativa syrningen kändes klumpig.
		\item Jag tycker att det var lätt att se vad som skedde på den stora skärmen.
		\item Jag hade svårt att följa vad som skedde på den stora skärmen.
		
	\end{enumerate}




	
\pagebreak

\printbibliography
\addcontentsline{toc}{section}{\refname}


\end{document}
