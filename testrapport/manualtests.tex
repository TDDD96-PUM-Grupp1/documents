
% TEST TEMPLATE
\begin{comment}
%START
\testid{}
\testpriority{}
\testmodule{}
\testtitle{}
\testby{}
\testdate{}
\testexecby{}
\testexecdate{}
\testdesc{}
\testpreq{}
\testdep{}

\begin{manualtest}
    %Tillvägagångssätt - Förväntat resultat - Faktiskt resultat - Status - Anteckningar
    \teststep{}{}{}{}{}
    \teststep{}{}{}{}{}
\end{manualtest}
%END
\end{comment}





% SCOREBOARD TEST
\testid{ui-highscoresort}
\testpriority{Låg}
\testmodule{Spellogik}
\testtitle{Sortering av highscorelista}
\testby{Joel Oskarsson}
\testdate{2018-04-18}
\testexecby{Joel Oskarsson, David Kjellström}
\testexecdate{2018-04-18}
\testdesc{Test av att spelare får poäng som reflekterar deras position i listan på skärmen. Test av att spelare får poäng som reflekterar deras position i listan på skärmen.}
\testpreq{Spelare kan ansluta och röra sig runt på spelplanen. Ett spelläge tilldelar spelare poäng.}
\testdep{Grundspel, spelläge, kommunikation, fungerande kontrollapplikation}

\begin{manualtest}
    \teststep{Starta instans av Service och UI}{Systemet startar}{förväntat}{ok}{}
    \teststep{Skapa spelinstans}{Spelplanen visas på skärmen med tom highscorelista}{förväntat}{ok}{}
    \teststep{Två spelare starar kontrollapplikationer och går med i spelinstansen}{Spelarna kan gå med i instansen}{förväntat}{ok}{}
    \teststep{Spelare ett knuffar ut den andra så att den får mest poäng}{Spelare ett visas högst på listan}{förväntat}{ok}{}
    \teststep{Spelare två knuffar ut den andra så att den får mest poäng}{Spelare två visas högst på listan}{förväntat}{ok}{}
    \teststep{Spelarna lämnat instansen}{Spelarna försvinner från highscore-listan}{förväntat}{ok}{}
\end{manualtest}

%Test av spelkontroll
\testid{kontroll-styrning}
\testpriority{Hög}
\testmodule{sensoravläsning, spellogik}
\testtitle{spelarkontroll}
\testby{Joel Oskarsson}
\testdate{2018-02-22}
\testexecby{Joel Oskarsson, David Kjellström}
\testexecdate{2018-02-22}
\testdesc{ Test av ansluta från kontrollapplikationen till en spelinstans. Efter anslutningen är upprättad ska sensordata på kontrollen röra en seplare som visas på UI-appikationen.}
\testpreq{Sensoravläsning, kommunikation, enkel spelimplementation}
\testdep{}

\begin{manualtest}
    %Tillvägagångssätt - Förväntat resultat - Faktiskt resultat - Status - Anteckningar
    \teststep{Starta service och UI-Applikation }{ Systemet startar}{ förväntat}{OK}{I Google Chrome}
    \teststep{Starta spelet på UI}{spelet startas}{förväntat}{OK}{}
    \teststep{Starta kontrollapplikation}{kontrollapplikation startas}{förväntat}{OK}{I Google Chrome}
    \teststep{Spelare ansluter till spelet}{spelarna visas i UI}{förväntat}{OK}{}
    \teststep{Spelare lutar sin telefon}{spelare rör sig i spelet}{förväntat}{OK}{}
    \teststep{Spelaren stänger kontroll-applikationen}{spelarens figur försvinner från spelplanen}{förväntat}{OK}{inom 5 sekunder (förväntat beteende)}
\end{manualtest}

\testid{flera-spelare-simultant}
\testpriority{Medium}
\testmodule{Hela systemet}
\testtitle{Flera spelare}
\testby{David Kjellström}
\testdate{2018-04-16}
\testexecby{Gruppen med Cybercomgroup}
\testexecdate{2018-04-16}
\testdesc{Skapa en session där flera spelare kör samtidigt. Testades direkt med kund genom en demonstration av produkten.}
\testpreq{Spelare kan knuffa ut varandra och en highscorelista visar poäng}
\testdep{Grundspel}

\begin{manualtest}
    %Tillvägagångssätt - Förväntat resultat - Faktiskt resultat - Status - Anteckningar
    \teststep{Starta service och UI-applikation}{Systemet startar}{förväntat}{OK}{}
    \teststep{Starta spelet på UI}{spelet startas}{förväntat}{OK}{}
    \teststep{Starta kontrollapplikation}{kontrollapplikationen startas}{förväntat}{OK}{}
    \teststep{Spelare ansluter}{spelare ansluts}{förväntat}{OK}{}
    \teststep{Spelare väljer utseende på karaktär}{Spelare får utseendet}{långsammare än väntat}{OK}{Långa laddtider mellan de olika utseendena}
    \teststep{Spelare går med i spelet}{spelare är med i spelet}{förväntat}{OK}{Spelare tappade förvald färg när knapp trycktes ned på kontroll}
    \teststep{Flera spelare är med i spelet}{flera spelare i samma spel utan problem}{flera spelare i samma spel med fördröjningsproblem}{EJ OK}{flera spelare upplevde fördröjningar och bristande responsivitet}
\end{manualtest}

%START
\testid{spelare-namn}
\testpriority{Låg}
\testmodule{Kontroll, UI}
\testtitle{Spelarnamn visas i highscorelista}
\testby{Joel Oskarsson, David Kjellström}
\testdate{2018-04-23}
\testexecby{Joel Oskarsson, David Kjellström}
\testexecdate{2018-04-23}
\testdesc{Skapa spelare med slumpgenererat användarnamn och inskrivet användarnamn. Se till att de dyker upp på highscorelistan.}
\testpreq{kommunikation, poängsystem, namnval}
\testdep{}

\begin{manualtest}
    %Tillvägagångssätt - Förväntat resultat - Faktiskt resultat - Status - Anteckningar
    \teststep{Starta UI-applikation}{UI startar}{förväntat}{OK}{}
    \teststep{Starta kontroll-applikation}{kontroll-applikation startar}{förväntat}{OK}{}
    \teststep{Ange användarnamn och gå med i spel}{Rätt användarnamn syns på highscorelistan}{förväntat}{OK}{}
    \teststep{Starta ny kontroll-applikation}{kontroll-applikation startar}{förväntat}{OK}{}
    \teststep{Ange inget användarnamn och gå med i spel}{Slumpat användarnamn syns på highscorelistan}{förväntat}{OK}{}
\end{manualtest}

\testid{knapp-funktion}
\testpriority{Medium}
\testmodule{UI, kontroll}
\testtitle{Knapptryckningar ger konsekvenser i spelet}
\testby{Joel Oskarsson, David Kjellström}
\testdate{2018-04-23}
\testexecby{Joel Oskarsson, David Kjellström}
\testexecdate{2018-04-23}
\testdesc{Spelare trycker på en knapp och spelfiguren byter färg.}
\testpreq{Knappar, kommunikation, färger i spelet}
\testdep{}

\begin{manualtest}
    %Tillvägagångssätt - Förväntat resultat - Faktiskt resultat - Status - Anteckningar
    \teststep{Starta UI- och kontroll-applikation}{Startas}{förväntat}{OK}{}
    \teststep{Spelare går med i spelet}{Spelaren dyker upp på spelplanen}{förväntat}{OK}{}
    \teststep{Spelaren trycker på knapp på kontrollen}{Spelarens figur byter färg}{förväntat}{OK}{}
\end{manualtest}
