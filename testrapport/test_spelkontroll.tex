\documentclass[10pt]{article}
\usepackage[utf8]{inputenc}
\usepackage[margin=2cm]{geometry}
\usepackage[swedish]{babel}
\usepackage{hyperref}
\usepackage{color}
\usepackage{scrextend}
\usepackage[backend=bibtex,sorting=none,style=numeric,natbib=true]{biblatex}
\usepackage{multirow}

\begin{document}




\begin{tabular}{| p{1cm}|  p{3cm} | p{3cm}| p{3cm}| p{2cm}| p{3cm}|}
  \hline
	\multicolumn{3}{|l|}{Testfall ID:}&\multicolumn{3}{|l|}{Test designat av: Joel Oskarsson}\\
	\hline
	\multicolumn{3}{|l|}{Test prioritet: Hög}&\multicolumn{3}{|l|}{Test designat datum: 22-02-2018}\\
	\hline
	\multicolumn{3}{|l|}{Modul: sensoravläsning, spellogik}&\multicolumn{3}{|l|}{Test utfört av: Joel Oskarsson}\\
	\hline
	\multicolumn{3}{|l|}{Test titel: spelarkontroll}&\multicolumn{3}{|l|}{Test utfört: 22-02-2018}\\
	\hline
	\multicolumn{6}{|p{\textwidth}|}{Beskrivning: Test av ansluta från kontrollapplikationen till en spelinstans. Efter anslutningen är upprättad ska sensordata på kontrollen röra en seplare som visas på UI-appikationen.}\\
	\hline
	\multicolumn{6}{|p{\textwidth}|}{Förutsättningar: Sensoravläsning, kommunikation, enkel spelimplementation}\\

	\hline
	\multicolumn{6}{|l|}{}\\
	\multicolumn{6}{|l|}{}\\
      	\hline
	Steg&Tillvägagångssätt&Förväntat Resultat&Faktiskt Resultat&Status&Anteckningar \\
	\hline
	1& Starta service och UI-Applikation & Systemet startar & förväntat & ok &\\
      	\hline
	2& Starta spelet på UI & spelet startas & förväntat & ok &\\
      	\hline
	3& Starta kontrollapplikation & kontrollapplikation startas & förväntat & ok &\\
      	\hline
  4& Spelare ansluter till spelet & spelarna visas i UI & förväntat & ok &\\
        \hline
  5& Spelarna lutar sina telefoner & spelarna rör sig i spelet & förväntat & ok &\\
      	\hline
\end{tabular}




\end{document}
