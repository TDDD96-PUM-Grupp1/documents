\section{Definitioner}
\begin{itemize}
  \item Cachning -- Temporär lagning av data för snabb åtkomst
  \item Instans -- En spelsession som startas från UI-applikationen och spelare kan gå med i för att spela spelet tillsammans
  \item IoT-backend -- Existerande system som kan dirigera data mellan många uppkopplade enheter
  \item Kontroll-applikation -- Applikation som körs på en mobil eller surfplatta och styr spelet
  \item Progressive Web Apps -- Ett mellanting mellan en hemsida och en applikation. Med en PWA behöver man inte ladda ner en app, men den ger viss funktionallitet som appar har. \cite{bib-pwa}
  \item Resurs -- Media som används i spelet, t.ex. bilder och ljud
  \item Sensor -- En sensor som sitter på kontroll-applikationen och inte är en pekskärm, t.ex. en accelerometer
  \item Server-klient-modell -- Struktur på ett system där någon enhet tillhandahåller resurser, information eller tjänster och flera andra enheter interagerar med denna
  \item Spelläge -- En utökning av grundspelet som definierar speciella regler och spelmekanik
  \item Spelmekanik -- Regler och möjligheter som definierar ett spel
  \item Tunn klient -- Specialfall av server-klient-modell där mycket få beräkningar sker på klienten
  \item UI-applikation -- Applikationen som kör själva spelet och visar upp spelplanen
  \item Use Case Map -- Diagram som illustrerar hur olika händelser interagerar med arkitekturen \cite[p.~30--33]{bib-architecture-primer}
\end{itemize}
