\subsection{Dokument}
Under projektets gång kommer olika dokument att produceras för att underlätta utvecklingen.
Dessutom kommer en sista kandidatrapport, som beskriver allt om och runtomkring projektet,
produceras.

\subsubsection*{Projektplan}
Detta dokument beskriver allt som finns runtomkring projektet, beskrivning utav projektet, 
resurser som teamet har tillgång till, processer som kommer användas och risker inom projektet.
Dessutom beskrivs en aktivitetsplan som översiktligt tar upp alla aktiviteter som kommer att
göras under projektet.

\subsubsection*{Kravspecifikation}
Kravspecifikationen har som mål att detaljerat beskriva de viktigaste kraven för kunden.
Den ska även beskriva andra krav som har mindre mening för kunden, men dessa behöver inte vara
like detaljerade.

\subsubsection*{Kvalitetsplan}
För att få en meningsfull process kommer detta dokumentet beskriva dessa för att få ut ett 
kvalitativt projekt i slutändan.

\subsubsection*{Statusrapports}
Statusrapporten kommer fungera som en reflektion på hur förstudierna har gått, vad som kommer
hända härnäst och vad för risker som finns framöver. Det fungera som en mindre projektplan för
de olika iterationerna.

\subsubsection*{Systemanatomi}
Anatomin för systemet beskriver hur systemet är uppbyggt på olika nivåer, såsom funktioner, 
mjukvara, hårdvara. Det ger en större bild för hur systemet fungera och i vilken miljö den kommer
befinna sig i.

\subsubsection*{Arkitekturdokument}
Här beskrivs själva arkitetkturen för systemet i mer detalj, hur olika submoduler kommer hänga
ihop. Dessutom beskrivs även alternativa arkitekturer och varför dessa inte är relevanta för
projektet.

\subsubsection*{Testplan}
Detta dokument beskriver hur teamet ska testa sina olika funktioner och även hur de följer upp på
dessa.

\subsubsection*{Kandidatrapport}
Kandidaten som ska produceras i slutet utav projektet har som mål att beskriva hela projektet i
sin helhet. Allt som mål, arbetsprocess, slutprodukt och individuella forskningsområden.

