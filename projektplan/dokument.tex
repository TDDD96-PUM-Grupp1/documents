\subsection{Dokument}
Under projektets gång kommer olika dokument att produceras för att underlätta utvecklingen.
Dessutom kommer en sista kandidatrapport, som kommer tydligt beskriva projekets bakgrund, arbetsgång och resultat.

\subsubsection*{Projektplan}
Detta dokument beskriver allt som finns runtomkring projektet, beskrivning utav projektet, 
resurser som teamet har tillgång till, processer som kommer användas och risker inom projektet.
Dessutom beskrivs en aktivitetsplan som översiktligt tar upp alla aktiviteter som kommer att
göras under projektet.

\subsubsection*{Kravspecifikation}
Kravspecifikationen har som mål att detaljerat beskriva de viktigaste kraven för kunden.
Den ska även beskriva andra krav som har mindre betydelse för kunden, men dessa behöver inte vara
lika detaljerade.

\subsubsection*{Kvalitetsplan}
För att produkten som utvecklas ska hålla hög kvalitet kommer teamet arbeta mot att applicera kvalitetsprocesser som är definierade i kvalitetsplanen\cite{bib-kvalitetsplan}.

\subsubsection*{Statusrapport}
Statusrapporten kommer fungera som en reflektion på hur allt förarbete har gått, vad som kommer
hända härnäst och vilka risker som finns framöver. Den fungerar som en mindre projektplan för
de olika iterationerna.

\subsubsection*{Systemanatomi}
Anatomin för systemet beskriver hur systemet är uppbyggt på olika nivåer, såsom funktioner, 
mjukvara, hårdvara. Den ger en större bild för hur systemet fungerar och i vilken miljö den kommer
befinna sig i.

\subsubsection*{Arkitekturbeskrivning}
Här beskrivs själva arkitekturen för systemet i mer detalj, hur olika submoduler kommer hänga ihop. Arkitekturens fördelar, nackdelar och möjliga utökningar kommer också diskuteras.

\subsubsection*{Testplan}
Detta dokument beskriver hur teamet ska testa de olika delarna av produkten och även hur de följer upp på tester.

\subsubsection*{Kandidatrapport}
Kandidatrapporten som ska produceras i slutet av projektet har som mål att beskriva projektet i sin helhet. Detta innefattar mål, arbetsprocess, slutprodukt och individuella forskningsområden.