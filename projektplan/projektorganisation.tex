\section{Projektorganisering}

\subsection{Roller}
\begin{center}
    \begin{tabular}{| c | c |}
        \hline
        \textbf{Namn} & \textbf{Roll} \\
        \hline
        \centering Alexander Wilkens & Teamledare\\
        \hline
        \centering Joel Almqvist & Kvalitetssamordnare\\
        \hline
        \centering Tim Håkansson & Dokumentansvarig\\
        \hline
        \centering Joel Oskarsson & Arkitekt\\
        \hline
        \centering Lieth Wahid & Utvecklingsledare\\
        \hline
        \centering Axel Löjdquist & Analysansvarig\\
        \hline
        \centering David Kjellström & Testledare\\
        \hline
        \centering Björn Detterfelt & Konfigurationsansvarig\\
        \hline
    \end{tabular}
\end{center}
\risksection{Teamledare}
Teamledaren ska se till att samtliga processer som ska utföras i projektets gång följs. Denna person representerar också gruppen utåt och har kontakt med handledaren. Om det behövs har teamledaren sista ordet.

\risksection{Kvalitetssamordnare}
Ansvarar för arbetsprocesser som ska hålla kvalitén av projektet på en hög nivå. Gör en budget av vad kvalitet får kosta och ansvarar för kvalitetsplanen.

\risksection{Dokumentansvarig}
Ser till att ansvara för samtliga dokument som gruppen ska producera. Även ansvarig för gruppens logotyp och dokumentmallar.

\risksection{Arkitekt}
Ansvarar för att arkitekturen av den tekniska delen av projektet. Gör övergripande teknikval och har det sista ordet på tekniska beslut.

\risksection{Utvecklingsledare}
Ansvarar för den mer detaljerade designen av den tekniska produkten. Leder utvecklingsarbetet och ser till att resten av gruppen har något att arbeta med.

\risksection{Analysansvarig}
Ansvarar för majoriteten av kundkontakt och jobbar ständigt med att ta reda på kundens verkliga behov. Har huvudansvar för kravspecifikationen.

\risksection{Testledare}
Beslutar systemets status genom att arbeta tillsammans med kvalitetssamordnaren för att testa så systemet uppnår kraven. Skriver testplan och testrapport.

\risksection{Konfigurationsansvarig}
Ansvarar för generell versionshantering i projektet. Arbetar mycket med utvecklingledaren och dokumentansvarig för att bestämma vilka arbetsprodukter som ska ingå i en utgåva(release).


\subsection{Kunskap och erfarenheter}
Alla projektmedlemmar har läst flera år på cilivingenjörsprogrammet i datateknik respektive mjukvaruteknik. Gruppen förväntas ha kunskaper och erfarenheter från de tidigare kurser och projekt som utförs i utbildning. Extra fokus ligger på de kurser som studiehandboken listar som förkunskapskrav.SKRIV REFERENS HÄR


\subsection{Utbildning}

\subsection{Kommunikation och rapporter}