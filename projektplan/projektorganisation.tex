\section{Projektorganisering}
Detta kapitel beskriver hur teamet är strukturerat, hur kommunikationen går tillväga samt hur rapporteringen kommer att ske.
\subsection{Generell struktur}
Generellt sett är projektet strukturerat genom att teamet utvecklar en produkt till kunden. Teamet använder sig av
handledaren för generella frågor kring projektstruktur samt för att se till att ett stadigt tempo upphålls.
Kommunikationen med handledaren går via teamledaren och kommunikation med kund sker via analysansvarig. Dessa roller
beskrivs vidare under \textit{\ref{subsec:roles} Roller}. Teamet har också skrivit ett gruppkontrakt\cite{bib-gruppkontrakt} som alla gått med på att följa.

\begin{figure}[h]
    \centering
    \includegraphics[scale=0.4]{struktur}
    \caption{Projektstruktur}
    \label{fig:struktur}
\end{figure}



\subsection{Roller}
\label{subsec:roles}
Nedan beskrivs vad de olika rollerna omfattar. Relationerna mellan dessa beskrivs i figur \ref{fig:struktur}.
\begin{center}
    \begin{tabular}{| l | l |}
        \hline
        \textbf{Namn} & \textbf{Roll} \\
        \hline
        \centering Alexander Wilkens & Teamledare\\
        \hline
        \centering Joel Almqvist & Kvalitetssamordnare\\
        \hline
        \centering Tim Håkansson & Dokumentansvarig\\
        \hline
        \centering Joel Oskarsson & Arkitekt\\
        \hline
        \centering Lieth Wahid & Utvecklingsledare\\
        \hline
        \centering Axel Löjdquist & Analysansvarig\\
        \hline
        \centering David Kjellström & Testledare\\
        \hline
        \centering Björn Detterfelt & Konfigurationsansvarig\\
        \hline
    \end{tabular}
\end{center}
\subsubsection*{Teamledare}
Teamledaren ska se till att samtliga processer som ska utföras under projektets gång följs. Denna person representerar också teamet utåt och har kontakt med handledaren. Om det behövs har teamledaren sista ordet.

\subsubsection*{Kvalitetssamordnare}
Kvalitetssamordnaren ansvarar för arbetsprocesser som ska hålla kvaliteten av projektet på en hög nivå. Samordnaren gör en budget av vad kvalitet får kosta, samtidigt som han ansvarar för kvalitetsplanen.

\subsubsection*{Dokumentansvarig}
Dokumentansvarig ansvarar för samtliga dokument som teamet ska producera. Rollen är även ansvarig för gruppens logotyp och dokumentmallar.

\subsubsection*{Arkitekt}
Arkitekten ansvarar för arkitekturen av den tekniska delen av projektet. Gör övergripande teknikval och har det sista ordet på tekniska beslut.

\subsubsection*{Utvecklingsledare}
Utvecklingsledaren ansvarar för den mer detaljerade designen av den tekniska produkten, leder utvecklingsarbetet och ser till att resten av teamet har något att arbeta med.

\subsubsection*{Analysansvarig}
Analysansvarig ansvarar för majoriteten av kundkontakt och jobbar ständigt med att ta reda på kundens verkliga behov. Huvudansvaret för kravspecifikationen ligger på denna roll.

\subsubsection*{Testledare}
Testledaren beslutar systemets status genom att arbeta tillsammans med kvalitetssamordnaren för att testa så systemet uppnår kraven. Testledaren skriver testplan och testrapport.

\subsubsection*{Konfigurationsansvarig}
Konfigurationsansvarig ansvarar för generell versionshantering i projektet. Rollen arbetar mycket med utvecklingledaren och dokumentansvarig för att bestämma vilka arbetsprodukter som ska ingå i en utgåva.



\subsection{Kunskap och erfarenheter}
Alla projektmedlemmar har åtminstone läst 2 år på civilingenjörsprogrammet i datateknik respektive mjukvaruteknik. Teamet förväntas ha kunskaper och erfarenheter från de tidigare kurser och projekt som slutförts i utbildning. Extra fokus ligger på de kurser som studiehandboken listar som förkunskapskrav till TDDD96\cite{bib-tddd96}.


\subsection{Utbildning}
Mycket av den kunskap som teamet kommer behöva ha för att utföra projektet kommer införskaffas via självstudier. Innan implementationsdelen av projekt drar igång kommer teamet också få chans att ha en genomgång av Cybercoms API och backend.
\\
Varje projektmedlem har ett ansvar att bli bekant med ett eventuellt ramverk eller en annan teknik som tänkts att användas inför en sprint.

\subsection{Kommunikation och rapportering}
Varje vecka under projektets gång ska teamet producera veckorapporter som lämnas in till gruppens handledare. Veckorapporten beskriver vad teamet jobbat med under den senaste veckan, vad de ska jobba med kommande vecka samt beskriva eventuella risker för projektet. Innan rapporten skickas in ska teamet ha möte för att bestämma innehållet och diskutera arbete inför veckan. En uppdaterad tidsrapport bifogas också.\\
\\
Kommunikationen i teamet sker antingen personligen eller via Slack. Kommunikation mellan teamet och handledaren sker via mail genom teamledaren. Kundkommunikation sker antingen i Slack eller via mail genom analysansvarig.

\pagebreak
