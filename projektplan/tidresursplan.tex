\section{Tids- och resursplan}

\subsection{Tidsrapportering}
För att hålla koll så att alla i teamet följer våran antalet timmar som ska läggas ner under
projektet kommer en tidsrapportering användas i form utav ett exceldokument. Dokumentet
innehåller nio olika blad, ett för varje teammedlem och ett för burndown charts.
De individuella används till att schemalägga alla arbetspass som jobbas under en vecka.
Varje arbetspass får vara max 2h och ska ge en kort beskrivning på vad som har gjorts och vem
man har sammarbetat med. Burndown används för att få en bättre bild utav hur alla ligger till
och få en uppfattning om någon halkar efter.
Dokumentet finns på \url{https://docs.google.com/spreadsheets/d/1mW3OsZbzP4Fu7PKR7oXfcZxurDALwAOuxchXx6rXuLE/edit?usp=sharing}


\subsection{Iterationsplan}
Projektet är uppdelat i tre projektinlämningar samt en inlämning för kandidatrapport. Den första inlämningen fokuserar på några av de dokument som nämns under inlämningar, medan de andra två är en blandning av dokument och kod. Tiden mellan inlämning ett och två är endast två veckor, alltså lika lång tid som en sprint, vilket betyder att endast en iteration av produkten kommmer utvecklas. Mellan inlämning två och tre finns det betydligt mycket mer tid vilket gör att gruppen hinner med fler sprints, och i sin tur fler iterationer av produkten.\\
\\
Under projektets andra iteration, alltså inför inlämning två, kommer gruppen arbeta med att skapa ett "skal" till produkten. Med detta menas en något abstrakt struktur där implementationen av de mer ingående delarna lämnas åt senare iterationer.

\subsection{Arbetssätt och sprintplan}
Gruppen har valt att arbeta på ett sätt som efterliknar Scrum. Efter första iterationen kommer kommer gruppen arbeta i två veckor långa sprints, där man i slutet av varje i sprint ska ha en produkt redo att visa kunden. Just från Scrum kommer gruppen använda sig av sprints, sprintmöten före och efter sprints samt en scrum-board. Resterande saker som finns i Scrum kommer ej användas då gruppen känner att det ej behövs.

Inför varje sprint kommer gruppen ha ett möte att bestämma aktiviteter som ska utföras. Varje aktivitet ska ha: \begin{itemize}
    \item Namn och beskrivning
    \item Prioritet
    \item Tidsuppskattning
    \item Referenser
    \item Utförare
    \item Tids arbetat samt uppskattade tid kvar
\end{itemize}
Om det visar sig att aktiviteterna tar slut innan sprinten är över ska ett möte bokas in för att snabbt komma på nya aktiviteter. Om det istället skulle vara så att det blir över aktiviteter så övergår dessa till nästa sprint med reviderade prioriteringar.

Efter varje varje sprint har gruppen ett sprintmöte där sprinten utvärderas. 


\subsection{Milstolpar}
Gruppens interna deadlines.

\begin{center}
    \begin{tabular}{| c | c | c | }
        \hline
        \textbf{\#} & \textbf{Namn} & \textbf{Datum} \\
        \hline
        \centering 1 & Dokument klara för inlämning 1 & Februari 12, 2018\\
        \hline
    \end{tabular}
\end{center}

\subsection{Tollgates}
\begin{center}
    \begin{tabular}{| c | c | c | }
        \hline
        \textbf{\#} & \textbf{Namn} & \textbf{Datum} \\
        \hline
        \centering 1 & Inlämning kopia kontrakt & Februari 2, 2018\\
        \hline
        \centering 2 & Inlämning 1 & Februari 19, 2018\\
        \hline
        \centering 3 & Inlämning 2 & Mars 5, 2018\\
        \hline
        \centering 4 & Inlämning 3 & April 23, 2018\\
        \hline
        \centering 5 & Kandidatrapport & Maj 7, 2018\\
        \hline
    \end{tabular}
\end{center}



\subsection{Saker att leverera}
Under projektets gång kommer diverse dokument levereras tillsammans med produkten som utvecklas. I tabellen nedan följer vilka saker som ska levereras vid de olika inlämningarna.

\begin{center}
    \begin{tabular}{| l | l l l |}
        \hline
        \textbf{Namn} & \multicolumn{3}{|c|}{ \textbf{Inlämning} } \\
        \hline
        \centering Produkt & & 2 & 3 \\
        \hline
        \centering Statusrapport & 1 & &\\
        \hline
        \centering Systemanatomi & 1 & &\\
        \hline
        \centering Arkitekturdokument & 1* & 2 &\\
        \hline
        \centering Testplan & 1* & 2 &\\
        \hline
        \centering Projektplan & 1 & 2 &\\
        \hline
        \centering Kravspecifikation & 1 & 2 &\\
        \hline
        \centering Kvalitetsplan & 1 & 2 &\\
        \hline
        \centering Testrapport & & 2 & \\
        \hline
        \centering Utvärdering av iteration 2 & & 2 &\\
        \hline
        \centering Kandidatrapport & & 2* &\\
        \hline
    \end{tabular}
\end{center}

*Dokumenten ska vara påbörjade till denna inlämning.

\subsection{Aktiviteter}

\subsubsection*{Testning}
Produkten som utvecklas under projektets gång kommer kontinuerligt bli testad på olika nivåer. Vidare information om hur produkten ska testas finns i projektets testplan \cite{bib-testplan}.




\subsubsection*{Processer för kvalité}


\subsection{Resurser}
Projektgruppen består av 8 medlemmar som har 400 timmar var att spendera på projektet. Ekonomin som finns är därmed totalt 3200 timmar. Projektgruppen har tillgång till en handledare som kan bistå med erfarenhet inom relevanta områden. Det finns även mycket kunskap hos anställda vid Cybercom som till viss del står till gruppens förfogande.\\

För användning i projektet finns ett existerande system i form av ett IoT-gränssnitt skapat av Cybercom. Detta system, dess dokumentation och kunskap hos Cybercom om systemet är resurser som kommer att användas i projektet. Förutom detta system finns flera open-source ramverk som kommer användas för att underlätta projektet. Dokumentation för dessa existerar online och kommer vara en viktig resurs.\\

Under arbetet har gruppen tillgång till lokaler hos kunden. Detta innefattar både arbetsplatser och konferensrum för möten. Kunden tillhandahåller också två datorer för utvecklingsarbetet. \\

\pagebreak
