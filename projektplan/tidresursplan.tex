\section{Tids- och resursplan}
Vad har vi för tids och resurs restriktioner och hur kan vi ta hjälp utav dessa för att utföra 
projektet?

\subsection{Tidsrapportering}
För att hålla koll så att alla i teamet följer våran antalet timmar som ska läggas ner under 
projektet kommer en tidsrapportering användas i form utav ett exceldokument. Dokumentet 
innehåller nio olika blad, ett för varje teammedlem och ett för burndown charts. 
De individuella används till att schemalägga alla arbetspass som jobbas under en vecka. 
Varje arbetspass får vara max 2h och ska ge en kort beskrivning på vad som har gjorts och vem 
man har sammarbetat med. Burndown används för att få en bättre bild utav hur alla ligger till 
och få en uppfattning om någon halkar efter.
Dokumentet finns på \url{https://docs.google.com/spreadsheets/d/1mW3OsZbzP4Fu7PKR7oXfcZxurDALwAOuxchXx6rXuLE/edit?usp=sharing}


\subsection{Iterationsplan}
Projektet är uppdelat i tre projektinlämningar samt en inlämning för kandidatrapport. Den första inlämningen fokuserar på några av de dokument som nämns under inlämningar, medan de andra två är en blandning av dokument och kod.

\subsection{Milstolpar}
Vilka deadlines har vi internt?

\begin{center}
    \begin{tabular}{| c | c | c | }
        \hline
        \textbf{\#} & \textbf{Namn} & \textbf{Datum} \\
        \hline
        \centering 1 & Dokument klara för inlämning 1 & Februari 12, 2018\\
        \hline
        \centering 2 & Dokument klara för inlämning 1 & Februari 12, 2018\\
        \hline
    \end{tabular}
\end{center}


När ska saker och ting vara gjort?

\subsection{Tollgates}
\begin{center}
    \begin{tabular}{| c | c | c | }
        \hline
        \textbf{\#} & \textbf{Namn} & \textbf{Datum} \\
        \hline
        \centering 1 & Inlämning kopia kontrakt & Februari 2, 2018\\
        \hline
        \centering 2 & Inlämning 1 & Februari 19, 2018\\
        \hline
        \centering 3 & Inlämning 2 & Mars 5, 2018\\
        \hline
        \centering 4 & Inlämning 3 & April 23, 2018\\
        \hline
        \centering 5 & Kandidatrapport & Maj 7, 2018\\
        \hline
    \end{tabular}
\end{center}



\subsection{Saker att leverera}
Vad ska vi leverera för något? Återkommer...

\begin{center}
    \begin{tabular}{| c | c | c | }
        \hline
        \textbf{\#} & \textbf{Namn} & \textbf{Inlämning} \\
        \hline
        \centering 1 & Projektplan & 1 \\
        \hline
        \centering 2 & Kravspecifikation & 1\\
        \hline
        \centering 3 & Kvalitetsplan & 1\\
        \hline
        \centering 4 & Statusrapport & 1\\
        \hline
        \centering 5 & Systemanatomi & 1\\
        \hline
        \centering 2 & Kravspecifikation & 1\\
        \hline
    \end{tabular}
\end{center}


\subsection{Resurser}
Vilka resurser har vi att jobba med?
laptop,mobil, backend...\\
\\
I projektet har gruppen tillgång till en hel del olika resurser. Nedan följer en lista på de mest relevant i detta projekt.\\
\begin{itemize}
\item 8 gruppmedlemmar
\item Cybercom API
\item Dokumentation
\item Open-source ramverk
\item Handledare
\item Anställda på Cybercom
\end{itemize}
