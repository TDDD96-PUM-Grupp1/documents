\section{Tid- och resursplan}
Detta kapitel beskriver hur projektgruppen ska arbeta och planera, samt utnyttja de resurser som finns till hands.

\subsection{Tidrapportering}
För att bevaka att alla i teamet lägger antalet timmar som förväntas under
projektet, kommer ett kalkylark\cite{bib-tidsrapportering} att användas för att tidrapportera. Dokumentet innehåller nio olika blad, ett för varje teammedlem och ett för 
burndown charts. De individuella används till att rapportera alla arbetspass som jobbas under 
en vecka. Varje arbetspass får vara max 2h och ska ge en kort beskrivning om vad som har gjorts 
och vem man har samarbetat med. Burndown charts används för att få en bättre bild av hur alla
ligger till i tidplanen och få en uppfattning om någon halkar efter.


\subsection{Iterationsplan}
Projektet är uppdelat i tre projektinlämningar samt en inlämning för kandidatrapport. Den första inlämningen fokuserar
på några av de dokument som nämns under \textit{\ref{subsec:deliverables} saker att leverera}, medan de andra två är en blandning av dokument och kod. Tiden mellan inlämning ett och två är endast två veckor, alltså lika lång tid som en sprint, vilket betyder att endast en iteration av produkten kommmer utvecklas. Mellan inlämning två och tre finns det betydligt mycket mer tid vilket gör att teamet hinner med fler sprints, och i sin tur fler iterationer av produkten.\\
\\
Under projektets andra iteration, alltså inför inlämning två, kommer teamet arbeta med att skapa ett ''skal'' till produkten. Detta innebär en något abstrakt struktur där implementationen av de mer ingående delarna lämnas åt senare iterationer.

\subsection{Arbetssätt och sprintplan}

Teamet har valt att arbeta på ett sätt som efterliknar Scrum\cite{bib-scrum}. Efter första iterationen kommer teamet arbeta i två veckor långa sprints, där man i slutet av varje i sprint ska ha en produkt redo att visa kunden. Just från Scrum kommer teamet använda sig av sprints, sprintmöten före och efter sprints samt en scrum-board. Resterande saker som finns i Scrum kommer ej användas då teamet känner att det ej behövs och endast kommer tillföra onödig overhead.\\
\\

Inför varje sprint kommer teamet ha ett möte att bestämma aktiviteter som ska utföras. Varje aktivitet ska ha: \begin{itemize}
    \item Namn och beskrivning
    \item Prioritet
    \item Tidsuppskattning
    \item Referenser
    \item Utförare
    \item Tid arbetat samt uppskattad tid kvar
\end{itemize}
Om det visar sig att aktiviteterna tar slut innan sprinten är över ska ett möte bokas in för att snabbt komma på nya aktiviteter. Om det istället skulle bli aktiviteter över övergår dessa till nästa sprint med reviderade prioriteringar.\\
\\
Teamet kommer sträva efter att arbeta i par, där varje par jobbar med en egen del. Dock uppmuntras
gruppen att sitta tillsammans för att snabbt kunna diskutera problem under projektets gång.\\
\\
Utöver sprintmöten kommer teamet träffas i början av varje vecka för att diskutera eventuella frågor samt sammanställa en veckorapport. Informella möten kan också förkomma under veckans gång.

\subsection{Milstolpar}
Teamets interna deadlines.

\begin{center}
    \begin{tabular}{| l | l | l | }
        \hline
        \textbf{\#} & \textbf{Namn} & \textbf{Datum} \\
        \hline
        \centering 1 & Dokument klara för inlämning 1 & Februari 12, 2018\\
        \hline
        \centering 2 & Slut utvecklingssprint 1 & , Mars 10, 2018\\
        \hline
        \centering 3 & Slut utvecklingssprint 2 & , Mars 29, 2018\\
        \hline
        \centering 4 & Slut utvecklingssprint 3 & , April 12, 2018\\
        \hline
    \end{tabular}
\end{center}

\subsection{Tollgates}
Externa deadlines för projketet.
\begin{center}
    \begin{tabular}{| l | l | l | }
        \hline
        \textbf{\#} & \textbf{Namn} & \textbf{Datum} \\
        \hline
        \centering 1 & Inlämning kopia av kontrakt & Februari 2, 2018\\
        \hline
        \centering 2 & Inlämning 1 & Februari 19, 2018\\
        \hline
        \centering 3 & Inlämning 2 & Mars 5, 2018\\
        \hline
        \centering 4 & Inlämning 3 & April 23, 2018\\
        \hline
        \centering 5 & Kandidatrapport & Maj 7, 2018\\
        \hline
    \end{tabular}
\end{center}



\subsection{Saker att leverera}
\label{subsec:deliverables}
Under projektets gång kommer diverse dokument levereras tillsammans med produkten. I tabellen nedan följer vilka saker som ska levereras vid de olika inlämningarna.

\begin{center}
    \begin{tabular}{| l | l | l | l |}
        \hline
        \textbf{Namn} & \multicolumn{3}{|c|}{ \textbf{Inlämning} } \\
        \hline
        \centering Produkt & & 2 & 3 \\
        \hline
        \centering Statusrapport & 1 & &\\
        \hline
        \centering Systemanatomi & 1 & &\\
        \hline
        \centering Arkitekturdokument & 1* & 2 &\\
        \hline
        \centering Testplan & 1* & 2 &\\
        \hline
        \centering Projektplan & 1 & 2 &\\
        \hline
        \centering Kravspecifikation & 1 & 2 &\\
        \hline
        \centering Kvalitetsplan & 1 & 2 &\\
        \hline
        \centering Testrapport & & 2 & \\
        \hline
        \centering Utvärdering av iteration 2 & & 2 &\\
        \hline
        \centering Kandidatrapport & & 2* &\\
        \hline
    \end{tabular}
\end{center}
*: Dokumnetet ska vara påbörjat. 

\subsection{Aktiviteter}

\subsubsection*{Testning}
Produkten som utvecklas under projektets gång kommer kontinuerligt bli testad på olika nivåer. Vidare information om hur produkten ska testas finns i projektets testplan\cite{bib-testplan}.

\subsubsection*{Processer för kvalité}
Teamet kommer tillämpa olika arbetsprocesser för att uppnå hög kvalité på produkten. Dessa beskrivs mer ingående i projekets kvalitetsplan\cite{bib-kvalitetsplan}.
Dokumenten ska vara påbörjade till denna inlämning.

\subsection{Resurser}
Projektgruppen består av 8 medlemmar som har 400 timmar var att spendera på projektet, vilket leder till totalt 3200 timmar. Projektgruppen har tillgång till en handledare som kan bistå med erfarenhet inom relevanta områden. Det finns även mycket kunskap hos anställda vid Cybercom som till viss del står till gruppens förfogande.\\
\\
För användning i projektet finns ett existerande system i form av ett IoT-gränssnitt skapat av Cybercom. Detta system,
dess dokumentation och Cybercoms kunskap om systemet är resurser som kommer att användas i projektet. Förutom detta
system finns flera open-source ramverk som kommer användas för att underlätta projektet. Dokumentation för dessa finns online och kommer vara en viktig resurs.\\
\\
Teamet har tillgång till lokaler hos kunden. Detta innefattar både arbetsplatser och konferensrum för möten. Kunden tillhandahåller också två datorer för utvecklingsarbetet. \\

\pagebreak
