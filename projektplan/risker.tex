\section{Risker och riskhantering}
Inom projektet behövs risker frambringas och evalueras för att få så lite motstånd som möjligt. 
Magnituden hos dessa risker beräknas genom produkten utav sannolikheten och inverkan.
Riskerna inför projektet är nedan:

\risk{Cybercoms backend är för långsam för våra krav}{1}{4}
{
    För att minimera inverkan på risken så ska backenden användas så lite som möjligt, och applikationen kommer inte följa arkitekturen \textit{tunn klient}.
    Om det skulle visa sig att backenden fortfarande inte klarar av att skicka information tillräckligt snabbt, 
    så skulle en lösning vara att få spelet att gå långsammare.
}

\risk{Sensordata varierar mycket mellan enheter}{2}{2}
{
    Om sensordatan varierar mycket mellan olika enheter, såsom precision eller brus,
    kan det leda till att utvecklingen blir långsammare då det kan ge oönskade problem som måste jobbas runt.
    För att undvika att detta blir ett problem så behövs tydligt efterforskning om hur enheterna fungerar när
    det gäller sensordata. För att fixa problemet om det skulle bli ett behövs koden struktureras upp för att
    hantera problemen tidigt, såsom att motverka brus genom interpolation.
}

\risk{Projektmedlemmar saknar laptops vilket leder till långsam utveckling}{1}{3}
{
    Två projektmedlemmar saknar laptops i teamet vilket kan tillföra med att utvecklingen tar längre tid
    då dessa eventuellt får sitta tillsammans med någon annan.
    Cybercom kommer förhoppningsvis att ordna två laptops som kan användas utav medlemmarna, men om det skulle
    visa sig vara ett problem kan medlemmarna utveckla mer i SU-salar med eller utan de andra.
}

\risk{Körtidsfel i JavaScript ger dålig utveckling}{3}{2}
{
    Under projektets gång kommer troligvis JavaScript att ge upphov till oförutsedda körtidsfel då språket inte är typat.
    Detta kan leda till långsammare utveckling då konstiga problem måste letas upp.
    Genom att skapa bra tester kan detta problemet minskas då testerna ska utformas så att oönskade typer hittas snabbt.
    Om problemen kvarstår ska mer resurser läggas på att testerna fungerar.
}

\risk{Arkitekturen lämpar sig inte för realtid}{1}{4}
{
    Då projektets största krav är att spelet ska vara i realtid är det viktigt att arkitekturen fungerar för just
    detta ändamål.
    För att mitigera risken kommer det läggas fokus på arkitekturen i början för att den ska klara av realtid, 
    men även tillsammans hålla koll så att arkitekturen följs.
    Skulle det vissa sig att våran arkitektur ändå inte håller så ska vi så snabbt som möjligt titta över den och bygga om.
}

\risk{Teamledare åker bort en vecka som leder till dålig utveckling}{1}{2}
{
    Då teamledaren lämnar en vecka i slutet utav april kan det leda till att utvecklingen kan
    hindras. För att undvika att detta blir ett problem ska teamledaren hålla alla andra i
    teamet uppdaterade på diverse områden som andra i teamet inte har koll på. I fallet då det skulle bli
    problematiskt finns det inte så mycket att göra, då teamledaren eventuellt inte kan ha någon form utav kontakt,
    teamet får med andra ord jobba på så gott det går under veckan.
}

\risk{Någon i teamet respekterar inte deadlines}{1}{3}
{
   I alla projekt i team finns det en risk att vissa inte respekterar deadlines, både interna och
   externa. För att mitigera detta så kommer teamet se till att alla är uppdaterade på
   diverse deadlines och att på möten hålla koll på vilka som halkar efter. Om det skulle bli
   ett problem ska det tas upp med personen i fråga, om det inte hjälper kontaktar vi handledaren
   eller eventuellt examinatorn för råd.
}

\risk{Oförutsedd dataförlust}{1}{4}
{
    En sak som inte får hända under projektets gång är att all kod eller data försvinner och 
    inte går att få tillbaka. Ett enkelt sätt att undvika att vi förlorar kod är att regelbundet
    versionshantera sin kod med git, även om den inte fungerar så ska den finnas i en personlig
    branch. Detta leder till att kod alltid finns utanför ens personliga dator och därmed mindre
    chans för förlust. Om det skulle vara ett problem med förlust så kan teamet endast gå 
    tillbaka till en tidigare version och fortsätta därifrån.
}

\risk{Kod som inte fungerar blir pushat till master}{1}{3}
{
    Tanken med master-branchen är att allt ska vara körbart och fungera korrekt, så om teamet ska
    demonstrera projektet ska allt köras som tänkt. För att hålla master-branchen fungerande
    kommer tester sättas upp på git som behöver gå igenom för att bli pushat. Dessutom kommer
    konfigurationsansvarig att behöva godkänna pull requests till mastern. Vid problem kan teamet
    gå tillbaka till en tidigare version på mastern vid demonstration och därefter utöka tester
    för att minimera risken att det händer igen.
}

\risk{Brist på kunskap ger dålig kommunikation}{2}{3}
{
   När teamet diskuterar saker finns det risk till att vissa i teamet inte har tillräckligt med
   kunskap inom ämnet för att leda en meningsfull diskussion, vilket kan leda till missförstånd
   eller att personen inte medför till diskussionen. Genom att redan i början av projektet ha en uppfattning om teamets erfarenheter kan medlemmarna bättre förbereda sig
   inför diskussionen. Personen i fråga ska även meddela när det kommer upp ämnen som den inte
   har erfarenheter utav. Vid tillfället då en person har problem och inte meddela detta så
   ska teamet hålla koll så att alla tillför till diskusionen.
}

\pagebreak
