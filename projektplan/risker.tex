\section{Risker}
Inom projektet behövs risker frambringas och evalueras för att få så lite motstånd som möjligt. 
Magnituden hos dessa risker beräknas genom produkten utav sannolikheten och inverkan. 
Riskerna inför projektet är nedan:

\risk{Cybercoms backend för långsam för våra krav}{1}{4}
{
    För att minimera inverkan på risken så ska backenden användas så lite som möjligt.
    Backended kommer inte att fungera som en fat-server utan all "server" kod kommer ligga på UI:n. 
    Om det skulle vissa sig att backenden fortfarande inte klarar av att skicka information tillräckligt snabbt, 
    så skulle en lösning vara att få spelet att gå långsammare beträffande kontroller.
}

\risk{Sensordata varierar mycket mellan enheter}{2}{2}
{
    Om sensordatan varierar mycket mellan olika enheter, såsom precision eller brus,
    kan det leda till att utvecklingen blir långsammare då det kan ge oönskade problem som måste jobbas runt.

    För att undvika att detta blir ett problem så behövs tydligt efterforskning om hur enheterna fungera när
    det gäller sensordata. För att fixa problemet om det skulle bli ett behövs koden struktureras upp för att
    hantera problemen tidigt, såsom att motverka brus genom interpolation.
}

\risk{Projektmedlemmar saknar laptops}{1}{3}
{
    Två projektmedlemmar saknar laptops i gruppen vilket kan tillföra med att utvecklingen blir tar längre tid
    då dessa eventuellt får sitta tillsammans med någon annan.
    
    Cybercom kommer förhoppningsviss att ordna två laptops som kan användas utav medlemmarna, men om det skulle
    Visar det sig vara ett problem kan medlemmarna utveckla mer i SU-salar med eller utan de andra.
}

\risk{Körtidsfel i JavaScript ger dålig utveckling}{3}{2}
{
    Under projektets gång kommer troligviss JavaScript att ge upphov till konstiga körtidsfel då språket inte är typat.
    Detta kan leda till långsammare utveckling då konstiga problem måste letas upp.

    Genom att skapa bra tester kan detta problemet minskas då testerna ska utformas så att oönskade typer hittas snabbt.
    Om problemen kvarstår ska mer resurser läggas på att testerna fungerar.
}

\risk{Arkitekturen lämpar sig inte för realtid}{1}{4}
{
    Då projektets största krav är att spelet ska vara i realtid är det viktigt att arkitekturen fungerar till just
    detta endamålet.

    För att mitigera risken kommer det läggas fokus på arkitekturen i början för att den ska klara av realtid, 
    men även tillsammans hålla koll så att arkitekturen följs.
    Skulle det vissa sig att våran arkitektur ändå inte håller så ska vi så snabbt som möjligt titta över den och bygga om.
}

\risk{Teamledare åker bort en vecka}{1}{2}
{

}
