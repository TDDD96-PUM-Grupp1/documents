\documentclass[10pt]{article}
\usepackage{calc}
\usepackage{subfiles}
\usepackage{hyperref}
\usepackage[utf8]{inputenc}
\usepackage[swedish]{babel}
\usepackage[margin=2cm]{geometry}
\usepackage[backend=bibtex,sorting=none,style=numeric,natbib=true]{biblatex}
\usepackage{graphicx}
\graphicspath{ {images/} }
\usepackage{filecontents}
\selectlanguage{swedish}
\addbibresource{references.bib}

% Variable för att räkna ut magnituden på en risk.
\newcounter{riskmagnitude}
% Macros för risker, för att strukturera upp dem.
\newcommand{\risk}[4]{
    \risksection{#1}
    \riskmagnitude{#2}{#3}
    \newline #4
}
\newcommand{\risksection}[1]{
    \subsection*{#1}
    \addcontentsline{toc}{subsection}{#1}
}

\newcommand{\riskmagnitude}[2]{
    \setcounter{riskmagnitude}{#1*#2} 
    \noindent Sannolikhet: #1 
    \newline Inverkan: #2 
    \newline Magnitud: \arabic{riskmagnitude}
}

\tolerance=1
\emergencystretch=\maxdimen
\hyphenpenalty=10000
\hbadness=10000

\begin{document}

\pagenumbering{gobble}
\title{Arkitekturbeskrivning\\
    \large Projektgrupp 1}
\author{
    Joel Almqvist\\
    \texttt{joeal360@student.liu.se}
    \and
    Björn Detterfelt\\
    \texttt{bjode786@student.liu.se}
    \and
    Tim Håkansson\\
    \texttt{timha404@student.liu.se}
    \and
    David Kjellström\\
    \texttt{davkj168@student.liu.se}
    \and
    Axel Löjdquist\\
    \texttt{axelo225@student.liu.se}
    \and
    Joel Oskarsson\\
    \texttt{joeos014@student.liu.se}
    \and
    Lieth Wahid\\
    \texttt{liewa893@student.liu.se}
    \and
    Alexander Wilkens\\
    \texttt{alewi684@student.liu.se}
}

\date{\today \\Version 2.0}

\maketitle


\pagenumbering{arabic}
\section{Projektbeskrivning}

Detta kapitel redogör syftet till varför det här projektet utförs samt beskriver vilka dokument som ska produceras.

\subsection{Definition}
\begin{itemize}[leftmargin=5cm]
  \definition{Scrum-board}
  \definition{Burndown-chart}
  \definition{Tunn klient}
\end{itemize}


\subsection{Bakgrund}
Detta projekt utförs som del av kursen TDDD96 - Kandidatprojekt i mjukvaruutveckling. Teamet har fått i uppdrag av Cybercom att utföra projektet \textit{Realtidsmultiplayerspel på IoT-backend}. Cybercom vill ha ett spel som kan demonstrera deras backend på ett snyggt sätt och visa upp hur bra det fungerar.

\subsection{Begränsningar}
Eftersom projektet utförs som del av en universitetskurs finns det en del tidsbegränsning som måste uppmärksammas. Projektet utförs under hela vårterminen men avslutas efter det. En del hårdvarubegränsningar finns, till exempel har ingen i projektgruppen tillgång till en iPhone vilket leder till begränsad testning på den enheten. Utvecklingen är också något begränsad till de egna datorer projektgruppen äger, förutom de eventuella datorer Cybercom kan låna ut.

\subsection{Mål och syfte}
Målet med projektet är att utveckla ett spel som använder sig av Cybercoms backend. Realtidsdelen av spelet används som en verifiering av Cybercom att deras backend uppfyller de prestandakrav som företaget strävar efter. Projektet ska också producera de nödvändiga dokument som krävs för att uppfylla kursens krav\cite{bib-tddd96}. Projektgruppen kommer sträva mot att produkten som utvecklas uppfyller de krav som finns i kravspecifikationen\cite{bib-kravspec}. Projektet och kursen avslutas med en gemensam kandidatrapport.

\subsection{Dokument}
Under projektets gång kommer olika dokument att produceras för att underlätta utvecklingen.
Dessutom kommer en sista kandidatrapport, som beskriver allt om och runtomkring projektet,
produceras.

\subsubsection*{Projektplan}
Detta dokument beskriver allt som finns runtomkring projektet, beskrivning utav projektet, 
resurser som teamet har tillgång till, processer som kommer användas och risker inom projektet.
Dessutom beskrivs en aktivitetsplan som översiktligt tar upp alla aktiviteter som kommer att
göras under projektet.

\subsubsection*{Kravspecifikation}
Kravspecifikationen har som mål att detaljerat beskriva de viktigaste kraven för kunden.
Den ska även beskriva andra krav som har mindre mening för kunden, men dessa behöver inte vara
like detaljerade.

\subsubsection*{Kvalitetsplan}
För att få en meningsfull process kommer detta dokumentet beskriva dessa för att få ut ett 
kvalitativt projekt i slutändan.

\subsubsection*{Statusrapports}
Statusrapporten kommer fungera som en reflektion på hur förstudierna har gått, vad som kommer
hända härnäst och vad för risker som finns framöver. Det fungera som en mindre projektplan för
de olika iterationerna.

\subsubsection*{Systemanatomi}
Anatomin för systemet beskriver hur systemet är uppbyggt på olika nivåer, såsom funktioner, 
mjukvara, hårdvara. Det ger en större bild för hur systemet fungera och i vilken miljö den kommer
befinna sig i.

\subsubsection*{Arkitekturdokument}
Här beskrivs själva arkitetkturen för systemet i mer detalj, hur olika submoduler kommer hänga
ihop. Dessutom beskrivs även alternativa arkitekturer och varför dessa inte är relevanta för
projektet.

\subsubsection*{Testplan}
Detta dokument beskriver hur teamet ska testa sina olika funktioner och även hur de följer upp på
dessa.

\subsubsection*{Kandidatrapport}
Kandidaten som ska produceras i slutet utav projektet har som mål att beskriva hela projektet i
sin helhet. Allt som mål, arbetsprocess, slutprodukt och individuella forskningsområden.



\subsection{Start och slut}
Projektet började den 15 januari och förväntas avslutas den 28 maj med en slutversion av produkten.

\pagebreak

\section{Tids- och resursplan}
Vad har vi för tids och resurs restriktioner och hur kan vi ta hjälp utav dessa för att utföra
projektet?

\subsection{Tidsrapportering}
För att hålla koll så att alla i teamet följer våran antalet timmar som ska läggas ner under
projektet kommer en tidsrapportering användas i form utav ett exceldokument. Dokumentet
innehåller nio olika blad, ett för varje teammedlem och ett för burndown charts.
De individuella används till att schemalägga alla arbetspass som jobbas under en vecka.
Varje arbetspass får vara max 2h och ska ge en kort beskrivning på vad som har gjorts och vem
man har sammarbetat med. Burndown används för att få en bättre bild utav hur alla ligger till
och få en uppfattning om någon halkar efter.
Dokumentet finns på \url{https://docs.google.com/spreadsheets/d/1mW3OsZbzP4Fu7PKR7oXfcZxurDALwAOuxchXx6rXuLE/edit?usp=sharing}


\subsection{Iterationsplan}
Projektet är uppdelat i tre projektinlämningar samt en inlämning för kandidatrapport. Den första inlämningen fokuserar på några av de dokument som nämns under inlämningar, medan de andra två är en blandning av dokument och kod.

\subsection{Milstolpar}
Vilka deadlines har vi internt?

\begin{center}
    \begin{tabular}{| c | c | c | }
        \hline
        \textbf{\#} & \textbf{Namn} & \textbf{Datum} \\
        \hline
        \centering 1 & Dokument klara för inlämning 1 & Februari 12, 2018\\
        \hline
        \centering 2 & Dokument klara för inlämning 1 & Februari 12, 2018\\
        \hline
    \end{tabular}
\end{center}


När ska saker och ting vara gjort?

\subsection{Tollgates}
\begin{center}
    \begin{tabular}{| c | c | c | }
        \hline
        \textbf{\#} & \textbf{Namn} & \textbf{Datum} \\
        \hline
        \centering 1 & Inlämning kopia kontrakt & Februari 2, 2018\\
        \hline
        \centering 2 & Inlämning 1 & Februari 19, 2018\\
        \hline
        \centering 3 & Inlämning 2 & Mars 5, 2018\\
        \hline
        \centering 4 & Inlämning 3 & April 23, 2018\\
        \hline
        \centering 5 & Kandidatrapport & Maj 7, 2018\\
        \hline
    \end{tabular}
\end{center}



\subsection{Saker att leverera}
Vad ska vi leverera för något? Återkommer...

\begin{center}
    \begin{tabular}{| c | c | c | }
        \hline
        \textbf{\#} & \textbf{Namn} & \textbf{Inlämning} \\
        \hline
        \centering 1 & Projektplan & 1 \\
        \hline
        \centering 2 & Kravspecifikation & 1\\
        \hline
        \centering 3 & Kvalitetsplan & 1\\
        \hline
        \centering 4 & Statusrapport & 1\\
        \hline
        \centering 5 & Systemanatomi & 1\\
        \hline
        \centering 2 & Kravspecifikation & 1\\
        \hline
    \end{tabular}
\end{center}


\subsection{Resurser}
Projektgruppen består av 8 medlemmar som har 400 timmar var att spendera på projektet. Ekonomin som finns är därmed totalt 3200 timmar. Projektgruppen har tillgång till en handledare som kan bistå med erfarenhet inom relevanta områden. Det finns även mycket kunskap hos anställda vid Cybercom som till viss del står till gruppens förfogande.\\

För användning i projektet finns ett existerande system i form av ett IoT-gränssnitt skapat av Cybercom. Detta system, dess dokumentation och kunskap hos Cybercom om systemet är resurser som kommer att användas i projektet. Förutom detta system finns flera open-source ramverk som kommer användas för att underlätta projektet. Dokumentation för dessa existerar online och kommer vara en viktig resurs.\\

Under arbetet har gruppen tillgång till lokaler hos kunden. Detta innefattar både arbetsplatser och konferensrum för möten. Kunden tillhandahåller också två datorer för utvecklingsarbetet. \\

\section{Projektorganisering}
Detta kapitel beskriver hur teamet är strukturerat, hur kommunikationen går tillväga samt hur rapporteringen kommer att ske.
\subsection{Generell struktur}
Generellt sett är projektet strukturerat genom att teamet utvecklar en produkt till kunden. Teamet använder sig av
handledaren för generella frågor kring projektstruktur samt för att se till att ett stadigt tempo upphålls.
Kommunikationen med handledaren går via teamledaren och kommunikation med kund sker via analysansvarig. Dessa roller
beskrivs vidare under \textit{\ref{subsec:roles} Roller}. Teamet har också skrivit ett gruppkontrakt\cite{bib-gruppkontrakt} som alla gått med på att följa.

\begin{figure}[h]
    \centering
    \includegraphics[scale=0.4]{struktur}
    \caption{Projektstruktur}
    \label{fig:struktur}
\end{figure}



\subsection{Roller}
\label{subsec:roles}
Nedan beskrivs vad de olika rollerna omfattar. Relationerna mellan dessa beskrivs i figur \ref{fig:struktur}.
\begin{center}
    \begin{tabular}{| l | l |}
        \hline
        \textbf{Namn} & \textbf{Roll} \\
        \hline
        \centering Alexander Wilkens & Teamledare\\
        \hline
        \centering Joel Almqvist & Kvalitetssamordnare\\
        \hline
        \centering Tim Håkansson & Dokumentansvarig\\
        \hline
        \centering Joel Oskarsson & Arkitekt\\
        \hline
        \centering Lieth Wahid & Utvecklingsledare\\
        \hline
        \centering Axel Löjdquist & Analysansvarig\\
        \hline
        \centering David Kjellström & Testledare\\
        \hline
        \centering Björn Detterfelt & Konfigurationsansvarig\\
        \hline
    \end{tabular}
\end{center}
\subsubsection*{Teamledare}
Teamledaren ska se till att samtliga processer som ska utföras under projektets gång följs. Denna person representerar också teamet utåt och har kontakt med handledaren. Om det behövs har teamledaren sista ordet.

\subsubsection*{Kvalitetssamordnare}
Kvalitetssamordnaren ansvarar för arbetsprocesser som ska hålla kvaliteten av projektet på en hög nivå. Samordnaren gör en budget av vad kvalitet får kosta, samtidigt som han ansvarar för kvalitetsplanen.

\subsubsection*{Dokumentansvarig}
Dokumentansvarig ansvarar för samtliga dokument som teamet ska producera. Rollen är även ansvarig för gruppens logotyp och dokumentmallar.

\subsubsection*{Arkitekt}
Arkitekten ansvarar för arkitekturen av den tekniska delen av projektet. Gör övergripande teknikval och har det sista ordet på tekniska beslut.

\subsubsection*{Utvecklingsledare}
Utvecklingsledaren ansvarar för den mer detaljerade designen av den tekniska produkten, leder utvecklingsarbetet och ser till att resten av teamet har något att arbeta med.

\subsubsection*{Analysansvarig}
Analysansvarig ansvarar för majoriteten av kundkontakt och jobbar ständigt med att ta reda på kundens verkliga behov. Huvudansvaret för kravspecifikationen ligger på denna roll.

\subsubsection*{Testledare}
Testledaren beslutar systemets status genom att arbeta tillsammans med kvalitetssamordnaren för att testa så systemet uppnår kraven. Testledaren skriver testplan och testrapport.

\subsubsection*{Konfigurationsansvarig}
Konfigurationsansvarig ansvarar för generell versionshantering i projektet. Rollen arbetar mycket med utvecklingledaren och dokumentansvarig för att bestämma vilka arbetsprodukter som ska ingå i en utgåva.



\subsection{Kunskap och erfarenheter}
Alla projektmedlemmar har åtminstone läst 2 år på civilingenjörsprogrammet i datateknik respektive mjukvaruteknik. Teamet förväntas ha kunskaper och erfarenheter från de tidigare kurser och projekt som slutförts i utbildning. Extra fokus ligger på de kurser som studiehandboken listar som förkunskapskrav till TDDD96\cite{bib-tddd96}.


\subsection{Utbildning}
Mycket av den kunskap som teamet kommer behöva ha för att utföra projektet kommer införskaffas via självstudier. Innan implementationsdelen av projekt drar igång kommer teamet också få chans att ha en genomgång av Cybercoms API och backend.
\\
Varje projektmedlem har ett ansvar att bli bekant med ett eventuellt ramverk eller en annan teknik som tänkts att användas inför en sprint.

\subsection{Kommunikation och rapportering}
Varje vecka under projektets gång ska teamet producera veckorapporter som lämnas in till gruppens handledare. Veckorapporten beskriver vad teamet jobbat med under den senaste veckan, vad de ska jobba med kommande vecka samt beskriva eventuella risker för projektet. Innan rapporten skickas in ska teamet ha möte för att bestämma innehållet och diskutera arbete inför veckan. En uppdaterad tidsrapport bifogas också.\\
\\
Kommunikationen i teamet sker antingen personligen eller via Slack. Kommunikation mellan teamet och handledaren sker via mail genom teamledaren. Kundkommunikation sker antingen i Slack eller via mail genom analysansvarig.

\pagebreak

\section{Risker}
Inom projektet behövs risker frambringas och evalueras för att få så lite motstånd som möjligt. 
Magnituden hos dessa risker beräknas genom produkten utav sannolikheten och inverkan. 
Riskerna inför projektet är nedan:

\risk{Cybercoms backend för långsam för våra krav}{1}{4}
{
    För att minimera inverkan på risken så ska backenden användas så lite som möjligt.
    Backended kommer inte att fungera som en fat-server utan all "server" kod kommer ligga på UI:n. 
    Om det skulle vissa sig att backenden fortfarande inte klarar av att skicka information tillräckligt snabbt, 
    så skulle en lösning vara att få spelet att gå långsammare beträffande kontroller.
}

\risk{Sensordata varierar mycket mellan enheter}{2}{2}
{
    Om sensordatan varierar mycket mellan olika enheter, såsom precision eller brus,
    kan det leda till att utvecklingen blir långsammare då det kan ge oönskade problem som måste jobbas runt.

    För att undvika att detta blir ett problem så behövs tydligt efterforskning om hur enheterna fungera när
    det gäller sensordata. För att fixa problemet om det skulle bli ett behövs koden struktureras upp för att
    hantera problemen tidigt, såsom att motverka brus genom interpolation.
}

\risk{Projektmedlemmar saknar laptops}{1}{3}
{
    Två projektmedlemmar saknar laptops i gruppen vilket kan tillföra med att utvecklingen blir tar längre tid
    då dessa eventuellt får sitta tillsammans med någon annan.
    
    Cybercom kommer förhoppningsviss att ordna två laptops som kan användas utav medlemmarna, men om det skulle
    Visar det sig vara ett problem kan medlemmarna utveckla mer i SU-salar med eller utan de andra.
}

\risk{Körtidsfel i JavaScript ger dålig utveckling}{3}{2}
{
    Under projektets gång kommer troligviss JavaScript att ge upphov till konstiga körtidsfel då språket inte är typat.
    Detta kan leda till långsammare utveckling då konstiga problem måste letas upp.

    Genom att skapa bra tester kan detta problemet minskas då testerna ska utformas så att oönskade typer hittas snabbt.
    Om problemen kvarstår ska mer resurser läggas på att testerna fungerar.
}

\risk{Arkitekturen lämpar sig inte för realtid}{1}{4}
{
    Då projektets största krav är att spelet ska vara i realtid är det viktigt att arkitekturen fungerar till just
    detta endamålet.

    För att mitigera risken kommer det läggas fokus på arkitekturen i början för att den ska klara av realtid, 
    men även tillsammans hålla koll så att arkitekturen följs.
    Skulle det vissa sig att våran arkitektur ändå inte håller så ska vi så snabbt som möjligt titta över den och bygga om.
}

\risk{Teamledare åker bort en vecka}{1}{2}
{

}


\printbibliography
\addcontentsline{toc}{section}{\refname}

\end{document}
