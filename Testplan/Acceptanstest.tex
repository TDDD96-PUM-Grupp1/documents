\documentclass[10pt]{article}

\usepackage[utf8]{inputenc}
\usepackage{color}
\usepackage{scrextend}

\title{Acceptanstestplan}

\author{
	Projektgrupp 1\\
	Joel Oskarsson\\
	\texttt{joeos014@student.liu.se}
	\and
	LastName, FirstName\\
	\texttt{first.last@xxxxx.com}
	\and
	LastName, FirstName\\
	\texttt{first.last@xxxxx.com}
	\and
	LastName, FirstName\\
	\texttt{first.last@xxxxx.com}
	\and
	LastName, FirstName\\
	\texttt{first.last@xxxxx.com}
  	\and
  	LastName, FirstName\\
  	\texttt{first.last@xxxxx.com}
  	\and
  	LastName, FirstName\\
  	\texttt{first.last@xxxxx.com}
  	\and
  	LastName, FirstName\\
  	\texttt{first.last@xxxxx.com}
}

\begin{document}

\maketitle
\pagebreak
\tableofcontents
\pagebreak
\section{Inledning}
	I den här testplanen kommer det mest vara information angående krav och hur de kan testas av kund.
	\subsection{Revisionshistorik}

	
	\begin{center}
 	   \begin{tabular}{| l | l | l |  l | }
 	       \hline
 	       \textbf{Version} & \textbf{Datum} & \textbf{Förändring och kommentar} & \textbf{Ansvarig} \\
 	       \hline
 	       \centering 0.1 & 2018-02-18 & Första utkast & David Kjellström\\
 	       \hline
 	   \end{tabular}
	\end{center}


	\subsection{Definitioner}
		\begin{itemize}
		\item [Acceptanstest --]Slutgiltiga testet som kund utför för att se att produkten lever upp till förväntningarna
		\item [Enhetstest --]Testa varje enhet så den fungerar när den är färdig
		\item [Integrationstest --]Testa att en ny enhet som läggs till i projektet fungerar som den ska tillsammans med de andra enheterna
		\item [Kontroller --]En mobil eller surfplatta som kör kontrolldelen av applikationen
		\item [Kund --]Cybercom Sweden
		\item [Regressionstest --]Testa ny kod enligt gamla parametrar för att säkerställa att ingen funktionalitet försvunnit
		\item [Systemtest --]Test för att säkerställa att enheten uppfyller kraven för projektet		
		\item [UI --]User Interface - Den del av applikationen som visar spelplanen
		\end{itemize}
	
	\subsection{Referenser}
		\begin{itemize}
		\item [1] Kravspecifikation
		\item [2] Projektplan
		\item [3] Kvalitetsplan
		\end{itemize}

	
\section{Testföremål}
	Precis som i systemtestsplanen innefattar acceptanstesterna alla krav på kravspecifikationen.
	



\section{Tillvägagångssätt}
	Kund kommer få komplett kontroll över systemet och på egen hand testa det efter sina egna kriterier, och se så det lever upp till de överrenskomna kraven. \\
	
	

\section{Föremåls godkännande/misslyckande kriterier}
	För godkännande krävs det att systemet lever upp till de överrenskomna kraven.


\section{Avbrott och fortsättningskriterier}
	Om ett av kraven inte uppfylls, ska det ha stoppats redan vid systemtesterna. Om projektgruppen ändå kommer till denna punkt är det viktigt att föra dialog med kund angående kravet i fråga.\\
	\\
	Största anledningen till att man kan hamna här beror på tvetydiga krav. Det är har förhoppningsvis mitigierats till en viss grad med hjälp av en grundlig genomgång utav kravspecifikationen.


\section{Test leveranser {\color{red}TODO}}
	

	
\section{Ansvarsområden}
	Under acceptanstestet är ansvaret på kund att de verifierar att de fått den produkt de frågat efter. Analysansvarig har extra ansvar att ta eventuella diskussioner som kan uppstå runt krav.
	

	

	
	
\section{Godkännande}
	För att Testplanen ska vara avklarad behöver följande vara godkänt:
	\begin{itemize}
	 \item Samtliga krav godkända direkt från kund.
	 \item {\color{red}???}
	\end{itemize}
	






\end{document}
