\title{Projektplan\\
    \large Projektgrupp 1}

\author{
    Joel Almqvist\\
    \texttt{xxxxxxxx@student.liu.se}
    \and
    Björn Detterfelt\\
    \texttt{xxxxxxxx@student.liu.se}
    \and
    Tim Håkansson\\
    \texttt{timha404@student.liu.se}
    \and
    David Kjellström\\
    \texttt{xxxxxxxx@student.liu.se}
    \and
    Axel Löjdquist\\
    \texttt{xxxxxxxx@student.liu.se}
    \and
    Joel Oskarsson\\
    \texttt{joeos014@student.liu.se}
    \and
    Lieth Wahid\\
    \texttt{xxxxxxxx@student.liu.se}
    \and
    Alexander Wilkens\\
    \texttt{xxxxxxxx@student.liu.se}
}

\date{Februari 19, 2018}

\maketitle
\pagebreak

\titel{Acceptanstestplan}

\begin{document}

\pagenumbering{gobble}

\maketitle
\pagebreak
	\section*{Dokumenthistorik}

	
	\begin{center}
 	   \begin{tabular}{| l | l | p{12cm} |  }
 	       \hline
 	       \textbf{Version} & \textbf{Datum} & \textbf{Förändring och kommentar} \\
 	       \hline
 	       \centering 0.1 & 2018-02-12 & Första utkast\\
		\hline
 	       \centering 1.0 & 2018-02-12 & Iteration 2\\
 	       \hline
 	   \end{tabular}
	\end{center}
\pagebreak
\tableofcontents
\pagebreak

\pagenumbering{arabic}
\section{Inledning}
	I den här testplanen kommer det mest vara information angående krav och hur de kan testas av kund.

	


  \section{Definitioner}
\begin{itemize}[leftmargin=3cm]
  \item [Cachning] Temporär lagning av data för snabb åtkomst
  \item [Instans] En spelsession som startas från UI-applikationen och spelare kan gå med i för att spela spelet tillsammans
  \item [IoT-backend] Existerande system som kan dirigera data mellan många uppkopplade enheter
  \item [Kontroll-applikation] Applikation som körs på en mobil eller surfplatta och styr spelet
  \item [Progressive Web Apps] Ett mellanting mellan en hemsida och en applikation. Med en PWA behöver man inte ladda ner en app, men den ger viss funktionalitet som appar har. \cite{bib-pwa}
  \item [Resurs] Media som används i spelet, t.ex. bilder och ljud
  \item [Sensor] En sensor som sitter på kontroll-applikationen och inte är en pekskärm, t.ex. en accelerometer
  \item [Server-klient-modell] Struktur på ett system där någon enhet tillhandahåller resurser, information eller tjänster och flera andra enheter interagerar med denna
  \item [Spelläge] En utökning av grundspelet som definierar speciella regler och spelmekanik
  \item [Spelmekanik] Regler och möjligheter som definierar ett spel
  \item [Tunn klient] Specialfall av server-klient-modell där mycket få beräkningar sker på klienten
  \item [UI-applikation] Applikationen som kör själva spelet och visar upp spelplanen
  \item [Use Case Map] Diagram som illustrerar hur olika händelser interagerar med arkitekturen \cite[p.~30--33]{bib-architecture-primer}
\end{itemize}

	

	
\section{Testföremål}
	Precis som i systemtestsplanen innefattar acceptanstesterna alla krav på kravspecifikationen.
	
	\begin{tabular}{| p{1.5cm} | p{6cm} | p{8cm}|}
	
  \hline
    \multicolumn{2}{|c|}{Krav}&{Test}\\
    \hline


		Krav 35& Enkätens\cite {bib-kvalitetsplan} genomsnittliga betyg för responsivitet ska åtminstone vara 6 av 10 vid en undersökning av 20 deltagare. & En grupp på 20 deltagare ska utvärdera responsiviteten i slutet av projektet med betyg 6 av 10. \\
		\hline
		Krav 36& Enkätens\cite {bib-kvalitetsplan} genomsnittliga betyg för användbarhet ska åtminstone vara 6 av 10 vid en undersökning av 20 deltagare. & En grupp på 20 deltagare ska utvärdera användarbarheten i slutet av projektet med betyg 6 av 10. \\
		\hline
		Krav 37& Enkätens\cite {bib-kvalitetsplan} genomsnittliga betyg för den alternativa styrningen ska åtminstone vara 6 av 10 vid en undersökning av 20 deltagare. & En grupp på 20 deltagare ska utvärdera den alternativa styrningen i slutet av projektet med betyg 6 av 10. \\
		\hline
		Krav 38&Enkätens\cite{bib-kvalitetsplan} genomsnittliga betyg för ''lätt att förstå vad som händer'' ska åtminstone vara 6 av 10 vid en undersökning av 20 deltagare. & En grupp på 20 deltagare ska utvärdera hur ''lätt spelet är att förstå'' i slutet av projektet med betyg 6 av 10. \\
		\hline

   
  \end{tabular}
  


\section{Tillvägagångssätt}
	Kund kommer få komplett kontroll över systemet och på egen hand testa det efter sina egna kriterier, och se så det lever upp till de överrenskomna kraven. Det kommer även tillsättas en kontrollgrupp som ska få testa produkten och svara på en enkät för att säkerställa produktens kvalitet.
	
	

\section{Föremåls godkännande/misslyckande kriterier}
	För godkännande krävs det att systemet lever upp till de överrenskomna kraven.


\section{Avbrott och fortsättningskriterier}
	Om ett av kraven inte uppfylls, ska det ha stoppats redan vid systemtesterna. Om projektgruppen ändå kommer till denna punkt är det viktigt att föra dialog med kund angående kravet i fråga.\\
	\\
	Största anledningen till att man kan hamna här beror på tvetydiga krav. Det har förhoppningsvis mitigierats till en viss grad med hjälp av en grundlig genomgång utav kravspecifikationen.


\section{Test leveranser }
	Produkten ska vara levererad till kund där de säkerställer att produkten fungerar som överrenskommet.
\\	
	Vidare information angående leveransdatum kan återfinnas i Master Testplan.
	

	
\section{Ansvarsområden}
	Under acceptanstestet är ansvaret på kund att de verifierar att de fått den produkt de frågat efter. Analysansvarig har extra ansvar att ta eventuella diskussioner som kan uppstå runt krav.
	

	

	
	
\section{Godkännande}
	För att Testplanen ska vara avklarad behöver följande vara godkänt:
	\begin{itemize}
	 \item Samtliga krav från kravspecifikationen ska vara godkända direkt från kund.
	\item Produkten ska vara godkänd enligt testföremålens kriterier.
	\end{itemize}
	


\printbibliography




\end{document}
