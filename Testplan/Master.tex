\title{Projektplan\\
    \large Projektgrupp 1}

\author{
    Joel Almqvist\\
    \texttt{xxxxxxxx@student.liu.se}
    \and
    Björn Detterfelt\\
    \texttt{xxxxxxxx@student.liu.se}
    \and
    Tim Håkansson\\
    \texttt{timha404@student.liu.se}
    \and
    David Kjellström\\
    \texttt{xxxxxxxx@student.liu.se}
    \and
    Axel Löjdquist\\
    \texttt{xxxxxxxx@student.liu.se}
    \and
    Joel Oskarsson\\
    \texttt{joeos014@student.liu.se}
    \and
    Lieth Wahid\\
    \texttt{xxxxxxxx@student.liu.se}
    \and
    Alexander Wilkens\\
    \texttt{xxxxxxxx@student.liu.se}
}

\date{Februari 19, 2018}

\maketitle
\pagebreak

\titel{Master Test Plan}
\begin{document}

\pagenumbering{gobble}



\maketitle
\pagebreak
	\section*{Dokumenthistorik}

	
	\begin{center}
 	   \begin{tabular}{| l | l | p{12cm} |  }
 	       \hline
 	       \textbf{Version} & \textbf{Datum} & \textbf{Förändring och kommentar} \\
 	       \hline
 	       \centering 1.0 & 2018-02-12 & Första utkast\\
		\hline
 	       \centering 2.0 & 2018-03-05 & Iteration 2\\
 	       \hline
 	   \end{tabular}
	\end{center}

	
	
\pagebreak
\tableofcontents
\pagebreak
\pagenumbering{arabic}

\section{Inledning}
	Den här testplanen är framtagen för att kvalitetssäkra utvecklingen av ett IoT-spel som skapas av en studentgrupp som läser kursen TDDD96 på Linköpings universitet. Planen kommer endast innefatta tester som påverkar spelet både direkt och indirekt, och som är relaterade till utvecklingen. Primärt fokus kommer ligga på att säkerställa att spelet lever upp till de förväntningar som kund har på oss. Se kravspecifikationen \cite{bib-kravspec}. Testplanen följer standard IEEE829. \\

Det här projektet kommer ha 4 olika nivåer av tester, vilket innefattar enhetstester, integrationstester, systemtester och acceptanstest. Detaljer angående testerna kan finnas under respektive testplan.

Projektet förväntas vara färdigt för leverans vid 28-5-2018 och testplanen kommer med störst sannolikhet modifieras under arbetets gång då arbetssättet är agilt.




  \section{Definitioner}
\begin{itemize}[leftmargin=3cm]
  \item [Cachning] Temporär lagning av data för snabb åtkomst
  \item [Instans] En spelsession som startas från UI-applikationen och spelare kan gå med i för att spela spelet tillsammans
  \item [IoT-backend] Existerande system som kan dirigera data mellan många uppkopplade enheter
  \item [Kontroll-applikation] Applikation som körs på en mobil eller surfplatta och styr spelet
  \item [Progressive Web Apps] Ett mellanting mellan en hemsida och en applikation. Med en PWA behöver man inte ladda ner en app, men den ger viss funktionalitet som appar har. \cite{bib-pwa}
  \item [Resurs] Media som används i spelet, t.ex. bilder och ljud
  \item [Sensor] En sensor som sitter på kontroll-applikationen och inte är en pekskärm, t.ex. en accelerometer
  \item [Server-klient-modell] Struktur på ett system där någon enhet tillhandahåller resurser, information eller tjänster och flera andra enheter interagerar med denna
  \item [Spelläge] En utökning av grundspelet som definierar speciella regler och spelmekanik
  \item [Spelmekanik] Regler och möjligheter som definierar ett spel
  \item [Tunn klient] Specialfall av server-klient-modell där mycket få beräkningar sker på klienten
  \item [UI-applikation] Applikationen som kör själva spelet och visar upp spelplanen
  \item [Use Case Map] Diagram som illustrerar hur olika händelser interagerar med arkitekturen \cite[p.~30--33]{bib-architecture-primer}
\end{itemize}

	


	
\section{Testföremål}
	Nedan följer en lista över de stora delarna projektet är uppdelat i, som måste testas var för sig.
	\begin{itemize}
	\item [Backend] Gruppen kommer använda sig utav Cybercoms backend för kommunikation mellan Kontroller och UI, backend i sig kommer inte att testas utan den kod som läggs till på backend för att hantera kommunikationen.
	\item [Kontroller] Spelet ska kontrolleras med hjälp av accelerometern i t.ex. en mobil eller surfplatta.
	\item [UI] Spelet ska kunna visas på en skärm som får all användardata från Cybercoms backend.
	\end{itemize}
	
	
\section{Funktioner som inte ska testas}
	Nedan följer en lista på de delar av projektet som ligger utanför fokus av olika anledningar.
	\begin{itemize}
	\item Chrome
	\item React
	\item Mobila enheter som kör ett annat OS än Android eller iOS
	\end{itemize}
	Vi kommer ej att testa Google Chrome eller React som systemet kommer vara beroende utav, detta för att vi utgår från att det är rigoröst testat av sina respektive utvecklare och på grund av den tiden vi har, ej valt att lägga vår begränsade tid på det. Den tredje punkten har att göra med hur ovanliga dessa mobila enheter är och att vi inte har resurserna att införskaffa en sådan enhet för testning. 




\section{Tillvägagångssätt}
	Testerna för IoT-spelet kommer bestå av enhetstester, regressionstester, integrationstester och acceptanstester. Planen är att alla personer ska vara delaktiga i de olika 		testmomenten i utbildningssyfte. Med den tidsbegränsning som är satt på projektet är risken relativt hög att testningen kan bli lidande. \\
	\\
	Enhetstestningen kommer utföras utav den person som utvecklat enheten och kommer utföras under hela projektets gång. Bereonde på storlek av enheten kan flera utvecklare behöva hjälpas åt för att enheten ska bli godkänd.\\
	\\
	Integrationstestningen kommer i huvuddel styras av konfigurationsansvarig för projektet. När en ny enhet ska integreras ska den först ha klarat enhetstestet och beroende på storleken kan fler projektmedlemmar kopplas in och hjälpa till med att testa samtliga delar av de olika interaktionerna mellan enheterna. \\
	\\
	Systemtester kommer utföras i slutet av varje iteration för att säkerställa att spelet uppfyller kraven. Detta kommer utföras dels internt till en början för att sedan visa demo inför kund för att få feedback.  \\ 
	\\
	Acceptanstest utförs i slutet vid leverans för att låta kund avgöra om produkten lever upp till förväntningar. De kommer själva få köra igenom spelet och fylla i ett formulär enligt Kvalitetsplanen där vi kan bocka av de sista kraven. \\
	\\
	Regressionstester är något som kommer fortgå under hela projektet med hjälp av automatisering. Automatiska tester kommer göras när nya delar läggs till på Github med hjälp av Travis. Detta är ett extra säkerhetssteg för att försäkra att ingen essentiell funktionalitet går förlorad som existerade i en tidigare version.\\
	För mer information om varje modul, se respektive testplan.
	
\subsection{Iteration 2}
	Under Iteration 2 kommer utvecklingen ha påbörjats och kommer till största del bestå utav enhetstester där utvecklarna gör sina egna manuella tester. Vidare ska något eller några automatiska tester ha påbörjats för att bygga upp ett bibliotek av funktionstester som körs vid regressionstestning. Planen för första iterationen är att skapa grundläggande kommunikation mellan de olika delarna så testen kommer skapas därutefter.
	\\
	Väldigt få tester kommer att genomföras under denna iteration då de kommer vara svåra att definiera pga att det är en web-applikation som utvecklas. Tester kommer bli mer genomgående när servern är klar.
	
\subsection{Iteration 3}
	I iteration 3 kommer huvudfokus ligga på att vidareutveckla applikationen för att få mer funktionalitet. Systemet ska ha de flesta funktionerna implementerade vid slutet av iterationen och samtliga ska testas. 
	
	
	

\section{Föremåls godkännande/misslyckande kriterier}
	\subsection{Backend}
		För att backend ska vara godkänd måste den uppnå de krav på hastighet som kan hittas i kravspecifikationen \cite{bib-kravspec}. All kommunikation mellan UI och kontroller ska gå genom kundens backend och för att samtliga krav ska bli uppfyllda måste alla lösningar följa kvalitetsplanen  \cite{bib-kvalitetsplan} krav.

	\subsection{Kontroller och UI}
		För att kontroller och UI-delen ska vara godkänd krävs det att den kan köras på Chrome, version 64. För godkännande av Kontroller måste den godkännas enligt kvalitetsplanen \cite{bib-kvalitetsplan}, en enkät genomförs mot en testgrupp som besvarar frågor angående responsivitet, stabilitet och användarvänlighet.



\section{Avbrott och fortsättningskriterier}
	Om något av prioritet 1 kraven i kravspecifikationen \cite{bib-kravspec} misslyckas, avbryts vidareutveckling av produkten tills kravet åter är uppfyllt. Om kravet av någon anledning skulle visa sig vara ouppnåeligt kommer problemet först diskuteras internt för att sedan förhandlas om med kund. 



\section{Test leveranser}
	I slutet av varje iteration ska en testrapport lämnas. Rapporten kommer innefatta olika testfall, testdata och anomaliteter.
	\\
	I samband med leveranser kommer även testfall levereras på de olika tester som genomförts. 


	

\section{Miljöbehov}
	För att genomföra flera utav testerna behövs viss hårdvara. Nedan listas det som behövs för att genomföra de olika testerna. 

	\begin{itemize}
	\item Stabil nätverksuppkoppling
	\item Dator med tillhörande mus och tangentbord
	\item Mobiltelefon av både typ Android och typ Iphone 
	\item Surfplatta
	\end{itemize}

\section{Ansvarsområden}
	Samtliga gruppmedlemmar har ett eget ansvarsområde i projektet \cite{bib-projektplan}. Dock betyder inte detta att t.ex. testledaren ska utföra alla test själv, utan mer att testledaren har en mer ingående förståelse i rollen. \\
	En av poängerna med projektet är att utbilda sig, så var och en i projektet kommer få prova på det mesta. För stora beslut gällande testning ska vi rösta demokratiskt över vad som är mest lämpligt, vilket fördelar ansvaret på samtliga medlemmar.
	

\section{Personal och utbildningsbehov}
	För att utföra testerna på detta system krävs inga extra utbildningar, men det uppmuntras att vara kunnig i automatisering utav tester och känna till Javascripts stil. Testverktyget som kommer användas är Jest som väldigt få har tidigare kunskap om, därför kommer testledaren sätta sig in i verktygets styrkor och svagheter för att sedan hjälpa resten av gruppen att skriva tester för att optimera arbetsflödet.

	
	
\section{Godkännande}
	För att Testplanen ska vara avklarad behöver följande vara godkänt:
	\begin{itemize}
	 \item Enhetstestplan ska ha använts och godkänts på samtliga punkter.
	 \item Integrationstestplanen ska ha använts och godkänts på samtliga punkter.
	 \item Systemtestplanen ska ha använts och godkänts på samtliga punkter.
	 \item Regressionstestplanen ska ha använts och godkänts på samtliga punkter.
	 \item Acceptanstestplanen ska ha använts och godkänts på samtliga punkter.
	\end{itemize}
\noindent
	Sista ordet lämnas till gruppens handledare Carl Brage. För att denna testplan ska vara godkänd krävs hans godkännande.
	
\printbibliography	

\end{document}
