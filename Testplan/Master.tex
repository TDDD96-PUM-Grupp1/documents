\documentclass[10pt]{article}

\usepackage[utf8]{inputenc}
\usepackage{color}
\usepackage{scrextend}

\title{Master Test plan}

\author{
	Projektgrupp 1\\
	Joel Oskarsson\\
	\texttt{joeos014@student.liu.se}
	\and
	LastName, FirstName\\
	\texttt{first.last@xxxxx.com}
	\and
	LastName, FirstName\\
	\texttt{first.last@xxxxx.com}
	\and
	LastName, FirstName\\
	\texttt{first.last@xxxxx.com}
	\and
	LastName, FirstName\\
	\texttt{first.last@xxxxx.com}
  	\and
  	LastName, FirstName\\
  	\texttt{first.last@xxxxx.com}
  	\and
  	LastName, FirstName\\
  	\texttt{first.last@xxxxx.com}
  	\and
  	LastName, FirstName\\
  	\texttt{first.last@xxxxx.com}
}

\begin{document}

\maketitle
\pagebreak
\tableofcontents
\pagebreak
\section{Inledning}
	Den här testplanen är framtagen för att kvalitetssäkra utvecklingen av ett IoT-spel som skapas av en studentgrupp som läser kursen TDDD96 på Linköpings universitet. Planen kommer endast innefatta tester som påverkar spelet både direkt och indirekt, och som är relaterade till utvecklingen. Primärt fokus kommer ligga på att säkerställa att spelet lever upp till de förväntningar som kund har på oss. {\color{red}Se kravspecifikationen}. Testplanen följer standard IEEE829. \\

Det här projektet kommer ha 4 olika nivåer av tester, vilket innefattar enhetstester, integrationstester, systemtester och acceptanstest. Detaljer angående testerna kan finnas under respektive testplan.

Projektet förväntas vara färdigt för leverans vid {\color{red}28-5-2018} och testplanen kommer med störst sannolikhet modifieras under arbetets gång då arbetssättet är agilt.

	\subsection{Revisionshistorik}

	
	\begin{center}
 	   \begin{tabular}{| l | l | l |  l | }
 	       \hline
 	       \textbf{Version} & \textbf{Datum} & \textbf{Förändring och kommentar} & \textbf{Ansvarig} \\
 	       \hline
 	       \centering 0.1 & 2018-02-12 & Första utkast & David Kjellström\\
 	       \hline
 	   \end{tabular}
	\end{center}


	\subsection{Definitioner}
		\begin{itemize}
		\item [Acceptanstest --]Slutgiltiga testet som kund utför för att se att produkten lever upp till förväntningarna
		\item [Enhetstest --]Testa varje enhet så den fungerar när den är färdig
		\item [Integrationstest --]Testa att en ny enhet som läggs till i projektet fungerar som den ska tillsammans med de andra enheterna
		\item [Kontroller --]En mobil eller surfplatta som kör kontrolldelen av applikationen
		\item [Kund --]Cybercom Sweden
		\item [Regressionstest --]Testa ny kod enligt gamla parametrar för att säkerställa att ingen funktionalitet försvunnit
		\item [Systemtest --]Test för att säkerställa att enheten uppfyller kraven för projektet		
		\item [UI --]User Interface - Den del av applikationen som visar spelplanen
		\end{itemize}
	
	\subsection{Referenser}
		\begin{itemize}
		\item [1] Kravspecifikation
		\item [2] Projektplan
		\item [3] Kvalitetsplan
		\end{itemize}

	
\section{Testföremål}
	Nedan följer en lista över de stora delarna projektet är uppdelat i, som måste testas var för sig.
	\begin{itemize}
	\item [Backend] Eftersom målet med projektet är att visa hastigheten av Cybercoms backend är detta en central punkt i projektet.
	\item [Kontroller] Spelet ska kontrolleras med hjälp av accelerometern i t.ex. en mobil eller surfplatta.
	\item [UI] Spelet ska kunna visas på en skärm som får all användardata från Cybercoms backend.
	\end{itemize}
	
	
\section{Funktioner som inte ska testas}
	Nedan följer en lista på de delar av projektet som ligger utanför fokus av olika anledningar.
	\begin{itemize}
	\item Chrome
	\item React
	\item Mobila enheter som kör ett annat OS än Android eller iOS
	\end{itemize}
	Vi kommer ej att testa Google Chrome eller React som systemet kommer vara beroende utav, detta för att vi utgår från att det är rigoröst testat av sina respektive utvecklare och på grund av den tiden vi har, ej valt att lägga vår begränsade tid på det. Den tredje punkten har att göra med hur ovanliga dessa mobila enheter är och att vi inte har resurserna att införskaffa en sådan enhet för testning. 




\section{Tillvägagångssätt}
	Testerna för IoT-spelet kommer bestå av enhetstester, regressionstester, integrationstester och acceptanstester. Planen är att alla personer ska vara delaktiga i de olika 		testmomenten i utbildningssyfte. Med den tidsbegränsning som är satt på projektet är risken relativt hög att testningen kan bli lidande. \\
	\\
	Enhetstestningen kommer utföras utav den person som utvecklat enheten och kommer utföras under hela projektets gång. Bereonde på storlek av enheten kan flera utvecklare behöva hjälpas åt för att enheten ska bli godkänd.\\
	\\
	Integrationstestningen kommer i huvuddel styras av konfigurationsansvarig för projektet. När en ny enhet ska integreras ska den först ha klarat enhetstestet och beroende på storleken kan fler projektmedlemmar kopplas in och hjälpa till med att testa samtliga delar av de olika interaktionerna mellan enheterna. \\
	\\
	Systemtester kommer utföras i slutet av varje iteration för att säkerställa att spelet uppfyller kraven. Detta kommer utföras dels internt till en början för att sedan visa demo inför kund för att få feedback.  \\ 
	\\
	Acceptanstest utförs i slutet vid leverans för att låta kund avgöra om produkten lever upp till förväntningar. De kommer själva få köra igenom spelet och fylla i ett formulär enligt Kvalitetsplanen där vi kan bocka av de sista kraven. \\
	\\
	Regressionstester är något som kommer fortgå under hela projektet med hjälp av automatisering. Automatiska tester kommer göras när nya delar läggs till på Github med hjälp av Travis. Detta är ett extra säkerhetssteg för att försäkra att ingen essentiell funktionalitet går förlorad som existerade i en tidigare version.\\
	För mer information om varje modul, se respektive testplan.
	
	

\section{Föremåls godkännande/misslyckande kriterier}
	\subsection{Backend}
		För att backend ska vara godkänd måste den uppnå de krav på hastighet som kan hittas i kravspecifikationen {\color{red}[1]}. All kommunikation ska gå över backend och för att samtliga krav ska bli uppfyllda måste de godkännas enligt kvalitetsplanen  {\color{red}[3]}.

	\subsection{Kontroller och UI}
		För att kontroller och UI-delen ska vara godkänd krävs det att den kan köras på Chrome, version 64. För godkännande av Kontroller måste den godkännas enligt kvalitetsplanen {\color{red}[3]}, där en enkät genomförs mot en testgrupp som besvarar frågor angående responsivitet, stabilitet och användarvänlighet.



\section{Avbrott och fortsättningskriterier}
	Om något av de hårda kraven i kravspecifikationen {\color{red}[1]} misslyckas, avbryts vidareutveckling av produkten tills kravet åter är uppfyllt. Om kravet av någon anledning skulle visa sig vara ouppnåeligt kommer kravspecifikationen kommer problemet diskuteras internt för att sedan ta diskussionen vidare till kund. 



\section{Test leveranser}
	I slutet av varje sprint ska en testrapport lämnas. Rapporten kommer innefatta olika testfall, testdata och anomaliteter.


	

\section{Miljöbehov {\color{red}TODO}}
	För att genomföra flera utav testerna behövs viss hårdvara. Nedan listas det som behövs för att genomföra de olika testerna. 

	\begin{itemize}
	\item stabil nätverksuppkoppling
	\item Dator med tillhörande mus och tangentbord
	\item Mobiltelefon av både typ Android och Iphone {\color{red}Windowsphone osv?}
	\item surfplatta {\color{red}vilka?}
	\end{itemize}

\section{Ansvarsområden}
	Samtliga gruppmedlemmar har ett eget ansvarsområde i projektet {\color{red}[2]}. Dock betyder inte detta att t.ex. testledaren ska utföra alla test själv, utan mer att testledaren har en mer ingående förståelse i rollen. \\
	En av poängerna med projektet är att utbilda sig, så var och en i projektet kommer få prova på det mesta. För stora beslut gällande testning ska vi rösta demokratiskt över vad som är mest lämpligt, vilket fördelar ansvaret på samtliga medlemmar.
	

\section{Personal och utbildningsbehov}
	För att utföra testerna på detta system krävs inga extra utbildningar, men det uppmuntras att vara kunnig i automatisering utav tester och känna till Javascripts stil.

	
	
\section{Godkännande}
	För att Testplanen ska vara avklarad behöver följande vara godkänt:
	\begin{itemize}
	 \item Enhetstestplan ska ha använts och godkänts på samtliga punkter.
	 \item Integrationstestplanen ska ha använts och godkänts på samtliga punkter.
	 \item Systemtestplanen ska ha använts och godkänts på samtliga punkter.
	 \item Regressionstestplanen ska ha använts och godkänts på samtliga punkter.
	 \item Acceptanstestplanen ska ha använts och godkänts på samtliga punkter.
	\end{itemize}
	
\end{document}
