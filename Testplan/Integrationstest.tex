\title{Projektplan\\
    \large Projektgrupp 1}

\author{
    Joel Almqvist\\
    \texttt{xxxxxxxx@student.liu.se}
    \and
    Björn Detterfelt\\
    \texttt{xxxxxxxx@student.liu.se}
    \and
    Tim Håkansson\\
    \texttt{timha404@student.liu.se}
    \and
    David Kjellström\\
    \texttt{xxxxxxxx@student.liu.se}
    \and
    Axel Löjdquist\\
    \texttt{xxxxxxxx@student.liu.se}
    \and
    Joel Oskarsson\\
    \texttt{joeos014@student.liu.se}
    \and
    Lieth Wahid\\
    \texttt{xxxxxxxx@student.liu.se}
    \and
    Alexander Wilkens\\
    \texttt{xxxxxxxx@student.liu.se}
}

\date{Februari 19, 2018}

\maketitle
\pagebreak

\titel{Integrationstestplan}

\begin{document}
\pagenumbering{gobble}
\maketitle
\pagebreak
	\section*{Dokumenthistorik}

	
	\begin{center}
 	   \begin{tabular}{| l | l | p{12cm} |  }
 	       \hline
 	       \textbf{Version} & \textbf{Datum} & \textbf{Förändring och kommentar} \\
 	       \hline
 	       \centering 1.0 & 2018-02-12 & Första utkast\\
		\hline
 	       \centering 2.0 & 2018-02-12 & Iteration 2\\
 	       \hline
 	       \centering 3.0 & 2018-02-12 & Iteration 3. Ändrat Versionshanteringen till 1.0... istället för 0.1... Under Testföremål, ändrat meningsbyggdnaden och expanderat t.ex. till till exempel. \\
 	       \hline
 	   \end{tabular}
	\end{center}
\pagebreak
\tableofcontents
\pagebreak

\pagenumbering{arabic}
\section{Inledning}
     I detta dokument kommer Integrationstesterna beskrivas i större detalj.


  \section{Definitioner}
\begin{itemize}[leftmargin=3cm]
  \item [Cachning] Temporär lagning av data för snabb åtkomst
  \item [Instans] En spelsession som startas från UI-applikationen och spelare kan gå med i för att spela spelet tillsammans
  \item [IoT-backend] Existerande system som kan dirigera data mellan många uppkopplade enheter
  \item [Kontroll-applikation] Applikation som körs på en mobil eller surfplatta och styr spelet
  \item [Progressive Web Apps] Ett mellanting mellan en hemsida och en applikation. Med en PWA behöver man inte ladda ner en app, men den ger viss funktionalitet som appar har. \cite{bib-pwa}
  \item [Resurs] Media som används i spelet, t.ex. bilder och ljud
  \item [Sensor] En sensor som sitter på kontroll-applikationen och inte är en pekskärm, t.ex. en accelerometer
  \item [Server-klient-modell] Struktur på ett system där någon enhet tillhandahåller resurser, information eller tjänster och flera andra enheter interagerar med denna
  \item [Spelläge] En utökning av grundspelet som definierar speciella regler och spelmekanik
  \item [Spelmekanik] Regler och möjligheter som definierar ett spel
  \item [Tunn klient] Specialfall av server-klient-modell där mycket få beräkningar sker på klienten
  \item [UI-applikation] Applikationen som kör själva spelet och visar upp spelplanen
  \item [Use Case Map] Diagram som illustrerar hur olika händelser interagerar med arkitekturen \cite[p.~30--33]{bib-architecture-primer}
\end{itemize}


	
\section{Testföremål}
	Varje gång en ny enhet läggs till på projektet behöver integrationstester genomföras, till exempel kommunikation mellan UI och Backend.\\
	\\
	Nedan visas de test som behöver göras för att säkerställa att de implementerade funktionerna fungerar som de ska efter implementation.\\
	
	
\noindent
	\begin{tabular}{| p{2.8cm}| p{2cm} | p{8cm}|}
	
      \hline
      NR&Enheter&Test\\
      \hline
    
		Integrationstest 1&UI och backend&Kommunikation mellan UI och backend.\\
		\hline
		Integrationstest 2&backend och kontroll&Kommunikation mellan backend och kontroll.\\
		\hline
		Integrationstest 3&UI, backend och kontroll&Kommunikation mellan UI och kontroll via backend.\\
		\hline


  \end{tabular}
	



\section{Tillvägagångssätt}
	Under projektets gång kommer de enheter som utvecklas parallellt skapas på olika grenar i git. När enheterna börjar bli redo att implementeras på projektet kommer konfigurationsansvarig att överse mergen mellan de olika grenarna.\\
	\\
	Testerna som följer kommer skapas med konfigurationsansvarig i spetsen uppbackad av de gruppmedlemmar som hjälpt till att utveckla den del av produkten som är i fokus. Under testningen kommer det ske whiteboxtesting för att se till att samtliga funktioner sammankopplade fungerar som de ska. 



\section{Avbrott och fortsättningskriterier}
	Om ett whiteboxtest inte fungerar som det ska kommer testningen avbrytas tills felet är åtgärdat och testningen kommer börja om från början. 
	


\section{Test leveranser}
	Vid varje integration kommer det levereras nya automatiska tester designade för att testa samtliga moduler inblandade. Testerna ska vara utformade på så sätt att de testar övergripande funktionalitet för regressionstestning. Problemrapporter kommer skapas om integrationen stöter på större problem.

	

	

\section{Ansvarsområden}
	Integrationstesterna kommer konfigurationsansvarig och de som skapade enheterna till största del att implementera. Samtliga andra gruppmedlemmar kommer som vid andra tester att vara behjälpliga vid integrationen för att den ska gå smidigt. 


	
	
	
\section{Godkännande}
	För att Testplanen ska vara avklarad behöver följande vara godkänt:
	\begin{itemize}
	 \item Enhetstesterna som godkändes innan integration ska godkännas efter.
	\item Automatiska test ska ha skapats för regressionstestning mellan enheterna i fråga. 
	\end{itemize}
	



\printbibliography

\end{document}
