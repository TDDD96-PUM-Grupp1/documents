\documentclass[10pt]{article}

\usepackage[utf8]{inputenc}
\usepackage{color}
\usepackage{scrextend}

\title{Integrationstestplan}

\author{
	Projektgrupp 1\\
	Joel Oskarsson\\
	\texttt{joeos014@student.liu.se}
	\and
	LastName, FirstName\\
	\texttt{first.last@xxxxx.com}
	\and
	LastName, FirstName\\
	\texttt{first.last@xxxxx.com}
	\and
	LastName, FirstName\\
	\texttt{first.last@xxxxx.com}
	\and
	LastName, FirstName\\
	\texttt{first.last@xxxxx.com}
  	\and
  	LastName, FirstName\\
  	\texttt{first.last@xxxxx.com}
  	\and
  	LastName, FirstName\\
  	\texttt{first.last@xxxxx.com}
  	\and
  	LastName, FirstName\\
  	\texttt{first.last@xxxxx.com}
}

\begin{document}

\maketitle
\pagebreak
\tableofcontents
\pagebreak
\section{Inledning}
     I detta dokument kommer Integrationstesterna beskrivas i större detalj.
	\subsection{Revisionshistorik}

	
	\begin{center}
 	   \begin{tabular}{| l | l | l |  l | }
 	       \hline
 	       \textbf{Version} & \textbf{Datum} & \textbf{Förändring och kommentar} & \textbf{Ansvarig} \\
 	       \hline
 	       \centering 0.1 & 2018-02-18 & Första utkast & David Kjellström\\
 	       \hline
 	   \end{tabular}
	\end{center}


	\subsection{Definitioner}
		\begin{itemize}
		\item [UI] User Interface - Den del av applikationen som visar spelplanen
		\item [Kontroller] En mobil eller surfplatta som kör kontrolldelen av applikationen
		\item [Kund] Cybercom Sweden
		\item [regressionstest] Testa ny kod enligt gamla parametrar för att säkerställa att ingen funktionalitet försvunnit
		\item [enhetstest] Testa varje enhet så den fungerar när den är färdig
		\item [integrationstest] Testa att en ny enhet som läggs till i projektet fungerar som den ska tillsammans med de andra enheterna
		\item [systemtest] Test för att säkerställa att enheten uppfyller kraven för projektet
		\item [acceptanstest] Slutgiltiga testet som kund utför för att se att produkten lever upp till förväntningarna
		\end{itemize}
	
	\subsection{Referenser}
		\begin{itemize}
		\item [1] Kravspecifikation
		\item [2] Projektplan
		\item [3] Kvalitetsplan
		\end{itemize}

	
\section{Testföremål}
	Varje gång en ny enhet läggs till på projektet behöver integrationstester genomföras. t.ex. Kommunikation mellan UI och Backend.
	
	
	
	



\section{Tillvägagångssätt {\color{red}TODO}}
	Under projektets gång kommer de enheter som utvecklas parallellt skapas på olika branches i git. När enheterna börjar bli redo att implementeras på projektet kommer konfigurationsansvarig att överse mergen mellan de olika grenarna.\\
	\\
	Testerna som följer kommer skapas med konfigurationsansvarig i spetsen uppbackad av de gruppmedlemmar som hjälpt till att utveckla den del av produkten som är i fokus.


	
	

\section{Föremåls godkännande/misslyckande kriterier {\color{red}TODO}}
	


\section{Avbrott och fortsättningskriterier {\color{red}TODO}}
	


\section{Test leveranser {\color{red}TODO}}


	

	

\section{Ansvarsområden}
	Integrationstesterna kommer konfigurationsansvariga till största del att implementera. Samtliga andra gruppmedlemmar kommer som vid andra tester att vara behjälpliga vid integrationen för att den ska gå smidigt. 


	
	
	
\section{godkännande {\color{red}TODO}}
	För att Testplanen ska vara avklarad behöver följande vara godkänt:
	\begin{itemize}
	 \item Viktiga integrationssaker!!!
	\end{itemize}
	






\end{document}
