\documentclass[10pt]{article}
\usepackage[utf8]{inputenc}
\usepackage[margin=2cm]{geometry}
\usepackage[swedish]{babel}
\usepackage{hyperref}
\usepackage{color}
\usepackage{scrextend}
\selectlanguage{swedish}

\title{Integrationstestplan\\
    \large Projektgrupp 1}
\author{
    Joel Almqvist\\
    \texttt{joeal360@student.liu.se}
    \and
    Björn Detterfelt\\
    \texttt{bjode786@student.liu.se}
    \and
    Tim Håkansson\\
    \texttt{timha404@student.liu.se}
    \and
    David Kjellström\\
    \texttt{davkj168@student.liu.se}
    \and
    Axel Löjdquist\\
    \texttt{axelo225@student.liu.se}
    \and
    Joel Oskarsson\\
    \texttt{joeos014@student.liu.se}
    \and
    Lieth Wahid\\
    \texttt{liewa893@student.liu.se}
    \and
    Alexander Wilkens\\
    \texttt{alewi684@student.liu.se}
}

\begin{document}
\pagenumbering{gobble}
\maketitle
\pagebreak
\tableofcontents
\pagebreak

\pagenumbering{arabic}
\section{Inledning}
     I detta dokument kommer Integrationstesterna beskrivas i större detalj.
	\subsection{Revisionshistorik}

	
	\begin{center}
 	   \begin{tabular}{| l | l | l |  l | }
 	       \hline
 	       \textbf{Version} & \textbf{Datum} & \textbf{Förändring och kommentar} & \textbf{Ansvarig} \\
 	       \hline
 	       \centering 0.1 & 2018-02-18 & Första utkast & David Kjellström\\
 	       \hline
 	   \end{tabular}
	\end{center}


	\subsection{Definitioner}
  \begin{labeling}{\textbf{Integrationstest}}
  \item [\textbf{Acceptanstest}]Slutgiltiga testet som kund utför för att se att produkten lever upp till förväntningarna
    \item [\textbf{Enhetstest}] Testa varje enhet så den fungerar när den är färdig
    \item [\textbf{Integrationstest}] Testa att en ny enhet som läggs till i projektet fungerar som den ska tillsammans med de andra enheterna
    \item [\textbf{Kontroller}] En mobil eller surfplatta som kör kontrolldelen av applikationen
    \item [\textbf{Kund}] Cybercom Sweden
    \item [\textbf{Regressionstest}] Testa ny kod enligt gamla parametrar för att säkerställa att ingen funktionalitet försvunnit
    \item [\textbf{Systemtest}] Test för att säkerställa att enheten uppfyller kraven för projektet		
    \item [\textbf{UI}] User Interface - Den del av applikationen som visar spelplanen
  \end{labeling}
	
	\subsection{Referenser}
		\begin{itemize}
		\item [1] Kravspecifikation
		\item [2] Projektplan
		\item [3] Kvalitetsplan
		\end{itemize}

	
\section{Testföremål}
	Varje gång en ny enhet läggs till på projektet behöver integrationstester genomföras. t.ex. Kommunikation mellan UI och Backend.
	
	
	
	



\section{Tillvägagångssätt {\color{red}TODO}}
	Under projektets gång kommer de enheter som utvecklas parallellt skapas på olika branches i git. När enheterna börjar bli redo att implementeras på projektet kommer konfigurationsansvarig att överse mergen mellan de olika grenarna.\\
	\\
	Testerna som följer kommer skapas med konfigurationsansvarig i spetsen uppbackad av de gruppmedlemmar som hjälpt till att utveckla den del av produkten som är i fokus.


	
	

\section{Föremåls godkännande/misslyckande kriterier {\color{red}TODO}}
	


\section{Avbrott och fortsättningskriterier {\color{red}TODO}}
	


\section{Test leveranser {\color{red}TODO}}


	

	

\section{Ansvarsområden}
	Integrationstesterna kommer konfigurationsansvariga till största del att implementera. Samtliga andra gruppmedlemmar kommer som vid andra tester att vara behjälpliga vid integrationen för att den ska gå smidigt. 


	
	
	
\section{Godkännande {\color{red}TODO}}
	För att Testplanen ska vara avklarad behöver följande vara godkänt:
	\begin{itemize}
	 \item Viktiga integrationssaker!!!
	\end{itemize}
	






\end{document}
