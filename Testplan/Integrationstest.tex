\title{Arkitekturbeskrivning\\
    \large Projektgrupp 1}
\author{
    Joel Almqvist\\
    \texttt{joeal360@student.liu.se}
    \and
    Björn Detterfelt\\
    \texttt{bjode786@student.liu.se}
    \and
    Tim Håkansson\\
    \texttt{timha404@student.liu.se}
    \and
    David Kjellström\\
    \texttt{davkj168@student.liu.se}
    \and
    Axel Löjdquist\\
    \texttt{axelo225@student.liu.se}
    \and
    Joel Oskarsson\\
    \texttt{joeos014@student.liu.se}
    \and
    Lieth Wahid\\
    \texttt{liewa893@student.liu.se}
    \and
    Alexander Wilkens\\
    \texttt{alewi684@student.liu.se}
}

\date{\today \\Version 2.0}

\maketitle

\titel{Integrationstestplan}

\begin{document}
\pagenumbering{gobble}
\maketitle
\pagebreak
	\section*{Dokumenthistorik}

	
	\begin{center}
 	   \begin{tabular}{| l | l | p{12cm} |  }
 	       \hline
 	       \textbf{Version} & \textbf{Datum} & \textbf{Förändring och kommentar} \\
 	       \hline
 	       \centering 0.1 & 2018-02-12 & Första utkast\\
		\hline
 	       \centering 1.0 & 2018-02-12 & Iteration 2\\
 	       \hline
 	   \end{tabular}
	\end{center}
\pagebreak
\tableofcontents
\pagebreak

\pagenumbering{arabic}
\section{Inledning}
     I detta dokument kommer Integrationstesterna beskrivas i större detalj.


  \section{Definitioner}
\begin{itemize}
  \item Progressive Web Apps -- Ett mellanting mellan en hemsida och en applikation. Med en PWA behöver man inte ladda ner en app, men den ger viss funktionallitet som appar har. \cite{bib-pwa}
  \item IoT-backend -- Existerande system som kan dirigera data mellan många uppkopplade enheter.
  \item Kontroll-applikation -- Applikation som körs på en mobil eller surfplatta som styr spelet
  \item UI-applikation -- Applikationen som kör själva spelet och visar upp spelplanen
  \item Use Case Map -- Diagram som illustrerar hur olika händelser interagerar med arkitekturen. \cite[p.~30--33]{bib-architecture-primer}
  \item Resurs -- Media som används i spelet, t.ex. bilder och ljud.
  \item Spelmekanik -- Regler och möjligheter som definierar ett spel.
  \item Cachning -- Temporär lagning av data för snabb åtkomst.
  \item Spelläge -- En utökning av grundspelet som definierar speciella regler och spelmekanik.
  \item Instans -- En spelsession som startas från UI-applikationen och spelare kan gå med i för att spela spelet tillsammans.
  \item Sensor -- En sensor som sitter på kontroll-applikationen och inte är en touch-skärm (t.ex. en accelerometer).
\end{itemize}


	
\section{Testföremål}
	Varje gång en ny enhet läggs till på projektet behöver integrationstester genomföras. t.ex. Kommunikation mellan UI och Backend.
	
	
	
	



\section{Tillvägagångssätt}
	Under projektets gång kommer de enheter som utvecklas parallellt skapas på olika branches i git. När enheterna börjar bli redo att implementeras på projektet kommer konfigurationsansvarig att överse mergen mellan de olika grenarna.\\
	\\
	Testerna som följer kommer skapas med konfigurationsansvarig i spetsen uppbackad av de gruppmedlemmar som hjälpt till att utveckla den del av produkten som är i fokus. Under testningen kommer det ske whiteboxtesting för att se till att samtliga funktioner sammankopplade fungerar som de ska. 



\section{Avbrott och fortsättningskriterier}
	Om ett whiteboxtest inte fungerar som det ska kommer testningen avbrytas tills felet är åtgärdat och testningen kommer börja om från början. 
	


\section{Test leveranser}
	Vid varje integration kommer det levereras nya automatiska tester designade för att testa samtliga moduler inblandade. Testerna ska vara utformade på så sätt att de testar övergripande funktionalitet för regressionstestning. Problemrapporter kommer skapas om integrationen stöter på större problem.

	

	

\section{Ansvarsområden}
	Integrationstesterna kommer konfigurationsansvarig och de som skapade enheterna till största del att implementera. Samtliga andra gruppmedlemmar kommer som vid andra tester att vara behjälpliga vid integrationen för att den ska gå smidigt. 


	
	
	
\section{Godkännande}
	För att Testplanen ska vara avklarad behöver följande vara godkänt:
	\begin{itemize}
	 \item Enhetstesterna som godkändes innan integration ska godkännas efter.
	\item Automatiska test ska ha skapats för regressionstestning mellan enheterna i fråga. 
	\end{itemize}
	



\printbibliography

\end{document}
