\documentclass[10pt]{article}
\usepackage[utf8]{inputenc}
\usepackage[margin=2cm]{geometry}
\usepackage[swedish]{babel}
\usepackage{hyperref}
\usepackage{color}
\usepackage{scrextend}
\usepackage[backend=bibtex,sorting=none,style=numeric,natbib=true]{biblatex}
\usepackage{multirow}

\newcounter{indexcounter}
\newcommand{\Krav}[2]{
	\stepcounter{indexcounter}
	Krav \arabic{indexcounter} & #1 & #2 \\ \hline
}
\selectlanguage{swedish}
\usepackage{xifthen}
\usepackage{enumitem}
\usepackage{scrextend}
\newcounter{switchcase}

\newcommand{\ifequals}[3]{\ifthenelse{\equal{#1}{#2}}{\stepcounter{switchcase} #3}{}}
\newcommand{\case}[2]{#1 #2} % Dummy, so \renewcommand has something to overwrite...
\newenvironment{switch}[1]{
  %Executed at \begin{switch}
  \setcounter{switchcase}{0}
  \renewcommand{\case}{\ifequals{#1}}
}{
 % Executed at \end{switch}
\ifthenelse{\equal{\value{switchcase}}{0}}{
  \PackageError{ProjectDefinitions}{Could not find given definition}{}}{}
}

\newcommand{\definition}[1]
{
  \begin{switch}{#1}
    \case{Cachning}{\item [\textbf{#1}]
      Temporär lagning av data för snabb åtkomst.}
    \case{Instans}{\item [\textbf{#1}]
      En spelsession som startas från UI-applikationen och spelare kan gå med i för att spela spelet tillsammans.}
    \case{IoT-backend}{\item [\textbf{#1}]
      Existerande system som kan dirigera data mellan många uppkopplade enheter.}
    \case{Kontroll-applikation}{\item [\textbf{#1}]
      Applikation som körs på en mobil eller surfplatta och tar input från användare.}
    \case{Progressive Web Apps}{\item [\textbf{#1}]
      Ett mellanting mellan en hemsida och en applikation. Med en PWA behöver man inte ladda ner en app, men den ger viss funktionalitet som appar har. \cite{bib-pwa}}
    \case{Resurs}{\item [\textbf{#1}]
      Media som används i spelet, t.ex. bilder och ljud.}
    \case{Sensor}{\item [\textbf{#1}]
      En sensor som sitter på kontroll-applikationen och inte är en pekskärm, t.ex. en accelerometer.}
    \case{Server-klient-modell}{\item [\textbf{#1}]
      Struktur på ett system där någon enhet tillhandahåller resurser, information eller tjänster och flera andra enheter interagerar med denna.}
    \case{Spelläge}{\item [\textbf{#1}]
      En utökning av grundspelet som definierar speciella regler och spelmekanik.}
    \case{Spelmekanik}{\item [\textbf{#1}]
      Regler och möjligheter som definierar ett spel.}
    \case{Tunn klient}{\item [\textbf{#1}]
      Specialfall av server-klient-modell där mycket få beräkningar sker på klienten.}
    \case{UI-applikation}{\item [\textbf{#1}]
      Applikationen som kör spelet och visar spelplanen.}
    \case{Use Case Map}{\item [\textbf{#1}]
      Diagram som illustrerar hur olika händelser interagerar med arkitekturen. \cite[p.~30--33]{bib-architecture-primer}}
    \case{Scrum-board}{\item [\textbf{#1}]
      En tavla med post-it lappar som innehåller aktiviteter som ska göras under
      projektet. Detta komplementeras med olika kolumner i tavlan såsom planerad, pågående,
      testning och utgåva. Dessa bestämmer i vilket stadie lapparna befinner sig i.}
    \case{Burndown-chart}{\item [\textbf{#1}]
      En graf som visar hur många timmar medlemmarna har lagt ner i förhållande till vad som krävs för att hinna med projektet.}
    \case{Acceptanstest}{\item [\textbf{#1}]
      Slutgiltiga testet som kund utför för att se att produkten lever upp till förväntningarna.}
    \case{Enhetstest}{\item [\textbf{#1}]
      Testa varje enhet så den fungerar när den är färdig.}
    \case{Integrationstest}{\item [\textbf{#1}]
      Testa att en ny enhet som läggs till i projektet fungerar som den ska tillsammans med de andra enheterna.}
    \case{Kund}{\item [\textbf{#1}]
      Cybercom Sweden.}
    \case{Regressionstest}{\item [\textbf{#1}]
      Testa ny kod enligt gamla parametrar för att säkerställa att ingen funktionalitet försvunnit.}
    \case{Systemtest}{\item [\textbf{#1}]
      Test för att säkerställa att enheten uppfyller kraven för projektet.}
    \case{Cybercom}{\item [\textbf{#1}]
      Kortare variant av Cybercom Sweden, företaget produkten utvecklas åt.}
    \case{Enkäten}{\item [\textbf{#1}]
      Den enkät som ska användas för att utvärdera användarupplevelsen, se avsnitt  3.3 Demo och enkät.}
    \case{Kvalitet}{\item [\textbf{#1}]
        I likhet med IEEE 730 definierar denna rapport kvalitet som konformitet till projektets krav.}
    \case{Projektet}{\item [\textbf{#1}]
        Processen att framställa en produkt  Cybercom Sweden.}
    \case{SQA}{\item [\textbf{#1}]
    	Software Quality Assurance}
    \case{SQA-process}{\item [\textbf{#1}]
      I likhet med IEEE 730 definieras en SQA-process som aktiviteten att samla underlag för att med säkerhet ta
      beslutet av produkten uppnår sina kvalitetskrav}
    \case{Teamet}{\item [\textbf{#1}]
      Det team av åtta studenter som tillsammans ska utföra projektet}
    \case{Trello}{\item [\textbf{#1}]
      En hemsida för att lägga till och fördela uppgifter bland flera personer, kan liknas till en whiteboard där
      postit lappar fästs på.}
    \case{Speldata}{\item [\textbf{#1}]
      Information om handlingar och status i spelet samt nödvändig teknisk data för
      att upprätthålla kommunikation.}
    \case{Realtidsmultiplayerspel}{\item [\textbf{#1}]
      Spel där flera användares handlingar har en direkt inverkan på spelets tillstånd.}
    \case{Gamemode}{\item [\textbf{#1}]
      En variant av basspelet med eventuellt andra funktioner och regler.}
    \case{Vanliga nätverksförhållanden}{\item [\textbf{#1}]
      En enhet med en stabil internetuppkoppling utan yttre störningar.}
    \case{React}{\item [\textbf{#1}]
      Javascript-bibliotek för att bygga hemsidor och mer avancerade webbsystem.\cite{bib-react}}
    \case{Deep Stream}{\item [\textbf{#1}]
    Kommunikationssystem som tillåter synkronisering av data mellan många enheter i realtid. Tillgängligt i många olika programmeringsspråk, bland annat javascript.\cite{bib-deepstream}}
    \case{Impact Map}{\item [\textbf{#1}]
    Diagram som visar inverkan av händelser under ett mjukvarusystems livstid. Kan visa på effekterna av implementation av ny funktionalitet, fel i systemet eller säkerhetsintrång.\cite[p.~91--93]{bib-architecture-primer}}
    \case{IoT, Internet of things}{\item [\textbf{#1}]
    Internet of things -- Ett begrepp som beskriver den tekniska och samhällsliga utveckling då fler och fler saker blir uppkopplade mot internet.}
    \case{Gitrepo}{\item [\textbf{#1}]
    En datastruktur för att lagra och hantera olika versioner av kod i git.}
    \case{Master-branch}{\item [\textbf{#1}]
    Standardgrenen till ett gitrepo som vanligtvis reflekterar repot i ett fungerande lägge.}

  \end{switch}
}



\title{Systemtestplan\\
    \large Projektgrupp 1}
\author{
    Joel Almqvist\\
    \texttt{joeal360@student.liu.se}
    \and
    Björn Detterfelt\\
    \texttt{bjode786@student.liu.se}
    \and
    Tim Håkansson\\
    \texttt{timha404@student.liu.se}
    \and
    David Kjellström\\
    \texttt{davkj168@student.liu.se}
    \and
    Axel Löjdquist\\
    \texttt{axelo225@student.liu.se}
    \and
    Joel Oskarsson\\
    \texttt{joeos014@student.liu.se}
    \and
    Lieth Wahid\\
    \texttt{liewa893@student.liu.se}
    \and
    Alexander Wilkens\\
    \texttt{alewi684@student.liu.se}
}

\begin{document}

\pagenumbering{gobble}
\maketitle
\pagebreak
\section*{Revisionshistorik}

	\begin{center}
 	   \begin{tabular}{| l | l | p{12cm} |  }
 	       \hline
 	       \textbf{Version} & \textbf{Datum} & \textbf{Förändring och kommentar} \\
 	       \hline
 	       \centering 0.1 & 2018-02-12 & Första utkast\\
		\hline
 	       \centering 1.0 & 2018-02-12 & Iteration 2\\
 	       \hline
 	   \end{tabular}
	\end{center}
\pagebreak

\tableofcontents
\pagebreak
\pagenumbering{arabic}
\section{Inledning}
	Det här dokumentet kommer mer ingående definiera hur systemtesterna kommer gå tillväga.


  \section{Definitioner}
\begin{itemize}
  \item Progressive Web Apps -- Ett mellanting mellan en hemsida och en applikation. Med en PWA behöver man inte ladda ner en app, men den ger viss funktionallitet som appar har. \cite{bib-pwa}
  \item IoT-backend -- Existerande system som kan dirigera data mellan många uppkopplade enheter.
  \item Kontroll-applikation -- Applikation som körs på en mobil eller surfplatta som styr spelet
  \item UI-applikation -- Applikationen som kör själva spelet och visar upp spelplanen
  \item Use Case Map -- Diagram som illustrerar hur olika händelser interagerar med arkitekturen. \cite[p.~30--33]{bib-architecture-primer}
  \item Resurs -- Media som används i spelet, t.ex. bilder och ljud.
  \item Spelmekanik -- Regler och möjligheter som definierar ett spel.
  \item Cachning -- Temporär lagning av data för snabb åtkomst.
  \item Spelläge -- En utökning av grundspelet som definierar speciella regler och spelmekanik.
  \item Instans -- En spelsession som startas från UI-applikationen och spelare kan gå med i för att spela spelet tillsammans.
  \item Sensor -- En sensor som sitter på kontroll-applikationen och inte är en touch-skärm (t.ex. en accelerometer).
\end{itemize}

	

	
\section{Testföremål}
	För ett godkänt systemtest ska samtliga av nedanstående tester vara godkända. \\
  
	\begin{tabular}{| p{1.5cm} | p{6cm} | p{8cm}|}
	
  \hline
    \multicolumn{2}{|c|}{Krav}&{Test}\\
    \hline
		Krav 2&UI-applikationen ska endast kommunicera med \newline kontrollapplikationen via Cybercoms IoT-backend.&Verifiera att ingen kommunikation går direkt mellan UI-applikationen och kontrollapplikationen.\\
		\hline
		Krav 3& En användare ska kunna ansluta sig till en spelinstans. &Starta upp en spelinstans och låta en användare ansluta. \\
		\hline
		Krav 7& En spelare ska ha möjligheten att rösta på vilket gamemode som ska spelas. & Testa låta samtliga spelare i en session rösta på nästa spel och låta det spel med flest röster vara nästa spel.\\
		\hline
		Krav 12& Kontrollapplikationen ska ansluta sig till en spelinstans genom att välja en spelinstans från en lista. & Verifiera att kontrollapplikationen har en lista på samtliga tillgängliga sessioner och att spelare kan ansluta till dem direkt. \\
		\hline
		Krav 14& Kontrollapplikationen ska kommunicera med Cybercoms IoT-backend. & Verifiera att kontrollapplikationen har direkt kontakt med backend. \\
		\hline
		Krav 21& UI-applikationen ska kommunicera med Cybercoms IoT-backend. & Verifiera att UI-applikationen har direkt kontakt med backend. \\
		\hline
		Krav 24& UI-applikationen ska kunna visa det som andra UI-applikationer visar i realtid. & Testa gå in i en aktiv spelsession med ett nytt UI. \\
		\hline
		Krav 32& UI-applikationen ska kunna köras i 4 timmar utan avbrott. & Testa hålla en UI-session aktiv i minst 4 timmar utan nertid. \\
		\hline
		Krav 34& Tiden att ansluta sig till en spelsession får inte överskrida 10 sekunder under vanliga nätverksförhållanden. & Mäta tiden en kontrollapplikation tar på sig att ansluta till en aktiv spelsession. \\
		\hline

   
  \end{tabular}
  
  

	
	


\section{Tillvägagångssätt}
	Systemtester ska göras för de krav från kravspecifikationen, nuvarande iteration är byggd för att klara. Testen görs framåt slutet av varje iteration och när ett systemtest klarats, bör det läggas till som ett regressionstest i Travis.\\
	\\
	Samtliga systemtester ska följa Enhetstestens standard, och vara automatiska. 		
	
	
	

\section{Föremåls godkännande/misslyckande kriterier}
	För att ett systemtest ska vara godkänt, måste det uppfylla det krav det är specifierat efter. Krav med prioritet 2 är inte nödvändigt att de är avklarade för ett komplett projekt.\\


\section{Avbrott och fortsättningskriterier}
	Om ett test ej blir godkänt och ej går att fixa inom en rimlig tid måste gruppen kalla till möte för diskussion över vilken väg projektet ska ta härnäst. Om ett krav skulle vara näst intill omöjligt, kan kund behöva kallas till möte.
	



\section{Test leveranser}
	I slutet av varje sprint kommer en testrapport lämnas skapas. Rapporten kommer innefatta olika testfall, testdata och anomaliteter.




\section{Ansvarsområden}
	Ansvaret för välspecifierade krav har fallit på analysansvarige, sedan är det upp till projektledaren att se till att projektet lever upp till kraven medan testledaren ska visa att projektet uppfyller kraven.

  

	
	
\section{Godkännande}
	För att Testplanen ska vara avklarad behöver följande vara godkänt:
	\begin{itemize}
	 \item Samtliga kvantifierbara krav ska testas, automatiseras och implementeras med hjälp av Travis.
	 \item Ett test ska motsvara ett krav i kravspecifikationen.
	\end{itemize}
	


\section{Referenser}
	\begin{itemize}
	\item [1] Grupp 1. Kravspecifikation. Tekn. rapport. Linköpings universitet, febr. 2018.
	\item [2] Grupp 1. Projektplan. Tekn. rapport. Linköpings universitet, febr. 2018.
	\item [3] Grupp 1. Kvalitetsplan. Tekn. rapport. Linköpings universitet, febr. 2018.
	\end{itemize}


\end{document}
