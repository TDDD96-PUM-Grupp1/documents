\title{Projektplan\\
    \large Projektgrupp 1}

\author{
    Joel Almqvist\\
    \texttt{xxxxxxxx@student.liu.se}
    \and
    Björn Detterfelt\\
    \texttt{xxxxxxxx@student.liu.se}
    \and
    Tim Håkansson\\
    \texttt{timha404@student.liu.se}
    \and
    David Kjellström\\
    \texttt{xxxxxxxx@student.liu.se}
    \and
    Axel Löjdquist\\
    \texttt{xxxxxxxx@student.liu.se}
    \and
    Joel Oskarsson\\
    \texttt{joeos014@student.liu.se}
    \and
    Lieth Wahid\\
    \texttt{xxxxxxxx@student.liu.se}
    \and
    Alexander Wilkens\\
    \texttt{xxxxxxxx@student.liu.se}
}

\date{Februari 19, 2018}

\maketitle
\pagebreak

\titel{Systemtestplan}

\begin{document}

\pagenumbering{gobble}
\maketitle
\pagebreak
\section*{Dokumenthistorik}

	\begin{center}
 	   \begin{tabular}{| l | l | p{12cm} |  }
 	       \hline
 	       \textbf{Version} & \textbf{Datum} & \textbf{Förändring och kommentar} \\
 	       \hline
 	       \centering 1.0 & 2018-02-12 & Första utkast\\
		\hline
 	       \centering 2.0 & 2018-03-05 & Iteration 2\\
 	       \hline
 	   \end{tabular}
	\end{center}
\pagebreak

\tableofcontents
\pagebreak
\pagenumbering{arabic}
\section{Inledning}
	Det här dokumentet kommer mer ingående definiera hur systemtesterna kommer gå tillväga.


  \section{Definitioner}
\begin{itemize}[leftmargin=3cm]
  \item [Cachning] Temporär lagning av data för snabb åtkomst
  \item [Instans] En spelsession som startas från UI-applikationen och spelare kan gå med i för att spela spelet tillsammans
  \item [IoT-backend] Existerande system som kan dirigera data mellan många uppkopplade enheter
  \item [Kontroll-applikation] Applikation som körs på en mobil eller surfplatta och styr spelet
  \item [Progressive Web Apps] Ett mellanting mellan en hemsida och en applikation. Med en PWA behöver man inte ladda ner en app, men den ger viss funktionalitet som appar har. \cite{bib-pwa}
  \item [Resurs] Media som används i spelet, t.ex. bilder och ljud
  \item [Sensor] En sensor som sitter på kontroll-applikationen och inte är en pekskärm, t.ex. en accelerometer
  \item [Server-klient-modell] Struktur på ett system där någon enhet tillhandahåller resurser, information eller tjänster och flera andra enheter interagerar med denna
  \item [Spelläge] En utökning av grundspelet som definierar speciella regler och spelmekanik
  \item [Spelmekanik] Regler och möjligheter som definierar ett spel
  \item [Tunn klient] Specialfall av server-klient-modell där mycket få beräkningar sker på klienten
  \item [UI-applikation] Applikationen som kör själva spelet och visar upp spelplanen
  \item [Use Case Map] Diagram som illustrerar hur olika händelser interagerar med arkitekturen \cite[p.~30--33]{bib-architecture-primer}
\end{itemize}




\section{Testföremål}
	För ett godkänt systemtest ska samtliga av nedanstående tester vara godkända. Dessa test är definierade utefter kravspecifikation. \\

	\begin{tabular}{| p{1.5cm} | p{6cm} | p{8cm}|}

  \hline
    \multicolumn{2}{|c|}{Krav}&{Test}\\
    \hline
		Krav 2&UI-applikationen ska endast kommunicera med kontrollapplikationen via Cybercoms IoT-backend.&Verifiera att ingen kommunikation går direkt mellan UI-applikationen och kontrollapplikationen.\\
		\hline
		Krav 3& En användare ska kunna ansluta sig till en spelinstans. &Starta upp en spelinstans och låta en användare ansluta. \\
		\hline
		Krav 7& En spelare ska ha möjligheten att rösta på vilket gamemode som ska spelas. & Testa att låta samtliga spelare i en session rösta på nästa spel och låta det spel med flest röster vara nästa spel.\\
		\hline
		Krav 12& Kontrollapplikationen ska ansluta sig till en spelinstans genom att välja en spelinstans från en lista. & Verifiera att kontrollapplikationen har en lista på samtliga tillgängliga sessioner och att spelare kan ansluta till dem direkt. \\
		\hline
		Krav 14& Kontrollapplikationen ska kommunicera med Cybercoms IoT-backend. & Verifiera att kontrollapplikationen har direkt kontakt med backend. \\
		\hline
		Krav 21& UI-applikationen ska kommunicera med Cybercoms IoT-backend. & Verifiera att UI-applikationen har direkt kontakt med backend. \\
		\hline
		Krav 24& UI-applikationen ska kunna visa det som andra UI-applikationer visar i realtid. & Testa gå in i en aktiv spelsession med ett nytt UI. \\
		\hline
		Krav 32& UI-applikationen ska kunna köras i 4 timmar utan avbrott. & Testa hålla en UI-session aktiv i minst 4 timmar utan nertid. \\
		\hline
		Krav 34& Tiden att ansluta sig till en spelsession får inte överskrida 10 sekunder under vanliga nätverksförhållanden. & Mäta tiden en kontrollapplikation tar på sig att ansluta till en aktiv spelsession. \\
		\hline


  \end{tabular}
    \\  \\ \\
  Dessa test nedanför är test som behöver göras men som ej är definierade utefter kravspecifikation.
  \\

  \noindent
	\begin{tabular}{| p{2.1cm}| p{8cm}|}

      \hline
      NR&Test\\
      \hline

		Systemtest 1&Styrning från kontroll ska motsvara styrning på UI.\\
		\hline
		Systemtest 2&Användning av knapp i kontroller ska motsvara en händelse i spelet.\\
		\hline
		Systemtest 3&Användarnamn i kontroll ska visas på UI.\\
		\hline
		Systemtest 4&En spelare ska kunna lämna ett spel och dess representation i spelet försvinner.\\
		\hline
		Systemtest 5&Spelare ska bli utkastade ur sessionen när UI stängs ned.\\
		\hline
		Systemtest 6&Om poäng ges till spelare ska highscorelistan uppdateras i kontrollen.\\
		\hline




  \end{tabular}




\section{Tillvägagångssätt}
	Systemtester ska göras för de krav från kravspecifikationen nuvarande iteration är byggd för att klara. Testen görs framåt slutet av varje iteration och när ett systemtest klarats, bör det läggas till som ett regressionstest i Travis.\\

	Samtliga systemtester ska följa Enhetstestens standard, och vara automatiska.




\section{Föremåls godkännande/misslyckande kriterier}
	För att ett systemtest ska vara godkänt, måste det uppfylla det krav det är specifierat efter. Krav med prioritet 2 är inte nödvändigt att de är avklarade för ett komplett projekt.\\


\section{Avbrott och fortsättningskriterier}
	Om ett test ej blir godkänt och ej går att fixa inom en rimlig tid måste gruppen kalla till möte för diskussion över vilken väg projektet ska ta härnäst. Om ett krav skulle vara näst intill omöjligt, kan kund behöva kallas till möte.




\section{Test leveranser}
	I slutet av varje iteration kommer en testrapport lämnas. Rapporten kommer innefatta olika testfall, testdata och anomaliteter.




\section{Ansvarsområden}
	Ansvaret för välspecifierade krav har fallit på analysansvarig, sedan är det upp till projektledaren att se till att projektet lever upp till kraven medan testledaren ska visa att projektet uppfyller kraven.





\section{Godkännande}
	För att Testplanen ska vara avklarad behöver följande vara godkänt:
	\begin{itemize}
	 \item Samtliga krav ska vada godkända.
	\item Samtliga test ska vara automatiserade och implementerade för regressionstestning när de är godkända.

	\end{itemize}



\section{Referenser}
	\begin{itemize}
	\item [1] Grupp 1. Kravspecifikation. Tekn. rapport. Linköpings universitet, febr. 2018.
	\item [2] Grupp 1. Projektplan. Tekn. rapport. Linköpings universitet, febr. 2018.
	\item [3] Grupp 1. Kvalitetsplan. Tekn. rapport. Linköpings universitet, febr. 2018.
	\end{itemize}


\end{document}
