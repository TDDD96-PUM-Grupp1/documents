\documentclass[10pt]{article}
\usepackage[utf8]{inputenc}
\usepackage[margin=2cm]{geometry}
\usepackage[swedish]{babel}
\usepackage{hyperref}
\usepackage{color}
\usepackage{scrextend}
\selectlanguage{swedish}
\usepackage{xifthen}
\usepackage{enumitem}
\usepackage{scrextend}
\newcounter{switchcase}

\newcommand{\ifequals}[3]{\ifthenelse{\equal{#1}{#2}}{\stepcounter{switchcase} #3}{}}
\newcommand{\case}[2]{#1 #2} % Dummy, so \renewcommand has something to overwrite...
\newenvironment{switch}[1]{
  %Executed at \begin{switch}
  \setcounter{switchcase}{0}
  \renewcommand{\case}{\ifequals{#1}}
}{
 % Executed at \end{switch}
\ifthenelse{\equal{\value{switchcase}}{0}}{
  \PackageError{ProjectDefinitions}{Could not find given definition}{}}{}
}

\newcommand{\definition}[1]
{
  \begin{switch}{#1}
    \case{Cachning}{\item [\textbf{#1}]
      Temporär lagning av data för snabb åtkomst.}
    \case{Instans}{\item [\textbf{#1}]
      En spelsession som startas från UI-applikationen och spelare kan gå med i för att spela spelet tillsammans.}
    \case{IoT-backend}{\item [\textbf{#1}]
      Existerande system som kan dirigera data mellan många uppkopplade enheter.}
    \case{Kontroll-applikation}{\item [\textbf{#1}]
      Applikation som körs på en mobil eller surfplatta och tar input från användare.}
    \case{Progressive Web Apps}{\item [\textbf{#1}]
      Förkortat PWA, är ett mellanting mellan en hemsida och en applikation.
      Med en PWA behöver man inte ladda ner en app, men den ger viss funktionalitet som appar har. \cite{bib-pwa}}
    \case{Resurs}{\item [\textbf{#1}]
      Media som används i spelet, t.ex. bilder och ljud.}
    \case{Sensor}{\item [\textbf{#1}]
      En sensor som sitter på kontroll-applikationen och inte är en pekskärm, t.ex. en accelerometer.}
    \case{Server-klient-modell}{\item [\textbf{#1}]
      Struktur på ett system där någon enhet tillhandahåller resurser, information eller tjänster och flera andra enheter interagerar med denna.}
    \case{Spelläge}{\item [\textbf{#1}]
      En utökning av grundspelet som definierar speciella regler och spelmekanik.}
    \case{Spelmekanik}{\item [\textbf{#1}]
      Regler och möjligheter som definierar ett spel.}
    \case{Tunn klient}{\item [\textbf{#1}]
      Specialfall av server-klient-modell där mycket få beräkningar sker på klienten.}
    \case{UI-applikation}{\item [\textbf{#1}]
      Applikationen som kör spelet och visar spelplanen.}
    \case{Use Case Map}{\item [\textbf{#1}]
      Diagram som illustrerar hur olika händelser interagerar med arkitekturen. \cite[p.~30--33]{bib-architecture-primer}}
    \case{Scrum-board}{\item [\textbf{#1}]
      En tavla med post-it lappar som innehåller aktiviteter som ska göras under
      projektet. Detta komplementeras med olika kolumner i tavlan såsom planerad, pågående,
      testning och utgåva. Dessa bestämmer i vilket stadie lapparna befinner sig i.}
    \case{Burndown-chart}{\item [\textbf{#1}]
      En graf som visar hur många timmar medlemmarna har lagt ner i förhållande till vad som krävs för att hinna med projektet.}
    \case{Acceptanstest}{\item [\textbf{#1}]
      Slutgiltiga testet som kund utför för att se att produkten lever upp till förväntningarna.}
    \case{Enhetstest}{\item [\textbf{#1}]
      Testa varje enhet så den fungerar när den är färdig.}
    \case{Integrationstest}{\item [\textbf{#1}]
      Testa att en ny enhet som läggs till i projektet fungerar som den ska tillsammans med de andra enheterna.}
    \case{Kund}{\item [\textbf{#1}]
      Cybercom Sweden.}
    \case{Regressionstest}{\item [\textbf{#1}]
      Testa ny kod enligt gamla parametrar för att säkerställa att ingen funktionalitet försvunnit.}
    \case{Systemtest}{\item [\textbf{#1}]
      Test för att säkerställa att enheten uppfyller kraven för projektet.}
    \case{Cybercom}{\item [\textbf{#1}]
      Kortare variant av Cybercom Sweden, företaget produkten utvecklas åt.}
    \case{Enkäten}{\item [\textbf{#1}]
      Den enkät som ska användas för att utvärdera användarupplevelsen, se avsnitt  3.3 Demo och enkät.}
    \case{Kvalitet}{\item [\textbf{#1}]
        I likhet med IEEE 730 definierar denna rapport kvalitet som konformitet till projektets krav. \cite{ieee730}}
    \case{Projektet}{\item [\textbf{#1}]
        Processen att framställa en produkt åt Cybercom Sweden.}
    \case{Software Quality Asssurance}{\item [\textbf{#1}]
    	Förkortat SQA, är en samling aktiviteter som bedömmer lämpligheten och inger förtroende
    	för utvecklingsmetodiken som används.}
    \case{SQA-process}{\item [\textbf{#1}]
      I likhet med IEEE 730 definieras en SQA-process som aktiviteten att samla underlag för att med säkerhet ta
      beslutet av produkten uppnår sina kvalitetskrav}
    \case{Teamet}{\item [\textbf{#1}]
      Det team av åtta studenter som tillsammans ska utföra projektet}
    \case{Trello}{\item [\textbf{#1}]
      En hemsida för att lägga till och fördela uppgifter bland flera personer, kan liknas till en whiteboard som
      postit lappar fästs på.}
    \case{Speldata}{\item [\textbf{#1}]
      Information om handlingar och status i spelet samt nödvändig teknisk data för
      att upprätthålla kommunikation.}
    \case{Realtidsmultiplayerspel}{\item [\textbf{#1}]
      Spel där flera användares handlingar har en direkt inverkan på spelets tillstånd.}
    \case{Gamemode}{\item [\textbf{#1}]
      En variant av basspelet med eventuellt andra funktioner och regler.}
    \case{Vanliga nätverksförhållanden}{\item [\textbf{#1}]
      En enhet med en stabil internetuppkoppling utan yttre störningar.}
    \case{React}{\item [\textbf{#1}]
      Javascript-bibliotek för att bygga hemsidor och mer avancerade webbsystem.\cite{bib-react}}
    \case{Deep Stream}{\item [\textbf{#1}]
    Kommunikationssystem som tillåter synkronisering av data mellan många enheter i realtid. Tillgängligt i många olika programmeringsspråk, bland annat javascript.\cite{bib-deepstream}}
    \case{Impact Map}{\item [\textbf{#1}]
    Diagram som visar inverkan av händelser under ett mjukvarusystems livstid. Kan visa på effekterna av implementation av ny funktionalitet, fel i systemet eller säkerhetsintrång.\cite[p.~91--93]{bib-architecture-primer}}
    \case{IoT, Internet of things}{\item [\textbf{#1}]
    Internet of things -- Ett begrepp som beskriver den tekniska och samhälleliga utveckling då fler och fler saker blir uppkopplade mot internet.}
    \case{Gitrepo}{\item [\textbf{#1}]
    En datastruktur för att lagra och hantera olika versioner av kod i git.}
    \case{Master-branch}{\item [\textbf{#1}]
    Standardgrenen till ett gitrepo som vanligtvis reflekterar repot i ett fungerande tillstånd.}
	\case{Kursen}{\item [\textbf{#1}]
    Den kurs som detta projekt utförs inom, det vill säga LiTHs kurs ''Kandidatprojekt i programvaruutveckling'' med kurskod TDDD96}
  \case{npm}{\item [\textbf{#1}]
  Node Package Manager -- En pakethanterare för Javascripts ekosystem}
  \case{npm-paket}{\item [\textbf{#1}]
  Ett paket med Javascript-kod som finns tillängligt i npm}


  \end{switch}
}


\title{Enhetstestplan\\
    \large Projektgrupp 1}
\author{
    Joel Almqvist\\
    \texttt{joeal360@student.liu.se}
    \and
    Björn Detterfelt\\
    \texttt{bjode786@student.liu.se}
    \and
    Tim Håkansson\\
    \texttt{timha404@student.liu.se}
    \and
    David Kjellström\\
    \texttt{davkj168@student.liu.se}
    \and
    Axel Löjdquist\\
    \texttt{axelo225@student.liu.se}
    \and
    Joel Oskarsson\\
    \texttt{joeos014@student.liu.se}
    \and
    Lieth Wahid\\
    \texttt{liewa893@student.liu.se}
    \and
    Alexander Wilkens\\
    \texttt{alewi684@student.liu.se}
}

\begin{document}
\pagenumbering{gobble}
\maketitle
\pagebreak
	\section*{Revisionshistorik}

	
	\begin{center}
 	   \begin{tabular}{| l | l | p{12cm} |  }
 	       \hline
 	       \textbf{Version} & \textbf{Datum} & \textbf{Förändring och kommentar} \\
 	       \hline
 	       \centering 0.1 & 2018-02-12 & Första utkast\\
		\hline
 	       \centering 1.0 & 2018-02-12 & Iteration 2\\
 	       \hline
 	   \end{tabular}
	\end{center}
\pagebreak
\tableofcontents
\pagebreak

\pagenumbering{arabic}
\section{Inledning}
	Det här dokumentet går igenom hur Enhetstestningen kommer ske.


  \section{Definitioner}
\begin{itemize}[leftmargin=3cm]
  \item [Cachning] Temporär lagning av data för snabb åtkomst
  \item [Instans] En spelsession som startas från UI-applikationen och spelare kan gå med i för att spela spelet tillsammans
  \item [IoT-backend] Existerande system som kan dirigera data mellan många uppkopplade enheter
  \item [Kontroll-applikation] Applikation som körs på en mobil eller surfplatta och styr spelet
  \item [Progressive Web Apps] Ett mellanting mellan en hemsida och en applikation. Med en PWA behöver man inte ladda ner en app, men den ger viss funktionalitet som appar har. \cite{bib-pwa}
  \item [Resurs] Media som används i spelet, t.ex. bilder och ljud
  \item [Sensor] En sensor som sitter på kontroll-applikationen och inte är en pekskärm, t.ex. en accelerometer
  \item [Server-klient-modell] Struktur på ett system där någon enhet tillhandahåller resurser, information eller tjänster och flera andra enheter interagerar med denna
  \item [Spelläge] En utökning av grundspelet som definierar speciella regler och spelmekanik
  \item [Spelmekanik] Regler och möjligheter som definierar ett spel
  \item [Tunn klient] Specialfall av server-klient-modell där mycket få beräkningar sker på klienten
  \item [UI-applikation] Applikationen som kör själva spelet och visar upp spelplanen
  \item [Use Case Map] Diagram som illustrerar hur olika händelser interagerar med arkitekturen \cite[p.~30--33]{bib-architecture-primer}
\end{itemize}

	

	
\section{Testföremål}
	Varje enhet som skapas kommer mer eller mindre genomgå enhetstest. Beroende på dess storlek kan de testas olika. T.ex. en liten service som skrivs på backend kan testas manuellt medan för större enheter t.ex. Kontrollerdelen, kan behöva automatiska tester.
	\\
	
	


\section{Tillvägagångssätt}
	Hurvida en enhets testning ska gå till är på fall till fall basis, där testaren ansvarig över enheten själv bestämmer om den är stor nog att meritera automatisk testning.
	Om så är fallet, bör {\color{red}Jasmine}, Javascript testframework, användas. Test som automatiseras skall övervägas att läggas in som regressionstester i Travis för att funktionalitet inte går förlorad i utvecklingsprocessen. \\
	\\
	För att ha en bra teststandard bör följande punkter has i åtanke:
	\begin {itemize}
	 \item [Genomgående] Testa alla vägar och möjliga scenarion för varje enhet och försök uppnå maximal täckning.
	 \item [Repeterbar] Ett test som lyckats en gång bör lyckas varje gång. Tester bör alltid producera samma resultet.
	 \item [Oberoende] Varje test bör endast testa en sak i taget. Flera antaganden är acceptabelt så länge de testar en funktion. Tester ska ej vara beroende av varandra, varje test ska kunna köras separat.
	 \item [Kodstandard] Tester ska skriva på så sätt att det ser professionellt ut. Bra funktionsnamn, inga dubletter, osv.
	\end {itemize}

	

\section{Föremåls godkännande/misslyckande kriterier}
	För att en enhet ska vara helt godkänd ska samtliga tester den genomgått vara godkända. Upptäcks problem av mindre magnitud som har försumbar inverkan på projektet kan det förbigås.



\section{Avbrott och fortsättningskriterier}
	Om en enhet trots flertalet revederingar och omskrivningar fortfarande misslyckas sina tester bör de avbrytas och ett möte med majoriteten utav gruppen sammankallas. 
	Gruppen ska bestämma om det testerna berör är viktiga nog så att de måste behandlas, och sedan fatta ett gemensamt beslut om hur de ska hanteras, t.ex. skriva om en enhet från grunden.


\section{Test leveranser}

Leveranser från enhetstesten kommer bestå utav automatiska tester för regressionstestning som kommer implementeras direkt och köras vid varje pushning till git. 
\\
Datum för dessa är samma som återfinnes i Master Testplan.



\section{Ansvarsområden}
	En enhets utvecklare är ansvarig över de tester som utförs på enheten för att se till att den lever upp till kraven. Vid testning av större enheter kan fler medlemmar behövas för att sammanställa de nödvändiga testerna.
	
	
	
\section{Godkännande}
	För att Testplanen ska vara avklarad behöver följande vara godkänt:
	\begin{itemize}
	 \item Varje enhet ska ha testas enligt den standard beskriven under Tillvägagångssätt.
	 \item större enheter ska testas automatiskt genom {\color{red}Jasmine} och sedan skapa regressionstester att lägga till i Travis.
	\end{itemize}
	


\section{Referenser}
	\begin{itemize}
	\item [1] Grupp 1. Kravspecifikation. Tekn. rapport. Linköpings universitet, febr. 2018.
	\item [2] Grupp 1. Projektplan. Tekn. rapport. Linköpings universitet, febr. 2018.
	\item [3] Grupp 1. Kvalitetsplan. Tekn. rapport. Linköpings universitet, febr. 2018.
	\end{itemize}
	



\end{document}
