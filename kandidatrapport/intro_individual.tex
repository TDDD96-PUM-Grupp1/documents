\chapter{Översikt över individuella delar}
Projektgruppens medlemmar har utfört mindre undersökningar av specifika aspekter hos projektet. Dessa presenteras som bilagor till denna rapport. Nedan följer korta introduktioner till de olika undersökningarna.

\subsection*{\fullref{individual:joel_a}}

\subsection*{\fullref{individual:bjorn}}

\subsection*{\fullref{individual:tim}}
IoT är ett koncept som har stor utbredning i vårt samhälle idag. Med detta behövs bra algoritmer och strukturer för att skicka och hantera data mellan en stor mängd enheter. Deepstream är en tjänst som erbjuder kommunikation samt hantering av data i realtid. I denna delen av rapporten granskas då denna tjänsten genom att göra en undersökning av hur deepstream serialisera sin data för minska nätverksbelastning och hur detta påverkar rundturstider. Efter att ha undersökt tjänsten så visade det sig att deepstream inte gör någon speciell serialisering utan skickar JSON-objekt i läsbart format. Dessutom såg man att RPC gick något snabbare att exekvera gentemot events, men att båda växer linjärt med datastorleken.

\subsection*{\fullref{individual:david}}
Testning är något som alla vet är ett ont måste. Men måste det vara en utdragen process som man bara behöver få avklarad? Det den här studien kommer gå in djupare på är en undersökning i automatiska tester, speltestning och hur de lite mer knepiga testerna går till. Fördjupningen kommer ligga i Jest då det kommer grundas i många egna erfarenheter, men det kommer dras paralleller mellan olika testverktyg som Jasmine och Mocha. Jämförelser kommer göras som diskuterar skillnader mellan verktyg, vad som är grundläggande inom testvärlden och hur man kan få ut det mesta ur sina tester. De jämförda verktygen visar sig dock inte innehålla några stora skillnader, utan det kommer mer ner till vad man känner sig bekväm med för vilket syntax, där vissa verktyg har lite mer kött på benen gällande vilka bibliotek man använder sig utav i Javascript.

\subsection*{\fullref{individual:axel}}
Utvecklingen av webbapplikationer har genom gått stora förändringar på bara de senaste 10 åren. Flera nya ramverk och bibliotek har introducerats för att underlätta utvecklingsprocessen. En konsekvens av detta är att det har blivit svårare för en utvecklare att bestämma vilket ramverk eller bibliotek lämpar sig bäst för sin webbapplikation. I denna delen av rapporten jämförs de två mest populära verktygen, React och Angular. Rapporten fokuserar på hur lätt det är för en utomstående utvecklare att bidra till ett projekt och jämför även strukturella skillnader. Baserat på erfarenheter som samlats från projektet, diskuteras att React passade bättre för det utförda projektet. Resultaten visar på att React är ett lättare verktyg att lära sig men att Angular har en tydligare struktur för något ska implementeras. Resultatet visar även på att Angular är ett lämpligare verktyg för större project medan React lämpas mer för mindre projekt, då användbarhet sätts i fokus.

\subsection*{\fullref{individual:joel_o}}
Stora ekosystem av färdigskrivna paket med mjukvara har på senare år nått stor popularitet. Det största arkivet av sådana paket är npm för Javascript. I detta arbete undersöks hur beroenden till färdiga paket påverkar utvecklingsprocessen och kvalitet hos slutprodukten för mjukvara skriven i Javascript. En kvantitativ analys av över 6~000 populära Javascript-projekt har utförts för att ta reda på hur utbredd denna praxis är. Resultaten från analysen visar att de undersökta Javascript-projekten i genomsnitt har 6.76 direkta beroenden. Baserat på projekterfarenheter presenteras också hur beroenden kan förändra mjukvaruutvecklingsprocessen. Slutsatser dras även om hur beroenden bör hanteras för att maximera säkerhet och tillförlitlighet hos ett system.

\subsection*{\fullref{individual:alexander}}
Javascript, det primära programmeringsspråket gruppen utvecklat i, är ett dynamiskt typat språk avsätt för webbprogrammering. Den dynamiska typningen språket har leder till en mer uttrycksfull programmeringsupplevelse. Men hur påverkar egentligen den dynamiska typningen utvecklingsarbete i språket, och hur står det sig mot strikt typade alternativ som Facebooks Flow eller Microsofts Typescript? Undersökningen i denna del använder sig av forskning från tidigare studier för att bestämma detta, tillsammans med projektgruppens egna erfarenheter i Javascript. Resultatet från undersökningen pekar mot att utveckling i ett typat alternativ leder till färre publika buggar och reducerad utvecklingstid.

\subsection*{\fullref{individual:lieth-wahid}}
Dagens arbetsmarknad kräver ingenjörer som kan tillämpa teorin som de har lärt sig under sina högskolestudier i praktiken. För att förbereda sådana ingenjörer måste man ha hänsyn till utvecklingsmetodiken som används under studietiden. Vattenfall är det mest använda utvecklingsmetodiken i projektbaserade kurser inom mjukvaruutveckling som ges vid högskolestudier. Scrum, å andra sidan, är ett ramverk som är väldigt populärt inom agil utveckling. Metoden har börjat används vid fler universitet runt om världen. Både metoder har sina för-och nackdelar. Denna undersökning jämför vattenfall arbetsmetoden med Scrum och diskutera vilken är mer fördelaktig för studentprojekt. Undersökning använder sig av rundfrågor samt ett frågeformulär på en studentgrupp som består av åtta studenter. Resultatet från undersökningen visar att en kombination av både metoden skulle vara mer fördelaktig för studentprojekt.
