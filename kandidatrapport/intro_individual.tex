\chapter{Översikt över individuella delar}
Projektgruppens medlemmar har utfört mindre undersökningar av specifika aspekter hos projektet. Dessa presenteras som bilagor till denna rapport. Nedan följer korta introduktioner till de olika undersökningarna.

\subsection*{\fullref{individual:bjorn}}

\subsection*{\fullref{individual:tim}}
IoT är ett koncept som har stor utbredning i vårt samhälle idag. Med detta behövs bra algoritmer och strukturer för att skicka och hantera data mellan en stor mängd enheter. Deepstream är en tjänst som erbjuder kommunikation samt hantering av data i realtid. I denna delen av rapporten granskas då denna tjänsten genom att göra en undersökning av hur deepstream serialisera sin data för minska nätverksbelastning och hur detta påverkar rundturstider. Efter att ha undersökt tjänsten så visade det sig att deepstream inte gör någon speciell serialisering utan skickar JSON-objekt i läsbart format. Dessutom såg man att RPC gick något snabbare att exekvera gentemot events, men att båda växer linjärt med datastorleken.

\subsection*{\fullref{individual:david}}

\subsection*{\fullref{individual:axel}}

\subsection*{\fullref{individual:joel_o}}
Stora ekosystem av färdigskrivna paket med mjukvara har på senare år nått stor popularitet. Det största arkivet av sådana paket är npm för Javascript. I detta arbete undersöks hur beroenden till färdiga paket påverkar utvecklingsprocessen och kvalitet hos slutprodukten för mjukvara skriven i Javascript. En kvantitativ analys av över 6~000 populära Javascript-projekt har utförts för att ta reda på hur utbredd denna praxis är. Resultaten från analysen visar att de undersökta Javascript-projekten i genomsnitt har 6.76 direkta beroenden. Baserat på projekterfarenheter presenteras också hur beroenden kan förändra mjukvaruutvecklingsprocessen. Slutsatser dras även om hur beroenden bör hanteras för att maximera säkerhet och tillförlitlighet hos ett system.

\subsection*{\fullref{individual:alexander}}
Javascript, det primära programmeringsspråket gruppen utvecklat i, är ett dynamiskt typat språk avsätt för webbprogrammering. Den dynamiska typningen språket har leder till en mer uttrycksfull programmeringsupplevelse. Men hur påverkar egentligen den dynamiska typningen utvecklingsarbete i språket, och hur står det sig mot strikt typade alternativ som Facebooks Flow eller Microsofts Typescript? Undersökningen i denna del använder sig av forskning från tidigare studier för att bestämma detta, tillsammans med projektgruppens egna erfarenheter i Javascript. Resultatet från undersökningen pekar mot att utveckling i ett typat alternativ leder till färre publika buggar och reducerad utvecklingstid.

\subsection*{\fullref{individual:joel_a}}

\subsection*{\fullref{individual:lieth}}
