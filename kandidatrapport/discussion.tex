\chapter{Diskussion}
\label{cha:discussion}

I följande del presenteras olika tankar och idéer som uppkommit baserat på denna rapports innehåll och det genomförda projektet. De resultat som har uppnåtts väcker flera intressanta tankar och frågor. Det finns möjliga förbättringar till metoden som i efterhand har uppdagats. Arbetet existerar även i en samhällelig och etisk kontext som bör lyftas.

\section{Resultat}
\label{sec:discussion-results}

%Fanns alternativa implementationssätt?
Det fanns i det genomförda projektet många val som kunde leda till annorlunda resultat. Många teknikval har behövt tas för att färdigställa produkten. En stor andel av dessa har tagits enligt önskemål från kunden och det är inte troligt att kundens värde skulle kunna öka om dessa inte följdes. I denna kategori faller användning av React samt att koden skrevs i ren Javascript.

Många av de tekniska val som gruppen har tagit relaterar till speldelen av produkten. Ett stort sådant var att skapa spelet som ett mer allmänt ramverk för olika spellägen istället för ett specifikt spel. Det är troligt att projektet hade förändrats mycket om det andra alternativet hade valts. Då ett av kundens viktigaste önskemål var att enkelt kunna vidareutveckla projektet bedömdes dock det mer generella valet ge mer värde. Val av bibliotek för spelet var också ett viktigt avgörande i projektet. Flera alternativ diskuterades som till olika grad förlitade sig på färdiga paket för fysik, grafik och logik. Den slutgiltiga lösningen med PIXI har givit stora möjligheter att skapa det system som önskades och gav även mycket stöd i rendering av spelets grafik. Utan färdiga bibliotek hade arbetet troligen tagit alldeles för lång tid. Användning av många stora bibliotek kan istället introducera oönskade begränsningar på projektets struktur.

%Vad återstår för att kunden skall få ut fullt värde av produkten?
Den slutgiltiga produkten innehåller några enkla spellägen som fungerar till fullo. Dessa kräver ingen vidareutveckling från kundens sida utan kan användas som tilltänkt efter leverans. Det är dock troligt att kunden vill introducera fler spellägen. En viss del av produktens värde ligger i hur hög utökningsbarhet som har uppnåtts för att underlätta denna process. Utförlig dokumentation har även skapats för att minimera tiden som krävs för att sätta sig in i systemet. Med detta är förhoppningen att skapandet av nya spellägen ska gå mycket smidigt och inte kräva förståelse för implementationsdetaljer.

%Lyckades ni förbättra/fortsätt något från tidigare projekt?

%Viktigaste lärdomar inför framtiden.

% Övrig mumbo-jumbo :S
Det fanns genom projektet en viss förvirring och osäkerhet gällande arbetet med systemanatomin. Ofta fanns stora skillnader i olika medlemmars uppfattning av hur anatomin skulle vara konstruerad. Det är möjligt att en otillräcklig förståelse för konceptet kan ha lett till att den framtagna modellen tappat mycket värde. Det är möjligt att svaret på frågeställning \ref{fs:fs_3} kunde haft ett större värde om det fanns fler ingående erfarenheter av arbete med systemanatomier. Gruppen diskuterade även huruvida relevansen av detta dokument kan förändras mycket mellan olika typer av projekt och utvecklingsmetodik. Det genomförda projektet befinner sig på en hög abstraktionsnivå med liten hårdvarukoppling. Det är av intresse hur systemanatomins nytta skulle se ut i mer hårdvarunära projekt, men detta har inte undersökts vidare.

Det är mycket troligt att resultat av projekterfarenheter har påverkats av att denna studie av processen har pågått parallellt. Gruppmedlemmarna har haft en medvetenhet om att projektet ligger till grund för detta arbete och även tagit del av seminarier som skapat en djupare reflektion. Detta delade fokus har troligen påverkat projektets utveckling. Det är dock troligt att utförandet av denna undersökningen parallellt är nödvändigt för att ge en tillräcklig insikt i projektet. Även om gruppens erfarenheter är att projektet ej kan anses till fullo representera typisk mjukvaruutveckling bedöms detta inte påverka resultaten i alltför stor utsträckning.

Testningen som gjorts under projektets gång har lämnat en lite önskande efter mer. De flesta test har utförts direkt efter koden som testats skrivits och av samma person och därmed inte varit en så utarbetad process man hade kunnat önskat. Detta berodde till stor del på avsaknad av ordentlig struktur från början och okunnighet från gruppens sida då få gruppmedlemmar var vana användare av automatisk testning.

En bit in i projektet började testningen komma ifatt och ett antal automatiska tester började formas. Möjligtvis blev testerna inte optimala då tidigare erfarenheter av testning var bristande. En lärdom dras av detta att påbörja testningen i ett tidigare skede för att verkligen komma igång med det, kanske till och med testdriven utveckling vore ett bra alternativ \cite{TDD}. Om man formar produkten efter testerna blir man tvungen att skriva tester och får som en slags checklista att bocka av när testerna klarar sig. Dock hade det inte fungerat med vår nuvvarande setup då vi har en regel som säger att man inte kan synka med versionhanteringssystemet om test inte går igenom. 

Manuella tester är det som använts flitigast under projektets gång då manuella tester går snabbt och gruppen vet väl sedan tidigare hur de ska genomföras. De flesta har skett inofficiellt av samma testare vilket kan vara en nackdel men utvecklingen har ändå gått framåt med bra hastighet. Det finns ett par officiella tester som gjorts för att kolla så produkten lever upp till vad som är utlovat i kravspecifikationen. Exempel då ett manuellt test finns i figur \ref{fig:manual_test} i appendixet. Dessa har gjorts av en testare i samband med testledaren för att säkerställa att allting gått rätt till.

Resultatet från den utförda enkätundersökningen uppnåde till stor del de önskade resultaten. Målet av undersökningen var att få ett snittbetyg över sex. Detta uppfylldes på alla frågor utom en, "Jag tycker det var svårt att förstå spelets regler", som fick betyget 5.8. För att åtgärda detta problem implementerade gruppen en inforuta på både kontrollern och UI:t. Inforutan beskriver reglerna för spelläget som spelas. Enkätundersökningen utfördes inte ytterigare en gång efter inforutan hade implementerats. Detta hade varit intressant för att undersöka inforutans effekt på betyget. När medianen undersökts för de övriga resultaten syns det att detta resultatet är högre än medelvärdet. Detta tyder på att det var en större andel som röstade högt än lågt, men att de som röstade lågt, satte väldigt låga poäng.

\section{Metod}
\label{sec:discussion-method}
%Vilka konsekvenser fick de valda metoderna för resultaten?
Den rolluppdelning som har gjorts har troligen påverkat delar av projektet och därmed resultatet. Rollerna har fört med sig dokumentationsbeslut och arbetsprocesser. Vissa ansvarsområden överlappar delvis och det är tänkbart att andra resultat hade nåtts om dessa hade slagits ihop till färre roller. Det finns även specifika roller som inte använts. Exempel på detta är tekniska experter över vissa områden i projektet. Denna sorts roll har i det utförda projektet till viss del uppstått spontant genom personer som är mer insatta i viss teknik. Det är möjligt att resultaten hade sett annorlunda ut om denna uppdelning hade skett i förtid och mer explicit.

%Fanns det alternativ?
Ett alternativ till den iterativa, agila arbetsmetodiken i projektet hade varit en mer traditionell vattenfallsmodell. Detta går dock mot gruppmedlemmarnas tidigare erfarenheter av problem med denna sorts arbetsprocess. Det skulle även finnas organisatoriska problem med detta utifrån planeringen av den kurs projektet ingår i. Denna jämförelse undersöks vidare i individuell rapport \ref{individual:lieth-wahid}. Även inom iterativa arbetsmetoder finns det olika alternativ som har övervägts. Istället för att basera arbetsmetodiken på Scrum skulle exempelvis Kanban \cite{kanban} eller Rational Unified Process \cite{RUP} kunnat användas.

Inom den variant av Scrum som har använts genom projektet finns många valmöjligheter som har förändrat utvecklingsprocessen. Längden på sprints, tid avsatt för vissa möten och uppdelning av arbetsuppgifter är saker som troligen har påverkat arbetet märkbart. Vilka delar av Scrum som har tagits med i projektets utförande har troligen också påverkat mycket. Exkludering av stand-up meetings är ett sådant centralt beslut. I det fallet ansågs det inte effektivt logistiskt. Här hade andra alternativ till daglig kommunikation kunnat övervägts för att uppnå samma effekter.

Arbetet med enkäterna och demonstrationerna påverkade troligen resultatet på helt olika sätt. Enkätundersökningen gjordes under projektets senare skede, under en tid då utvecklingsarbetets fokus var på kvalitet och inte funktioner. Alltså var det svårt för gruppen att använda sig av resultatet från denna undersökning på ett sätt som faktiskt visades i produkten. En möjlig förbättring på denna metod skulle vara att hålla flera enkätundersökningar under projektets gång och i sin tur använda sig av feedbacken i utvecklingsarbetet. Nu användes istället resultatet för att verifiera kvaliteten av en nästan färdigställd produkt. Detta skilde sig dock med demonstrationerna för kund. Denna feedback kom kontinuerligt under utvecklingsfaserna och feedbacken ledde till direkta ändringar av utvecklingen. En eventuell förbättring till denna metoden är att mer formellt dokumentera den feedback som gruppen får, för att senare verifiera dess påverkan på produkten.

%Källkritik.
En del av de använda källorna kommer från dokumentation över produkter skapade av företag. Här finns anledning att tro att ett visst intresse finns för att presentera dessa produkter med viss positiv vinkling. Dock har dessa källor endast används för tekniska specifikationer och i detta anses inte någon möjlig vinkling påverka korrektheten.

% This is where the applied method is discussed and criticized.
% Taking a self-critical stance to the method used is an
% important part of the scientific approach.
%
% A study is rarely perfect. There are almost always things one
% could have done differently if the study could be repeated or
% with extra resources. Go through the most important
% limitations with your method and discuss potential
% consequences for the results. Connect back to the method
% theory presented in the theory chapter. Refer explicitly to
% relevant sources.
%
% The discussion shall also demonstrate an awareness of methodological
% concepts such as replicability, reliability, and validity. The concept
% of replicability has already been discussed in the Method chapter
% (\ref{cha:method}). Reliability is a term for whether one can expect
% to get the same results if a study is repeated with the same method. A
% study with a high degree of reliability has a large probability of
% leading to similar results if repeated. The concept of validity is,
% somewhat simplified, concerned with whether a performed measurement
% actually measures what one thinks is being measured. A study with a
% high degree of validity thus has a high level of credibility. A
% discussion of these concepts must be transferred to the actual context
% of the study.
%
% The method discussion shall also contain a paragraph of
% source criticism. This is where the authors’ point of view on
% the use and selection of sources is described.
%
% In certain contexts it may be the case that the most relevant
% information for the study is not to be found in scientific
% literature but rather with individual software developers and
% open source projects. It must then be clearly stated that
% efforts have been made to gain access to this information,
% e.g. by direct communication with developers and/or through
% discussion forums, etc. Efforts must also be made to indicate
% the lack of relevant research literature. The precise manner
% of such investigations must be clearly specified in a method
% section. The paragraph on source criticism must critically
% discuss these approaches.
%
% Usually however, there are always relevant related research.
% If not about the actual research questions, there is certainly
% important information about the domain under study.

\section{Samhälleliga och etiska aspekter}
\label{sec:work-wider-context}

Projektet som har utvecklats har helt publicerats som open-source under licensen MIT \cite{MIT-license}. Detta är en mycket öppen licens som tillåter fri användning av programvaran för de flesta syften. Genom att dela med sig av all kod kan andra utvecklare välja att använda delar av den producerade koden i egna projekt eller lära sig av de designbeslut som tagits. Denna spridning av information och erfarenheter gör att mängden resurser för liknande framtida projekt ökar. Eftersom projektet är helt öppet och inte dolt bakom någon betaltjänst eller liknande hinder erbjuds alla samma möjlighet att ta del av det.

Projektet som har genomförts har från kundens sida syftet att demonstrera ett system för Internet of Things. Detta gör att projektet till viss del kan anses marknadsföra IoT som koncept. Direkta effekter av den utvecklade produkten på samhälle och miljö anses mycket små. Istället läggs här ett större fokus på indirekt påverkan från den mer uppkopplade värld som slutprodukten bidrar till att demonstrera.

%Samhälleliga
\subsection{Samhälle}
Introduktionen av IoT-koncept innebär stora förändringar av samhället. På en stor skala kan bussar och bilar tänkas vara uppkopplade mot servrar som kan erbjuda många logistiska optimeringsmöjligheter. Även i andra viktiga samhällsinstanser som sjukvård och räddningstjänst kan en mer uppkopplad värld erbjuda många möjligheter. På en mindre skala påverkar IoT varje enskild person och interaktionen mellan människor. Allt fler av objekten i vårt hem kopplas mot internet. Detta påverkar hur vi använder dessa och därmed våra levnadsmönster.

%Miljö
\subsection{Miljö}
Då fler och fler saker kopplas upp mot internet skapas nya behov av nätverk. Både i de uppkopplade apparaterna och i den kringliggande infrastrukturen krävs hårdvara som klarar av större mängder nätverkstrafik. Detta introducerar en ökad energianvändning på flera nivåer i nätverksstrukturen. Små objekt som tidigare innehållit ingen eller endast väldigt enkel elektronik behöver kunna upprätthålla nätverkskommunikation. Även behandling av de stora mängder data som kan skapas är en stor del av IoT-konceptet. Servrar som under lång tid utför tunga dataanalyser kräver stora mängder energi för att utföra sina uppgifter.

IoT erbjuder även många möjligheter för miljöarbete. Insamling av data genom många små sensorer och behandling av denna kan tänkas möjliggöra övervakning av processer som påverkar miljön. IoT tillåter också effektivisering av många industrier, vilket kan tänkas leda till en mer hållbar utveckling. Exempel på en industri med stor miljöpåverkan och möjligheter inom IoT är jordbruk. Här kan olika uppkopplade lösningar optimera resursanvändning och effektivisera processer, vilket ger en positiv miljöpåverkan. \cite{IoT-agriculture}

%Etiska
\subsection{Etik}
Med många små uppkopplade moduler skapas stora mängder data. Denna data har ofta en obetydlig natur i sig självt, men kan få värde i kombination med annan. I de fall denna data beskriver egenskaper hos personer eller deras aktiviteter introducerar detta flera etiska dilemman. Det blir aktuellt att ställa frågor om vem som äger datan och vad den får användas till. Många IoT-lösningar erbjuds som tjänster där all data lagras och hanteras på företags servrar. Det är då förståeligt att användare känner en viss oro för hur den används och sprids. Här läggs ett stort ansvar på företag och organisationer att ta fram riktlinjer för hur man behandlar denna data.

% There must be a section discussing ethical and societal
% aspects related to the work. This is important for the authors
% to demonstrate a professional maturity and also for achieving
% the education goals. If the work, for some reason, completely
% lacks a connection to ethical or societal aspects this must be
% explicitly stated and justified in the section Delimitations in
% the introduction chapter.
%
% In the discussion chapter, one must explicitly refer to sources
% relevant to the discussion.
