%Hackfix, pls no h8
{\justify
I denna rapport presenteras ett projekt för företaget Cybercom utfört av åtta studenter från Linköpings universitet. Projektet har gått ut på att utveckla ett realtidsspel som använder sig av ett existerande system för kommunikation mellan enheter. Spelet har utvecklats som en webbapplikation och innehåller flera olika spellägen. I det genomförda projektet har en modifierad, nedskalad variant av arbetsmetodiken Scrum följts och denna presenteras i rapporten. Utvecklingen har därmed varit iterativ och agil. Resultatet av projektet är en väl fungerande produkt som direkt skapar värde för kunden, men även tillåter smidig vidareutveckling.\\[1in]\par
}
{\centering
  \indent\textbf{Abstract}\par
}
This report presents a project carried out for the company Cybercom by eight students from Linköping University. The aim of the project has been to develop a real-time multiplayer game using an existing system for communication between different devices. The game has been developed as a web app that contains multiple game modes. The specific development methodology that has been used throughout the project is presented in this report. This methodology has been iterative, agile and followed a simplified version of the Scrum framework. The end result of the project is a well functioning product that directly creates value for the customer, but also allows for further development.
