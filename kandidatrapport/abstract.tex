I denna rapport presenteras ett projekt för företaget Cybercom Group utfört av 8 studenter från Linköpings Universitet. Projektet har gått ut på att utveckla ett realtidsspel som använder sig av ett existerande system för kommunikation mellan enheter. Spelet har utvecklats som en webbapplikation och innehåller flera olika spellägen. I det genomförda projektet har en viss arbetsmetodik följts och denna presenteras i denna rapport. Utvecklingen har varit iterativ och agil. En modifierad, nedskalad variant av metodiken Scrum har använts. Resultatet av projektet är en väl fungerande produkt som direkt skapar värde för kunden, men även tillåter enkel vidareutveckling.
