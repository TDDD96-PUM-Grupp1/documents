%%% lorem.tex ---
%%
%% Filename: lorem.tex
%% Description:
%% Author: Ola Leifler
%% Maintainer:
%% Created: Wed Nov 10 09:59:23 2010 (CET)
%% Version: $Id$
%% Version:
%% Last-Updated: Wed Nov 10 09:59:47 2010 (CET)
%%           By: Ola Leifler
%%     Update #: 2
%% URL:
%% Keywords:
%% Compatibility:
%%
%%%%%%%%%%%%%%%%%%%%%%%%%%%%%%%%%%%%%%%%%%%%%%%%%%%%%%%%%%%%%%%%%%%%%%
%%
%%% Commentary:
%%
%%
%%
%%%%%%%%%%%%%%%%%%%%%%%%%%%%%%%%%%%%%%%%%%%%%%%%%%%%%%%%%%%%%%%%%%%%%%
%%
%%% Change log:
%%
%%
%% RCS $Log$
%%%%%%%%%%%%%%%%%%%%%%%%%%%%%%%%%%%%%%%%%%%%%%%%%%%%%%%%%%%%%%%%%%%%%%
%%
%%% Code:

\chapter{Method}
\label{cha:method}
Det här kapitlet beskriver hur projektet har utförts samt beskrivningar av de metoder och verktyg som använts för de olika områden projektet omfattar.

\section{Utvecklingsverktyg för produkt}
Det här avsnittet redovisar vilka verktyg projektgruppen har använt för att utveckla produkten samt hur projektgruppen har använt sig av dessa verktyg.

\subsection*{React}
Ramverket React användes för att utveckla UI-applikationen samt kontroller-applikationen. Ett krav från kunden var att produkten skulle skrivas i JavaScript. React är ett Javascript biblotek som bidrar med mycket funktionallitet vilket underlättar utveckladet av användargränssnittet.  Detta bestämdes tidigt i projektet då vissa projektmedlemmar hade tidigare erfarenheter med React ramverket.

\subsection*{Yarn}

\subsection*{Node.js}
Node.js är en exekveringsmiljö för Javascript. Nodes.js ger funktionallitet att emulera en server med sin applikation på lokalt på sin dator. Att Node.js skulle användas var ett krav från kunden.

\section{Utvecklingsmetodik}
Det här avsnittet förklarar hur projektgruppen är strukturerad. Avsnittet beskriver även vilka metoder och verktyg som har använts för att sköta den interna kommunikationen och utvecklingssmetodiken för projektgruppen.

\subsection*{Scrum}
\subsection*{Trello}
Trello användes för att organisera varje sprint i projektet. Det skapades en product-backlog 

\subsection*{Slack}
Slack användes för både den interna- och externa kommunikationen under projektetsgång. En Slack arbetsplats skapades, där enbart projektgruppen var medverkande i, skedde majoriteten av all den interna kommunikationen. Olika kanaler skapades på denna Slack arbetsplats för att diskutera olika ärenden och hålla kommunikationen strukturerad. Det fanns ytterligare en slack arbetsplats där både projektgruppen och kunden medverkade i, på denna arbetsplats skedde majoriteten av den externa kommunikationen med kunden. Även denna arbetsplats var uppdelad i olika kanaler, frågor och allmänt. Tanken bakom frågor kanalen var att projektgruppen kunde fylla kanalen med frågor rörande utvecklingen och kunden kunde bearbeta frågorna när de hade tid. Allmänt kanalen användes för all övrig kommunikation.


\subsection*{IEEE-Standarder}
\subsection*{Github}
\subsection*{Google Calendar}
En gemensam Google kalender skapades där alla möten samt viktiga moment lades till.

\subsection*{Projekt Orginisation}
Projektetgruppen är strukturerad av förbestämda roller. Varje gruppmedlem tilldelades en roll vid projektetsstart efter en intern valprocess. Rollerna är väldefinierade och dess ansvarsområden beskrivs vidare under \textit{Roller}. Kommunikationen mellan externa parter och projektgruppen gick i huvudsak genom rollerna team ledare och analysansvarig. Projektgruppen undertecknade ett gruppkontrakt[gruppkontrakt] som skrevs vid ett tidigt stadie för att förtydliga vad som förväntades av varje gruppmedlem.

\subsection*{Roller}
Nedan beskrivs ansvarsområden rollerna omfattar.

\subsubsection*{Teamledare}
Teamledaren ska se till att samtliga processer som ska utföras under projektets gång följs. Denna person representerar också teamet utåt och har kontakt med handledaren. Om det behövs har teamledaren sista ordet.

\subsubsection*{Kvalitetssamordnare}
Kvalitetssamordnaren ansvarar för arbetsprocesser som ska hålla kvaliten av projektet på en hög nivå. Samordnaren gör en budget av vad kvalitet får kosta, samtidigt som han ansvarar för kvalitetsplanen.

\subsubsection*{Dokumentansvarig}
Dokumentansvarig ser till att ansvara för samtliga dokument som teamet ska producera. Även ansvarig för gruppens logotyp och dokumentmallar.

\subsubsection*{Arkitekt}
Arkitekten ansvarar för arkitekturen av den tekniska delen av projektet. Gör övergripande teknikval och har det sista ordet på tekniska beslut.

\subsubsection*{Utvecklingsledare}
Utvecklingsledaren ansvarar för den mer detaljerade designen av den tekniska produkten. Leder utvecklingsarbetet och ser till att resten av teamet har något att arbeta med.

\subsubsection*{Analysansvarig}
Analysansvarig ansvarar för majoriteten av kundkontakt och jobbar ständigt med att ta reda på kundens verkliga behov. Har huvudansvar för kravspecifikationen.

\subsubsection*{Testledare}
Testledaren beslutar systemets status genom att arbeta tillsammans med kvalitetssamordnaren för att testa så systemet uppnår kraven. Skriver testplan och testrapport.

\subsubsection*{Konfigurationsansvarig}
Konfigurationsansvarig ansvarar för generell versionshantering i projektet. Arbetar mycket med utvecklingledaren och dokumentansvarig för att bestämma vilka arbetsprodukter som ska ingå i en utgåva.


\section{Dokumentation}
Det här avsnittet beskriver vilka verktyg som har använts för att dokumentera projektgruppens arbete samt vilka dokument som har producerats och dess syften.
\subsection*{Document}
\subsection*{Google Drive \& Google Docs}
\subsection*{Latex}
\subsection*{Eslint}
\subsection*{Prettier}

\section{Testning}
Det här avsnittet beskriver hur produkten har testats samt vilka verktyg som har använts för att testa.
\subsection*{Testningsmetodik}
\subsection*{Travis}
\subsection*{Jsdom}

\section{Metod för att fånga erfarenheter}



%%%%%%%%%%%%%%%%%%%%%%%%%%%%%%%%%%%%%%%%%%%%%%%%%%%%%%%%%%%%%%%%%%%%%%
%%% lorem.tex ends here

%%% Local Variables:
%%% mode: latex
%%% TeX-master: "demothesis"
%%% End:
