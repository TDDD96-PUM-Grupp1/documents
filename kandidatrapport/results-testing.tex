\section{Testning}
Många tester har genomförts under projektets gång. I huvudsak har de flesta tester genomförts av personen som skrivit koden. Flera strukturerade tester har också genomförts tillsammans för att koordinera och verifiera att produkten lever upp till kravspecifikationen. För de mer officiella testerna har testaren i fråga skrivit en testrapport om det utförts manuellt och ett automatiskt test i den mån tid har funnits. Användning av automatiska tester har underlättat arbetet markant då samtliga tester körs när koden synkar med Github. Om ett test skulle misslyckas innebär det att synkningen nekas och koden måste kollas över och fixas innan det går att synka.
\subsection{Jest}
Under utvecklingen användes testverktyget Jest för att skriva de automatiska testerna. Besultet fattades då gruppen bestämt att produkten skulle utvecklas i React då båda har utvecklats utav samma företag för att fungera bra tillsammans. Ingen i gruppen hade tidigare erfarenhet utav Jest, vilket ledde till att många tester tog lång tid att skriva och kan ha blivit mindre konventionella. Mer om hur Jest fungerar och hur det skiljer sig från andra verktyg utvecklas vidare under den individuella delen \ref{individual:david}.
\subsection{Travis}
Travis har använts flitigt under projektets gång och då det är svårt att kvantifiera hur mycket det har hjälpt vid utvecklingen har det hindrat ett flertal olyckor från att ske. Det har hjälpt gruppen att bättre kolla igenom sin kod innan de försöker synka och varit uppskattat överlag. Var gång en synkning har skett har Travis kollat igenom och godkänt koden enligt vad som för gruppen är acceptabelt. 