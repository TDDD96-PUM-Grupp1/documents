\section{Testning}
Tester som har gjorts och utförts har på olika sätt format produkten produkten och skapat värde för kunden. Nedan presenteras resultatet av att testa och hur det har hjälpt teamet att få fram en värdeful produkt.

\subsection{Automatisk}
Genom att ha utfört diversa automatiska tester har teamet lättare kunnat hitta och analysera olika former problem. Till största del var dessa kompilerings- och eslintfel, vilket hjälpte till stor del att framföra en kvalitativ produkt, då dessa inte behövdes köras manuellt för varje \textit{Pull request}. Detta ledde till att teamet kunde lägga mindre tid på att testa och felsöka problem som ger okörbar eller dålig kod, vilket i sin tur ger mer tid till att utveckla en bra produkt som ger större värde för kunden.  

\subsection{Manuell}
Vid de tillfällerna då manuella tester användes var det lättare för buggar att gå igenom till master-grenen, detta på grund av att det fanns många olika tester för olika delar av produkten och alla dessa kunde inte manuellt göras för varje ändring som gjordes. Utöver det fick teamet problem när många manuella tester inte blev dokumenterade eller blev utdaterade på grund av att kodstrukturen ändrades. Vilket då leder till att mer tid blev spenderad på att skriva och uppdatera triviala tester. 

%Självklart har manuella tester givit teamet en bättre produkt, saker som att spela på kontrollern har i många fall visat (även om subjektivt) hur responsivt systemet känns och teamet kunde då justera parametrar för att optimera responsivitet. Just denna testen kunde inte göras ensam utan behövdes göras av stora delar av teamet då många har olika åsikter om responsivitet. 
