\section{Testning}
Många tester har genomförts under projektets gång. I huvudsak har de flesta tester genomförts av personen som skrivit koden. Många strukturerade tester har även genomförts tillsammans för att koordinera att produkten lever upp till kravspecifikationen.

\subsection{Jest}
Under utvecklingen användes testverktyget Jest för att skriva de automatiska testerna. Beslutet fattades då gruppen bestämt att produkten skulle utvecklas i React då båda har utvecklats av samma företag för att fungera bra tillsammans. Ingen i gruppen hade tidigare erfarenhet av Jest, vilket ledde till att många tester tog lång tid att skriva och kan ha blivit mindre konventionella. Mer om hur Jest fungerar och hur det skiljer sig från andra verktyg utvecklas vidare under den individuella delen \ref{individual:david}.

\subsection{Travis}
Travis har använts flitigt under projektets gång i och med att dessa tester har utförts automatiskt så fort en teammedlem lagt till sina ändringar i Git-repot. Detta ledde till att många kompilerings- och Eslint-fel kunde fångas utan att en annan teammedlem behövde kolla över dessa triviala fel. Genom att då ha kört dessa tester automatiskt minimerades den tid som spenderades på att verifiera koden och mer tid kunde då läggas på själva utvecklingen.

\subsection{Manuell}
Vid de tillfällena då manuella tester användes var det lättare för buggar att gå igenom till master-grenen. Detta på grund av att det fanns många olika tester för olika delar av produkten och alla dessa kunde inte testas manuellt för varje ändring som gjordes. Utöver det fick teamet problem när många manuella tester inte blev dokumenterade eller blev utdaterade på grund av att kodstrukturen ändrades. Vilket då leder till att mer tid spenderades på att skriva och uppdatera triviala tester. Anledningen till att alla tester inte dokumenterades var för att många är väldigt triviala, såsom att undersöka om en knapp tar användaren till rätt meny.
