\chapter{Introduktion}
\label{cha:introduction}

Detta projekt utfördes som en del av kursen ''Kandidatprojekt i Programvaruutveckling'' på LiTH våren 2018. Det genomfördes av en grupp studenter som går civilingenjör i datateknik samt civilingenjör i mjukvaruteknik.
Denna introduktion är uppdelad i fyra delar som tillsammans framför vad denna rapport kommer att gå igenom.

\section{Motivering}
\label{sec:motivation}
Trenden Internet of Things (IoT) har blivit mer populär den senaste tiden \cite{IoT-ecosystem}, och fler aktörer har möjligheten att koppla upp sin verksamhet till internet. Utrustning kan kopplas till IoT av många olika anledningar, till exempel kan fabriker använda sig av IoT för att få en bättre överblick av slitaget på sina maskiner. För privatpersonen kan produkter kopplas till IoT såsom ett kylskåp där innehållet kan ses genom en smarttelefon. En central del i IoT är hur slussandet av data sker, för det måste hamna på rätt ställe samtidigt som det ska gå snabbt. Storleksordningen av dessa egenskaper för ett bra resultat är dock inte genomskinligt för en person som inte är insatt i området. Därför gavs gruppen uppdraget att skapa ett interaktivt system för att demonstrera responsiviteten hos Cybercoms IoT-system. Detta system är ett realtidsspel där spelarens handlingar omgående påverkar dennes pjäs på spelplanen, samtidigt som datan går igenom Cybercoms IoT-servrar.


\section{Frågeställning}

\begin{enumerate}
	\item \label{fs:fs_1} Hur kan ett realtidsspel som använder sig av Cybercoms backend implementeras så att man skapar värde för kunden?
	\item \label{fs:fs_2} Vilka erfarenheter kan dokumenteras från programvaruprojektet som kan vara intressanta för framtida projekt?
	\item \label{fs:fs_3} Vilket stöd kan man få genom att skapa och följa upp en systemanatomi?
	\item \label{fs:fs_4} Hur kan kontinuerliga användardemonstrationer användas i utvecklingsfasen för att förbättra ett spels kvalitet?

\end{enumerate}

\section{Syfte}
\label{sec:aim}
Syftet med denna rapport är att dokumentera erfarenheter som teamet har fått genom processen av produktens utveckling, samt projektets utförande. Dessutom så
 har rapporten som syfte att utreda hur utvecklingen av ett realtidsmultiplayerspel skapar värde för kunden, och eventuellt möta deras behov av att testa hastigheten samt skalbarheten hos kundens backend.


Projektets syfte är att skapa ett realtidsmultiplayerspel för att demonstrera hastigheten, responsiviteten, och skalbarheten på Cybercom IoT-backend vilket görs på deras begäran.
\section{Avgränsning}
\label{sec:delimitations}

Projektet utförs som en del av kursen ''Kandidatprojekt i Programvaruutveckling''. Detta innebär att projektet har en begränsad tidsbudget på 400 timmar
per gruppmedlem, det vill säga 3200 timmar totalt. Kursen innefattar obligatoriska seminarier, föreläsningar, workshops och dokumentskrivning som också är inräknade i tidsbudgeten.
Projektet utför en testning av Cybercoms backend och därför är projektet direkt beroende av backendens funktionalitet.
