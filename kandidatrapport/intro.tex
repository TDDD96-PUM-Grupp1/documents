\chapter{Introduktion}
\label{cha:introduction}

The introduction shall be divided into these sections:

\section{Motivering}
\label{sec:motivation}


Trenden Internet of Things (IoT) har blivit mer populär på den senaste tiden(TODO Källa), och fler aktörer har möjligheten att koppla upp sin verksamhet till internet. Utrustning kan kopplas till IoT av många olika anledningar, till exempel så kan fabriker använda sig av IoT för att få en bättre överblick av slitaget på sina maskiner. För privatpersonen så kan produkter kopplas till IoT såsom ett kylskåp där innehållet kan ses genom en smarttelefon. En central del i IoT är hur slussandet av data sker, för det måste hamna rätt samtidigt som det ska gå snabbt. Vilken siffra dessa egenskaper bör ha för ett bra resultat är dock inte genomskinliga för en person som inte är insatt i området. Därför gavs gruppen uppdraget att skapa ett interaktivt system för att demonstrera responsitiviteten hos Cybercoms IoT system. Detta system är ett realtidsspel där spelarens handlingar omgående påverkar dennes pjäs på spelplanen, samtidigt som datan går igenom Cybercoms IoT servrar.



\section{Frågeställning}


*Hur kan ett realtidsspels som använder sig av Cybercoms backend implementeras så att man skapar värde för kunden?

*Vilka erfarenheter kan dokumenteras från programvaruprojektet som kan vara intressanta för framtida projekt?

*Vilket stöd kan man få genom att skapa och följa upp en systemanatomi?

*V




*Hur gör vi ett spel som tydligt demonstrerar realtidsförändringar hos spelaren?

*Hur gör vi ett spel som är lätthanterligt och lättförståerligt nog att användas av personer som inte spelar spel

*Hur gör vi ett spel som är underhållande för personer som inte är spelintresserade

*Hur gör vi en arkitektur som hanterar utökningar

*


\section{Research questions}
\label{sec:research-questions}


This is where the research questions are described.
Formulate these as explicit questions, terminated with a
question mark. A report will usually contain several different
research questions that are somehow thematically connected.
There are usually 2-4 questions in total.

Examples of common types of research questions (simplified
and generalized):

\begin{enumerate}
\item How does technique X affect the possibility of achieving the
  effect Y?

\item How can a system (or a solution) for X be realized so
  that the effect Y is achieved?

\item What are the alternatives to
  achieving X, and which alternative gives the best effect considering
  Y and Z? (This research question is normally broken down in to 2
  separate questions.)

\end{enumerate}


Observe that a very specific research question almost always
leads to a better thesis report than a general research question
(it is simply much more difficult to make something good
from a general research question.)

The best way to achieve a really good and specific research
question is to conduct a thorough literature review and get
familiarized with related research and practice. This leads to
ideas and terminology which allows one to express oneself
with precision and also have something valuable to say in the
discussion chapter. And once a detailed research question
has been specified, it is much easier to establish a suitable
method and thus carry out the actual thesis work much faster
than when starting with a fairly general research question. In
the end, it usually pays off to spend some extra time in the
beginning working on the literature review. The thesis
supervisor can be of assistance in deciding when the research
question is sufficiently specific and well-grounded in related
research.

\section{Delimitations}
\label{sec:delimitations}

This is where the main delimitations are described. For
example, this could be that one has focused the study on a
specific application domain or target user group. In the
normal case, the delimitations need not be justified.

% Definitionerna finns i ../projectdefinition.tex om man vill lägga till flera eller se vilka som finns.
\section{Definitioner} % Vet ej om detta ska vara här eller eget kapitel
\label{sec:definitions}
\begin{enumerate}[leftmargin=5cm]
    \definition{IoT, Internet of things}
    \definition{Teamet}
    \definition{Cybercom}
    \definition{Projektet}
    \definition{Gitrepo}
    \definition{Master-branch}
    \definition{Trello}
    \definition{Scrum-board}
    \definition{Burndown-chart}
\end{enumerate}
