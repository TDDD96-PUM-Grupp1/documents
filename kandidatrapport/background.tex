\chapter{Bakgrund}
\label{cha:background}
Projektet som har utförts har varit en del i kursen ''Kandidatprojekt i programvaruutveckling'' som ges för studenter vid programmen civilingenjör i datateknik och civilingenjör i mjukvaruteknik vid Linköpings universitet~\cite{tddd96}. Projektet har utförts på uppdrag av företaget Cybercom. Detta kapitel kommer ge en kort beskrivning av Cybercom samt ta upp gruppens erfarenheter från föregående studier.

\section{Kundens syfte}
\label{sec:customer-aim}
Cybercom är ett nordiskt IT-konsultbolag som arbetar inom hela kedjan av uppkopplade lösningar. Företaget har ca 1300 anställda som arbetar mot både lokala och globala marknader~\cite{cybercomgroup}. Projektet ''Realtidsmultiplayerspel på IoT-backend'' har drivits mot Cybercoms kontor i Linköping.

Ett av Cybercoms arbetsområden är Internet of Things. De arbetar med bland annat rådgivning, etablering och drift av system med IoT-lösningar~\cite{cybercomiot}. I Linköping finns ett team som arbetar med en egenutvecklad IoT-backend. Detta är ett system som tillåter sammankoppling mellan uppkopplade enheter, server-tjänster och externa tjänster från andra leverantörer.

Cybercoms motivation till projektet är att skapa ett verktyg för att demonstrera denna IoT-lösning. Med just ett spel i realtid vill man visa på effektivitet och responsitivitet i sin lösning. Spelet ska kunna användas för att demonstrera systemets realtidskapacitet för kunder. När Cybercom befinner sig på mässor vill man även kunna låta besökare vid sin monter testa spelet för att ge en mer minnesvärd upplevelse av företaget. Med detta vill man locka både potentiella kunder och nya anställda.

\pagebreak

\section{Gruppens tidigare erfarenheter}
\label{sec:earlier-experience}
Samtliga gruppmedlemmar har erfarenhet från kursen ''Programutvecklingsmetodik teori'' som ger en god inblick i storskalig programvaruutveckling i projektform. Från denna kurs har gruppmedlemmarna fått med sig kunskap om kravhantering, arbetsprocesser, design, testning och mjukvarukvalitet~\cite{tddc93}. Dessa kunskaper har under projektet legat som grund för vidareutbildning inom både generella och rollspecifika områden.

Studenter i projektgruppen som läser programmet civilingenjör i datateknik har praktisk erfarenhet från projektkursen ''Konstruktion med mikrodatorer, projektkurs''~\cite{tsea29}. Kursen har givit medlemmarna mer praktiska erfarenheter gällande planering och strukturering av projekt samt arbete med dokumentation i större projekt. Projektet som utförs i ''Konstruktion med mikrodatorer, projektkurs'' följer projektmodellen Lips~\cite{lips}, som ej är framtagen specifikt för mjukvaruutveckling. Fokuset i projektet från denna kurs är en kombination av hård- och mjukvara och följer, till skillnad från det projekt denna rapport berör, inte en iterativ arbetsprocess.

Studenter i projektgruppen som läser programmet civilingenjör i mjukvaruteknik har istället erfarenheter av mjukvaruprojekt från kursen ''Artificiell intelligens - projekt''~\cite{tddd92}. Från denna kurs har det erfarits kunskaper om planering och rapportskrivning. Projektet i kursen utfördes i större grupper bestående av sex personer och följde en arbetsmetodik liknande vattenfallsmetoden. Målet med projektet var att undersöka och applicera tekniker inom området av artificiell intelligens.

Inom de tekniska områden och tekniker som projektet berör har gruppen ingen direkt erfarenhet från tidigare kurser i utbildningen. Däremot har samtliga medlemmar en god programmeringsvana som snabbt låtit dem ta till sig nödvändiga kunskaper för projektet. Vissa gruppmedlemmar har även erfarenhet av webbprogrammering genom projekt utanför utbildningen.

Från tidigare projekt har gruppens medlemmar tagit med sig positiva och konstruktiva erfarenheter som tillämpats under det genomförda projektet. Att arbeta i mindre grupper på samma plats var något som flera lyfte som en bra arbetsmetod, eftersom detta ansågs ge mindre missförstånd inom gruppen. För att samordna gruppens arbete hade flera gruppmedlemmar positiva erfarenheter med gemensamma scheman, vilket togs med in i detta projekt.

Projektgruppen har också erfarenheter från problematiska aspekter av tidigare projekt och har därför arbetat för att undvika dessa. Från projekt utan iterativ utveckling tyckte projektmedlemmarna att många problem och därmed mycket extraarbete uppkom i slutskedet av utvecklingen. Detta ansågs kunna lösas genom att fokusera på att hålla utvecklingen iterativ och inkrementell. En annan viktig erfarenhet var att inte skriva all dokumentation som ett sista steg av utvecklingsprocessen. Istället har dokumentation producerats tillsammans med koden kontinuerligt genom projektet. Från tidigare grupparbeten fanns också idéer om hur gruppdynamik och kommunikation kan förbättras. En viktig aspekt som togs upp var att lyfta samarbetsproblem och irritationsmoment tidigt och ordentligt diskutera igenom dessa för att kunna lösa dem.
