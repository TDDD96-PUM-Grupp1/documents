\chapter{Teori}
\label{cha:theory}
I detta avsnitt beskrivs grundläggande teoretiska begrepp och ramverk som följdes under projektets gång. Här beskrivs de essentiella programmeringsverktyg och ramverk för projektet samt utvecklingsmetoden Scrum. 
\section{Utvecklingsverktyg för produkt}
Det här avsnittet redovisar vilka verktyg projektgruppen har använt för att utveckla produkten samt hur projektgruppen har använt sig av dessa verktyg.

\subsection*{React}
Ramverket React användes för att utveckla UI-applikationen samt kontroller-applikationen. Ett krav från kunden var att produkten skulle skrivas i JavaScript. React är ett Javascript bibliotek som bidrar med mycket funktionallitet vilket underlättar utvecklandet av användargränssnittet.  Detta bestämdes tidigt i projektet då vissa projektmedlemmar hade tidigare erfarenheter med React ramverket \cite{ReactAJa67:online}.

\subsection*{Yarn}
Yarn är ett Javascript bibliotek. Biblioteket fungerar som en pakethanterare som skapar beroende mellan paket och underhåller dessa. Detta gör det enklare att lägga till och ta bort bibliotek till ett projekt \cite{GettingS85:online}.

\subsection*{Prettier}
Detta verktyget hjälper en programmerare att formattera koden på ett sätt så alla filer håller en enhetlig standard oavsätt vilken utvecklingsmiljö man befinner sig i.

\subsection*{Eslint}
Utöver prettier finns eslint som kommer att klaga på dåliga sätt att skriva kod. Detta gäller mindre om strukturen utav koden och mer om funktioner som inte ska användas för bättre kod.

\subsection*{Node.js}
Node.js är en exekveringsmiljö för Javascript. Nodes.js ger funktionallitet att emulera en server med sin applikation på lokalt på sin dator \cite{Nodejs11:online}. Att Node.js skulle användas var ett krav från kunden.

\subsection*{PIXI}
PIXI är ett kraftfullt opensource renderingsbiblotek för Javascript\cite{PixiJSv473:online}.

\subsection*{Scrum}
Scrum är en populär agil utvecklingsmetodik inom mjukvaruutveckling. Metoden är anpassad för en mindre grupp på 5-9 medlemmar. Scrum använder ett iterativt, inkrementellt tillvägagångssätt som är relevant för detta projekt då den tillåter ändringar att ske när det behövs\cite{TheScrum81:online}. I Scrum delas arbetet i mindre iterationer kallas för \textit{sprint}. En sprint kan vara 1-4 veckor lång. Varje sprint planeras vid dess början och då bestäms mål som ska vara färdiga vid sprintensslut. Efter en sprint så utvärderas hur väl arbetet gick under denna sprint samt vad som bör göras annorlunda till nästa sprint. Scrum har många inslag som är typiska för just Scrum, de som tas upp i denna rapport är följande:

\begin{itemize}
	\item \textit{Scrum-bräde} är ett bräde med alla uppgifter som ska göras under en sprint, varje medlem kan sedan ta en egen uppgift och flytta den till rätt kategori beroende på hur det går i arbetet med den. Vanliga kategorier är: att göras, pågående, testing, färdig.
	
	\item \textit{Burndown chart} är en graf som mäter hur mycket tid som är kvar på sprinten
	
	\item \textit{Produkt-backlog} uppgifter att göra på produkten
	
	\item \textit{Sprint-backlog} nedbrutna uppgifter ifrån produkt-backlogen, tillräckligt många uppgifter som anses hinnas med i en sprint
	
\end{itemize}

\subsection*{Trello}
Trello är en hemsida för att lägga till och fördela uppgifter bland flera personer, kan liknas en whiteboard som postit lappar fästs på.

\subsection{Git}
Ett opensource projekt som används för versionhantering av kod \cite{Git52:online}. Med hjälp av Git kund filer som var nödvändiga för projektet versionhanteras. För att versionhantera filerna följande kommandon användes: 
\begin{itemize}
	\item \textit{Pull request}:
	\item \textit{Push}:
	\item \textit{Gren}:
\end{itemize}

\subsection*{Github}


\subsection*{Travis}
Ett opensource testningstjänst som erbjudst av gratis Github på deras plattform. Travis används i kontinuerlig integration för att köra tester på alla grenar innan de kan mergeas in i huvudgrenen.

\subsection*{Slack}
Slack är ett kommunikationsverktyg som ofta används proffesionelt inom IT och mjukvaruutveckling. Funktionsmässigt är det ett chatprogram för datorer och mobila enheter.

\subsection*{Latex}
Latex är ett gratis verktyg som kombinerat med en latex-editor används för att skapa artiklar och rapporter av olika slag. Det är ett väldigt kraftigt verktyg som med många formateringsmmöjligheter och tillskillnad ifrån många andra text-editors så separerar Latex mellan det du skriver och slutprodukten som ska läsas.
