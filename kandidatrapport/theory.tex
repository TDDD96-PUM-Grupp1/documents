\chapter{Theory}
\label{cha:theory}

\section{Utvecklingsverktyg för produkt}
Det här avsnittet redovisar vilka verktyg projektgruppen har använt för att utveckla produkten samt hur projektgruppen har använt sig av dessa verktyg.

\subsection*{React}
Ramverket React användes för att utveckla UI-applikationen samt kontroller-applikationen. Ett krav från kunden var att produkten skulle skrivas i JavaScript. React är ett Javascript biblotek som bidrar med mycket funktionallitet vilket underlättar utveckladet av användargränssnittet.  Detta bestämdes tidigt i projektet då vissa projektmedlemmar hade tidigare erfarenheter med React ramverket.

\subsection*{Yarn}
JavaScript bygger på ett flertal olika bibliotek som kan utöka och förenkla JavaScripts funktioner. Yarn fungerar som en pakethanterare som skapar beroende mellan paket och underhåller dessa. Detta gör det enklare att lägga till och ta bort bibliotek till ett projekt.

\subsection*{Prettier}
Detta verktyget hjälper en programmerare att formattera koden på ett sätt så alla filer håller en enhetlig standard oavsätt vilken utvecklingsmiljö man befinner sig i.

\subsection*{Eslint}
Utöver prettier finns eslint som kommer att klaga på dåliga sätt att skriva kod. Detta gäller mindre om strukturen utav koden och mer om funktioner som inte ska användas för bättre kod.

\subsection*{Node.js}
Node.js är en exekveringsmiljö för Javascript. Nodes.js ger funktionallitet att emulera en server med sin applikation på lokalt på sin dator. Att Node.js skulle användas var ett krav från kunden.
