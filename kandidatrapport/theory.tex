\chapter{Teori}
\label{cha:theory}
I detta avsnitt beskrivs grundläggande teoretiska begrepp och ramverk som följdes under projektets gång. Här beskrivs de essentiella programmeringsverktyg och ramverk för projektet samt utvecklingsmetoden Scrum. 
\section{Utvecklingsverktyg för produkt}
Det här avsnittet redovisar vilka verktyg projektgruppen har använt för att utveckla produkten samt hur projektgruppen har använt sig av dessa verktyg.

\subsection*{React}
Ramverket React användes för att utveckla UI-applikationen samt kontroller-applikationen. Ett krav från kunden var att pro
\subsection*{Github}


\subsection*{Travis}
Ett opensource testningstjänst som erbjudst av gratis Github på deras plattform. Travis används i kontinuerlig integration för att köra tester på alla grenar innan de kan mergeas in i huvudgrenen.

\subsection*{Slack}
Slack är ett kommunikationsverktyg som ofta används proffesionelt inom IT och mjukvaruutveckling. Funktionsmässigt är det ett chatprogram för datorer och mobila enheter.

\subsection*{Latex}
Latex är ett gratis verktyg som kombinerat med en latex-editor används för att skapa artiklar och rapporter av olika slag. Det är ett väldigt kraftigt verktyg som med många formateringsmmöjligheter och tillskillnad ifrån många andra text-editors så separerar Latex mellan det du skriver och slutprodukten som ska läsas.
