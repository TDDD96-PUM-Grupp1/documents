\chapter{Teori}
\label{cha:theory}
I detta avsnitt beskrivs grundläggande teoretiska begrepp och ramverk som följdes under projektets gång. Här beskrivs de essentiella programmeringsverktyg och ramverk för projektet samt utvecklingsmetoden Scrum. 
\section{Utvecklingsverktyg för produkt}
Det här avsnittet redovisar vilka verktyg projektgruppen har använt för att utveckla produkten samt hur projektgruppen har använt sig av dessa verktyg.

\subsection*{React}
Ramverket React användes för att utveckla UI-applikationen samt kontroller-applikationen. Ett krav från kunden var att produkten skulle skrivas i JavaScript. React är ett Javascript bibliotek som bidrar med mycket funktionallitet vilket underlättar utvecklandet av användargränssnittet.  Detta bestämdes tidigt i projektet då vissa projektmedlemmar hade tidigare erfarenheter med React ramverket \cite{ReactAJa67:online}.

\subsection*{Yarn}
Yarn är ett Javascript bibliotek. Biblioteket fungerar som en pakethanterare som skapar beroende mellan paket och underhåller dessa. Detta gör det enklare att lägga till och ta bort bibliotek till ett projekt \cite{GettingS85:online}.

\subsection*{Prettier}
Prettier är ett verktyg för att formatera kod på ett standardiserat och lättläsigt sätt\cite{prettier}. Prettier går igenom koden och lägger till eller tar bort blanksteg och nya rader enligt förutbestämda regler. Syntaktiskt blir koden samma före och efter att Prettier körts, men läsbarheten lär ha förändrats. Prettier tar även bort stilpreferenser som olika kodskribenter då den alltid formaterar koden enligt samma regler.


\subsection*{Eslint}
Eslint är ett \textit{open source} program som definerar stilregler för hur Javascriptkod skrivs. Dessa stilregler handlar  om hur kod ska formateras och att följa bra programmeringspraxis. Eslint söker sen igenom koden för rader som bryter mot de satta reglerna och påpekar alla fall som deta sker. Oftast så integreras Eslint in till programet där koden skrivs för att få Eslints varningar direkt när koden skrivs. De regler Eslint efterföljer är bra praxis för kodning i Javascript och de går att ändra på efter behov.


\subsection*{Node.js}
Node.js är en exekveringsmiljö för Javascript. Nodes.js ger funktionallitet att emulera en server med sin applikation på lokalt på sin dator \cite{Nodejs11:online}. Att Node.js skulle användas var ett krav från kunden.

\subsection*{PIXI}
PIXI är ett kraftfullt \textit{opensource} renderingsbiblotek för Javascript\cite{PixiJSv473:online}. Den erbjuder funktioner som att rendera geometriska former och bilder samt en uppdateringsloop varje gång ett objekt ritas ut igen. PIXI är väldigt populärt och används av många stora företag såsom Google, Ubisoft och Spotify. 

\subsection*{Scrum}
Scrum är en populär agil utvecklingsmetodik inom mjukvaruutveckling. Metoden är anpassad för en mindre grupp på 5-9 medlemmar. Scrum använder ett iterativt, inkrementellt tillvägagångssätt som är relevant för detta projekt då den tillåter ändringar att ske när det behövs\cite{TheScrum81:online}. I Scrum delas arbetet i mindre iterationer kallas för \textit{sprint}. En sprint kan vara 1-4 veckor lång. Varje sprint planeras vid dess början och då bestäms mål som ska vara färdiga vid sprintensslut. Efter en sprint så utvärderas hur väl arbetet gick under denna sprint samt vad som bör göras annorlunda till nästa sprint. Scrum har många inslag som är typiska för just Scrum, de som tas upp i denna rapport är följande:

\begin{itemize}
	\item \textit{Scrum-bräde} är ett bräde med alla uppgifter som ska göras under en sprint, varje medlem kan sedan ta en egen uppgift och flytta den till rätt kategori beroende på hur det går i arbetet med den. Vanliga kategorier är: att göras, pågående, testing, färdig.
	
	\item \textit{Burndown chart} är en graf som visar kvarstående arbetet. Grafen hjälper teamet att ta reda på om de ligger bra eller dåligt till för att leverera dem uppgifter som de har åtagit sig. 
	
	\item \textit{Produkt-backlog}: En lista av prioriterade önskemål som visar vad kundens önskemål vid slutprodukten.
	
	\item \textit{Sprint-backlog} nedbrutna uppgifter ifrån produkt-backlogen som teamet åtar sig att leverera under en sprint. 	
	
\end{itemize}

\subsection*{Trello}
Trello är en hemsida för att skapa och fördela uppgifter bland flera personer. Det kan liknas till en anslagstavla där lappar med uppgifter kan klistras på. Anslagstavlan i sig är uppdelad i kategorier som användarna själva kan skapa och modifiera. Oftast så har dessa kategorier namn som att ''att göra'', ''pågående'' och ''färdigt''. Lappar fästs sen vid den första kategorien ''att göra'' och flyttas sedan till de andra när det anses passande. Varje lapp har sedan möjlighet att innehålla extra information bland annat: vem som arbetar med den, en beskrivning av lappen samt en tidsuppskattning. Alla fält på en lapp bortsett dess namn är friviliga, och därav så kan en Trello-lapp innehålla väldigt mycket, eller väldigt lite, information.

\subsection{Git}
Ett opensource projekt som används för versionhantering av kod \cite{Git52:online}.
\begin{itemize}
	\item \textit{Push}: Skickar utvecklingsfiler från det lokala repot som man utvecklar i till Gitrepot.
	\item \textit{Pull request}: En begäran som skickas när en utvecklare laddar ner en kod från Git, modiferar den och vill \textit{pusha} tillbaka den. När en pull request är gjord intresserade partier kan se över modifieringar som har gjorts och sammanfoga den i huvudgrenen, begära ytterligare modifieringar, eller pushar egna modifieringar. 
	\item \textit{Gren}: En oberoende utvecklingslinje som en utvecklare kan uttnytjan när hen vill lägga till nya funktionalitet. 
\end{itemize}

\subsection*{Github}
Github är en hemsida som integrerar Git och tillåter användare att lägga upp sin kod där gratis förutsatt att den är offentlig. Github erbjuder även en visualisering av många av Gits funktionaliteter såsom en grafisk vy över hur koden förändrats över tid eller skillnaderna mellan två versioner. Github erbjuder även verktyg för att hantera koden såsom begränsning av rättigheter av //NOTE THIS IS NOT DONE 

Github är gratis att använda och visualiserar många av de funktionaliteter som Git har. Bland annat så ges en visualisering av hur koden förändrats vid varje tilläg och möjlighet att se skillnader mellan olika versioner av den. Denna funktionalitet finns redan i Git men Github ger detta UI som är lättare att ta till sig. 

ger denna funktionalitet ett UI som är lättare att ta till sig. 

Github är gratis att använda och ger en grafisk vy över hur Gitrepot ser ut och har förändrats sen det startades. 

\subsection*{Travis}
Ett opensource testningstjänst som erbjuder autmatisering för projekt som använder Github. Travis är gratis för opensource projek. Tjänsten används i kontinuerlig integration för att köra tester på alla grenar innan de kan sammanfogas in i huvudgrenen\cite{Travis:online}.

\subsection*{Slack}
Slack är ett kommunikationsverktyg som ofta används proffesionelt inom IT och mjukvaruutveckling. Funktionsmässigt är det ett chatprogram för datorer och mobila enheter.

\subsection*{Latex}
Latex är ett gratis verktyg som kombinerat med en latex-editor används för att skapa artiklar och rapporter av olika slag. Det är ett väldigt kraftigt verktyg som med många formateringsmmöjligheter och tillskillnad ifrån många andra text-editors så separerar Latex mellan det du skriver och slutprodukten som ska läsas.
