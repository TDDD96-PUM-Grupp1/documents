\section{Metod}
\label{sec:joel_a-method}

För att sammanställa hur stor miljöpåverkan produkten haft så analyseras den utefter Greensoft modellen. Detta är inte hur Greensoft egentligen avsett att sin modell används, utan den är egentligen tänkt att användas för att både förebygga och analysera miljöpåverkan. Men eftersom utvecklingsfasen av projektet är över så kan enbart delen gällande analys av miljöpåverkandeeffekter användas. I diskussionsdelen ska dessa effekter sedan vägas mot varandra för att på så sätt se hur stor miljöpåverkan produkten haft.

\subsection{Molntjänster}
Under utvecklingsprocessen använde sig teamet av flertal olika molntjänster såsom Google drive, Github och Googles sökmotor. Det är extremt svårt att uppskatta hur mycket ström som används när man använder dessa tjänster av flera skäl. Bland annat så påverkar tid på dygnet hur effektiva Googles data centrum är\cite{google-warehouse}. Ju effektivare datacenter desto mindre ström används av en enskild användare och desto mindre miljöutsläpp. Enligt Google själva så släpper en sökning ut ungefär 0.2 gram koldioxid\cite{google-blog}. Hur Googles arkitektur kring deras sökmotor ser ut är oklart, men i teorin så tar de emot en begäran, söker igenom deras databas och skickar ett resultat till användaren. Ingen data behöver förändras så en sökning borde klara sig utan någon WRITE begäran. I Google drive historiken finns åtta typer av operationer: \textit{redigera, kopiera, kommentera, byta namn, ladda upp, skapa, flytta, ta bort}. Alla dessa operationer förändrar eller skapar en datastruktur och kräver mer arbete än en sökning som enbart läser. Uppskattningen denna rapport utgår ifrån är att dessa operationer kräver dubbelt så mycket arbete som en sökning. Men sen så tillkommer också det faktumet att innan ett objekt modifieras så har det alltid hämtats ut först. Vilket enligt tidigare resonemang släpper ut 0.2 gram \ce{CO2}. Så totalt räknar denna rapport med att alla de uppräknade operationerna på Google drive släpper ut 0.6 gram \ce{CO2}. Github har tillskillnad ifrån Google inte sagt hur mycket koldioxid de släpper ut så vi antar att de ligger på samm nivå som Google med 0.6 gram \ce{CO2} per push operation. Förutom dessa molntjänster finns långt fler hemsidor som utvecklarna använts sig av som inte tas med i denna analys. Sidor såsom Stackoverflow, Pixis API dokumentation och Mozillas Javascript dokumentation. Även delar av Github som antalet pull-begäran räknas inte på, allt detta av den simpla anledningen att det inte finns någon historik på det.