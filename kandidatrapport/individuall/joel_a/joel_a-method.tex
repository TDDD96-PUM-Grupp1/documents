\section{Metod}
\label{sec:joel_a-method}

\subsection{Val av applikationer}
I denna undersökning


Undersökning är tänkt att genomföras genom att använda sig av Trepn för att jämföra hur mycket ström vår produkt drar jämför med native appar som då inte använder sig av någon browser. Dessa appar ska delar in i 3 kategorier, appar som troligtvis drar mycket mer ström (tex ett spel eller videospelare av något slag), appar som troligtvis drar mycket mindre (tex en simpel text editor av något slag), och appar som drar ungefär lika mycket. Problem kan uppkomma med den sista kategorin eftersom det blir lite godtyckligt vad som borde dra ungefär lika mycket, men skulle tro tex google maps borde vara liknande då den likt vår app konstant skickar ut sensor data.

De valda apparna ska sen användas kontinuerligt i 10s, 30s, 2min, och 5min och skriva ner hur mycket processorkraft och minne de använder samt hur mycket information de skickar ut mot internet. Möjligtvis ska Trepns inbyggda estimering av batteri-användning också användas.

\textbf{Här följer några källor som kan användas}\\

\begin{enumerate}
	\item ''Empirical evaluation of two best practices for energy- efficient software development'' -- Artikel 1 ifrån hållbarhetsseminariet
	\item ''Framing sustainability as a property of software'' -- Artikel 2 ifrån hållbarhetsseminariet
	\item ''Developing a sustainability non-functional requirements framework'' -- Artikel 3 ifrån hållbarhetsseminariet
	\item ''The GREENSOFT model: A reference model for green and sustainable software and its engineering'' -- Artikel 4 ifrån hållbarhetsseminariet
\end{enumerate}

Antagligen så kommer inte alla dessa hålbarhetsartiklar användas utan det som kommer att lyftas ifrån dem är framförallt hur man kan optimera olika operationer för att minska ström användning (artikel 1 nämner detta) vilket relaterar till webläsare vs native app. Ifrån de andra tre artiklarna är det framförallt hur mycket ström som mjukvara användar och miljöpåverkan av detta, samt hur miljöpåverkan av att producera ny hårdvara och vikten att stödja gammal hårdvara är ifrån ett hållbarhetsperspektiv.