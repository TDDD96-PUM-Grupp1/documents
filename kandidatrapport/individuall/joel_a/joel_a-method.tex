\section{Metod}
\label{sec:joel_a-method}
Tanken är att samla in data genom att skriva ett script som går igenom repositiories för open-source javascript projekt. En begränsad mängd av projekt att undersöka kan väljas genom att exempelvis se till de högst rankade projekten på github. Hur detta urval görs är dock viktigt och bör poängteras noga.

Själva datainsamlingen kan göras genom att i dessa  projekt läsa av deras beroenden från filen package.json, som används av npm. Detta ger direkta beroenden.

Hela trädet av beroenden kan fås från package-lock.json, som ofta lagras tillsammans med package.json. Detta ger indirekta beroenden och det kan vara möjligt att göra vidare undersökningar på denna graf (exempelvis nivåer av beroenden)

För att få fram inforamtion om effekterna av beroenden får en större informationssökning genomföras. Det verkar finnas gott om akademiska källor av intresse för detta och troligen även bloggposter från personer med stor erfarenhet av javascript-projekt (som kan användas ur ett kritiskt perspektiv).

Till viss del kan den egna gruppens erfarenheter presenteras. Det vore även möjligt att intervjua någon/några från en annan grupp inom kursen och därmed få in en större bredd av erfarenheter.\\

\textbf{Här följer några källor som kan användas}\\
\href{https://dspace.cvut.cz/bitstream/handle/10467/68195/F8-DP-2017-Zitny-Jakub-thesis.pdf?sequence=1&isAllowed=y}{Liknande undersökning}\\
\href{https://arxiv.org/pdf/1709.04638.pdf}{Effekter av micro-paket i npm}\\
\href{https://repository.tudelft.nl/islandora/object/uuid:3a15293b-16f6-4e9d-b6a2-f02cd52f1a9e?collection=education}{Effekter av beroenden i npm för säkerhet}\\ \href{https://arxiv.org/pdf/1710.04936.pdf}{Jämförelse mellan pakethantering i olika programmeringsspråk}\\ \href{http://www.scotthenry.ca/wp-content/uploads/2018/01/Report.pdf}{Undersökning om uppdatering av beroenden i npm}\\
