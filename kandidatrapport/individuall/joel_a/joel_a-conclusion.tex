\section{Slutsatser}
\label{sec:joel_a-conclusion}

De slutsatser man kan dra av denna rapport är att småskalig utveckling av mjukvara inte har en stor påverkan på miljön. I resultatet kan vi se att CO2 utsläppet för belysning av gruppensarbetsrum var flera gånger mer än användningen av gruppmedlemmarnas alla datorer. Detta visar att gruppens beslut att ignorera miljömässiga beslut gällande hur produkten ska utvecklas är rimlig till följd av hur småskalig produkten är. Detta gäller dock enbart småskaliga projekt som är tänka att köras av en kund maximalt ett fåtal gånger i månaden, för större system med en stor användbar kan dessa slutsatser inte appliceras. Gällande vilka förbättringar gruppen kunde gjort under utvecklingen av produkten så hittas inga specifika för mjukvaruutveckling eftersom det inte behövdes göras några sådana i detta projekt. Sammanfattningsvis så är det dock värt att notera att även om utvecklingsprocessen påverkan på miljön var liten så var den fortfarande strikt negativ. Utvecklingen generade utsläpp och uppmuntrade inte hållbar utvecklingen, varken med hur själva arbetsprocessen utfördes eller med vad målet med själva produkten var.