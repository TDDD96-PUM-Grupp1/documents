\section{Slutsatser}
\label{sec:joel_a-conclusion}

\subsection{Produktens miljöpåverkan}
Sammanfattningsvis så visar denna rapport svårigheten i att avgöra ifall mjukvara kan anses vara miljövänlig eller inte. Mjukvaras livscykel kan delas in i flera delar, denna rapport har framförallt fokuserat på utvecklingsdelen. Trots detta så har processen att analysera framförallt denna del varit svår då fullständig information om hur mycket energi som används vid användningen av olika tjänster saknas. Analysen enligt Greensofts modell gjord i avsnitt~\ref{joel_a-disc-team} visar hur följdeffekterna av mjukvara är både positiva och negativa. Detta gör det ännu svårare att avgöra ifall mjukvaran har en positiv eller negativ inverkan på miljön då dessa effekter måste vägas mot varandra. Detta görs ännu svårare av att effekter kan uppkomma när som under mjukvarans livslängd, vilket betyder att icke miljövänlig mjukvara kan bli det senare i dess livsskede. Enligt denna princip så skulle det inte gå att säga ifall teamets produkt kommer ha en negativ eller positiv miljöpåverkan förän efter den tas ur bruk.

\subsection{Utvecklingsprocessen}
Även om det är svårt att dra slutsatser gällande produktens miljöpåverkan så kan slutsatser gällande rapportens huvudfokus göras. Utvecklingsprocessen hade en direkt negativ miljöpåverkan, men inte en stor sådan. I resultaten så uppskattas utvecklingstiden till $400 * 0.3$ timmar per person, vilket ger en total tidsinstats på 960 timmar. Trots detta så generade utvecklingsprocessen mindre utsläpp än förbränning av 200 ml bensin enligt diskussions avsnitt \ref{joel_a-disc-team}. Jämför vi teamets utsläpp med förbränning av bensin så skulle teamet kunna arbeta i över 4800 timmar per liter bensin och fortfarande generera en likvärdig mängd koldioxidutsläpp.

Trots de många förenklingar och antaganden som gjorts i denna rapport så tyder denna siffra på att teamets utvecklingsprocess påverkat miljön högst marginellt. Även ifall den approximerade siffran för teamets koldioxidutsläpp var fel med en magnitudsstorlek så skulle fortfarande utsläppet fortfarande vara marginellt. Slutsatsen man kan dra av detta är att småskalig mjukvaruutveckling inte har en signifikant påverkan i form av koldioxid.

\subsection{Förbättringsmöjligheter}
Trots utvecklingsprocessens låga miljöinverkan så utforskades metoder för att minska den. Det metoder som kom upp handlade alla om att minimera teamets användning av molntjänster, då dessa stod för en överväldigande majoritet av teamets utsläpp. Metoder såsom att använda molntjänster under deras mest trafikerade och även mest effektiva tidpunkter utforskades. Dessa metoder skulle dock minska teamets effektivitet. Den åtgärd som inte hade denna effekt var att inte lägga upp dokument på molntjänsten Google Drive som inte behövde vara där. Detta var den enda rimliga åtgärden som hittades eftersom utvecklingsprocessens utsläpp var så låga och de andra åtgärderna skulle minskade teamets effektivitet. En avvägning mellan effektivitet eller minskat utsläpp lutar mot effektivitet i detta fall då utsläppen redan är så låga.

Slutsatsen man kan ta med sig gällande denna undersökning är att koldioxidutsläppen av teamets molntjänster var tre magnituder större än koldioxidutsläppen för strömanvändningen. Detta är intressant för framtida undersökningar inom hållbar utveckling med inriktning av mjukvaruutveckling. Denna rapport tar endast fram approximerade siffror, men trots detta så kan man se vilken del av utvecklingsprocessen där den negativa miljöpåverkan enklast kan minskas.