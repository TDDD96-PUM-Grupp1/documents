\section{Slutsatser}
\label{sec:joel_a-conclusion}

De slutsatser man kan dra av denna rapport är att småskalig utveckling av mjukvara inte har en stor påverkan på miljön. I resultatet kan vi se att \ce{CO2} utsläppet för belysning av gruppensarbetsrum var flera gånger mer än användningen av gruppmedlemmarnas alla datorer. Samtidigt så visade sig användingen av molntjänster, framförallt Google-sökningar, genera många gånger fler koldioxidutsläpp än detta. De aktiviteter som genererade dessa utsläpp är dock svåra undvika, till exempel så skulle en minskning av Google-sökningar leda till sämre effektivitet för gruppen. Mängden koldioxid som släpptes ut under utvecklingens gång kan motsvaras till förbränningen av en liter bensin och är minimal med tanka på projektets storlek. Då denna mängd är så liten så kan en minskning av den ses som överflödig, utvecklingen som gruppen utförde var redan miljövänlig. Detta motiverar även teamets beslut att inte använda ett miljömässigt perspektiv under utvecklingsprocessen då utsläppen blev så låga trots detta. Sammanfattningsvis så kan slutsatsen att småskalig mjukvaruutveckling inte har någon signifikant påverkan på miljön. Teamet kunde gjort många saker under detta projekt för att minska sina utsläpp, men de mest signifikanta sakerna skulle inte vara relaterade till utvecklandet av denna produkt. 

