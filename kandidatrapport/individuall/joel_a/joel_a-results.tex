\section{Resultat}
\label{sec:joel_a-results}
I denna del ska alla effekter på miljön eliciteras. 
\subsection{Nivå ett effekter}
Nivå ett effekter i Greensoft modellen är de effekter som direkt följer av utvecklingen eller användningen av produkten. Den första fasen i Greensoft modellen är utvecklingsfasen och i den kan följande effekter på miljön hittas:
\begin{itemize}
	\item Strömanvändning för utvecklarnas datorer
	\item Övrig strömanvändning av utvecklarna
	\item Transport till arbetsplats
	\item Utvecklarnas internetanvändning
\end{itemize}

Projektet att skapa produkten ingick i en större universitetskurs som tog 3200 arbetstimmar för hela gruppen, uppskattningsvis kan åtminstone hälften av denna dit sägas ha lagts på produkten. Gruppen bestod av åtta personer som utvecklade produkten på en laptop. All utvecklingstid gjordes dock inte vid en dator, visa moment såsom skapandet av arkitekturen och gruppmöten så användes ingen dator. Dessutom så skedde delar av utveckling med flera gruppmedlemmar vid samma dator. För att få in detta i uppskattningen kan vi säga att 90\% av utvecklingstiden skedde med en laptop. Så totalt blir strömanvändningen 1600 * 8 * 0.9x, där x står för den genomsnittliga strömanvändningen för en laptop. Eftersom en laptops strömanvändning beror på så många faktorer är det svårt att få en korrekt uppskattning av hur mycket ström som används. En ungefär värde på x kan fås genom observationen att en av medlemmarnas laptop kunde användas i ungefär fyra timmar innan den behövde laddas. Denna laptop är av modell ''Lenovo ideapad Y700'' och har enligt tillverkaren \cite{lenovo} ett batteri med 60 watt timmar. En urladdning på av detta på fyra timmar ger oss en timmes användning av 15 watt, viilket x kan approximeras till. Då får vi följande approximering för strömanvändning av datorerna gruppen använde sig av under utecklingsprocessen: $$1600 * 8 * 0.9 * 15 = 17.28KW$$

Utvecklarnas övriga strömanvändninge kommer framförallt ifrån strömanvändning i rummet där utvecklingen sker, saker såsom belysning, ventilation, uppvärmning, vatten och kaffemaskinen. Uppvärmningen av rummet kan ignoreras i detta fall då alla rum gruppen arbetade i skulle varit uppvärmda oavsett ifall de befann sig där eller inte. I stort sett all utveckling skedde i arbetsrum på Linköpings Universitet som är uppvärmda oavsett ifall de används och kontoret på Cybercom som också skulle vara uppvärmt oavsett gruppens närvaro. I energimyndighetens rapport ifrån 2007\cite{emynd} så använder det genomsnittliga svenska kontoret $23.0kWh/m^2$ för belysning och $2.6kWh/m^2$ för ventilation. En approximering av storleken på rummen gruppen arbetade i är 12 kvadratmeter, vilket ger oss en ungefärlig strömanvändning på: $$1600 * 25.6 * 12 = 491.52kW$$

De resterande nivå ett effekterna är inte nödvändigtvis försumbara, men de är svårare att bestämma konkreta siffror. Utvecklarnas transport till arbetsplatsen skedde framförallt genom gång eller cykel och är försumbar, men kategorin internetanvändning inte är det. I utvecklingen av produkten användes molntjänster såsom Githubs automatiska tester, Google drive och ett stort antal hemsidor. Användning av detta kräver både elektricitet från utvecklarnas datorer vilket syns i approximationen, men servrarna som ''hostar'' sidorna och deras strömanvändning  finns inte med i den. Strömanvändningen av alla routrar som skickar paketen till och från utvecklarna finns inte heller med i någon approximation.

\subsection{Nivå två}
Nivå två effekter är effekter på miljön som uppkommer tillföljd av nivå ett effekterna. 

\subsection{Sammanställning}
\begin{enumerate}
	\item Nivå ett effekter
	\begin{enumerate}
		\item En nivå ett effekt
	\end{enumerate}
	\item Nivå två effekter
	\begin{enumerate}
		\item En nivå två effekt
	\end{enumerate}
	\item Nivå tre effekter
	\begin{enumerate}
		\item En nivå tre effekt
	\end{enumerate}
\end{enumerate}