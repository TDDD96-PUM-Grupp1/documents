\section{Introduktion}
\label{sec:joel_a-introduction}
Denna del ska introducera varför denna rapport skrivs och vad den försöker uppnå.

\subsection{Bakgrund}
Spelet som utvecklades under detta projekt skrevs som en Progressive Web App (PWA) i Javascript för att göra det plattformsoberoende.  Detta är en väldigt positiv egenskap och betyder att appen enbart behöver skrivas i ett språk, men det betyder också att appen körs i en webläsare vars primära funktion är att visa hemsidor, inte att spela spel. 


\subsection{Syfte}
Det som ska utredas är hur resursanvändningen av minne, processor och mängden data skickat genom internet skiljer sig mellan mellan olika typer av appar. Med hjälp av detta ska det utredas ifall en applikation skriven i Javascript drar mer batterietid än en applikation skriven i ett annat språk.

\subsection{Frågeställning}
\label{subsec:joel_a-research-questions}

\begin{enumerate}
\item Hur många klockcykler använder vår applikation jämfört med andra typer av applikationer?

\item Hur mycket minne använder vår applikation jämfört med andra typer av applikationer?

\item Hur mycket data skickar vår app jämfört med andra typer av applikationer?

\item Hur mycket batteritid drar en applikation skriven i javascript jämfört med en applikation skriven i ett annat språk.

\end{enumerate}

\subsection{Avgränsingar}
\label{subsec:joel_a-delimitations}

Denna undersökning är en del av ett annat separat projekt och har därav en begränsad tidsbudget så produkten som utvecklades i projektet jämförs enbart med tre andra typer av appar. Detta är få applikationer för att kunna dra absoluta slutsatser gällande javascript och webläsares energianvändning, men kan fungera som en indikator.\\\\
Det tre kategorierna som produkten testas mot kommer enbart att representeras av en applikation, vilket kan skapa fel då applikationen som ska representera sin kategori kan vara en avvikare i hur den använder mobilens resurser.

