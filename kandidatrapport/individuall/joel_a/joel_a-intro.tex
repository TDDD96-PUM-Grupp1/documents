\section{Introduktion}
\label{sec:joel_a-introduction}
Denna del ska introducera varför denna rapport skrivs och vad den försöker uppnå.

\subsection{Bakgrund}
Vid skapandet av produkten gjorde gruppen många avvägningar, vilken typ av arkitektur uppfyller produktens syfte bäst, hur bör vi arbeta för att vara så effektiva som möjligt, vilka kvalitetsfaktorer är viktigast för kunden. En aspekt som inte togs upp var dock den miljömässiga aspekten av produkten. Varken kunden eller gruppen satte miljömässiga krav på produkten, så denna aspekt hamnade helt i skymundan. Mjukvara är tänkt att köras på hårdvara, så all mjukvara har en direkt negativ påverkan på miljön genom sin energiförbrukning. Men detta är inte en fullständig bild då mjukvara kan användas för att uppfylla ett behov som fanns ifrån tidigare, och göra det på ett mer hållbart sätt än tidigare. För att göra en djupgående analys av mjukvarans miljöpåverkan så skapade Green Software Engineering (GREENSOFT TODO) en modell för just detta syfte.

\subsection{Syfte}
Syftet med denna rapport är att använda Greensofts modell för att utvärdera miljöpåverkan utvecklingen av gruppens produkt har haft. Detta skapar en bild av en aspekt i utvecklingsproccessen som gruppen till stor del ignorerat tidigare och kan även lyfta fram vilka delar i utvecklingen som kunde ha gjorts annorlunda för att få en mer hållbar utvecklingsproccess.



\subsection{Frågeställning}
\label{subsec:joel_a-research-questions}

\begin{enumerate}
\item Vilken negativ påverkan på miljön har utvecklandet av produkten haft?

\item Vilken positiv påverkan på miljön har produkten haft eller kommer troligtvis att ha?

\item Vad i utvecklingsproccessen kunde gjorts annorlunda för att få en mer miljömässigt hållbar produkt eller utvecklingsproccess?

\end{enumerate}

\subsection{Avgränsingar}
\label{subsec:joel_a-delimitations}

*Lägg till