\section{Introduktion}
\label{sec:joel_a-introduction}

\subsection{Bakgrund}
Vid skapandet av produkten gjorde gruppen många avvägningar, vilken typ av arkitektur uppfyller produktens syfte bäst, hur bör vi arbeta för att vara så effektiva som möjligt, vilka kvalitetsfaktorer är viktigast för kunden. En aspekt som inte togs upp var dock den miljömässiga aspekten av produkten. Varken kunden eller gruppen satte miljörelaterade krav på produkten, så denna aspekt hamnade helt i skymundan. Mjukvara är tänkt att köras på hårdvara, så all mjukvara har en direkt negativ påverkan på miljön genom sin energiförbrukning. Men detta är inte en fullständig bild då mjukvara kan användas för att uppfylla ett behov som fanns ifrån tidigare, och göra det på ett mer hållbart sätt än tidigare. För att bestämma hur hållbar mjukvaran är så måste en mer gedigen analys göras. En mall för en sådan analys skapades av Green Software Engineering och kallas för Greensoft-modellen\cite{greensoft}, i den så analyseras miljö påverkan i flera steg.

\subsection{Syfte}
Syftet med denna rapport är att analysera utvecklingsprocessen av produkten för att undersöka hur hållbar denna var. För att göra denna analys ska Greensoft-modellen användas. Detta skapar en bild av en aspekt i utvecklingsprocessen som gruppen till stor del ignorerat tidigare. Det kan även lyfta fram vilka delar i utvecklingen som kunde ha gjorts annorlunda för att få en mer hållbar utvecklingsprocess.

\subsection{Frågeställning}
\label{subsec:joel_a-research-questions}

\begin{enumerate}
\item Vilken negativ påverkan på miljön har utvecklandet av produkten haft?

\item Vilken positiv påverkan på miljön har produkten haft eller kommer troligtvis att ha?

\item Vad i utvecklingsprocessen kunde gjorts annorlunda för att få en mer miljömässigt hållbar produkt eller utvecklingsprocess?

\end{enumerate}

\subsection{Avgränsingar}
\label{subsec:joel_a-delimitations}

*Lägg till