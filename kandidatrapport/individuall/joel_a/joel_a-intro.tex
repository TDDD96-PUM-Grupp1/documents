\section{Introduktion}
\label{sec:joel_a-introduction}
Vid skapandet av produkten gjorde gruppen många avvägningar, vilken typ av arkitektur uppfyller produktens syfte bäst, hur bör vi arbeta för att vara så effektiva som möjligt, vilka kvalitetsfaktorer är viktigast för kunden. En aspekt som inte togs upp var dock den miljömässiga aspekten av produkten. Varken kunden eller gruppen satte miljörelaterade krav på produkten, så denna aspekt hamnade helt i skymundan. Mjukvara är tänkt att köras på hårdvara, så all mjukvara har en direkt negativ påverkan på miljön genom sin energiförbrukning. Men detta är inte en fullständig bild då mjukvara kan användas för att uppfylla ett behov som fanns ifrån tidigare, och göra det på ett mer hållbart sätt än tidigare. För att bestämma hur hållbar mjukvaran är så måste en mer gedigen analys göras. En mall för en sådan analys skapades av Green Software Engineering och kallas för Greensoft-modellen\cite{greensoft}. I denna model så analyseras miljö påverkan i flera steg för att få en tydlig bild av mjukvarans påverkan i alla delar av dess livscykel.

\subsection{Syfte}
Syftet med denna rapport är att analysera utvecklingsprocessen av produkten för att undersöka hur hållbar denna var. Denna analys ska följa Greensofts modell för utvärdering av hållbar mjukvara. Detta ger en bild av en aspekt i utvecklingsprocessen som gruppen till stor del inte tagit upp tidigare. Denna rapport ska också konkretisera miljöpåverkan gruppen haft genom att lyfta fram faktiska värden på hur mycket koldioxid som släppts ut till följd av produktens utvecklande.

\subsection{Frågeställning}
\label{subsec:joel_a-research-questions}

\begin{enumerate}

\item Hur mycket koldioxid har släppts ut till följd av utvecklingsprocessens strömanvändning?

\item Hur mycket koldioxid har släppts ut till följd av teamets användning av molntjänster?

\item Vad i utvecklingsprocessen kunde gjorts annorlunda för att få en mer miljömässigt hållbar produkt eller utvecklingsprocess?

\end{enumerate}

\subsection{Avgränsingar}
\label{subsec:joel_a-delimitations}
Denna rapport använder sig av många uppskattningar och generaliseringar för att få fram sina värden. Detta sker eftersom mer exakt data inte finns tillgänglig, men det betyder att alla siffror som fås fram har en påtaglig felmarginal. De siffror som tas fram bör inte tolkas som något exakt utan borde snarare ses som indikator på vilken magnitudsskala det riktiga värdet ligger på.