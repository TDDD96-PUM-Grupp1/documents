\section{Introduktion}
\label{sec:joel_a-introduction}
Vid skapandet av produkten gjorde gruppen många avvägningar, vilken typ av arkitektur uppfyller produktens syfte bäst, hur bör vi arbeta för att vara så effektiva som möjligt, vilka kvalitetsfaktorer är viktigast för kunden. En aspekt som inte togs upp var dock den miljömässiga. Varken kunden eller gruppen satte upp miljörelaterade krav och denna aspekt hamnade helt i skymundan. Målet med denna rapport är att belysa denna bortprioriterade aspekt och noga utreda miljöpåverkan av teamets utvecklingsprocess.

\subsection{Syfte}
Syftet med denna rapport är att analysera utvecklingsprocessen ur ett miljömässigt perspektiv. Det som ska utredas är hur både produkten och dess framställning påverkat miljön samt hur dessa kan tänkas påverka miljön i framtiden. Dessutom ska det utredas ifall ett förebyggande arbete skulle kunna ha minskat teamets miljöinverkan.

\subsection{Frågeställning}
\label{subsec:joel_a-research-questions}
För att försöka uppnå rapportens syfte så har nedstående frågeställningar tagits fram. De har tagits med fram med åtankte given till att hålla de så konkreta och så mätbara som möjligt.

\begin{enumerate}
\item Hur mycket koldioxid har släppts ut till följd av utvecklingsprocessens strömanvändning?

\item Hur mycket koldioxid har släppts ut till följd av teamets användning av molntjänster?

\item Vad i utvecklingsprocessen kunde gjorts annorlunda för att minska koldioxidutsläppen under utvecklingsprocessen?

\end{enumerate}

\subsection{Avgränsningar}
\label{subsec:joel_a-delimitations}
Rapportens syfte är att mäta teamets miljöpåverkan, en väldigt komplicerat upp-
gift som kan göras oändligt detaljerad. För att förenkla problemet så mäter denna rapport
miljöpåverkan i form av koldioxidutsläpp generad vid strömanvändning. Detta betyder att andra former av utsläpp såsom tungmetaller eller metangas inte undersöks.