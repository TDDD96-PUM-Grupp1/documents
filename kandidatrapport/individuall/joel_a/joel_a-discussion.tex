
\section{Diskussion}
\label{sec:joel_a-discussion}

\subsection{Utvecklingsprocessens miljöpåverkan}
Gruppens approximerade utsläpp på 24.8 gram \ce{CO2} är grov approximering och missar utsläpp ifrån flera faktorer såsom strömmen kaffemaskinen använde och strömmen molntjänsterna använde. Så uppskattningen är sannolikt en underdrift, men trots det så är ett koldioxid utsläpp under 25 gram högst marginelt med tanke på omfattningen av utvecklingsprocessen. De mest sannolika faktorerna för det låga värmen är antagligen det låga \ce{CO2} utsläppet av energi producerat ifrån vatten och kärnkraft som representerar runt 90\% av sveriges eltillverkning (TODO källa). Den andra viktiga faktorn är hur småskalig utvecklingen av produkten var, med enbart åtta personer och utan stor utrustning såsom serverhallar så används helt enkelt inte en mängd elektricitet som är relevant ur ett storskaligt miljöperspektiv. Nivå ett effekterna verkar alltså inte ha haft någon större påverkan på hållbar utveckling.


\subsection{Produktens miljöpåverkan}
Produkten har ännu inte använts vid skrivandet av denna rapport så den har inte haft någon miljöpåverkan förutom dess utvecklingsprocess som beskrevs ovan. Men de nivå två effekterna som den kan tänka sig ha, såsom att den uppmuntrar fler företag att skapa ett spel för sina presentationer på mässor, verkar inte vara något att oroa sig över då den låga påverkan gruppens utvecklingsprocess haft på miljön. Dock är det värt att notera att ifall en sådan mjukvara utvecklas på en plats där elektriciteten framställs på ett mer omiljövänligt sätt så kan denna process vara påtagligt värre för miljön. Detta är dock väldigt spekulativt och det är osannolikt att ett studentarbete i praktiken skulle skapa stora vågeffekter i hur mjukvara utvecklas. Troligtvis så kommer inga nivå två eller tre effekter att skapas utifrån produkten gruppen sammanställt.


\subsection{Nivå två och tre effekter}
Nivå två effekter är effekter på miljön som uppkommer tillföljd av nivå ett effekterna eller tillfölja av användning av själva produkten. Vid skrivandetsstund har inte produkten använts av kunden än, så några nivå två eller tre effekter har inte ännu uppkommit utan de är rent spekulativa. En potentiell nivå två effekt skulle kunna vara att produkten blir populär på mässor och får personer att använda sin mobil för att spela spelet. Detta är syftet med produkten, men ur ett miljömässigt perspektiv är det en negativ konsekvens eftersom det kräver både ström och bredband. På grund av produktens snäva användningsområde och få nivå ett effekter finns det inte heller många nivå två effekter att finna. Nivå tre effekter är samhälliga förändringar som sker ur nivå två effekter. Produkten gruppen har skapat lär ha väldigt marginell påverkan på samhället i helhet, men man kan kan se den som en del av en större förändring. En förändring som kan ske ifall produkten är framgångsrik är att flera företag börjar använda sig av spel på sina mässor för att locka personer till deras presentationer. Det skulle betyda att många fler spel behöver utvecklas samt att ström och bandbredsanvändningen på mässor skulle gå upp, vilket inte är förenligt med hållbar utveckling. De andra nivå tre effekterna som produkten kan leda till är att den hjälper till att popularisera IoT, vilket är en del av dess syfte. Detta kan ses som positivt eller negativt och beror helt på hållbar utvecklingen inom den branschen är. 
