
\section{Diskussion}
\label{sec:joel_a-discussion}

\subsection{Metodikens problem}
Metodiken som används i denna rapport innehåller många antaganden och storskalig statistik använd på en mindre nivå. Till exempel så används statistik gällande hela Sveriges energiproduktion när koldioxidutsläppen för utvecklingsprocessen beräknas. Detta är inte helt korrekt då det som påverkade projektet var energiproduktionen i Linköping där det utfördes. Hur energin produceras varierar lokalt och rikssnittet behöver inte nödvändigtvis vara representativt för regionen. En annan förenkling är antagandet att hela datorns batteri går åt på fyra timmar, vilket är en grov approximation som dessutom varierar beroende på vilken aktivitet som utförs på den. Utöver detta så hade varje teammedlem en unik modell på sin arbetsdator. Sammantaget så bygger resultatet för teamets elektricitetsförbrukning på flera förenklingar. Till följd av detta så bör inte heller detta resultat ses som något annat än en grov överslagsräkning med väsentliga felmarginaler.

Ett annat problem metodiken hade är vilka delar som kan räknas till utvecklingsprocessens strömanvändning. Denna rapport hanterar detta synonymt med hur mycket elektricitet utvecklarnas datorer drog. Men detta är inte helt korrekt då en del aspekter utelämnas, till exempel elektriciteten som använts åt att producera all kaffe teamet konsumerat. Det är dock problematiskt att räkna på dessa siffror då de kan innehålla allt ifrån laddning av mobiler till elkonsumtionen som använts för att ta en utvecklare till dess arbetsplats. För en korrekt mätning av dessa faktorer skulle det krävts att teamet gjort dessa mätningar kontinuerligt under projektets gång. Då denna rapport skrivs i ett försök att väga upp ett bristande miljötänk under utvecklingsprocessen så bör det framgå att några sådana mätningar inte gjordes.

Sammanfattningsvis så kan anledningen till de många antagandena kopplas in till frågeställningen där det specificeras att konkreta siffror eftersträvas. Detta hade inte varit möjligt utan dessa antaganden. Det är dock värt att notera att det finns enorm utökningspotential i flera delar som denna rapport bygger på. Allt ifrån hur mycket koldioxid som släpps ut vid elkraftverken i Linköping till hur effektiva Google och Githubs databaser är.

\subsection{Potentiella förbättringar}
Ifall vi jämför resultaten så ser vi att det är en magnitudskillnad på koldioxidutsläppen, där molntjänster generade koldioxidutsläpp på ungefär 400 gram jämfört med strömanvändningens 0.5 gram. Rimligen så bör alla förslag för att minska koldioxidutsläppen rikta in sig på användningen av molntjänster på grund av detta. Ett förslag skulle vara att minska användningen av Google Drive. En del objekt som sparats på Google Drive användes väldigt sällan av gruppen, till exempel veckorapporter som skrevs under projektets gång. Dessa behövde inte vara på Google Drive då varje gruppmedlem kunde ha tillgång till dem genom sina mail-konton. Dessutom så skrevs dessa rapporter för att teamets handledare och examinator skulle få en tydligare bild av vad teamet gjorde. Teamet hade själva ingen användning av dessa rapporter då informationen kunde fås lättare genom till exempel Slack. 

En annan potentiell miljöförbättring skulle vara att enbart förändra dokument på Google Drive vid specifika tider på dygnet. I en rapport av Google~\cite{google-warehouse} så skriver de hur deras datacenter är som mest effektiva när de används som mest. Detta eftersom de då kan göra flera jobb på en gång. Detta betyder att ifall gruppen väntar med att göra ändringar, för att sedan göra en hop av dem samtidigt vid en tid då Googles datacenter är som mest effektiva, så skulle mindre elektricitet krävas och mindre koldioxid genereras. Problemet med detta förslag är dock att det minskar effektiviteten hos teamet och ökar risken att data försvinner då det går längre tid innan data  säkerhetskopieras till molnet.


\subsection{Teamets miljöpåverkan}
\label{joel_a-disc-team}
Den estimerade siffran för projektets koldioxidutsläpp var ungefär 400 gram. Detta kan jämföras med utsläppet ifrån förbränningen av en liter bensin. Enligt amerikanska Energy Information Administration så skapar förbränning av en ''gallon'' E10 17.6 ''pounds'' koldioxid~\cite{co2-bensin}. I SI-systemet så betyder detta att en liter E10 ger upphov till utsläpp av ungefär 2.1 kg koldioxid. Hela utvecklingsprocessen \ce{CO2} utsläpp är alltså mindre än utsläppen vid förbränning av 200 ml bensin. 

Denna jämförelse är dock enbart baserad på utvecklingsprocessen, vilket är en del av flera i mjukvarans livscykel. Under användningsfasen av mjukvaran så konsumeras mer elektricitet och miljöpåverkan kommer därav att öka. Enligt Greensofts modell så behöver mjukvara utvärderas i tre steg~\cite{greensoft}. Det första steget är direkta effekter, i teamets fall är dessa energianvändning som krävs för att köra produkten. Produkten i sig är ett spel tänkt att köras på mässor med en stor gemensam spelskärm och flertal användare som styr sina karaktärer med sina mobiler. Detta betyder att produkten potentiellt drar mycket ström då den tänkts köras på flera enheter samtidigt. Dessa effekter av produkten är strikt negativa.

Den andra nivåns av effekter som produkten skapar uppkommer till följd av dess nivå ett effekter. Det betyder alltså effekter som uppkommer till följd av användning av produkten. De kan vara saker som att fler personer går till Cybercoms utställning och lär sig om det företaget. Den sista nivån är då effekter som bygger på nivå två effekter. Det skulle kunna vara saker såsom att fler personer väljer att arbeta med Cybercom då de lärt sig mer om företaget. Dessa effekter kan vara både positiva och negativa, till exempel så profilerar sig Cybercom starkt som miljövänliga, någon som avspeglas i deras ’’code of conduct’’~\cite{cyber-cod}. En utökning av detta företag skulle kunna ses som positivt ifall de arbetar på ett sätt som är miljömässigt positivt. Men effekterna kan också vara negativa. En möjlig nivå tre effekt skulle vara att fler personer använder sig av konceptet att utveckla ett spel att demonstrera på mässor. Då skulle flera utvecklingsprocesser genomföras, något som har en direkt negativ påverkan på miljön.