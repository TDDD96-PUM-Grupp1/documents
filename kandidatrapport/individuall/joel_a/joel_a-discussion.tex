
\section{Diskussion}
\label{sec:joel_a-discussion}

\subsection{Utvecklingsprocessens miljöpåverkan}
Gruppens approximerade utsläpp på 24.8 gram C02 är grov approximering och missar utsläpp ifrån flera faktorer såsom strömmen kaffemaskinen använde och strömmen molntjänsterna använde. Så uppskattningen är sannolikt en underdrift, men trots det så är ett koldioxid utsläpp under 25 gram högst marginelt med tanke på omfattningen av utvecklingsprocessen. De mest sannolika faktorerna för det låga värmen är antagligen det låga CO2 utsläppet av energi producerat ifrån vatten och kärnkraft som representerar runt 90\% av sveriges eltillverkning (TODO källa). Den andra viktiga faktorn är hur småskalig utvecklingen av produkten var, med enbart åtta personer och utan stor utrustning såsom serverhallar så används helt enkelt inte en mängd elektricitet som är relevant ur ett storskaligt miljöperspektiv. Nivå ett effekterna verkar alltså inte ha haft någon större påverkan på hållbar utveckling.


\subsection{Produktens miljöpåverkan}
Produkten har ännu inte använts vid skrivandet av denna rapport så den har inte haft någon miljöpåverkan förutom dess utvecklingsprocess som beskrevs ovan. Men de nivå två effekterna som den kan tänka sig ha, såsom att den uppmuntrar fler företag att skapa ett spel för sina presentationer på mässor, verkar inte vara något att oroa sig över då den låga påverkan gruppens utvecklingsprocess haft på miljön. Dock är det värt att notera att ifall en sådan mjukvara utvecklas på en plats där elektriciteten framställs på ett mer omiljövänligt sätt så kan denna process vara påtagligt värre för miljön. Detta är dock väldigt spekulativt och det är osannolikt att ett studentarbete i praktiken skulle skapa stora vågeffekter i hur mjukvara utvecklas. Troligtvis så kommer inga nivå två eller tre effekter att skapas utifrån produkten gruppen sammanställt.
