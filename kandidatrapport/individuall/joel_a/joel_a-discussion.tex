
\section{Diskussion}
\label{sec:joel_a-discussion}

\subsection{Utvecklingsprocessen utsläpp}
Den estimerade siffran för utsläppen hamnade då på $24.8 + 2347.8 = 2372.6 \text{gram \ce{CO2}}$. Detta kan jämföras med utsläppet ifrån förbränningen av en liter bensin. Enligt amerikanska Energy Information Administration så skapar förbränning av en ''gallon'' E10 17.6 ''pounds'' koldioxid.\cite{co2-bensin} I SI-systemet så betyder detta att en liter E10 ger upphov till utsläpp av ungefär 2.1 KG koldioxid. Hela utvecklingsprocessen \ce{CO2} utsläpp är alltså jämförbart med förbränning av ungefär en liter bensin. Det är även värt att notera att uppskattningen är helt dominerad av utsläppen ifrån Googlesökningar som står för 1920 av 2373 gram \ce{CO2}. 

Potentiella förbättringar för att minska koldioxid utsläppen under utvecklingsprocessen bör vara riktad åt den dominerande siffran. Mängded Google-sökningar skulle kunna minskas genom att teamet sparar resultatet ifrån en sökning lokalt och på så sätt undviker dubbla sökningar av samma sak. En annan metod skulle vara att teammedlemmar alltid frågar andra medlemmar innan en sökning görs. Denna princip går dock emot en vanlig princip inom grupparbeten att alltid försöka att lösa problemet själv innan hjälp rådfrågas. Ett sådant självständigt försök bör i princip alltid innefatta en Google-sökning. Med tanke på hur lite utsläpp hela utvecklingsprocessen skapade så kan denna idé kastas bort då den förlorade effektiviteten bör väga mer än de minskande utsläppen. Utvecklingsprocesser som minskar Google-sökandet kommer oftast med kostnaden av mer ''overhead'' och minskad effektivitet. Ytterligare en sådan teknik skulle vara att ladda ner API dokumentationen för alla biblotek och ramverk som används och sedan söka i dem först. Nedladdningen i sig kräver elektricitet som generar koldioxid och sökningen i denna dokumentation kan göra utvecklingsprocessen långsammare.


\subsection{Produktens potentiella miljöpåverkan}
Produkten som inte använts av kunden ännu kan inte ha några nivå två eller tre effekter förän detta sker. Men då både syftet och användaren av produkten är känd kan dessa effekter förutses med måttlig säkerhet. Produkten är tänkt att locka personer till kunden under mässor, detta leder antagligen till fler kunder eller anställda hos kunden Cybercom. Fler kunder skulle potentielt leda till fler anställda, vilket betyder att en ökning av anställda kan bli en nivå två och ekker nivå tre effekt. Cybercom profilierar sig starkt som miljövänliga, vilket bland annat reflekteras i deras ''code of conduct''.\cite{cyber-cod} De har även en ''code of conduct'' för sina underleverantörer där ett miljötänk ingår som en rubrik.\cite{cyber-cod} Miljöinverkan som ett expanderande Cybercom skulle ha är troligtvis inte stort eller i alla fall sannolikt mindre än ett annat företag utan samma miljöfokus.

Det finns även potentielt negativa effekter ifrån produkten, till exempel ifall konceptet med ett spel på mässor blir populärt så utvecklas fler spel i detta syfte. Utvecklingen av sånna spel påverkar också mijön på samma sätt som spelet teamet framtog har haft. Spel som framförs på mässor visas med all sannolikhet på stora skärmar som i sig drar mycket ström och skapar koldioxidutsläpp.
