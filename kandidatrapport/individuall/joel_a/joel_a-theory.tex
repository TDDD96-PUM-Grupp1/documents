\section{Teori}
\label{sec:joel_a-theory}

Greensoft-modellen är ett verktyg för att utvärdera och minska miljöpåverkan av utvecklingsprocessen. Den innehåller processer och teorier för att både utvärdera och förebygga negativ miljöpåverkan. I denna rapport så används framförallt koncepten om hur miljöpåverkan kan utvärderas. Först delas mjukvarans livslängd in i fyra perioder: utvecklingsfasen, distributionsfasen, användningsfasen och deaktiveringsfasen. Sedan så delas effekter på miljön in i tre nivåer, nivå ett är direkta effekter ifrån mjukvaran eller dess utvecklingsprocess. Nivå ett innefattar då saker såsom strömmen som utvecklarnas datorer använder eller bensinen som förbränns vid transport till jobbet. Nivå ett effekter kan också gälla användarna av produkten, såsom strömmen de använder eller miljöpåverkan att framställa hårdvara som köps för att köra programmet. Nivå två effekter sker till följd av nivå ett effekter, de kan vara effekter såsom att hårdvaran vars primära syfte är att köra programet också kör andra program. Strömmen som dessa andra program sen använder sig av är en nivå två effekt. Det är svårare att hitta nivå två effekter jämför med nivå ett effekter, och ännå svårare att hitta nivå tre effekter. En nivå tre effekt uppstår till följd av en nivå två effekt. Ifall vi återanvänder exemplet ovan så kanske de andra programmen vi kör på hårdvaran ökar strömanvändningen till en sån mängd att elbolaget måste importera omiljövänlig el ifrån utlandet. Oftast så är dessa nivå tre effekter väldigt storskaliga och samhällspåverkande. Enligt Greensoft-modellen så får vi först en korrekt bild av mjukvarans miljöpåverkan efter att ha analyserat alla dessa nivåer och perioder. 