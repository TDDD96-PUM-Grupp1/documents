\section{Teori}
\label{sec:joel_a-theory}

Greensoft modellen är verktyg för att utvärdera och minska påverkan utvecklingen av mjukvara har på miljön. Den innehåller processer och teorier för både utvärdering och förebyggande av negativ miljöpåverkan, i denna rapport så används framförallt koncepten om hur miljöpåverkan kan utvärderas. Först delar mjukvarans livslängd in i fyra perioder: utvecklingsfasen, distributionsfasen, användningsfasen och deaktiveringsfasen. Sedan så delas effekter på miljön in i tre nivåer, nivå ett är direkta effekter ifrån mjukvaran eller dess utvecklingsprocess, till exempel strömmen som utvecklarnas datorer använder. Nivå ett effekter kan också gälla användarna av produkten, såsom strömmen de använder eller miljöpåverkan att framställa hårdvara som de använder för att köra mjukvaran. Nivå två effekter sker till följd av nivå ett effekter, de kan vara effekter såsom att användarna använder hårdvaran de köpt för att använda mjukvaran för att köra ytterliggare mjukvara, och på så sätt använda mer ström. Nivå tre effekter bygger sedan på dessa nivå två effekter och kan vara mycket svåra att förutse. De är ofta samhällsförändringar som sker till grund av mjukvaran, i det tidigare exemplet så kan de vara effekten på samhället av att flera personer har ny hårdvara och använder mer ström, och då behöver importera omiljövänlig strömm ifrån utlandet. 
