\section{Teori}
\label{sec:joel_a-theory}

Denna del avser att ge den information läsaren behöver för att förstå de uttryck och tekniker som används i denna rapport men inte i huvudrapporten.


\subsection*{Reguljära uttryck}
Reguljära uttryck är en term ifrån datavetenskap samt formella språk och förkortas ofta till ’’regex’’.  De beskriver strängar av en viss form och används ofta för att göra mer kraftfulla sökningar än vanliga exakta matchningar. Funktionalitet som sökningar av reguljära uttryck erbjuder är ett jokertecken som matchar vilket tecken som helst. Möjligheten finns också att låta en del av strängen repeteras noll eller oändligt många gånger med hjälp av en så kallad Kleene-stjärna. Ifall ’’.’’ benämner jokertecknet och ’’*’’ Kleene-stjärna kan följande exempel ges: $$\text{Du och (.)* andra har redigerat ett objekt}$$

Denna sökning hittar då alla strängar som börjar på ’’Du och ’’ och efterföljs av vilket tecken som helst, repeterat mellan noll och oändligt många gånger, och sedan avslutas med ’’ andra har redigerat ett objekt’’. I praktiken så skulle denna sökning hitta alla strängar som börjar med ’’Du och ’’ och slutar med ’’ andra har redigerat ett objekt’’ oavsett vad som står emellan dem.~\cite{kozen-automata}

\subsection*{Greensofts modell}
Greensofts modell är ett verktyg för att utvärdera och minska miljöpåverkan av mjukvara. Den innehåller processer och teorier för att både utvärdera och förebygga negativ miljöpåverkan. I denna rapport så används framförallt koncepten om hur miljöpåverkan kan utvärderas. Först delas mjukvarans livslängd in i fyra perioder: utvecklingsfasen, distributionsfasen, användningsfasen och deaktiveringsfasen. Sedan så delas effekter på miljön in i tre nivåer, nivå ett är direkta effekter ifrån mjukvaran eller dess utvecklingsprocess. Nivå ett innefattar då saker såsom strömmen som utvecklarnas datorer använder eller bensinen som förbränns vid transport till jobbet. Nivå ett effekter kan också gälla användarna av produkten, såsom strömmen de använder eller miljöpåverkan att framställa hårdvara som köps för att köra programmet. Nivå två effekter sker till följd av nivå ett effekter, de kan vara effekter såsom att hårdvaran vars primära syfte är att köra programet också kör andra program. Strömmen som dessa andra program sen använder sig av är en nivå två effekt. Det är svårare att hitta nivå två effekter jämför med nivå ett effekter, och ännå svårare att hitta nivå tre effekter. En nivå tre effekt uppstår till följd av en nivå två effekt. Ifall vi återanvänder exemplet ovan så kanske de andra programmen vi kör på hårdvaran ökar strömanvändningen till en sån mängd att elbolaget måste importera omiljövänlig el ifrån utlandet. Oftast så är dessa nivå tre effekter väldigt storskaliga och samhällspåverkande. Enligt Greensofts modell så får vi först en korrekt bild av mjukvarans miljöpåverkan efter att ha analyserat alla dessa nivåer och perioder. ~\cite{greensoft}