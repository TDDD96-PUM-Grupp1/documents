\section{Teori}
\label{sec:joel_a-theory}

Greensoft-modellen är verktyg för att utvärdera och minska påverkan utvecklingen av mjukvara har på miljön. Den innehåller processer och teorier för att både utvärdera och förebygga negativ miljöpåverkan, i denna rapport så används framförallt koncepten om hur miljöpåverkan kan utvärderas. Först delas mjukvarans livslängd in i fyra perioder: utvecklingsfasen, distributionsfasen, användningsfasen och deaktiveringsfasen. Sedan så delas effekter på miljön in i tre nivåer, nivå ett är direkta effekter ifrån mjukvaran eller dess utvecklingsprocess såsom strömmen som utvecklarnas datorer använder. Nivå ett effekter kan också gälla användarna av produkten, såsom strömmen de använder eller miljöpåverkan att framställa hårdvara som används för att köra mjukvaran. Nivå två effekter sker till följd av nivå ett effekter, de kan vara effekter såsom att hårdvara köpts för att köra mjukvaran och denna hårdvara används för att köra ytterliggare mjukvara. Strömanvändningen för denna ytterligare mjukvara är ett exempel på en nivå två effekt, och som detta exempel visar är nivå två effekter svårare att förutse jämtemot nivå ett effekter. På samma sätt så är nivå tre effekter ännu svårare att förutse än nivå två effekter, och de bygger även på nivå två effekter som uppkommit. Nivå tre effekter är potentielt samhällsföränderliga effekter, om vi återanvänder exemplet ifrån ovan så kan den nya hårdvaran använda så mycket ström att elleverantören måste importera omiljövänlig el ifrån utlandet.