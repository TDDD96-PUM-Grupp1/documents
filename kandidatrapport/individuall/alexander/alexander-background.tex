\section{Bakgrund}
\label{sec:alexander-background}

Javascript är idag ett mycket populärt programmeringsspråk som kan köras på en stor majoritet av alla enheter som har tillgång till en webbläsare. Språket har fått en hel del kritik genom åren, där mycket av kritiken fokuserat på språkets sätt att hantera odefinierade variabler, samt hur Javascript generellt är ganska dåligt på att meddela när någonting kan ha gått fel.

Debatten kring för- och nackdelarna med dynamiskt och strikt typade språk är en som varit vid liv länge \cite{old-type-debate}. Denna debatt har förstås följt med då Javascript, ett dynamiskt typat språk, blev det språk som dominerar programmering av webbsidor och webbapplikationer. På senare år har dock alternativ som Flow och Typescript, som beskrivs mer under \ref{sec:alexander-theory}, blivit mer och mer populära. Dessa introducerar strikt typade alternativ till Javascript.

\subsection{Projekterfarenheter}
I projektet som denna rapport tillhör har projektgruppen nästan exklusivt använt sig av Javascript. Detta berodde på projektmedlemmarnas tidigare erfarenheter med språket tillsammans med rekommendationer och önskemål från kunden. Rekommendationerna från kunden var baserade på deras egna erfarenheter med språket och det faktum att majoriteten av deras kodbas var skriven i det. Kunden uttryckte också att användningen av Typescript kunde ses som onödig, men gav ingen vidare motivering.