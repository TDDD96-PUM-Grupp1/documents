\section{Introduktion}
\label{sec:alexander-introduction}
Inom programmering finns en konstant debatt kring olika sorters typning i programmeringsspråk. Det finns dynamiskt typade språk, som till exempel Javascript eller Python, men också strikt typade språk, som Java eller Typescript. Utvecklingsarbetet i de olika språken skiljer åt just på grund av den olika typning, men trots det överlappar deras användningsområden väldigt ofta. Vad är de olika för- och nackdelarna av att använda ett dynamiskt typat språk som Javascript istället för ett strikt typat?

\subsection{Syfte}
\label{subsec:motivation}

Syftet med denna undersökning är att få en bättre bild över hur Javascripts dynamiska typning påverkar utveckling i språket, men också hur språket ställer sig mot alternativ med striktare typning. Arbetet kommer förhoppningsvis ge insikt i varför man skulle välja Javascript över alternativen som finns, eller tvärt om.

\subsection{Frågeställning}
\label{subsec:research-questions}

\begin{enumerate}
\item Vad hade skilt sig i utvecklingsprocessen om utvecklingsarbetet hade skett med ett strikt typat språk?

\item Hur hade valet av ett strikt typat påverkat slutprodukten?

\item När vill man använda ett strikt typat språk istället för Javascript?

\end{enumerate}


\subsection{Avgränsingar}
\label{subsec:delimitations}

Undersökningen kommer fokusera på kvantitativa data från projekt vars primära utvecklingsspråk är Javascript, Javacript med Flow eller Typescript, vilka beskrivs i mer detalj senare. Undersökningen kommer inte innefatta andra dynamiska och strikt typade språk, som Java eller Python, dock kan dessa förekomma i de källor som refereras.

%\nocite{scigen}
%We have included Paper \ref{art:scigen}

%%%%%%%%%%%%%%%%%%%%%%%%%%%%%%%%%%%%%%%%%%%%%%%%%%%%%%%%%%%%%%%%%%%%%%
%%% Intro.tex ends here


%%% Local Variables: 
%%% mode: latex
%%% TeX-master: "demothesis"
%%% End: 
