\section{Diskussion}
\label{sec:alexander-discussion}

I detta kapitel kommer både metoden och resultatet som tagits upp innan att diskuteras. Eftersom ingen egen kvantitativ undersökning gjorts, förutom de erfarenheter gruppen stött på, kommer metoden och resultatet från forskningen som använts också analyseras och värderas.  Något värt att nämna om båda de undersökningarna som studerats är att båda baserats på existerande kodbaser. Det är alltså svårt att dra ens slutsats kring utveckling av nybörjade projekt, precis som de gruppen utfört. Frågeställningen är dock något baserad på generellt utvecklingsarbete inom projekt med Javascript, vilket betyder att nystartade projekt inte nödvändigtvis är lika relevanta. Något som framkommer i både [ref1] och [ref2] är att författarna menar att nyttan av ett typat system ökar med projektets storlek. 


\subsection{Metod}
\label{subsec:alexander-discussion-method}

Metoden för att presentera resultaten till denna rapport har främst bestått av att sammanställa information från tidigare publicerad forskning. Som det nämndes i början av metoden påbörjades en egen undersökning kring stora open-source-projekt på Github. Tanken bakom denna undersökning var att försöka bestämma kvaliteten av ett projekt på ett kvantitativt sätt och jämföra Javascript och Typescript. Ett beslut om att inte fortsätta med undersökning togs dock när ingen rimlig storhet för kvalitet av projekt kunde bestämmas.

Generellt tycker jag att metoden med att basera resultaten på tidigare forskning fungerade väldigt bra. För det första var det väldigt enkelt att hitta relevanta publicering via bibliotekets hemsida och Google Scholar. Jag lyckades hitta texter som var mycket välskrivna och hade en riklig mängd med källor att vidare studera.

Metoden för att fånga gruppens egna erfarenheter kring typning i Javascript kan absolut förbättras. För denna undersökning blev det mer en informell insamling av erfarenheter under projektets gång, och en mer formell avslutande fråga. För att förbättra insamling borde grupp tillfrågats att notera varje gång de stötte på någon märkvärdigt med typsystemet. Detta hade lett till fler användbara exempel att redovisa i resultatet.

\subsection{Resultat}
\label{subsec:alexander-discussion-results}

Resultaten som presenterades från de kvantitativa undersökningarna ger en mycket tydlig bild hur Javascript står mot sina typade alternativ. Båda undersökningarnas resultat pekar mot att de dynamiskt typade alternativen påverkar utvecklingsarbetet på ett positivt sätt. Från den första undersökningen ser man tydligt hur typsystemet kan hitta en relativt stor del av de publika buggar som undersöktes. Något som är värt att poängtera, som författarna av undersökningen nämner återkommande, är att de undersökning som gjorts ger en undre approximation av hur många buggar som kan hittas. Eftersom endast publika buggar undersökts, alltså buggar som blivit versionshanterade, finns ingen statistik alls kring de privata buggar utvecklarna troligtvis stött på. Författarna menar alltså att effektiviteten av typsystemet blir underskattad.

Resultatet som den andra undersökningen påvisade kompletterar de resultatet från den första undersökningen.  Den första undersökningen presenterade inte något om hur utvecklingstid och andra faktorer påverkades av ett strikt typad språk. Den andra undersökningen presenterar dock att utvecklingstiden i Typescript är bättre än den i Javascript, med eller utan kodkomplettering. Dessa resultat tillsammans pekar mot att utvecklingsarbete i Javascripts typade alternativ är effektivare än utveckling i Javascript. 

Som tidigare nämnts är den forskning som använts baserad på redan existerande kodbaser. Den andra undersökningen använde sig av en kodbas på 1400 rader kod, vilket anses som en undre gräns för resultaten i denna undersökning. Erfarenheter från mindre kodbaser kan hämtas direkt från det projekt som gruppen utfört. Detta gäller dock endast början av utvecklingsprocessen för gruppen, då projektet överskridit gränsen av 1400 rader kod (4100 på UI-applikationen i skrivande stund).

Något ingen av undersökningarna tar upp är tiden lagt på aktiviteter kring koden som inte är utveckling, till exempel granskning av skriven kod. Om man använder ett strikt typsystem för utveckling kan personen som granskar koden förhoppningsvis lägga mindre fokus på att korrekt information skickas runt i systemet, och mer fokus på generell design och kodstruktur. Därmed har typsystemet också nytta utanför kodskrivningen.

Slutligen bör de erfarenheter projektgruppen haft diskuteras. Flera av medlemmarna har liknande problem som det som presenterades under resultatet. Det har också uttryckts att det vore hjälpsamt att veta vad för typ att data som ska in och ut ur funktioner, istället för att behöva lägga tid på tolkning av kontexten. Detta stämmer bra med resultatet från den andra undersökningen, och visar återigen att utvecklare värderar typning för att snabbt sätta sig in i koden. 

\pagebreak