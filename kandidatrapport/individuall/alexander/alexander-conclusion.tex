\section{Slutsatser}
\label{sec:alexander-conclusion}

Resultatet och diskussionen som framförts hjälper till att svara på de frågeställningar som först presenterades.  Frågorna i frågeställningen är något öppna, vilket leder till ett svar som naturligt blir något öppna.

\begin{enumerate}
    \item \textbf{Vad hade skilt sig i utvecklingsprocessen om utvecklingsarbetet hade skett med ett strikt typat språk?}

    Troligtvis hade den största skillnaden varit mindre tid lagt på buggar som orsakats av typsystemet. Trots att ingen statistik förts känns det som att majoriteten av de buggar projektgruppen råkat ut för var direkt kopplade till typsystemet. Störst nytta hade gruppen nog haft av en koll för odefinierade variabler. Om kodprocessen inte blivit avbruten av den här typen av buggar hade mer tid lagts på andra delar.

    \item \textbf{Hur hade valet av ett strikt typat påverkat slutprodukten?}

    Denna fråga går hand i hand med fråga 1. Om mindre tid lagts på undersökning av buggar kunde mer tid lagts på kvaliteten av produkten, eller utveckling av fler funktioner. Något som också är intressant är överlämning av koden till kund. Kunden har visat stort intresse i att vidareutveckla de projektgruppen skapat, och därför hade ett strikt typat språk potentiellt hjälpt kunden att snabbare sätta sig in den existerande kodbasen.

    \item \textbf{När vill man använda ett strikt typat språk istället för Javascript?}
    
    Svaret på denna fråga baseras direkt på de resultat som tagits fram. Vill man utveckla en kodbas snabbare och med färre buggar ska man använda sin av ett strikt typat alternativ istället för Javascript. Eftersom mycket pekar på att det är enklare att sätta sig i en strikt typad kodbas är ett strikt typat språk också att rekommendera om det förväntas att fler ska sätta sig in i kodbasen.
\end{enumerate}