\section{Slutsatser}
\label{sec:alexander-conclusion}

Resultatet och diskussionen som framförts hjälper till att svara på de frågeställningar som först presenterades.  Frågorna i frågeställningen är något öppna, vilket leder till svar som naturligt blir något öppna.

\subsection*{\ref{alexander-fs:1} Hur påverkades utvecklingsarbetet av Javascripts typsystem?}
Under utvecklingsprocessen har gruppen stött på ett flertal buggar, där många var direkt relaterade till Javascripts typsystem. Alltså har typsystemet direkt påverkat utvecklingsarbetets effektivitet då extra tid har behövts allokeras till att undersöka dessa buggar. Flera av dessa buggar var relaterade till odefinierade variabler vilket de andra undersökta typsystemen klarar av att analysera och varna för.

\subsection*{\ref{alexander-fs:2} Hur påverkades slutprodukten av Javascripts typsystem?}
Denna fråga går hand i hand med fråga \ref{alexander-fs:1}. Om mindre tid lagts på undersökning av buggar kunde mer tid lagts på att öka kvaliteten av produkten eller utveckling av fler funktioner. Detta kunde i sin tur lett till en mer komplett slutprodukt och en kund som är ännu nöjdare.

\subsection*{\ref{alexander-fs:3} När vill man använda ett strikt typat språk istället för Javascript?}
Svaret på denna fråga baseras direkt på de resultat som tagits fram. Vill man utveckla en kodbas snabbare och med färre buggar pekar resultatet från undersökningen mot att man ska använda sig av ett strikt typat alternativ istället för Javascript. Resultatet säger också att det kan vara enklare att sätta sig i en strikt typad kodbas, vilket tyder på att strikt typade språk är att rekommendera om det förväntas att fler ska sätta sig in i kodbasen.  
