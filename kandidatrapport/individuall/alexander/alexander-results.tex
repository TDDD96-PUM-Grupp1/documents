\section{Resultat}
\label{sec:alexander-results}

Under denna rubrik framförs de resultat som samtliga undersökningar producerat. Resultaten omfattar hur buggar kan hanteras av typsystem, hur ett typsystem hjälper utvecklingsmiljöer och utvecklare samt projektgruppens resultat med typsystemet som finns i Javascript. Dessa resultat diskuteras vidare i nästkommande kapitel.

\subsection{Detektion av buggar i olika typsystem}
För att kommentera på detektionen av buggar i olika typsystem presenteras resultat från [CONFERENCE PAPER]. AUTHOR et al. undersöker flera stora open-source-projekt skrivna i Javascript.  Undersökning går ut på att bestämma en undre approximation på hur stor del av historiska buggar kunde ha detekterats genom att introducera ett strikt typsystem. Typsystemen som använts i undersökningen är Flow och Typescript som beskrevs tidigare under Teori. En fullständig beskrivning av metoden kan ses i referens []. Metoden går kort sagt ut på att hitta en commit med en buggfix, plocka ut den ändrade delen av koden innan den blev ändrad och introducera explicita typer där det är möjligt. När detta är gjort analyseras koden av respektive kompilator där författarna kollar om typsystemet ger en indikation att något fel. Undersökningen innefattade endast publika buggar, alltså buggar som blivit introducerade i projektets versionshaneringsystem. Resultaten av undersökningen visar att båda typsystemen lyckas hitta 15\% av de undersökta buggarna. 

\subsection{Integration i utvecklingsmiljö och kodkomplettering}
Resultaten för denna del kommer från [CONFERENCE PAPER]. AUTHOR et al. undersöker hur typsystemet i ett språk påverkar utvecklingsmiljöns möjligheter att visa hjälpsam information till användaren. De undersökte specifik skillnaden mellan Javascript och Typescript i utvecklingsmiljön MS Visual Studio. I undersökningen fick deltagarna försöka sätta sig in en redan existerande kodbas, skriven i både Javascript och Typescript, och försöka implementera en viss funktionalitet. Den fullständiga metoden för undersökningen finns tillgänglig i referensen. Undersökningen innefattade Javascript och Typescript, både med och utan kodkomplettering igång. Resultaten visar på att utvecklingsarbetet gjort i Typescript var betydligt mycket snabbare i alla fall som testats. Skillnaderna med kodkomplettering visade resultat som var betydligt mer varierande. Trots det anser författaren att kodkomplettering ger en något marginell förbättring i utvecklingstid.

\subsection{Projektgruppens resultat}

Under projektets gång har gruppen stött på en hel del buggar som orsakats av Javascripts typsystem. Majoriteten av dessa buggar berodde på bristande indikation att variabler var odefinierade, då många operation fungerar utan problem ändå. Ett specifikt exempel kan ses i kodsnutten nedan:
setInterval(this.tick, 1000/this.options.tickrate);
Denna kodraden skulle se till att kontrollapplikationen skickar information till servern med ett visst intervall. Problemet var dock att gruppen hade missat att definiera this.options.tickrate i en fil vilket ledde till följande istället:

1. 1000 / undefined -> NaN // NaN behandlas som ett vanligt nummer i Javascript\\
2. Anropa setInterval(this.tick, NaN)\\
3. setInterval ger inget fel då det andra argumentet är ett giltigt nummer.\\
4. setInterval tolkar NaN som 0, vilket gör att den anropas så fort den kan.

Detta ledde till en svårhittad bugg och en något misslyckad demo då systemet blev överbelastat eftersom för mycket data skickades.