\section{Metod}
\label{sec:alexander-method}

För att försöka svara på frågeställningarna har information från tidigare publicerade källor och undersökningar analyserats och sammanställts. Information från tidigare forskning har också kompletterats med erfarenheter från det projekt som gruppen utfört. 

\subsection{Information från tidigare forskning}

Då de erfarenheter gruppen upplevt inom detta område under projektets gång har varit begränsade så har ytterligare forskning och undersökningar om fler och större projekt använts. Mycket av forskning har hittats med hjälp av sökfunktioner på Google Scholar och biblioteket vid Linköpings universitet. Den kvantitativa data som presenteras kommer främst från publicerade konferenshandlingar.

\subsection{Projektgruppens erfarenheter}
Trots att de erfarenheter som gruppen upplevt under projektets gång varit begränsade har det ändå varit viktigt att fånga på grund av deras direkta påverkan på den produkt som utvecklats. För att få med personliga erfarenheter har projektgruppen frågats i efterhand hur de kände att typsystemet som finns i Javascript har påverkat deras utvecklingsprocess och produkten. Frågorna kring detta ställdes något informellt i gruppens Slack, där de som kände att de hade unika erfarenheter besvarade frågan. Eventuella buggar och andra problem som orsakats av typsystemet ligger också till viss del kvar i versionshanteringssystemet som gruppen använder. Utöver dessa mer direkta metoder har också generella återkommande problem och frågor kring typsystemet noterats.

\subsection{Kritik av källor}
För att resultatet som presenteras ska anses vara trovärdigt och tillförlitligt kommer viss kritik av källor och egna undersökningar föras fram. Denna finns under kapitel \ref{sec:alexander-discussion}. Kritiken kommer innefatta validiteten hos de källor och undersökningar som använts, tillsammans med en presentation av för- och nackdelar av resultatet från gruppens erfarenheter. 