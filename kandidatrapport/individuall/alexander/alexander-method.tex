\section{Metod}
\label{sec:alexander-method}

För att försöka svara på den frågeställningen har information från tidigare publicerade källor och undersökningar analyserats och sammanställts. En egen undersökning med att analysera stora open-source-projekt på Github påbörjades men avslutades aldrig. Detta berodde främst på problem med att bestämma en rimlig kvantitativ storhet för varje projekt. Information från tidigare forskning har också kompletterats med erfarenheter från det projekt som gruppen utfört. 

\subsection{Information från tidigare forskning}

Då de erfarenheter gruppen upplevt inom detta område under projektets gång har varit begränsade så har ytterligare forskning och undersökningar om fler och större projekt använts. Mycket av forskning har hittats med hjälp av sökfunktioner på Google Scholar och biblioteket vid Linköpings universitet. Den kvantitativa data som presenteras kommer främst från publicerade konferenshandlingar.

\subsection{Projektgruppens erfarenheter}
Trots att de erfarenheter som gruppen upplevt under projektets gång varit begränsade har det ändå varit viktigt att fånga på grund av deras direkta påverkan på den produkt som utvecklats. För att få med personliga erfarenheter har projektgruppen frågats i efterhand hur de kände att typsystemet som finns i Javascript har påverkat deras utvecklingsprocess, och produkten som blivit till. Eventuella buggar och andra problem som orsakats av typsystemet ligger också till viss del kvar i versionshanteringssystemet som gruppen använder. Utöver dessa mer direkta metoder har också generella återkommande problem och frågor kring typsystemet noterats.
