\section{Diskussion}
\label{sec:bjorn-discussion}



\subsection{Resultat}
\label{subsec:bjorn-discussion-results}
Då det finns debatter om vilket versionshanteringsverktyg som är bäst och hur de bäst används så har författare ofta lagt mer vikt på positiva aspekter och hur de löser just de problem som författarna stöter på i deras situationer. Författarna har ofta en tendens att vara mer kritiska till andra arbetsmetoder än de som de själva använder och de är mer positiva till de metoder som de själva använder.
Atlassian har intresse av att sälja deras produkt Bitbucket vilket är en tjänst som tillhandahåller centrala datakataloger. Därmed har de intresse av att förespråka utvecklingsflöden som använder en central datakatalog och därför borde man vara något kritisk mot det de säger om olika utvecklingsflöden.
Martin Fowlers webinar gör ett bra jobb med att beskriva både för- och nackdelar med CI och PI. Han klämmer också in i slutet lite generell kritik mot feature branching och förespråkar att utveckla med ett modulärt konfigurationsbaserat system. Det han skriver on CI och PI är logiskt och trovärdigt. Hans kritik mot Feature Branching verkar ha grund i verkliga scenarion men Feature Branching borde fortfarande kunna tillämpas med några mindre ändringar i sättet funktionalitet utvecklas. Artikeln är också nio år gammal och under de åren som gått har också verktygen som används blivit mycket bättre och det som kritiseras har nog mindre påverkan än vad det hade då. Feature Branchings succé föreslår att de problem som togs upp troligtvis är lösta eller inte har särskilt stor påverkan.
De två rapporterna som tas upp undersöker båda hur de automatiska verktygen som används kan förbättras men det resultat som de tagit fram går att ta lärdom ifrån och applicera manuellt för att få bättre effektivitet.
Det resultat som funnits har stämmt överrens med de erfarenheter som finns från arbetet i kandidatprojektet.

\subsection{Metod}
\label{subsec:bjorn-discussion-method}
Det utvecklas hela tiden nya metoder för versionshantering och de metoder som finns vidareutvecklas. Problem som nämns i källor som bara är några år gamla kan vara fixade eller lösta genom nya arbetssätt. Därför har det lagts större vikt på att hitta nya källor för att få en bättre bild av hur versionshantering fungerar i dagsläget. På grund av det finns det ont om relevanta rapporter att studera och en stor del av det studerade materialet har varit artiklar.
Då studien är en undersökning utav den information och debatt som finns om metoder som används i nuläget blir studien svår att replikera i framtiden då de aktuella metoderna kan se väldigt annnorlunda.