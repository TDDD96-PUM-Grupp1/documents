\section{Resultat}
\label{sec:bjorn-results}
Här sammanställs kvalitativ information från de resurser som erfanns.

%Projektet:
%Feature branching, CI, Code reviews, Squashing
%Historik försvinner, svår blame

\subsection{Undersökning}
Enligt Atlassians jämförelse av centraliserade och distribuerade versionshanteringssystem kan man emulera centraliserade system med distribuerade. Genom att välja ett distribuerat system som Mercurial eller Git får man då tillgång till samma arbetsflöden som centraliserade system erbjuder samtidigt som man får de fördelarna som ett distribuerat system ger. Som sammanfattning ger de följande fördelar:
\begin{enumerate}
\item De flesta aktiviteter blir väldigt snabba då de kan utföras lokalt.
\item Det går att lagra ändringar utan bekymmer.
\item Ändringar kan delas, testas och diskuteras utan att skapa problem för andra utvecklare.
\end{enumerate}
Dock så blir man av med de fördelar som kommer av att lagra historiken på en centraliserad server, det vill säga att det kan bli problem med diskutrymme i projekt med väldigt många filer och mycket historik.\cite{central-vs-distributed}

Atlassian rekommenderar arbetsflödet "feature branching" för agila team. Feature branching bygger på att det finns en stabil huvudbranch som fungerar och att all ny funktionalitet utvecklas på egna brancher som utgår från huvudbranchen. Det här arbetsflödet passar bra ihop med popullära agila metoder som Scrum och Kanban genom att man skapar en branch för varje uppgift. Feature branching kombineras också ofta med den agila processen Continuous integration (CI) som innebär att man integrar ny funktionalitet så fort den är redo och använder automatiska tester. Den kombineras också med kodgranskning som ger möjlighet för andra utvecklare att ge feedback. Målet med de processerna är att minimera tiden som måste läggas på att kombinera utvecklingen från två branches och se till att huvudbranchen håller bra kvalité enom att koden granskas och testas.\cite{daly-agile-git, radigan-feature-branch, atlassian-feature-branch}

Martin Fowler tar upp några problem med feature branching och CI i sin webinar. Han beskriver att ofta fungerar det bra så länge det bara byggs nya features. Däremot så kan det uppstå stora problem när det sker stora ändringar utav den existerande koden, till exempel när det har utförts en omfaktorering. När omfaktoreringen ska slås ihop med huvudbranchen måste all kod som hamnat på huvudbranchen medans omfaktoreringen utfördes också omfaktoreras. Det kan vara tidskrävande då det kan vara svårt att hitta all användning av den omfaktorerade koden och det finns inget direkt stöd från versionshanteringssystemen förrutom förmågan att se de ändringar till huvudbranchen som hänt. När det är gjort och omfaktoreringen hamnat på huvudbranchen så blir det problem på alla branches som varit i utveckling parallelt då utvecklarna måste spendera tid på att uppdatera koden i deras branches.
Som en lösning på det här problemet föreslår han en variant av CI som han kallar Promiscuous Integration (PI). Den tar fördel av de distribuerade systemen och innebär att man integrerar mellan brancher direkt när man vet att det kommer ske stora förändringar eller när båda brancher utvecklar liknande funktionalitet så att de kan samarbeta. PI är inte perfekt och kräver att utvecklare pratar med varandra om vad de håller på med. Det ingår det i de flesta agila arbetsmetoder men kan vara problematiskt när man jobbar med utspridda teams. Det kan också göra det rörigt med att hålla koll på vilken funktionalitet som finns på vilken branch men för det mesta kan versionshanteringssystemen effektivt hjälpa till med det.\cite{fowler-feature-branch}

%git workflows:
%centralized - rebase istället för merge gör det lättare att se var buggar kom ifrån
%feature branch! - continous integration, testning
%gitflow
%forking

%korta branches
%enkla reverts
%matcha release cykeln

När kod ska kombineras så sker det oftast inga problem som inte kan lösas automatiskt av versionshanteringsvektygen när ett effektivt arbetsflöde följs. Men det förekommer ändå tillräckligt ofta problem som behöver lösas manuellt för att det ska vara intressant att studera hur de borde lösas. För att effektivast lösa de "merge conflicts" som uppstår när kod från olika brancher ska kombineras så är det viktigt att de som skrivit koden är involverade. Det är också hjälpsamt om koden är förstålig och om de verktyg som används visar bra information.\cite{8094445}

Kodgranskning är en viktig del av de flesta agila utvecklingsmetoderna och versionshanteringsverktygen hjälper genom att visa de förändringar som skett i koden. Moderna kodgranskningar är baserade till stor del på att titta på vad versionshanteringsverktygen visar. Sättet som förändringarna visas på kan påverka kvalitén av kodgranskningen. En nyligen utförd studie fann att den optimala ordningen att visa ändringarna på borde baseras på hur de relaterar till varandra. \cite{8094433}