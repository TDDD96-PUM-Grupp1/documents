\section{Teori}
\label{sec:bjorn-theory}
De grundläggande koncepten för rapporten defineras och förklaras i det här kapitlet.

\subsection{Git}
Git är ett distribuerat versionshanteringsverktyg. Det skapades ursprungligen av Linus Torvalds för att täcka behoven till utvecklingen av Linux efter att deras dåvarande versionshanteringsverktyg BitKeeper tog bort deras gratisklient. Git skapades med krav på att den skulle vara distribuerad och snabb samt ha ett skydd mot korruptering. Git släpptes först i december 2005 och har sedan varit i konstant utveckling och förbättring \cite{linux-google} När Git släpptes var SVN dominerande och Git skiljde sig åt på många sätt, främst genom den distribuerade modellen. Det tog ett par åt innan Git började bli populärt men sedan dess har Git vuxit i popularitet med rasande fart och är idag det populäraste versionshanteringsverktyget. Git används i nuläget för både enorma projekt som Windows och Linux men också för små hobby projekt.\cite{vcs-popularity}
Exempel på de största fördelarna med Git är att det finns starkt stöd för förgrenande utveckling där det bland annat är lätt att sedan kombinera ihop grenar av kodbasen i en så kallad merge. Ett kraftigt verktyg som finns till hjälp där är en så kallad pull-request som tillåter att koden kan granskas och testas innan man utför en merge.\cite{pull-request} En annan fördel som kommer från att den är distribuerad gör så att det blir enklare för utvecklare att utveckla lokalt och ger bättre verktyg för att gå bakåt i historiken eller utföra en merge lättare.

\subsection{Mercurial (hg)}
Mercurial är liksom Git ett distribuerat versionshanteringsverktyg som skapades av samma anledning med tanke att bli det nya versionshanteringssystemet för Linux utveckling. Trots att Git användes istället så fortsattes Mercurial att utvecklas och har sett en del använding inom open-source projekt som Netbeans\cite{netbeans-contribute} men också utav stora företag som Facebook\cite{facebook-mercurial}.
Mercurial är på många sätt väldigt likt Git i hur det fungerar men har fördelar som att kommandon är simplare och har bättre koll på specifik historik för en viss fil. En annan fördel är att den är lättare att utöka med egen funktionalitet.\cite{mercurial-book}

\subsection{Subversion (SVN)}
SVN är ett centraliserat versionshanteringsverktyg. Det skapades år 2000 med syfte av att vara en bättre version av den då populära CVS och det fanns ursprungligen inga tankar på att revolutionera versionshantering. Därmed blev det ett logiskt och simpelt val för användare av CVS att byta över till SVN då SVN fungerade ekvivalent eller bättre. Efter att SVN blev populärt så har den vidareutvecklats och används fortfarande av många företag\cite{vcs-popularity} och open source projekt som The Apache Software Foundation\cite{apache-dev-svn}.\cite{svn-book}

\subsection{Agila utvecklingsmetoder}
Agila utvecklingsmetoder handlar om att värdera flexibilitet och samarbete över de traditionella värdena som planer och processer. De menar inte att de traditionella värdena är meningslösa utan att de är värdefulla, men det är värdefullare att kunna jobba närmare kunden och att våga ändra på planer under projektets gång. Agila utvecklingsmetoder siktar på att produkten utvecklas i kortare perioder där man låter kunden få säga till om vad de vill. Det agila arbetssättet förespråkar att utvecklarna skriver väldigt lite dokumentation och istället mycket kod så att ifall kunden ändrar sig så har man inte lagt tid på att dokumentera onödiga saker. Det kan dock göra det svårare för nya utvecklare att sätta sig in i ett projekt.
Det finns många väl sätt att utveckla agilt på och några populära metoder är Scrum, Kanban och Extreme Programming.