\section{Teori}
\label{sec:bjorn-theory}
De grundläggande koncepten för rapporten defineras och förklaras i det här kapitlet.

\subsection{Versionshantering}

\subsection{Git}
Git är ett distribuerat versionshanteringsverktyg som ursprungligen grundades av Linus Torvalds för att täcka behoven till utvecklingen av Linux. Git skapades med krav på att den skulle vara distribuerad och snabb samt ha ett skydd mot korruptering. Git släpptes först i december 2005 och har sedan varit i konstant utveckling och förbättring. När Git släpptes var SVN dominerande och Git skiljde sig åt på många sätt. Det tog ett par åt innan Git började bli populärt men sedan dess har Git vuxit i popularitet med rasande fart och är idag det populäraste versionshanteringsverktyget. Git används i nuläget för både enorma projekt som Windows och Linux men också för små hobby projekt.
Exempel på de största fördelarna med Git är att det finns starkt stöd för förgrenande utveckling där det bland annat är lätt att sedan kombinera ihop grenar av kodbasen i en så kallad merge. Ett kraftigt verktyg som finns till hjälp där är en så kallad pull-request som tillåter att koden kan granskas och testas innan man utför en merge. En annan fördel som kommer från att den är distribuerad gör så att det blir enklare för utvecklare att utveckla lokalt och ger bättre verktyg för att gå bakåt i historiken eller utföra en merge lättare.

\subsection{Agila utvecklingsmetoder}
Agila utvecklingsmetoder handlar om att värdera flexibilitet och samarbete över de traditionella värdena som planer och processer. De menar inte att de traditionella värdena är meningslösa utan att de är värdefulla, men det är värdefullare att kunna jobba närmare kunden och att våga ändra på planer under projektets gång. Agila utvecklingsmetoder siktar på att produkten utvecklas i kortare perioder där man låter kunden få säga till om vad de vill. Det agila arbetssättet förespråkar att utvecklarna skriver väldigt lite dokumentation och istället mycket kod så att ifall kunden ändrar sig så har man inte lagt tid på att dokumentera onödiga saker. Det kan dock göra det svårare för nya utvecklare att sätta sig in i ett projekt.
Det finns många väl sätt att utveckla agilt på och några populära metoder är Scrum, Kanban och Extreme Programming.