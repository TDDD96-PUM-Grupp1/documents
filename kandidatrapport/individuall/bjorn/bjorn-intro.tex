\section{Introduktion}
\label{sec:bjorn-introduction}

Den här rapportens syfte är att undersöka hur versionshanteringsverktyget Git kan användas tillsammans med en agil uvecklingsmetod. Rapporten kommer att fokusera till stor del på erfarenheter kring användningen av Git i diverse projekt men också de erfarenheter vi har upplevt i kandidatarbetet.

\subsection{Motivering}
\label{subsec:motivation}

Versionshantering är inte ett krav för att kunna arbeta agilt men det kan göra det lättare. Det är också ett kraftfullt verktyg för samarbete inom open-source projekt men också för teambaserad produktutveckling. Dock så finns det inte bara ett sätt att versionshantera på, vilket kan ge upphov till oklarheter när det ska versionshanteras. Därför undersöker rapporten populära alternativ och i vilka situationer de kan vara bäst att använda.

\subsection{Syfte}
\label{subsec:reason}
Den här rapportens syfte är att undersöka vilka moderna alternativ som finns för versionshantering med fokus på hur de lämpar sig för agil utveckling.

\subsection{Frågeställning}
\label{subsec:research-questions}

För att uppfylla syftet så kommer rapporten baseras på följande frågeställningar:

\begin{enumerate}
\item Vilka alternativ finns för att åstadkomma agil versionshantering?

\item Vilket alternativ ger effektivast arbete och enklast integrering?
\end{enumerate}


%    Does this workflow scale with team size?
%    Is it easy to undo mistakes and errors with this workflow?
%    Does this workflow impose any new unnecessary cognitive overhead to the team?


\subsection{Avgränsningar}
\label{subsec:delimitations}

Undersökningen begränsas till populära alternativ.