\section{Metod}
\label{sec:david-method}
Detta avsnitt går igenom vilka metoder som använts för att få fram det resultat som studien presenterar.


\subsection{Erfarenheter från projekt}
Egna personliga erfarenheter från projektet är mycket varierande. Gruppen har inte sedan tidigare använt sig av så mycket testning i sina respektive utbildningar, och därför kan det mycket väl vara så att många misstag har begåtts och testningen inte utförts optimalt. Under projektets gång har dock många erfarenheter samlats in vilket ökat kvalitén på testen allt eftersom. Det har lagts mycket fokus på olika varianter utav tester för att få utökad förståelse runt ämnet. Det innebär en bredare kunskap istället för en fördjupning, vilket kommer återspeglas i den här studien.

\subsection{Informationsinsamling}
Mycket utav informationen har kommit från diverse källor funna på nätet. Många från olika jämförelser gjorda av vana testare, men även information direkt från testverktygens egna hemsidor, API:er och dokumentation. Då insamlingen kommer från flera olika källor, många från personer med flera år inom området bör det ge en bra bas att utgå ifrån. Det har även samlats in information från böcker runt ämnet för att få till ett par säkrare källor till information, vilket förhoppningsvis kan ge en mer objektiv vy om ämnet.