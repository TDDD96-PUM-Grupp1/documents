\section{Metod}
\label{sec:david-method}
Detta avsnitt går igenom vilka metoder som använts för att få fram det resultat som studien presenterar.


\subsection{Erfarenheter från projekt}
Egna personliga erfarenheter från projektet är mycket varierande. Gruppen har inte sedan tidigare använt sig av så mycket testning i sina respektive utbildningar, och därför kan det mycket väl vara så att många misstag har begåtts och testningen inte utförts optimalt. Under projektets gång har dock många erfarenheter samlats in vilket ökat kvalitén på testen allt eftersom. Det har lagts mycket fokus på olika varianter utav tester för att få utökad förståelse runt ämnet. Det innebär en bredare kunskap istället för en fördjupning, vilket kommer återspeglas i den här studien.

\subsection{Informationsinsamling}
Den mesta information har funnits med hjälp utav Google, med störst fokus på Google Scholar. Google Scholar har använts till att finna vetenskapliga publikationer och tidsskrifter för exempelvis hur testning bör genomföras korrekt och olika rekommendationer om vilka vinklar man kan angripa olika problem från. Här har funnits ett par böcker som hänvisas till i olika delar av rapporten men som mycket utav texten baseras på. 

Google Scholar är bra till att hitta säkra källor, men mycket utav informationen är inte den allra senaste och därför har ett medvetet beslut tagits om att göra sökningar via sökmotorn Google också. Det gjordes för att få fram de senaste metoderna inom testning och de nya finessera verktygen har att erbjuda. Då ämnet uppdateras kontinuerligt och vetenskapliga artiklar oftast tar lite längre tid att få publicerade valdes Google för att få tag på det senaste från olika bloggar och verktygens egna hemsidor, API:er och dokumentation. Verktygens egna hemsidor är pålitliga källor då de endast beskriver vad de har att erbjuda, medan bloggar ofta kan vara partiska  till de verktyg de själva använder och känner till väl och därför får tas med en nypa salt. Många erfarna testare är dock bra på att peka ut viktiga skillnader mellan verktyg som man sedan kan forska vidare på och verifiera själv.

