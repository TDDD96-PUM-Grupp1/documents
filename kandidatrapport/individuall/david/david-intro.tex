\section{Introduktion}
\label{sec:david-introduction}

Det sägs att när det kommer till webbutveckling finns det tre språk en utvecklare måste kunna. HTML för att hantera innehållet på webbsidor, CSS för layouten och Javascript för funktioner. Javascript är ett gammalt språk med många egenheter som gör att många tycker det är svårt att hantera. Det finns dock få alternativ till Javascript vilket gör att många stödprogram har utvecklats för att underlätta utveckling av Javascript. Det vi går igenom kommer huvudsakligen handla om Jest, hur det fungerar och hur man använder det optimalt för att få ut det mesta ur sina tester.

När man talar om att utveckla en ny mjukvaruprodukt tänker man oftast på de funktioner som ska skrivas och implementeras. Det som få tänker på är att testningen under utvecklingen svarar för runt 25-50\% av tiden nedlagt på ett projekt. Att testa för lite innebär att mjukvaran kan släppas med många buggar, medan att testa för mycket kan vara kostsamt. Här kommer automatiseringen av tester in, om man kan automatisera testning på ett bra sätt kan man spara både tid och pengar.

\subsection{Syfte}
Anledningen till att denna rapport skrivs är att ge läsaren ökad förståelse över hur man kan använda Jest när man utvecklar en produkt, exempelvis ett spel. Vid slutet av denna rapport bör man ha ökad föreståelse till för och nackdelarna med att använda just Jest till att testa sina produkter. Fokus kommer ligga på spelutvecklingen då många paralleller kommer dras till det projekt gruppen utvecklat vilket är ``realtidsmultiplayerspel på IoT-Backend". 




\subsection{Frågeställning}
\label{subsec:david-research-questions}

\begin{enumerate}
\item Hur kan man automatisera speltester?
\item 
\item 

\end{enumerate}

\subsection{Avgränsningar}
Arbetet kommer dra paralleller till andra testverktyg och göra jämförelser mot hur andra fungerar för att få en överblick över vad som skiljer de olika verktygen åt. Fokus kommer dock ligga på Jest och spelutvecklingen.

%\nocite{scigen}
%We have included Paper \ref{art:scigen}

%%%%%%%%%%%%%%%%%%%%%%%%%%%%%%%%%%%%%%%%%%%%%%%%%%%%%%%%%%%%%%%%%%%%%%
%%% Intro.tex ends here


%%% Local Variables:
%%% mode: latex
%%% TeX-master: "demothesis"
%%% End:
