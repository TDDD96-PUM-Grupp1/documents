\section{Introduktion}
\label{sec:david-introduction}
Det sägs att när det kommer till webbutveckling finns det tre språk en utvecklare måste kunna. HTML för att hantera innehållet på webbsidor, CSS för layouten och Javascript för funktioner. Javascript är ett gammalt språk med många egenheter som gör att många tycker det är svårt att hantera. Det finns dock få alternativ till Javascript vilket gör att många stödprogram har utvecklats för att underlätta utveckling av Javascript. Det vi går igenom kommer huvudsakligen jämföra de olika testverktygen som skapats för att underlätta testning av Javascript-applikationer. 


När man talar om att utveckla en ny mjukvaruprodukt tänker man oftast på de funktioner som ska skrivas och implementeras. Det som få tänker på är att testningen under utvecklingen svarar för runt 25-50\% av tiden nedlagt på ett projekt. Att testa för lite innebär att mjukvaran kan släppas med många buggar, medan att testa för mycket kan vara kostsamt. Här kommer automatiseringen av tester in, om man kan automatisera testning på ett bra sätt kan man spara både tid och pengar.

\subsection{Syfte}
Anledningen till att denna rapport skrivs är att ge läsaren ökad föreståelse över vilka olika testverktyg det finns till javascript och hur de fungerar. När det kommer till att utveckla mjukvara är det många frågor som måste besvaras innan man börjar, och efter att ha läst den här rapporten har man förhoppningsvis en fråga mindre att besvara. Så länge utvecklaren i fråga vet vilken målgrupp projektet riktar sig åt och hur applikationen kommer fungera kommer man kunna avgöra vilket testverktyg som är bäst lämpat för just det projektet. Sen är det självklart en del av en smakfråga vad personen i fråga själv tycker är bäst och tidigare erfarenheter. Det här är ett försök till en objektiv vy över de större testverktygen som för tillfället finns ute på marknaden.

\subsection{Frågeställning}
\label{subsec:david-research-questions}

\begin{enumerate}
\item Vilka är de populäraste testverktygen?
\item Vad skiljer verktygen åt?
\item 

\end{enumerate}

\subsection{Avgränsningar}

%\nocite{scigen}
%We have included Paper \ref{art:scigen}

%%%%%%%%%%%%%%%%%%%%%%%%%%%%%%%%%%%%%%%%%%%%%%%%%%%%%%%%%%%%%%%%%%%%%%
%%% Intro.tex ends here


%%% Local Variables:
%%% mode: latex
%%% TeX-master: "demothesis"
%%% End:
