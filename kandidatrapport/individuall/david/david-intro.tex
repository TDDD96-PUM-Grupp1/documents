\section{Introduktion}
\label{sec:david-introduction}
Det sägs att när det kommer till webbutveckling finns det tre språk en utvecklare måste kunna. HTML för att hantera innehållet på webbsidor, CSS för layouten och JavaScript för beteendet. Javascript är ett gammalt språk med många egenheter som gör att många tycker det är svårt att hantera. Det finns dock få alternativ till JavaScript vilket gör att många stödprogram har utvecklats för att underlätta utveckling av JavaScript. Det vi går igenom kommer huvudsakligen jämföra de olika testverktygen som skapats för att underlätta testning av JavaScript-applikationer. 


När man talar om att utveckla en ny mjukvaruprodukt tänker man oftast på de funktioner som ska skrivas och implementeras. Det som få tänker på är att testningen under utvecklingen svarar för runt 25-50\% av tiden nedlagt på ett projekt. Att testa för lite innebär att mjukvaran kan släppas med många buggar, medan att testa för mycket kan vara kostsamt. Här kommer automatiseringen av tester in, om man kan automatisera testning på ett bra sätt kan man spara både tid och pengar.

\subsection{Syfte}
Syftet med denna rapport är att på ett objektivt sätt se över skillnaderna mellan manuell och automatisk testning för att bedömma om automatisering är värt besväret. 

\subsection{Frågeställning}
\label{subsec:david-research-questions}


\begin{enumerate}
\item Vilket verktyg ska användas för vilket tillfälle?
\item 

\end{enumerate}



%\nocite{scigen}
%We have included Paper \ref{art:scigen}

%%%%%%%%%%%%%%%%%%%%%%%%%%%%%%%%%%%%%%%%%%%%%%%%%%%%%%%%%%%%%%%%%%%%%%
%%% Intro.tex ends here


%%% Local Variables:
%%% mode: latex
%%% TeX-master: "demothesis"
%%% End:
