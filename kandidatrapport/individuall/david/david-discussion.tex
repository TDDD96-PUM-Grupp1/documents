\section{Diskussion}
\label{sec:david-discussion}
Testning kan ske på många sätt med många olika verktyg. Det här arbetet har förhoppningsvis givit läsaren en överblick över vilken typ av verktyg som bäst lämpar sig för vilken situation. Med en viss översikt över hur man kan använda Jest kan läsaren själv välja om det är ett bra sätt att testa sina produkter eller om något annat passar bättre. Man kan gå djupare än vad som gjorts i denna rapport, men det här är mer menat som en översikt över hur just Jests olika finesser och vad de stora skillnaderna mellan de olika testverktygen.

\subsection{Metod}
\label{subsec:david-discussion-method}
Det första som gjordes för att skriva den här rapporten var att börja studera grundläggande testning för att få en överblick över hur det går till. Två böcker valdes ut \cite{ADP} \cite{book-sta} för att få fler perspektiv på testning. Den ena boken är dock väldigt gammal, publicerad år 1999, vilket är en evighet när det kommer till mjukvara. Därför valdes en till nyare bok för att kunna få lite mer uppdaterad information. Dessa två har lagt grunden till arbetetet. Den senaste boken är från år 2007, vilket ändå gör den 11 år gammal när den här rapporten skrivits, och kan ha påverkat testmetoderna negativt. Det bör dock ha mitigerats av att flera andra nyare källor har använts, som exempelvis bloggar.

För att komma fram till det resultatet som angivits i den här rapporten har ett par olika metoder använts. Dels baseras det en del på egna erfarenheter, flera av de olika testerna som gås igenom i rapporten har använts i projektet gruppen har jobbat med. Dock kan inte endast erfarenheter spegla denna rapport då gruppen endast använt sig utav Jest när det kommit till testskrivning, och för att rapporten inte ska verka allt för partisk har många undersökningar gjorts på många bloggar och forum där testverktyg diskuteras. Det innebär sannerligen att många åsikter är partiska om man går in på en blogg som förespråkar exempelvis Mocha. Det viktiga är då att dubbelkolla mot andra källor för att se till att den angivna informationen stämmer. Det har exempelvis gjorts genom att bekräfta att funktionaliteten stämmer på deras egna webbsida, eller att det finns andra argumenterande faktorer som styrker informationen.

\subsection{Testverktyg} 
\label{subsec:david-discussion-jest}
Det är svårt att få med alla olika finesser och skillnader när man jämför verktyg. Förhoppningsvis har vissa inblickar fåtts av vad Jest har att erbjuda och hur man kan testa med hjälp utav det. Den stora frågan blir hur man använder verktygen och hur ens egna preferenser ser ut. För vår grupp var Jest precis det verktyget som behövdes då vi utvecklade en React-app och de fungerar fint med varandra. Det är lite lurigt att sätta sig in i hur man testar om man inte gjort det tidigare, men det ska vara ett av de enklare verktygen ute på marknaden att använda \cite{bib-jest-easy}. När alla verktyg kommer färdigpaketerade är det bara att börja jobba med det direkt vilket sparar en hel del tid på att felsöka varför vissa paket inte fungerar tillsammans och ta reda på vad man saknar. Är man van och har arbetat med det ett tag är antagligen Mocha en favorit, vilket är det troliga fallet då det är väldigt populärt och minimalistiskt, då man inte får onödig extra funktionalitet som man inte använder. 

\subsection{Speltestning}
\label{subsec:david-discussion-speltestning}
Speltestning som gått igenom har till stor del skett på lågnivå, det vill säga att det har inte analyserats till stor del hur större företag har gjort sina tester, utan mest fokuserat på vad man kan göra själv när man utvecklar en applikation och testar den. Det större företag oftast gör är att de har flera testavdelningar endast för att testa olika funktionalitet utav varje aspekt ur sina spel. Det som gåtts igenom i detta arbete har mestadels varit på låg skala, när utvecklarna själva testar sin produkt och hur man får ut det mesta ur sina tester. 