\section{Slutsatser}
\label{sec:david-conclusion}
Detta kapitel går igenom de slutsatser man kan dra av det presenterade resultatet och besvarar frågeställningarna från första kapitlet.
\subsection{Testverktyg}
Det finns många olika testverktyg därute redo att användas för Javascrpitsapplikationer och den här studien har bara skrapat lite på ytan över hur kraftfulla verktyg kan vara och dess specialitéer. Till viss del handlar det inte om att ett visst testverktyg är bättre än andra, utan det handlar i stor sak om smak, arbetssätt och användningsområde. Viss funktionalitet skiljer sig, Mocha och Jasmine har funnits längre på marknaden vilket innebär större användarbas och mer användarskapad funktionalitet, medan Jest är ett yngre verktyg som uppdateras kontinuerligt. Förhoppningsvis har läsaren blivit lite mer insatt i vad Jest har att erbjuda kontra andra testverktyg, lite rekommendationer för hur man kan göra olika tester och hur just Jest skiljer ut sig ur mängden där ute. 

Studien har gått in på hur Jest arbetar väldigt bra tillsammans med React då de utvecklas parallellt, men det fungerar även bra att använda till andra bibliotek. De nämner nämligen det på sin egna hemsida, \textit{``Jest is used by teams of all sizes to test web applications, node.js services, mobile apps, and APIs.''}\cite{bib-jest}.  Det är endast utvecklarnas riktlinjer som speglas utan det handlar om att experimentera och testa sig fram tills man hittar ett språk som passar en själv bäst. 

\subsection{Automatisk testning}
Automatisk testning kan göras på många olika sätt. Det gäller att använda rätt test på rätt plats för att få ut det mesta ur sin tid. Använder man snapshottesting på ställen som praktiskt taget är avklarade, kan den kolla på att inga ändringar sker utan testarens vetskap. Använder man assertion testing kan man testa mer specifikt vilka värden man förväntar sig få ut och faktiskt få dem. Det är mer specifikt och snapshottesting och hjälper en att konfirmera mindre sektioner kod, bra att använda vid enhetstester. Att använda sig av mock-up funktioner, eller spies, är bra att använda till att testa en sektions kod då vissa delar av koden kan vara inkompletta eller inte ens påbörjade. 