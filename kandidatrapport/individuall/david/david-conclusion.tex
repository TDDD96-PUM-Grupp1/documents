\section{Slutsatser}
\label{sec:david-conclusion}
Detta kapitel går igenom de slutsatser man kan dra av det presenterade resultatet och besvarar frågeställningarna från första kapitlet.
\subsection{Testverktyg}
Det finns många olika testverktyg därute redo att användas för Javascript och den här studien har bara skrapat lite på ytan över hur kraftfulla verktyg kan vara och dess specialitéer. Till viss del handlar det inte om att ett visst testverktyg är bättre än andra, utan det handlar i stor sak om smak, arbetssätt och användningsområde. Viss funktionalitet skiljer sig, Mocha och Jasmine har funnits längre på marknaden vilket innebär större användarbas och mer användarskapad funktionalitet, medan Jest är ett yngre verktyg som uppdateras kontinuerligt. Förhoppningsvis har läsaren blivit lite mer insatt i vad Jest har att erbjuda kontra andra testverktyg, lite rekommendationer för hur man kan göra olika tester och hur just Jest skiljer ut sig ur mängden där ute. 

Studien har gått in på hur Jest arbetar väldigt bra tillsammans med React då de utvecklas parallellt, men det fungerar även bra att använda till andra bibliotek. De nämner nämligen det på sin egna hemsida, \textit{``Jest is used by teams of all sizes to test web applications, node.js services, mobile apps, and APIs.''}\cite{bib-jest}.  Det är endast utvecklarnas riktlinjer som speglas utan det handlar om att experimentera och testa sig fram tills man hittar ett språk som passar en själv bäst. 

\subsection{Automatisk testning}
Automatisk testning kan göras på många olika sätt. Det gäller att använda rätt test på rätt plats för att få ut det mesta ur sin tid. De enklare användningarna av automatiska tester är att använda sig av regressionstestning. Skriver man tester i ett tidigt skede kan man återanvända sig utav dem i ett senare skede, även om vissa modifikationer kan behöva göras. Det är enkelt utfört via exempelvis assertion tester för funktionalitet eller snapshottesting för att förhindra omedvetna förändringar av kod. 

Det finns även andra saker som är svåra att testa manuellt, vissa måste man automatisera av olika anledningar. Exempelvis om man vill se hur en produkt håller om flera hundratals användare vill använda sig av produkten samtidigt. Det kan vara nästan omöjligt att testa själv, då kan man simulera användare att köra igenom olika bitar av koden för att se hur den klarar av det.

Det som även diskuterats i denna rapport är att flera saker går inte att lösa med endast automatisk testning. Vissa saker måste man testa manuellt. Man kan inte automatisera tester som kollar estetisk, och man kan heller inte testa att ett spel har bra känsla. Man måste kombinera både manuella och automatiska tester för att få bästa resultat.