\section{Teori}
\label{sec:david-theory}
För att gå igenom fördelar och nackdelar med Jest kommer jämförelser göras mot andra testverktyg, eftersom det finns så många kommer det i första hand jämföras med de större konkurrerande verktygen som Mocha och Jasmine. I det här kapitlet kommer begrepp redas ut och sammanfattningar om testmetoder och andra testverktyg.

\subsection{Jest}
Jest är som tidigare nämnt ett testverktyg utvecklat för React och är i fokus i denna studie. Det kommer färdigt med paket flera paket installerade så det bara är att starta direkt. 

\subsection{Mocha}
Mocha är så som Jest och Jasmine ett testverktyg för Javascript. Mocha är till skillnad från Jest och Jasmine, ett väldigt modulärt testverktyg. Här kan man inte börja testa direkt, utan man måste själv fixa varje paket man vill ha, till skillnad från till exempel Jest som kommer färdigbyggt. Det blir självklart fördelaktigt att man kan välja ut de paket man behöver för att minimera laddtider, testtider och det inte lika uppsvällt av onödiga saker som inte används. Detta innebär att mocha är ett kraftfullt och flexibelt verktyg, om man använder det på rätt sätt.

\subsection{Jasmine}
Jasmine är som Jest, ett färdigt testverktyg med ett färdigt testverktyg redan implementerat. Jasmine är dock till skillnad från Jest, som är riktat mo React, inte utvecklat för något speciellt Javascript ramverk.

\subsection{React}
I projektet gruppen utvecklade användes 

\subsection{Snapshot}
Snapshot är en testmetod som går ut på att ta en ``snapshot'' av hur till exempel ett UI ser ut. Den sparar ner en kopia över hur det ser ut vid första testet för att sedan jämföra med hur det ser ut vid andra testkörningar. Om någonting har ändrats får man upp en prompt som man får välja om man vill se över sin kod, eller om man vill spara en nya snapshot istället. Ett exempel på ett enkelt snapshot test kan man se i \ref{fig:snapshot-test} nedan där testet tar en snapshot över hur startmenyn för hur UI ser ut. I \ref{fig:snapshot-shot} kan man se hur snapshoten sparas som en fil i projektet.

\lstset{language=JavaScript}
\begin{figure}[h]
  \center
  \begin{minipage}[c]{5cm}
    \begin{lstlisting}
...,
describe('FirstMenu', () => {
  it('matches the snapshot', () => {
    var showAbout = jest.fn();
    var showCreate = jest.fn();
    const tree = renderer.create(<FirstMenu showCreate={showCreate} showAbout={showAbout} />).toJSON();
    expect(tree).toMatchSnapshot();
  });
});
...
    \end{lstlisting}
  \end{minipage}

  \caption{Ett test för att ta en snapshot på startmenyn i projektet}
  \label{fig:snapshot-test}
\end{figure}


\lstset{language=Java}
\begin{figure}[h]
  \center
  \begin{minipage}[c]{5cm}
    \begin{lstlisting}
// Jest Snapshot v1, https://goo.gl/fbAQLP

exports[`FirstMenu matches the snapshot 1`] = `
<div>
  <button
    className="menu-button"
    onClick={[Function]}
  >
    Create Game
  </button>
  <button
    className="menu-button"
    onClick={[Function]}
  >
    About
  </button>
</div>
`;

    \end{lstlisting}
  \end{minipage}

  \caption{Snapshot som skapats av testet i \ref{fig:snapshot-test}}
  \label{fig:snapshot-shot}
\end{figure}

\subsection{Assertion}
Assertion är basen i testning. Detta används ofta för att se att utvärdet från en funktion stämmer med det som förväntas få ut. 
\subsection{Spies}
spies
\subsection{API}
API
