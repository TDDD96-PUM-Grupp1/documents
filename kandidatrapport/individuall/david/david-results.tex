\section{Resultat}
\label{sec:david-results}
Det här kapitlet går igenom resultatet utav studien.

\subsection{Speltestning}

\subsection{Testverktyg}
Tyvärr är det ju så att det inte finns ett perfekt verktyg som alltid är bäst att använda. Däremot finns det fördelar och nackdelar med de olika verktygen så man lättare kan veta när man ska välja vad för verktyg för sina tester. Till exempel, vi valde att utveckla vårt spel i React, och därför var Jest ett bra alternativ då det parallellutvecklades med React. 

Jest är bra för att det är snabbt att komma igång och börja arbeta direkt, till skillnad från flera andra bibliotek har det ett brett API som inte kräver att man inkluderar extra bibliotek om man inte absolut behöver det. Jest utnyttjar även något som kallas för \textit{workers}, de används för att optimera testningen genom att låta test köras parallellt vilket kan öka hastigheten enormt vid större projekt.