\section{Resultat}
\label{sec:david-results}
Det här kapitlet går igenom resultatet utav studien.

\subsection{Speltestning}
Det finns många sätt att automatisera speltester på, men många saker som fortfarande kräver mänsklig interaktion. Tester automatiseras till stor del i början av utvecklingsfasen för regressionstestning, ett sätt att testa att det som fungerar i en tidigare utgåva, även fungerar i en senare. Exempel på detta kan vara Snapshot\ref{sec:david-theory}. Mycket när det kommer till spel är själva känslan utav spelet. Det är svårt att kvantifiera och därför brukar större företag släppa tidigare versioner utav spel för att se hur allmänheten tycker om spelet i sig, så kallade beta-versioner. Projektgruppen skapade även en egen beta-version. Där vi under uppsikt av medlemmar lät folk prova på produkten med sina egna mobiler och sedan fylla i ett formulär med frågor om hur de tyckte om produkten. Som testledare tyckte jag att det var intressant att se hur de olika mobilerna folk använde sig utav fick applikationen att se annorlunda ut och gjorde att spelupplevelsen varierade lite på vissa telefoner.

\subsection{Testverktyg}
Tyvärr är det ju så att det inte finns ett perfekt verktyg som alltid är bäst att använda. Däremot finns det fördelar och nackdelar med de olika verktygen så man lättare kan veta när man ska välja vad för verktyg för sina tester. Till exempel, vi valde att utveckla vårt spel i React, och därför var Jest ett bra alternativ då det parallellutvecklades med React. 

Jest är bra för att det är snabbt att komma igång och börja arbeta direkt, till skillnad från flera andra bibliotek har det ett brett API som inte kräver att man inkluderar extra bibliotek om man inte absolut behöver det. Jest utnyttjar även något som kallas för \textit{workers}, de används för att optimera testningen genom att låta test köras parallellt vilket kan öka hastigheten enormt vid större projekt. Jest är baserat på Jasmine och har mycket av dess funktionalitet men med mer funktionalitet. De har till exempel lagt till bättre mock-funktionalitet och förbättrat funktionaliteten så att tester kan köras parallellt. En nackdel med Jest är att det är nyare och inte funnits lika länge som Mocha och skaffat sig samma användarbas. Det påverkar användarskapade bibliotek och kan innebära mindre varierande funktionalitet och potentiellt fler oupptäckta buggar.