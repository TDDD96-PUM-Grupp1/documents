\section{Slutsatser}
\label{sec:axel-conclusion}

Detta avsnitt presenterar de slutsatser man kan göra utifrån det presenterade resultatet för att besvara de ställda frågeställningarna. Eftersom frågeställningarna är öppna kommer även slutsatserna vara öppna.

\subsection*{\ref{axel-fs:1} Vilka skillnader och likheter innebär valet av Angular jämfört med React?}
Likheter mellan React och Angular är att båda verktygen är komponentbaserad. Båda verktygen underlättar utveckling av webbapplikationer. Skillnader mellan React och Angular är:
\begin{enumerate}
    \item Deras filosofi kring hur en komponent ska struktureras. Angular vill separera på alla teknologier medan React vill samla alla relevanta teknologier inom en komponent.

    \item React är lättare att lära sig medan Angular förser en utvecklare med en tydligare struktur för en webbapplikation.

    \item Angular är ett större ramverk och ger mer funktionalitet än React.

    \item React anses vara mer flexibelt där en utvecklare får en större frihet att välja vilken funktionalitet som ska importeras till ett projekt.
    
\end{enumerate}

\subsection*{\ref{axel-fs:2} Vid vilka typer av projekt lämpar sig Angular bättre än React och vice versa?}
Då kvalitetsfaktorn användbarhet sätts i fokus lämpar sig Angular bättre i större projekt. Detta eftersom React kommer med största sannolikhet ha fler beroenden till externa Javascript-bibliotek. Detta gör att tröskeln för att sätta sig in i ett stort projekt utvecklat med React kan bli större än Angular. Vid mindre projekt lämpar sig React bättre. Eftersom React, med få beroenden till andra Javascript-bibliotek, är ett simplare och lättare verktyg att lära sig.
 
\subsection*{\ref{axel-fs:3} Hade Angular lämpat sig bättre i det utförda projektet?}
I det utförda projekt lämpade nog React bättre än Angular. Delvis för att Reacts roll i projektet var relativt liten. Men också för att det utförda projektet hade en tidsbegränsning på 400 timmar. Eftersom React är både lättare att simplare och lättare att lära sig kunde antagligen projektgruppen komma igång med utvecklingen tidigare. Om React hade haft en större roll i projektet eller om projektet hade pågått under en längre tid skulle Angular ha lämpat sig bättre.


