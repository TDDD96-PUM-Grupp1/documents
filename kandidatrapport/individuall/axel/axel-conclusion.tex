\section{Slutsatser}
\label{sec:axel-conclusion}

Detta avsnitt presenterar de slutsatser man kan göra utifrån det presenterade resultatet för att besvara de ställda frågeställningarna. Eftersom frågeställningarna är öppna kommer även slutsatserna vara öppna.

\subsection*{\ref{axel-fs:1} Vilka skillnader och likheter innebär valet av Angular jämfört med React?}
Likheter mellan React och Angular är att båda verktygen är komponentbaserade. Båda verktygen underlättar utveckling av webbapplikationer. Skillnader mellan React och Angular är:
\begin{enumerate}
    \item Deras filosofi kring hur en komponent ska struktureras. Angular vill separera på alla teknologier medan React vill samla alla relevanta teknologier inom en komponent.

    \item React är lättare att lära sig medan Angular förser en utvecklare med en tydligare struktur för en webbapplikation.

    \item Angular är ett större ramverk och ger mer funktionalitet än React.

    \item React anses vara mer flexibelt där en utvecklare får en större frihet att välja vilken funktionalitet som ska importeras till ett projekt.
    
\end{enumerate}

\subsection*{\ref{axel-fs:2} Vid vilka typer av projekt lämpar sig Angular bättre än React och vice versa?}
Då kvalitetsfaktorn användbarhet sätts i fokus lämpar sig Angular bättre i större projekt. Detta eftersom React kommer med största sannolikhet ha fler beroenden till externa Javascript-bibliotek. Detta gör att tröskeln för att sätta sig in i ett stort projekt utvecklat med React kan bli större än Angular. Vid mindre projekt lämpar sig React bättre. Eftersom React, med få beroenden till andra Javascript-bibliotek, är ett simplare och lättare verktyg att lära sig.
 
\subsection*{\ref{axel-fs:3} Vilka positiva effekter hade Angular haft på det utförda projektet?}
Angular hade bidragit med en tydligare struktur för hur det utförda projektet skulle ha strukturerats på grund av att ramverket har en fördefinierad struktur för hur en webbapplikation ska utvecklas. I och med majoriteten av projektgruppen saknade tidigare erfarenheter av webbutveckling, hade en fördefinierad struktur varit fördelaktigt för utvecklingsarbetet. Det hade även varit sannolikt att Angular hade minskat antalet beroenden till andra externa Javascript-bibliotek på grund av att Angular är ett stort och välutvecklat ramverk jämfört med React. 




