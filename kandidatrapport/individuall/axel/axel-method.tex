\section{Metod}
\label{sec:axel-method}

Detta avsnitt presenterar vilka metoder som använts för att besvara frågeställningarna som presenterades i avsnitt D.1.2. Majoriteten av denna studie grundar sig i en litteraturstudie som har utförts. Studien baseras även på samlade erfarenheter från det utförda projektet.

\subsection{Litteraturstudie}
Den utförda litteraturstudien har genomgått två olika faser. Under den första fasen hämtades information från dokumentationen given av företagen som utvecklat verktygen. Hela teoriavsnittet är baserad på denna information. Denna metod valdes för att ge en saklig presentation av de verktyg som har undersökts. I och med att företagen har ett intresse i de verktyg som ska jämföras, krävs det att studien är källkritisk för att öka dess trovärdighet. Eftersom studien fokuserar på strukturella skillnader kan det däremot ses positivt att informationen är hämtad direkt från verktygens dokumentation. Detta för att företagen själva vet bäst hur deras verktyg är uppbyggda. Informationen som har samlats under denna fas har enbart fokuserats på hur verktygen är uppbyggda. 

Under den andra fasen samlades information från akademiska artiklar genom sökmotorn Google Scholar och konferenser inom området. Information under denna fas har även hämtats från bloggartiklar. Denna information har hittats genom sökmotorn Google med sökorden ``React vs Angular'' och ``Differences between React and Angular''. Alla jämförelser utförda i denna studie är baserade på dessa källor. Trovärdighet från de akademiska artiklarna och konferenserna anses generellt sett vara hög. Däremot behöver man vara källkritisk till att bloggartiklar har valts som källor. Detta var ett medvetet val för att undvika utdaterad information. Verktygen som jämförs, utvecklas och förändras väldigt hastigt. Det kan ta flera år innan att en akademisk artikel blir publicerad och blir därmed irrelevant till denna studien. Av denna anledning har bloggartiklar valts att användas som källor.

\subsection{Erfarenheter från utfört projekt}
Det utförda projektet utvecklades i React. Detta var ett krav som specificerades i projektets kravspecifikation. Erfarenheter som har samlats under projektets gång kommer att ligga till grund för att besvara frågeställning 3. Projektet pågick under hela vårterminen där majoriteten av projektmedlemmarna hade ingen tidigare erfarenhet av varken React eller Angular. Projektet var dessutom tidsbegränsat till 400 timmar per projektmedlem. 

I det utförda projektet utvecklades två kodbaser med React:

\begin{enumerate}
    \item En kontroll-applikation
    \item En UI-applikation
\end{enumerate} 

Se avsnitt \ref{moduler} för mer information kring dessa. I utvecklingen av båda webbapplikationerna användes React för att konstruera menyer och skicka data, baserat på diverse menyval.

Erfarenheter har samlats genom de möten och problem projektgruppen har diskuterat genom Slack. Andra erfarenheter har också samlats allteftersom projektet utvecklats i projektgruppen.



