\section{Metod}
\label{sec:axel-method}

Detta avsnitt presenterar vilka metoder som använts för att besvara frågeställningarna som presenterades i avsnitt D.1.2.

\subsection{Litteraturstudie}
Majoriteten av denna studie kommer grunda sig i den litteraturstudie som har utförts för att samla relevant teori för studien. Hela teoriavsnittet har grundats helt från dokumentationen given av företagen som utvecklat biblioteken och ramverken. Detta för att validera trovärdigheten i den information som presenterats samt ger en saklig presentation av ramverken och biblioteken. Alla jämförelser grundar till stor del i egna observationer från teori avsnittet samt från andra källor. För att säkerställa trovärdigheten har informationen hämtas från akademiska källor och konferenser inom området.

\subsection{Erfareneheter från utfört projekt}
Det utförda projektet utvecklades i React. Detta var ett krav som specificerades i projektets kravspecifikation. Erfarenheter som har samlats under projektets gång kommer att ligga i grund för att besvara frågeställning 3. Projektet pågick under hela vårterminen där majoriteten av projektmedlemmarna hade ingen tidigare erfarenhet av varken React eller Angular.
