\section{Introduktion}
\label{sec:axel-introduction}
Utvecklingen av webbapplikationer har genomgått stora förändringar på bara de senaste tio åren~\cite{changing}. När man ska utveckla en webbapplikation står utvecklaren inför många olika val av ramverk och bibliotek. Det uppkommer nya ramverk och bibliotek konstant och deras popularitet förändras kontinuerligt. Traditionellt sett har webbapplikationer utvecklats genom HTML, CSS och Javascript. Allt eftersom tiden har gått, har det ställts det högre krav på prestanda, användarinteraktivitet, säkerhet och effektivitet för att ge en bättre användarupplevelse. På grund av att det ställts högre krav introducerades många olika Javascript-ramverk och bibliotek som fokuserar på detta. Konsekvensen har blivit att en utvecklare idag har ett stort urval av bibliotek och ramverk att välja bland. Med ett ökat utbud, har det även blivit svårare att välja vilka av dessa bibliotek och ramverk som lämpar sig bäst för olika webbapplikationer.

\subsection{Syfte och Mål}
\label{subsec:axel-motivation}

Syftet med studien är att undersöka vilka skillnader och likheter det finns mellan React och Angular. I det utförda projektet användes Javascript-biblioteket React. Detta var ett krav från vår kund som specificerades i kravspecifikationen. Men det finns ett intresse att även undersöka vilka skillnader det hade inneburit för det utförda projektet om Angular hade valts istället. Målet med studien är att undersöka denna aspekt. Då det finns otroligt många aspekter att jämföra kommer studien att begränsas till några få kvalitetsfaktorer, se avsnitt \ref{subsec:axel-delimitations}.

\subsection{Frågeställning}
\label{subsec:axel-research-questions}

De frågeställningar som har undersökts är:

\begin{enumerate}
\item\label{axel-fs:1} Vilka skillnader och likheter innebär valet av Angular jämfört med React?

\item\label{axel-fs:2} Vid vilka typer av projekt lämpar sig Angular bättre än React och vice versa?

\item\label{axel-fs:3} Vilka positiva effekter hade Angular haft på det utförda projektet?


\end{enumerate}


\subsection{Avgränsningar}
\label{subsec:axel-delimitations}
Studien kommer att begränsas till biblioteket React samt till ramverket Angular. Då det finns väldigt många aspekter att undersöka när React jämförs med Angular kommer studien att begränsas till kvalitetsfaktorn användbarhet och strukturella skillnader. Användbarhet definieras i denna studie som hur lätt det är för en utomstående utvecklare att bidra till ett projekt.

