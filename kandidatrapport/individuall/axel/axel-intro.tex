\section{Introduktion}
\label{sec:axel-introduction}
Utvecklingen av webbapplikationer har genomgått stora förändringar på bara de senaste tio åren. När en utvecklare ska utveckla en webbapplikation står utvecklaren inför många olika val av ramverk och biblotek. Det uppkommer nya ramverk och biblotek konstant och deras popularitet förändras kontinuerligt. Traditionellt sett, har webbapplikationer utvecklats genom HTML, CSS och Javascript. Allt eftersom tiden gick ställdes det högre krav på prestanda, användarinteraktivitet, säkerhet och effektivitet för att ge en bättre användarupplevelse. På grund av att det ställdes högre krav, introducerades många olika Javascript ramverk och biblotek som fokuserade på dessa kvaliteter. Konsekvensen har blivit att en utvecklare har idag ett stort urval av biblotek och ramverk att välja bland. Eftersom utbudet har ökat, har det även blivit svårare att välja vilka av dessa biblotek och ramverk som lämpar sig bäst för sin webbapplikation.

\subsection{Syfte \& Mål}
\label{subsec:motivation}

Syftet med studien är att undersöka vilka skillnader och likheter det finns mellan ReactJS och Angular. Studien kommer även att undersöka vad dessa skillnader hade inneburit för vårt projekt. 

\subsection{Frågeställning}
\label{subsec:research-questions}

De frågeställningar som undersöks är:

\begin{enumerate}
\item Vilka skillnader och likheter innebär valet av Angular jämfört med ReactJS?

\item Vid vilka typer av projekt lämpar sig Angular bättre än ReactJS och vice versa?

\item På vilka sätt hade valet av Angular istället för ReactJS inneburit för projektet?


\end{enumerate}


\subsection{Avgränsingar}
\label{subsec:delimitations}
Studien kommer att begränsas till bibloteket ReactJS samt till ramverket Angular. Det är vanligt att man delar upp en webbapplikation i två delar, en klient sida(front-end) och en server sida(back-end). Den här studien kommer enbart fokusera på klient sidan av dessa ramverk och biblotek. 

