\section{Diskussion}
\label{sec:axel-discussion}
Det här avsnittet kommer att diskutera metoden och resultatet som presenterats. 

\subsection{Resultat}
\label{subsec:axel-discussion-results}
Det finns nästintill oändligt många aspekter och perspektiv man kan undersöka då man jämför React med Angular. För att bedöma huruvida den ena aspekten är bättre än den andra beror på vilka kvalitetsfaktorer som värdesätts högst. Resultatet i den utförda studien fokuserade på strukturella skillnader och hur lätt det är för en utomstående utvecklare att bidra till utvecklingen. Dessa faktorer valdes eftersom de kändes intressanta och relevanta till det utförda projektet. Majoriteten av projektmedlemmarna saknade tidigare erfarenhet av både React och Angular. Därav var det intressant att undersöka vilken inverkan valet av React och Angular har på ett projekt. Resultatet ger en god insikt för någon utan tidigare erfarenheter av Angular eller React om hur deras fundamentala koncept skiljer sig åt. Resultatet visar även på vilka typer av projekt Angular och React lämpar sig till då ovannämnda kvalitetsfaktorer sätts i fokus. Det hade även varit intressant att undersöka andra kvalitetsfaktorer så som prestanda- och säkerhetsskillnader.

\subsubsection{Angular i det utförda projektet}
Jag anser att React lämpade sig bättre till det utförda projektet än vad Angular hade gjort. På grund av den tidsbegränsning på 400 timmar som sattes. Då det går snabbare att lära sig React är det väldigt sannolikt att projektgruppen kunde komma igång med utvecklingen tidigare än om Angular hade valts. Majoriteten av projektgruppen saknade erfarenhet inom webbutveckling. Av denna anledning anser jag att simpliciteten hos React uppskattades mer än strukturen hos Angular. Ännu en fördel var att en projektmedlem hade tidigare erfarenhet med React och kunde hålla en workshop. 

Jag klassificerar det utförda projektet som ett mindre projekt. Detta ledde till att få externa Javascript-bibliotek behövde användas. Detta är en fördel för React då det är färre verktyg att sätta sig in i, för att bidra till projektet. Om projektet hade krävt mer funktionalitet från React skulle jag hävda att Angular hade lämpat sig bättre.

Om projektet hade pågått under en längre tid skulle jag argumentera att Angular hade lämpat sig bättre. Detta eftersom hanteringen av dataflödet blev klurigt mot projektets slut. Dessutom hade inlärningsfasen haft en mindre inverkan om det funnits mer tid att lära sig Angular.

\subsection{Metod}
\label{subsec:axel-discussion-method}
Metoden som har använts för att generera ett resultat har främst bestått av sammanställning av information och dokumentation. Utmaningen med metoden var delvis att presentera både Angular och React på ett objektivt sätt utan att jämföra dom. Detta för att ge en så neutral bild som möjligt av vad dessa ramverk och bibliotek. För att sedan ställa dom mot varandra under resultat delen av studien. 

Det är värt att nämna att det kan vara problematiskt att använda företagen som utvecklat verktygen som källor. Då företagen har ett eget intresse i sina verktyg, kan verktygen presenteras på ett glorifierat sätt. Jag tror denna effekt hade en liten inverkan på studien på grund av de kvalitetsfaktorer som valdes. Hade andra kvalitetsfaktorer valts, så som prestanda, hade man behövt vara mer källkritisk mot studien. Det är möjligt att företagen presenterade strukturen på ett över-simplifierat sätt som inte stämmer överens med verkligheten. Men eftersom studien har fokuserat mycket på hur man ska använda sig av verktygen, bör det ligga i företagens intresse att presentera denna information på ett korrekt och simpelt sätt. 

En annan aspekt man ska vara kritiskt till i metoden är att bloggar användes som källor. Detta var ett medvetet val eftersom både webbutvecklingsvärlden och verktygen förändras hastigt. Det kan ta några år för en akademisk källa att bli publicerad vilket gör att den information som presenteras kan bli utdaterad och irrelevant snabbt. Med det sagt har även akademiska källor använts vilket styrker trovärdigheten av vissa delar av resultatdelen.

\pagebreak

