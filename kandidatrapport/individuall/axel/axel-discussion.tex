\section{Diskussion}
\label{sec:axel-discussion}
Det här avsnittet kommer att diskutera metoden och resultatet som presenterats. 

\subsection{Resultat}
\label{subsec:axel-discussion-results}
Det finns nästintill oändligt många aspekter och perspektiv man kan undersöka då man jämför ReactJS med Angular. För att bedömma hurvida den ena aspekten är bättre än den andra beror på vilka kvalitetsfaktorer som värdesätts högst. Resultatet i den utförda studien fokuserade på strukturella skillnader och hur lätt det är för en utomstående utvecklare att bidra till utvecklingen. Dessa faktorer valdes eftersom de kändes intressanta och relevanta till det utförda projektet. Majoriteten av projektmedlemarna saknade tidigare erfarenhet av både ReactJS och Angular. Därav var det intressant att undersöka vilken inverkan valet av ReactJS och Angular har på ett projekt. Resultatet ger en god insikt för någon utan tidigare erfarenheter av Angular eller ReactJS om hur deras fundementala koncept skiljer sig. Resultatet visar även på vilka typer av projekt Angular lämpar sig bättre än ReactJS samt vice versa då ovannämnda kvalitetsfaktorer sätts i fokus. Det skulle även vara intressant att undersöka andra kvalietetsfaktorer så som prestanda skillnader. 

\subsection{Metod}
\label{subsec:axel-discussion-method}
Metoden som har använts för att generera ett resultat har främst bestått av sammanställning av information och dokumentation. Utmaningen med metoden var delvis att presentera både Angular och ReactJS på ett objektivt sätt utan att jämföra  dom. Detta för att ge en neutral bild av vad dessa ramverk och biblotek är för att sedan ställa dom mot varandra under resultat delen av studien. För att hålla en hög kvalitet, användes enbart akademiska källor och konferenser inom området.

Samlade erfarenheter av det utförda projektet...



