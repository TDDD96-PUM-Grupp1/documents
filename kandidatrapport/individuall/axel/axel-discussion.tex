\section{Diskussion}
\label{sec:axel-discussion}
Det här avsnittet kommer att diskutera metoden och resultatet som presenterats. 

\subsection{Resultat}
\label{subsec:axel-discussion-results}
Det finns nästintill oändligt många aspekter och perspektiv man kan undersöka då man jämför React med Angular. För att bedöma huruvida den ena aspekten är bättre än den andra beror på vilka kvalitetsfaktorer som värdesätts högst. Resultatet i den utförda studien fokuserade på strukturella skillnader och hur lätt det är för en utomstående utvecklare att bidra till utvecklingen. Dessa faktorer valdes eftersom de kändes intressanta och relevanta till det utförda projektet. Majoriteten av projektmedlemmarna saknade tidigare erfarenhet av både React och Angular. Därav var det intressant att undersöka vilken inverkan valet av React och Angular har på ett projekt. Resultatet ger en god insikt för någon utan tidigare erfarenheter av Angular eller React om hur deras fundamentala koncept skiljer sig. Resultatet visar även på vilka typer av projekt Angular lämpar sig bättre än React samt vice versa då ovannämnda kvalitetsfaktorer sätts i fokus. Det skulle även vara intressant att undersöka andra kvalitetsfaktorer så som prestanda-skillnader.

\subsection{Metod}
\label{subsec:axel-discussion-method}
Metoden som har använts för att generera ett resultat har främst bestått av sammanställning av information och dokumentation. Utmaningen med metoden var delvis att presentera både Angular och React på ett objektivt sätt utan att jämföra  dom. Detta för att ge en neutral bild av vad dessa ramverk och biblotek är för att sedan ställa dom mot varandra under resultat delen av studien. 

Det är värt att nämna att det kan vara problematiskt att använda företagen som utvecklat verktygen som källor. Det blir problematiskt då företagen har ett eget intresse i sina verktyg vilket gör att verktygen kan glorifieras. Jag tror denna effekt hade däremot en liten inverkan på studien på grund av de kvalitetsfaktorer som valdes. Hade andra kvalitetsfaktorer valts så som prestanda hade man behövt vara mer källkritisk mot studien. Det är absolut en möjlighet att företagen presenterar strukturen på ett över simplifierat sätt som inte stämmer överens med verkligheten. Men eftersom studien har fokuserats mycket på hur man ska använda sig av verktygen, bör det ligga i företagens intresse att presentera denna information på ett korrekt och simpelt sätt. Detta eftersom verktyget blir svårare att lära sig om informationen är inkorrekt.

För att hålla en hög kvalitet vid jämförelsen användes enbart akademiska källor och konferenser inom området.

Samlade erfarenheter av det utförda projektet...



