\section{Diskussion}
\label{sec:axel-discussion}
Det här avsnittet kommer att diskutera metoden och resultatet som presenterats. 

\subsection{Resultat}
\label{subsec:axel-discussion-results}
Det finns nästintill oändligt många aspekter och perspektiv man kan undersöka då man jämför React med Angular. För att bedöma huruvida den ena aspekten är bättre än den andra beror på vilka kvalitetsfaktorer som värdesätts högst. Resultatet i den utförda studien fokuserade på strukturella skillnader och hur lätt det är för en utomstående utvecklare att bidra till utvecklingen. Dessa faktorer valdes eftersom de kändes intressanta och relevanta till det utförda projektet. Majoriteten av projektmedlemmarna saknade tidigare erfarenhet av både React och Angular. Därav var det intressant att undersöka vilken inverkan valet av React och Angular har på ett projekt. Resultatet ger en god insikt för någon utan tidigare erfarenheter av Angular eller React om hur deras fundamentala koncept skiljer sig. Resultatet visar även på vilka typer av projekt Angular och React lämpar sig till då ovannämnda kvalitetsfaktorer sätts i fokus. Det hade även varit intressant att undersöka andra kvalitetsfaktorer så som prestanda- och säkerhetsskillnader.

\subsubsection{Angular i det utförda projektet}
Jag tror att React lämpade sig bättre till det utförda projektet än vad Angular hade gjort. Delvis på grund den tidsbegränsning på 400 timmar som sattes. I och med React är ett lättare bibliotek att lära sig är det väldigt sannolikt att projektgruppen kunde komma igång med utvecklingen tidigare än om Angular hade valts. Dessutom saknade majoriteten av projektgruppen i allmänhet erfarenhet inom webbutveckling. Just av denna anledning tror jag att för projektet att simpliciteten hos React uppskattas mer än strukturen hos Angular.

Rollen React hade i det utförda projektet var inte heller särskilt stor. Jag skulle klassificera det utförda projektet som ett mindre projekt och detta är ytterligare en anledning varför React lämpade sig bättre. Hade däremot projektet varit större och haft fler beroende till externa Javascript-bibliotek hade Angular nog lämpat sig bättre än React.

\subsection{Metod}
\label{subsec:axel-discussion-method}
Metoden som har använts för att generera ett resultat har främst bestått av sammanställning av information och dokumentation. Utmaningen med metoden var delvis att presentera både Angular och React på ett objektivt sätt utan att jämföra  dom. Detta för att ge en neutral bild av vad dessa ramverk och bibliotek är för att sedan ställa dom mot varandra under resultat delen av studien. 

Det är värt att nämna att det kan vara problematiskt att använda företagen som utvecklat verktygen som källor. Det blir problematiskt då företagen har ett eget intresse i sina verktyg vilket gör att verktygen kan glorifieras. Jag tror denna effekt hade däremot en liten inverkan på studien på grund av de kvalitetsfaktorer som valdes. Hade andra kvalitetsfaktorer valts så som prestanda hade man behövt vara mer källkritisk mot studien. Det är absolut en möjlighet att företagen presenterar strukturen på ett över simplifierat sätt som inte stämmer överens med verkligheten. Men eftersom studien har fokuserats mycket på hur man ska använda sig av verktygen, bör det ligga i företagens intresse att presentera denna information på ett korrekt och simpelt sätt. Detta eftersom verktyget blir svårare att lära sig om informationen är inkorrekt.

En annan aspekt man ska vara kritiskt mot i metoden är att bloggar användes som källor. Detta var en medvetet val eftersom både webbutvecklingsvärlden och verktygen förändras hastigt. Det kan ta några år för en akademisk källa att bli publicerad vilket gör att den information som presenteras kan bli utdaterad och irrelevant snabbt. Med det sagt har även akademiska källor använts vilket styrker trovärdigheten av vissa delar av resultatdelen.

\pagebreak