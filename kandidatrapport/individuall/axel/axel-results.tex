\section{Resultat}
\label{sec:axel-results}
Detta avsnitt presenterar resultatet av den utförda studien.

\subsection{Ramverk jämfört med bibliotek}
Till att börja med är det värt att notera att React är ett Javascript-bibliotek medan Angular är ett ramverk. Detta är en viktig detalj eftersom ett ramverk dikterar flödet av ett program. \cite[Avsnitt 2.4]{medium} Angular har en fördefinierad struktur till hur en webbapplikation ska utvecklas genom sitt ramverk. Denna struktur saknas hos React. Där har utvecklaren istället valet att strukturera upp flödet i webbapplikationen som utvecklaren vill. På så sätt kan React anses vara mer flexibelt än Angular. Huruvida denna flexibilitet är en fördel eller nackdel beror på vilken aspekt man undersöker och vilket stadium ett projekt befinner sig i. 

\subsection{Beroenden till andra bibliotek}
\label{beroenden-till-andra-bibliotek}
Då en applikation utvecklas med React är det väldigt troligt att utvecklaren kommer introducera andra Javascript-bibliotek för att komplettera viss funktionalitet. \cite[Features]{sitepoint} En fördel med detta är valmöjligheten som ges till utvecklaren, att kunna få välja vilken funktionalitet som ska importeras. Detta kan dock bli problematiskt utifrån två aspekter. Båda dessa är relevanta för Angular, men är mindre troliga eftersom Angular är ett större ramverk med mycket mer inbyggd funktionalitet än React. Aspekterna är:

\begin{enumerate}
    \item Ansvaret att sköta integreringar av uppdateringar från de externa biblioteken hamnar hos utvecklaren, vilket kan leda till kompatibilitetsproblem.

    \item Desto fler Javascript-bibliotek man introducerar till ett projekt, desto svårare blir det för en ny utvecklare att förstå och komma igång med utvecklingen för applikationen. Detta är en viktig aspekt beroende på hur stort utvecklingsprojektet är.
\end{enumerate}



\subsection{Filosofier}
Trots att både React och Angular är komponentbaserade ramverk och bibliotek, har de två helt olika filosofier kring hur en webbapplikation ska struktureras. Angulars filosofi är att separera alla teknologier i olika moduler, där varje modul gör en specifik sak väldigt väl. React filosofi är att samla alla relevanta teknologier till en komponent. Detta syns tydligt när man jämför Angulars vyer med React komponenter. Angular gör valet att dela upp en vy med en komponent och en template, där beteendet för vyn existerar i komponenten och utseendet existerar i templaten. React integrerar både utseendet och beteendet till en komponent och låter komponenten i sig bli mer självständig. 

Fördelen med att integrera allt relevant till en komponent som i React, är att man minskar beroenden mellan komponenterna och ökar sammanhållningen inom komponenten. Detta är fördelaktigt när modifikationer måste ske som Pete Hunt förklarar på JSconf EU 2013.\cite{JSConf} Han förklara att i Angular kan till exempel olika komponenter använda sig av samma template. Vid vissa modifikationer i templaten, skulle det kunna innebära att man även måste modifiera alla komponenter som är beroende av denna template. Detta gör det svårare att underhålla webbapplikationen. Det här problemet gäller även för Angulars services.

Nackdelen med att integrera allt relevant till en komponent är att implementationsaspekten kan bli svårare när webbapplikationen växer. När det hierarkiska djupet växer bland komponenter riskeras att data behöver flöda en längre sträcka i trädet av komponenter. Detta kan innebära att varje komponent i en kedja kommer behöva modifieras vid en ny implementation.

%https://www.youtube.com/watch?v=x7cQ3mrcKaY

\subsection{Användbarhet}
Hur lätt det är för en utvecklare att sätta sig in i och bidra till ett projekt beror på många faktorer. Vid större projekt är det viktigt att ha en tydlig struktur. Här skulle Angular kunna anses vara bättre på grund av det fördefinierade flödet ramverket bidrar med. Framförallt om utvecklaren har tidigare erfarenheter av Angular eftersom det allmänna flödet i webbapplikationen kommer likna tidigare projekt. Medan flödet i React kommer att variera i stor grad mellan olika projekt då utvecklarna har friheten att bestämma hur flödet ska se ut. 

Om en utvecklare däremot har varken tidigare erfarenhet av Angular eller React finns det ytterligare en aspekt att ha i åtanke. React anses vara lättare att lära sig än Angular.\cite[Adoption, Learning Curve and Development Experience]{sitepoint} Tröskeln för att lära sig Angular är större eftersom utvecklaren måste förstå alla Angular koncept för hur något ska implementeras. Utöver detta behöver även utvecklaren lära sig Typescript och all Angular specifik syntax så som directives. 

När en utvecklare ska lära sig React kommer utvecklaren att utsättas för JSX och några React specifika koncept. För att lära sig JSX behöver man lära sig Javascript och XML. Båda dessa språk används även i Angular i form av bland annat Typescript. På så sätt kan man argumentera att en utvecklare utsätts för färre verktyg och språk när en utvecklare ska lära sig React. React ger frihet och simplicitet \cite{react-angular-paper}, men problematiken som presenterades i \ref{beroenden-till-andra-bibliotek} får inte glömmas.

\subsection{Erfarenheter från det utförda projektet}
Projektgruppen lyckades komma igång med utvecklingen av produkten vid ett tidigt stadie. Det var en medlem i projektgruppen som hade tidigare erfarenhet av React. Han höll i en workshop vilket hjälpte gruppen att lära sig grunderna i React. Redan en vecka efter workshopen hade tagit plats, var majoriteten av gruppmedlemmarna bekväma med att utveckla i React. Reacts roll i det utförda projektet var att konstruera menyer av olika slag. Det vill säga, React hade en liten roll i projektet. En konsekvens av detta var att få externa bibliotek introducerades. 

Ett återkommande problem som uppstod för gruppen var när projektet började nå sitt slut och ny funktionalitet skulle implementeras. Vid detta stadium fanns det flera djupa lager av komponenter. När ett nytt \texttt{state} introducerades kunde det innebära att den behövde slussas genom flertalet komponenter. Detta ledde till att viss funktionalitet tog onödigt mycket tid att implementera. Det uppstod buggar och kunde vara svårt att felsöka vart i dataflödet det hade gått snett. 


