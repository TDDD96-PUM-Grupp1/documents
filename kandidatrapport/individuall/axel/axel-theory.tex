\section{Teori}
\label{sec:axel-theory}
Det här kapitlet beskriver fundementala begrepp och koncept för studien.

\subsection{HTML DOM}
När en hemsida laddas in, skapar webbläsaren en HTML DOM (Document object module) av hemsidan. HTML DOM är ett träd som representerar HTML kod som objekt, se figur xx. Den definierar:
\begin{itemize}
\item HTML element som objekt
\item HTML elementens attribut
\item Metoder för att modifera HTML element
\item Alla event för varje HTML element.
\end{itemize} 
Med andra ord kan HTML DOM beskrivas som en standard för hur man ska få tag på, lägga till och ta bort HTML element. Detta ger möjligheten för en hemsida att bli mer dynamisk eftersom dess innehåll och struktur kan modifieras och formas om. Ett programmeringsspråk, generellt sett JavaScript, brukas användas för att få tillgång till HTML DOMen. HTML DOM metoder är handlingar som utövas över HTML element medan HTML attribut är värden som kan sättas och ändras. 


\subsection{JavaScript språk}


\subsection{Angular 2.0}
Angular 2.0 är ett TypeScript baserad open-source ramverk som underhålls främst av Google men även andra individer och företag. Angular 2.0 lanserades i septemeber 2016 med sin initiala release och lanserade sin stabila release i decemeber 2018. Angular 2.0 är en total rekonstruktion av det tidigare ramverket AngularJS som lanserades i october 2010. Målet med AngularJS var att skapa ett ramverk som underlättade utvecklingen och testningen vid skapandet av en webbapplikation. AngularJS ger funktionaliteten för utvecklaren att skapa egna HTML taggar (AngularJS directives) för att göra koden mer läsbar och förstålig. En HTML tagg skulle till exempel kunna vara en menurad. Efter att en HTML tagg har konstruerats kan den återanvändas fritt inom applikationen, vilket tar bort repetativ kod om samma komponent återanvänds. 

\subsection{ReactJS}


Här kommer de olika ramverken som undersökts att presenteras och vad som gäller specifikt för dom samt vad som gör dom unika. Det kommer även att tas upp hur vi har använt oss av React i vårt projekt och vad det har bidragit med. 
