\section{Teori}
\label{sec:Lieth-Teori}
Agila metoder, såsom Scrum, har sett en märkbar tillväxt i popularitet inom mjukvaruutveckling kontext som karakteriseras av höga ostadighet[6]. I ett Scrum 
tillvägagångssätt team deltagare jobbar ihop för att föra projektet framåt genom att använda s.k Scrum verktyg som t.ex Scrum board, Scrum möte, burndown 
charts[7]. Detta möjliggöra för teamet att vara medvetna om senaste ändringar i kraven och om-prioriterar om det sådant behövs[7]. 

Eftersom Scrum metodiken är väldigt flexibel och kan ändras sådant att den passar varje grupps behov vi bestämde oss att har endast följande delar av scrum:

\begin{enumerate}
	\item Sprintplanering

	\item Scrum board  i form av Trello board
	
	\item Scrum möten i form av veckomöte online kommunikation via Slack
	
	\item Veckorapporter och burndown charts
	
\end{enumerate}
