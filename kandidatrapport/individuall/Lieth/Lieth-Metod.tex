\section{Metod}
\label{sec:Lieth-Metod}
Detta avsnitt går i detaljer om hur vår Scrum version är uppbyggd och vad som förväntas av varje del.

\subsection{Sprintplannering}
\label{subsec:Lieth-Sprintplannering}
Varje sprint är 2 veckor lång. under varje sprint så jobbade teamet mot en del problem. Detta gjorde det lättare 
för teamet att jobba effektivare, då stora problem är delade i små delproblem som gör det lättare att tackla problemen 
samt att berätta vad  har gjort vid veckomöten. Vid slutet av varje sprint gruppen höll ett möte för att skapa en 
statusrapport på vad som har utförts under sprinten och ifall den sprintents uppsatta målen är mötta eller inte.


\subsection{Scrum Board}
\label{subsec:Lieth-Scrum Board}
Scrum board är gjordes i form av Trello aktiviteter. Detta möjliggjorde spårning av målen samt underlättade för gruppen att ha 
koll på vad som finns kvar att utföra, vad som utförs, och vad som har utförts. inför start av varje sprint, teamet höll ett möte 
för att planera vad som behöver göras under kommande sprint, vilka skulle göra vad, och hur lång tid varje aktivitet skulle ta. 
Aktiviteterna skrevs i form Trello aktiviteter och las enligt dess status, dvs att utföra, under utförande, utfördes osv.

\subsection{Scrum Möten}
\label{subsec:Lieth-Scrum Möten}
Inför varje av varje sprint hölls ett Scrum möte. under de möten diskuterades vad som skulle utföras, hur det skulle utföras och 
vad vi har lärt oss från tidigare sprinten. De möten var väldigt viktiga för att veta exakt vart teamet befann oss i vår plan av att 
skapa ett värde för kunden. Dessutom, diskuterade teamet risker och hur de borde hanteras.  
\\
\\Vid slutet av varje sprint hölls ett annat möte. Under detta diskuterade teamet vad som har gått bra, vad som måste förbättras 
och hur skulle vi kunna använda vår erfarenheter från tidigare sprint för att skapa bättre värde för kunden i varande sprinten. 

\subsection{Veckorapporter}
\label{subsec:Lieth-Veckorapporter}
Vid början av varje vecka ett veckomöte hölls. Veckorapporter bestod av tidrapportering som gjordes löpande under projektets 
gång, burndown charts, och statusuppdatering om olika projekt relaterade aktiviteter. Under de möten diskuterade teammedlemmar 
vad har gjorts under förra veckan, vad ska göras nästa vecka och vilka risker finns det. En veckorapport skrevs vid mötestillfällen. 
Burndown charts gjorde det lättare att veta hur mycket jobb har lagts på vad samt hur många timmar har varje medlem i teamet 
kvar att göra. 
