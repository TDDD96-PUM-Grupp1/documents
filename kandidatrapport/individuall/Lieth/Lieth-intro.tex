\section{Introduktion}
\label{sec:Lieth-introduction}

Kommunikation och information expanderar snabbare än någonsin idag. Som ett resultat, detta lägger stor press på dagens mjukvaruutveckling. 
Detta innebär att för ett mjukvaruutvecklingsprojekt att nå sina uppsatta mål måste utvecklingsprocessen optimeras så långt som möjligt [8]. 
En av dem mest tillämpade arbetsmetodik för sådan optimering är agil systemutvecklingsmetodiken. Agil systemutveckling är av de mest populära 
arbetsmetodiker bland små lika väl som stora mjukvaruprojekt. Agil systemutveckling har sett en stor intresseökning under dem senast åren [6]. 
På dess enklaste form agility definieras som förmågan att skapa och anpassa sig till förändring för att uppnå en slutprodukt, tjänst eller mål av 
ett projekt . Denna rapport kommer handla om tillämpning av en specifik form agil mjukvaruutvecklingsmetodik, nämligen Scrum. 
\\
\\ Scrum är agil mjukvaruutveckling ramverk som syftar till att hjälpa utvecklingsteamen att effektiviseras. Scrum är en av de mest tillämpade agil 
arbetsmetodiker. Den förser ett smidigt arbetssätt som syftar till att förbättra produktivitet i ett utvecklingsprojekt och effektivisera utvecklingsprocessen [7].

\subsection{Syfte \& Mål}
\label{subsec:Lieth-aim}
Syftet med denna rapporten är att utforska hur kan agil arbetsmetodik såsom Scrum bidra till att effektivisera teamsamarbete och eventuellt hjälpa teamet att 
uppnå de uppsatta målen och vad kan användas för göra framsteg i processen.


subsection{Frågeställning}
\label{subsec:Lieth-research-questions}
För att göra denna undersökninen krävs några fundementala frågor som behöver besvaras, dessa presenteras nedan:

\begin{enumerate}
	\item Vilka delar av Scrum bör vara med i vår Scrum version som kan bidra till att effektivisera vårt teamsamarbete 
	och eventuellt skapa bättre värde för vår kund?
	\item Hur börScrum planeras på sådant sätt att det skapar värde för kunden samtidigt som kursens lärandemålen uppnås? 
	
	
	\item Finns det fler verktyg, t.ex. Trello eller Github som kan användas till att göra vår  Scrum-version bättre på att skapa värde till kunden? 
	
\end{enumerate}

\subsection{Avgränsingar}
\label{subsec:Lieth-delimitations}
Resultaten som presenteras i detta papper är baserat på som läser programmen civilingenjör i datateknik samt civilingenjör i mjukvarateknik och förutser att
ingen av gruppmedlemmerna har tidigare haft något kurs eller erfarenhet inom agil mjukvaruutveckling. Dessutom, var studien begränsad av tidsbudget på 
400 timmar per person där fler obligatirska kursmoment var en del av.


