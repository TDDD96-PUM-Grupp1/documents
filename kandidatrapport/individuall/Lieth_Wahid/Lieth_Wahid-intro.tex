\section{Introduktion}
\label{sec:Lieth_Wahid-introduction}
Många ingenjörskurser inom mjukvaruutveckling som ges vid de flesta högskolor och universitet i Sverige idag är projektbaserade kurser på grund av marknadens behov till ingenjör som har förmåga att tillämpa teorin i praktiken. Dock de allra flesta av sådana kurser följer fortfarande den traditionella arbetsmetodiken som t.ex. vattenfallsmodellen. Problemet med en sådan modell är att det är svårt att göra ändringar i efterhand då allt sker sekventiellt. Detta papperet är ett försök att analysera samt diskutera om Scrum, en form av agil utvecklingsmetodik, kan vara fördelaktig för studenter för att uppnå målen av sådana kurser.
 

\subsection{Bakgrund}
Agil utvecklingsmetodik började i början av 1990-talet. Den uppstod ur behovet till mer flexibla utvecklingsmetodik. Till skillnad från traditionella metoder som användes agil utvecklingsmetodik erbjuder mer flexibilitet genom att tillåta ändringar att ske när det behövs.

Scrum är ett välkänt ramverk inom agil mjukvaruutveckling som utvecklades i mitten av 1990-talet\cite{WhatisSc87:online}. Ramverket sägs vara tillämpbart för små team med 5-9 personer med mindre projekt. Trotts detta Scrum ramverket används oftast inte i projektkurser inom mjukvaruutveckling som ges vid högskolestudier. 

Det som har bidragit till att göra Scrum ett  populärt ramverk inom agil-mjukvaruutveckling är att den erbjuder ett par nyckelvärde. Dessa nyckelvärden kan sammanfattas i två viktiga delar \cite{TheScrum6:online} : 
\begin{enumerate}
	\item Transparens: Alla viktiga aspekter måste vara synliga för alla som ansvar för resultaten.
	\item Granskning : Användare av Scrum måste ofta granska \textit{scrumartefakter och progress}. Scrumartefakter kan inkludera bl.a. product backlog, sprint back m.m.
\end{enumerate} 
Dessutom använder Scrum ett iterativt, inkrementellt tillvägagångssätt för att optimera förutsägbarhet och hantera risk \cite{TheScrum6:online,}. 

\subsection{Syfte}
Syftet med denna rapporten är att undersöka, analysera och diskutera hur kan agil arbetsmetodiken Scrum kan vara till hjälp vid mindre studentprojekt med begränsad tidsbudget som tillkommer vid högskolestudier. 

\subsection{Frågeställning}
\label{subsec:Lieth_Wahid-research-questions}

\begin{enumerate}
\item Hur bra fungerar Scrum för små team som består av 8 studenter i praktiken? 
 
\item  Vilka delar av Scrum är relevanta för mindre projektbaserade kurser vid högskolestudier ? 

\item Kan How can using Scrum help small teams of student to make value for their target costumer?

\end{enumerate}

\subsection{Avgränsingar}
\label{subsec:Lieth_Wahid-delimitations}

Resultaten som publiceras i detta papperet är baserad på den emperiska kunskapen som drogs från de erfarenheter som har utvecklats under projektets gång. Dessuton är resultaten baserade på rundfrågor som gemfördes under gruppmöten under projektets gång. Vidare hade projektet tidsbudget på ungefär 400 timmar per person. 