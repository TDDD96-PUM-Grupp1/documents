\section{Introduktion}
\label{sec:Lieth_Wahid-introduction}
Många ingenjörskurser inom mjukvaruutveckling som ges idag vid de flesta högskolor och universitet i Sverige är projektbaserade
kurser. Detta kommer på grund av marknadens behov för ingenjörer som har förmåga att tillämpa teorin i praktiken. Dock följer de flesta  
projektbaserade kurser fortfarande den traditionella arbetsmetodiken som t.ex. vattenfallsmodellen. Problemet med en sådan modell är att 
det är svårt att göra ändringar i efterhand då allt sker sekventiellt. Denna undersökning försöker analysera samt diskutera vilken utvecklingsmetodik,
Scrum eller vattenfall, är mest fördelaktig för studenter för att uppnå målen av sådana kurser.

\subsection{Bakgrund}Agil utvecklingsmetodik började i 1990-talet. Den uppstod ur behovet till mer en flexibel utvecklingsmetodik. Till skillnad från traditionella metoder som användes då, erbjuder agil utvecklingsmetodik mer flexibilitet genom att tillåta ändringar att ske när det behövs.

Scrum är ett välkänt ramverk inom agil mjukvaruutveckling som utvecklades i mitten av 1990-talet \cite{TheScrum}. 
Ramverket sägs vara tillämpbart för små team som består av 5-9 personer med mindre utvecklingsprojekt. Det som har bidragit till att Scrum har blivit ett väldigt populärt ramverk inom agil-mjukvaruutveckling är att den erbjuder ett par nyckelvärden. Dessa nyckelvärden kan sammanfattas i två viktiga delar \cite{TheScrum}: 
\begin{enumerate}\label{two}
	\item Transparens: Alla viktiga aspekter måste vara synliga för alla som ansvar för resultaten\cite{TheScrum}.
	\item Granskning : Användare av Scrum måste ofta granska \textit{scrumartefakter och progress}. Scrumartefakter kan inkludera
	bl.a. produkterbjuder ett par nyckelvärde.log, sprintplanering m.m\cite{TheScrum}.
\end{enumerate} 
Dessutom använder Scrum ett iterativt, inkrementellt tillvägagångssätt för att optimera förutsägbarhet och hantera risk \cite{TheScrum}. Trots de nämnda fördelar av Scrum ramverket appliceras den oftast inte i projektbaserade kurser inom mjukvaruutveckling som ges vid högskolestudier. 

\subsection{Syfte}
Huvudsyftet av denna rapport är att undersöka, analysera och diskutera vilken utvecklingsmetodik, Vattenfall (\textit{eng. Waterfall}) eller Scrum,
är mest fördelaktig för projektbaserade kurser inom mjukvaruutveckling som ges vid högskolestudier. Dessutom har rapporten som syfte att undersöka vilka delar av den valda utvecklingsmetodiken som är sannerligen relevant för framgången av ett studentprojekt.

\subsection{Frågeställning}
\label{subsec:Lieth_Wahid-research-questions}

\begin{enumerate}
	
	\item Vilken utvecklingsmetodik, vattenfall eller Scrum, lämpar sig bäst för studentprojekt där tidsbudget och resurser är begränsad? \label{LeyQ1}
	
	\item  Vilka delar av den valda utvecklingsmetodiken är mest relevanta för sådana projekt? \label{LeyQ2}
	
\end{enumerate}

\subsection{Avgränsingar}
\label{subsec:Lieth_Wahid-delimitations}
Enbart ett team av åtta studenter från två olika program har studerats. För att behålla objektiviteten i denna undersökning valde jag, författaren av denna rapport, att inte delta i studien. Vidare är resultaten som publiceras i denna undersökning baserad på den empiriska kunskapen som drogs från de erfarenheter som har utvecklats under projektets gång. Dessutom är resultaten som publiceras här baserade på gruppmöten, rundfrågor, observationer under arbetet, samt en enkät med syfte på att få ett djupare förståelse av påverkan av den valde utvecklingsmetodiken Scrum. Utöver det hade projektet en begränsad tidsbudget på ungefär 400 timmar per person. En annan avgränsning är teamets kunskap om Scrum och dess diverse delar då fler medlemmar av teamet är relativt nya till Scrum.



