\section{Introduktion}
\label{sec:Lieth_Wahid-introduction}
Många ingenjörskurser inom mjukvaruutveckling som ges idag vid de flesta högskolor och universitet i Sverige är projektbaserade
 kurser. Detta kommer på grund av marknadens behov till ingenjör som har förmåga att tillämpa teorin i praktiken. Dock de allra flesta av sådana 
 kurser följer fortfarande den traditionella arbetsmetodiken som t.ex. vattenfallsmodellen. Problemet med en sådan modell är att 
 det är svårt att göra ändringar i efterhand då allt sker sekventiellt. Detta papperet försöker analysera samt diskutera vilken utvecklingsmetodik, Scrum eller vattenfall, är mest fördelaktig för studenter för att uppnå målen av sådana kurser.
 
\subsection{Bakgrund}
Agil utvecklingsmetodik började i 1990-talet. Den uppstod ur behovet till mer flexibla utvecklingsmetodik. Till skillnad 
från traditionella metoder som användes då, agil utvecklingsmetodik erbjuder mer flexibilitet genom att tillåta ändringar att ske när 
det behövs.

Scrum är ett välkänt ramverk inom agil mjukvaruutveckling som utvecklades i mitten av 1990-talet \cite{TheScrum6:online}. 
Ramverket sägs vara tillämpbart för små team som består av 5-9 personer med mindre utvecklingsprojekt. Det som har bidragit
till att Scrum har blivit ett väldigt populärt ramverk inom agil-mjukvaruutveckling är att den erbjuder ett par nyckelvärde. Dessa 
nyckelvärden kan sammanfattas i två viktiga delar \cite{TheScrum6:online} : 
\begin{enumerate}\label{two}
	\item Transparens: Alla viktiga aspekter måste vara synliga för alla som ansvar för resultaten\cite{TheScrum6:online,}.
	\item Granskning : Användare av Scrum måste ofta granska \textit{scrumartefakter och progress}. Scrumartefakter kan inkludera
	 bl.a. produktbacklog, sprintplanering m.m\cite{TheScrum6:online,}.
\end{enumerate} 
Dessutom använder Scrum ett iterativt, inkrementellt tillvägagångssätt för att optimera förutsägbarhet och hantera risk \cite{TheScrum6:online,}. Trotts
de nämnda fördelar av Scrum ramverket den appliceras oftast inte i projektkurser inom mjukvaruutveckling som ges vid högskolestudier. 

\subsection{Syfte}
Huvudsyftet av denna rapport är att undersöka, analysera och diskutera vilken utvecklingsmetodik, Vattenfall (\textit{eng. Waterfall}) eller Scrum,
 är mest fördelaktig för projektbaserade kurser, inom mjukvaruutveckling, som ges vid högskolestudier. Dessutom utforskar rapporten vad är det som
  avgör att den valde utvecklingsmetodik lämpar sig bäst för liknande projekt. Utöver har rapporten som syfte att undersöka vilka delar av den vald
  e utvecklingsmetodiken som är sannerligen relevanta för framgången av ett studentprojekt.
 \
\subsection{Frågeställning}
\label{subsec:Lieth_Wahid-research-questions}

\begin{enumerate}
	
	\item Vilken utvecklingsmetodik, vattenfall eller Scrum, lämpar sig bäst för studentprojekt där tidsbudget och resurser är begränsad? 
	
	\item  Vilka delar av den valde utvecklingsmetodiken är mest relevanta för sådana projekt? 
	
	\item Vad är det som avgör om en utvecklingsmetodik lämpar sig för en viss typ studentprojekt? 

\end{enumerate}

\subsection{Avgränsingar}
\label{subsec:Lieth_Wahid-delimitations}

\begin{enumerate}
\item [\textbullet ] Enbart ett team av 8 studenter från två olika program har studerats. För att behålla objektiviteten i denna undersökning valde jag, författaren av detta papper, att inte delta på något sätt i studien.

\item[\textbullet]  Resultaten som publiceras i detta papperet är baserad på den empiriska kunskapen som drogs från de erfarenheter som har utvecklats under projektets gång.

\item[\textbullet]  Resultaten som publiceras i detta papperet är baserade på gruppmöten, rundfrågor, observationer under arbetet, samt en enkät med syfte på att få ett djupare förståelse av påverkan av den valde utvecklingsmetodiken Scrum. 

\item [\textbullet]  projektet hade en begränsad tidsbudget på ungefär 400 timmar per person. 

\item[\textbullet] Teamts kunskap om Scrum och dess diversa delar då fler medlemmar av teamet är relativt nya till Scrum.
\end{enumerate}

 
