\section{Teori} \label{sec:Lieth_Wahid-theory}
I detta avsnitt presenteras backgrunder och teorier som är relevanta till undersökningen. 
\subsection{Vattenfall}
Vattenfall är den mest använda utvecklingsmetodik mjukvarurelaterade högskolekurser. Metoden var introducerad i 1970 \cite{Waterfal86:online} av Dr. Winston W. Royce \cite{royce1987managing}. Vattenfall är en sekventiell utvecklingsprocess dvs. att
all utveckling sker i ett sekventiellt sätt vilket innebär mindre flexibilitet exempelvis det kan vara svårt att lägga till nya funktioner efterom det kräver att göra ändringar i kravspecifikation och resten av stegen som kommer därefter. Figur \ref{H1} visar de typiska 
stegen som en sådan modell kräver att följa. I början av kedjan kommer kravspecifikation under vilken genereras alla relevanta dokumenten som sedan kommer vara avtalet mellan kunden och utvecklingsteamet\cite{ASEEPEER95:online,}. Därefter kommer designfasen under vilken en detaljerad designdokument är producerad sedan kommer Implementationfasen under vilken påbörjar implementationsprocessen. Därefter kommer veriferingsfasen under vilken valideras programmets funktionalitet och eventuellt kommer testfasen under vilken programmet testas och säkerställs att den möter kravspecifikationen.
\subsection{Scrum}
Scrum är ett det mest populär ramverken inom agil mjukvaruutveckling. Ramverket utvecklades i mitten av 1990-talet av Jeff Sutherland, John Scumniotales och Jeff McKenna \cite{AgileToo72:online}. Det är möjligt [perhaps??] att den viktigaste egenskapen om Scrum är dess så kallad iterativa process som tillåter förändringar att hända när det behövs. Under  Scrum utvecklingsprocess deltagare av ett utvecklingsteam jobbar ihop för att föra projektet framåt genom att använda s.k Scrum artefakter som t.ex \textit{product backlog} och \textit{sprint backlog}. Detta möjliggöra för teamet att vara medvetna om senaste ändringar i kraven och omprioriterar om  sådant behövs \cite{aamir2017incorporating}. Eftersom Scrum är flexibel och kan ändras sådant att den passar varje grupps behov bestämde vårt att har endast vissa delar av scrum som teamet ansåg vara relevanta. Dess delar är:
Scrumbräda, Scrummöte, och Veckorapporter samt burndown charts. Mer detaljerad beskrvning av Scrum som användes under projektets gång kan återfinnas under \ref{main:Scrum}.