\section{Teori}
\label{sec:Lieth_Wahid-theory}
Scrum är ett välkänt ramverk inom agil mjukvaruutveckling. Ramverket sägs vara tillämpbart för små team med 5-9 personer med mindre projekt.  Många ingejörskurser idag är projektbaserade men utvecklingsmetodiken som används i sådan kurser följer fortfarande den traditionella arbetsmetodiken snarare än den agila arbetsmetodiken.  

Många ingenjörskurser inom mjukvaruutveckling som ges vid de flesta högskolor och universitet i Sverige idag är projektbaserade kurser på grund av marknadens behov till ingenjör som har förmåga att tillämpa teorin i praktiken. Dock de allra flesta av sådana kurser följer fortfarande den traditionella arbetsmetodiken som t.ex. vattenfallsmodellen. Problemet med en sådan modell är att det är svårt att göra ändringar i efterhand då allt sker sekventiellt. Detta papperet är ett försök att analysera samt diskutera om Scrum, en form av agilutvecklingsmetodik, kan vara fördelaktig för studenter för att uppnå målen av sådana kurser.
Fler studier har publicerats kring korrelation mellan antal studenter som har misslyckats att klara en av fler projektbaserade kurser och kunskapen om agil mjukvaruutveckling \cite{Usingscr27:online}. Fallstudien förslår att Scrum kan hjälpa

\subsection{Datainsamling}
För att kunna besvara frågraställningarna och dra slutsataser var det viktigt att samla data. För att samla data 
Denna delen består av två essentiella delar: sprintmöte och frågeformulärer. sprintmöten bestod av sprint retrospective några frågor som syftar till att få bättre förstålse på vart vi var med projektet, vad som behöver 
\begin{enumerate}
	\item {Sprintutvädering}: Sprintmöten är essensiell del av processen. Under denna delen ställdes frågor till  team för att vidare utereda vad har gått bra, vad som 
	\item  {Frågeformulärer}
\end{enumerate}