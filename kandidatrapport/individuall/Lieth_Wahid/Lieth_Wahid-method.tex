\section{Metod} \label{sec:Lieth_Wahid-method}
Metoden i denna rapporten bygger på gruppdiskussioner samt frågeformulär. För att bestämma lämpliga frågor inom ämnet och eventuellt  få relevant svar utfördes en förstudie i ämnet. Resultaten av rundfrågorna och frågeformulären jämfördes sedan med några  fallstudier och en del forskningsartiklar för att kunna dra slutsatser. De fallstudierna som rapporten använder sig av är \textit{"Using Scrum to Teach Software Engineering"} ~\cite{Usingscr27:online}, \textit{ Game Design and Development Capstone Project Assessment Using Scrum} ~\cite{GameDesign} och \textit{Use of Agile Methods in Software Engineering Education}~\cite{UseofAgi59:online}. Dessa artiklar diskuterar fördelarna med att integrera agil utvecklingsmetodik i mjukvaruutvecklingskurser som ges vid högskolestudier.  
\subsection{Datainsamling}\label{ds}
Data samlades genom gruppdiskussioner som hölls under gruppmöten, observationer, och ett frågeformulär. 
\subsubsection {Scrum-utvärdering} \label{Lieth:scrumU}
Under sprintutvärdering gjordes en utvärdering av Scrum. Utvärderingen togs muntligt i form av gruppdiskussioner för att minimera misstolkningen. Teamet fick svara bland annat på följande frågor:
\begin{enumerate}
	\item Har vi koll på vilka utvecklingsuppgifter som görs nu det vill säga vilka uppgifter som har status \textit{pågående} och vilka jobbar med vad? 
	\item Är vi nöjda med resultaten från den föregående sprinten?
	\item Är vi effektiva nog med arbetet? Saknar vi kunskap inom ett visst område som är nödvändigt för utvecklingsprocessen? Vad gör vi i sådana fall? \label{f3}
\end{enumerate} 
Samt sprintutvärderingsfrågor som till exempel:
\begin{enumerate}
	\item Vad har gått bra? Vad har gått dåligt?
	\item Har du stött på problem (under den föregående sprinten) som var svåra eller tidskrävande att lösa? Hur löste du dem?
	\item Har alla koll på vad som ska göras under nästkommande sprint?
	\item Vad kan förbättras?
	\item Hur noggranna är vår tidsestimering?
\end{enumerate} 

\subsubsection {Frågeformulär}
Vid slutet av iteration tre av fyra skickades ett frågeformulär till teamet. Syftet med enkäten var att få bättre förståelse för hur teamet upplevde utvecklingsprocessen med Scrum. Frågeformuläret innehöll  flervalsfrågor exempelvis: \textit{hade du erfarenheter inom Scrum sedan tidigare? 1-Ja. 2-Nej. 3-Knappast.}

Dessutom innehöll frågeformuläret ett par frågor som kräver svartext t.ex. \textit{Hur skulle du beskriva dina upplevelser av att jobba agilt (med Scrum) under de senaste iterationerna? Känner du dig nöjd med det? Saknar du någon del?}
\newpage