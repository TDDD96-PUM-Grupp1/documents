\section{Metod} \label{sec:Lieth_Wahid-method}
Metoden i denna rapporten bygger på rundfrågor samt frågeformulär. För att bestämma lämpliga frågor inom ämnet och eventuellt få relevant svar utfördes en förstudie i ämnet. Resultaten av rundfrågorna och frågeformulären 
jämfördes sedan med en del fallstudier och en del forskningsartiklar för att kunna dra slutsatser. De fallstudierna som rapporten använder sig av är i stort sett \textit{"Using Scrum to Teach Software Engineering"} \cite{Usingscr27:online} \textit{ Game Design and Development Capstone Project Assessment Using Scrum} \cite{ASEEPEER95:online} , och \textit{Use of Agile Methods in Software Engineering Education}\cite{UseofAgi59:online}. Dessa papper diskuterar fördelarna med att integrera Scrum agil utvecklingsmetodik i mjukvaruutvecklingskurser som ges vid högskolestudier.  
\subsection{Datainsamling}\label{ds}
Data samlades genom gruppdiskussioner som hölls under gruppmöten, observationer, och ett frågeformulär. 
\subsubsection {Scrum-utvärdering under sprintuvärdering } \label{Lieth:scrumU}
Under sprintutvärdering gjordes en utvärdering av Scrum. Utvärderingen togs muntligt i form av gruppdiskussioner för att minimera missförstående. Teamet fick svara bl.a. på följande frågor:
\begin{enumerate}
	\item Har vi koll på vilka utvecklingsuppgifter är under utföring och vilka jobbar med dem? 
	\item Är vi nöjda med resultaten av före sprint?
	\item Är vi effektiva nog med arbetet? Saknar vi kunskap inom ett visst område som är nödvändigt för utvecklingsprocessen? Vad gör vi i sådana fall? \label{f3}
\end{enumerate} 
Samt sprintutvärderingsfrågor som till exempel:
\begin{enumerate}
	\item Vad har gått bra? Vad har gått dåligt?
	\item Har du stött på problem (under förgående sprinten) som var svåra eller tidskrävande att lösa? Hur löste du dem?
	\item Har alla koll på vad som ska göras under näst kommande sprint?
	\item Vad kan förbättras?
	\item hur ackurata är vår tidsestimering?
\end{enumerate} 

\subsubsection {Förågeformulär}
Vid slutet av iteration tre skickades ett frågeformulär till teamet. Syftet med enkäten var att få bättre Förståelse av hur teamet upplevde utvecklingsprocessen med Scrum. Frågeformuläret bestod av flervalsfrågor exempelvis \textit{Hade du erfarenheter inom Scrum sedan tidigare? 1-Ja. 2-Nej 3- Knappast} samt frågor som kräver svartext t.ex. \textit{Hur skulle du beskriva dina upplevelser av att Jobba agilt (med Scrum) under de senaste iterationer? känner du dig nöjd med det? saknar du någon del?}.