\section{Metod}
\label{sec:Lieth_Wahid-method}
Detta avsnitt beskriver metoden som användes under projektets gång för att komma fram till resultaten.

\subsection{Scrum}
Projektet använder sig av Scrum utvecklingsmetodik. Metoden som användes under projektets gång beskriver i detaljer i  \ref{main-Utvecklingsmetod}. Det är värt att nämna att Scrum som användes i projekt var en modiferad version av det väl kända Scrum ramverket. Teamet bestämde sig från början av projektet att använda bara dem delar som verkade vara relevanta för projektets framgång.

\subsection{sprintutvärdering}
Vid slutet av varje sprint hölls ett  möte under vilket fick teamet refelektera över sprinten. Teamet frågades båda under mötet och via frågefomulärer för att skapa en bättre förståelse inför kommande sprints. Frågorna som teamet bads att svara på  kan uppdelas i två delar://
\subsubsection{Sprint-specifka frågor}
här fick teamet besvara frågor antigen under mötet eller via fårgefomulär som är releterade till förgående
sprintent. Följande är frågorna som teamet bads att besvara: 
\begin{enumerate}
	\item[1] Vad har gått bra/ dåligt?//
	Syftet med att ställa denna fråga är att få ett djupare förståelse om vilka svårigheter teamet konferterar samt hur bra fungerar Scrum för teametm samt hur den skalar. 
	\item[2] description 
\end{enumerate}

   ill sprinten såsom "\textit{vad har gått bra, dåligt? och vad kan förbättras? } men också mer generalla frå 
vad som har gått bra, vad som måste förbättras 
\subsubsection{Projekt samt Scrum relaterade frågor}
och hur skulle vi kunna använda vår erfarenheter från tidigare sprint för att skapa bättre värde för kunden i varande sprinten. 

\label{sec:Lieth_Wahid-Frågefromulär}

Metoden som har använts under projekts gång var en blandning av sprintmöte där frågor kring förgående och uppkommande sprinten diskuterades samt frågeformulär som skickades till gruppmedlemmeran vid slutet av iteration 3 och 4. Frågeformuläret hade i syfte att få djubare förståelse av vad som gick bra, dåligt, vad som saknas 

