
\section{Diskussion}
\label{sec:Lieth_Wahid-discussion}
Scrum har varit till stor nytta för vårt projekt och underlättade arbetet delvis. Många fallstudier också bevisar att Scrum är fördelaktig för studentprojekten inom mjukvaruutveckling. En av dem fallstudien som stödjer detta är
en fallstudie som genomfördes av \textit{Federal University of São Carlos} som visade att Scrum ramverket var fördelaktigt för studenter och hjälpte att alla studenter i fallstudien att leverera ett komplett projekt vid kursavslutning. en annan fallstudie gemfördes av\textit{DanielWebster College} som involverade en liknande studentprojektkurs.\cite{GameDesign} Reslusten också visade att Scrum kom till sort nytta för studenterna då alla studenter som följde Scrum ramverket har lyckats att uppnå kursens mål och leverera ett fungerande spelprojekt. Å andra sidan i ett relaterat arbeta av andra TDDD96-kursstudenter rapportera de att Scrum har varit svårt för dem att följa på grund av dess begränsningar som exempelvis den tillåter inte förändringar att ske i sprintbacklog under sprinten.\cite{overvakn73:online}


\subsection{Analys av resultaten}
Resultaten av undersökningen visar att Scrum var till stor nytta till teamet och kunden. Följning av Scrum gjorde det enklare för teammedlemmarna att ha koll på vad som ska göras genom att titta på Scrum-bräde i Trello-board. Dessutom var majoriteten av teamet nöjda med sitt arbete och tyckte att Scrum hjälpte dem att jobba effektivare. Å andra sidan Scrum metoden som följdes under projektets gång var en mycket minder version av Scrum ramverket. Man kan säga att de delar som valdes in i vår version av Scrum inkluderade bara dem allra essentiella delar av Scrum.

Dessutom visade resultaten att vissa Scrumartefakter är nödvändig för ett framgångsrikt projekt. Anledning till att den har valts ut kan vara på grund av att väldigt få av teammedlemmarna har bra kunskap om ramverket. Därför ser vi i resultaten av enkäten att teamet tyckte att stå-upp möten kunde ha varit till nytta till teamet. 

Resultatet också visar att två teammedlemmar tyckte att Scrum var inte givande och inte har gjort en skillnad. Detta kan vara på grund av att ramverket inte följdes helt d.v.s. som nämndes tidigare många delar valdes bort utan vidare undersökning. Därför blev vår version av Scrum mindre strukturerad. Dessutom sprinten blev mindre än vanligt p.g.a. dokumentskrivning, seminarier, presentationer men också på grund av helg och lov.
\subsection{Utvärdering}
Trots att Scrum har fungerat bra för det mesta har vi ändå stött på problem och svårigheter. Ett av dem problemen som vi har stött på är planering av aktiviteter. Anledning till detta var att vi hade väldigt korta sprints och väldigt mycket att göra både projektmässing och skolmässigt. I vissa fall hade vi många aktiviteter från en förgående sprinten flyttade till den innevarande sprinten för att kompensera för tiden som togs för dokumentskrivning, seminarier och liknande. Ett annat problem som vi har stött på var när vi valde bort fler viktiga delar av Scrum som exempelvis stå-upp möten som visade sig var fördelaktig hos andra grupper. Anledning till att detta har skett kan vara brist på kunskap inom agil mjukvaruutveckling då få av oss har faktiskt haft chansen att applicera agil mjukvaruutveckling i praktiken. 

\subsubsection{Metod}
Metoden var baserad på observation, rundfrågor, och ett frågeformulär därför finns det alltid ett rum för misstolkning. Detta gör det väldigt utmanande att dra generella slutsatser som kan gälla för en och varje grupp från liknande undersökningar. Trots att extrema åtgärdar togs för att minimera misstolkning när det gällde rundfrågor var emellertid svårt att få teammedlemmarna att inte missuppfatta frågorna samt att inte misstolka svaren som har fåtts från dem. 

En annan sak som kan ha påverkat resultaten av undersökningen är val av frågor både i rundfrågorna och frågeformuläret. Trots att en förstudie har utförts innan frågorna valdes var det ändå svårt att hitta rätt frågor och formulera dem på ett rätt sätt. En anledning till detta kan vara brist på kunskaper inom detta området samt brist på tidsbudget för att kunna utföra djupare studier inom området i och med att utvecklingsprocessen skedde parallellt med förstudien. 

\subsubsection{Val av vetenskapliga artiklar}
Artiklarna som valdes för jämförelse för undersökningen var en hel del av fallstudier samt äldre kandidatarbete. Ett problem som kan finnas med artiklarna är att var och en av dem argumenterar för agil mjukvaruutveckling och emot den traditionella utvecklingsmetoder. Därför är det naturligt att alla de artiklarna rekommenderar agil utveckling över traditionella metoder. Vidare läsning borde ha genomförts för att hitta balans i val av litteratur och hålla neutraliteten.
