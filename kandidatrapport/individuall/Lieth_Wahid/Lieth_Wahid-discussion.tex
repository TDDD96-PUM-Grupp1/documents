
\section{Diskussion}
\label{sec:Lieth_Wahid-discussion}
Scrum har varit av stort nytta till vårt projekt och underlättade arbetet delvis. Många fallstudier också bevisar att Scrum är fördelkatig för studentprojekten inom mjukvaruutveckling. En av dem fallstudien som stödjer detta är 
en fallstudie som genomfördes av \textit{Federal University of São Carlos} som visade att agil ramverket Scrum var fördelaktigt för studenter och hjälpte att alla studenter i fallstudien att leverare ett komplett projekt vid kursavslutning. en annan fallstudie gemfördes av\textit{DanielWebster College} \cite{ASEEPEER95:online} som involverade en liknande studentprojektkurs. Resulten också visade att Scrum kom till till sort nytta för studerna då alla studenter som följed Scrum ramverket har lyckats att uppnå kursens mål och leverera ett fungerande spelprojekt. Å andra sidan i ett relaterat arbeta av andra TDDD96-kursstudenter \cite{overvakn73:online } rapportera de att Scrum har varit svårt för dem att följa på grund av dess begränsningar som exempelvis den tillåter inte förändringar att ske i sprintbacklog under sprinten. 

\subsection{Analys av resultaten}
Resultaten av forskningen visar att Scrum var av stor nytta till teamet och kunden. Följning av Scrum gjorde det lätt att för teamets medlemmar att ha kolla på vad some ska göras genom att titta på Scrumbräda i Trello-board. Dessutom majoriteten av teamet nöjda med sitt arbete och tyckte att Scrum hjälpte dem att jobba effektivare. Å andra sidan Scrum metoden som följdes under projektets gång var en mycket minder version av Scrum. Man kan
säga att delar att valdes in i vår version av Scrum inkulderade bara dem allra måste ha delar.




Teamts kunskap om Scrum och dess vektiga delar då fler medlemmar av teamet är relativt nya till Scrum. detta kan förklarar varför vissa delar av Scrum valdes
bort utan vidare undesökningar. 
