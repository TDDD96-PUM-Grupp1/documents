
\section{Diskussion}
\label{sec:Lieth_Wahid-discussion}
Scrum har varit till stor nytta för vårt projekt och underlättade arbetet delvis. Många fallstudier bevisar också att Scrum är fördelaktigt för studentprojekten inom mjukvaruutveckling. En av de fallstudierna som stödjer detta är
en fallstudie som genomfördes av \textit{Federal University of São Carlos} som visade att ramverket Scrum var fördelaktigt för studenter och hjälpte alla studenter i fallstudien att leverera ett komplett projekt vid kursavslutning. En annan fallstudie gemfördes av\textit{DanielWebster College} som involverade en liknande studentprojektkurs.~\cite{GameDesign} Resultaten visade också att Scrum kom till stor nytta för studenterna då alla studenter som följde ramverket Scrum har lyckats att uppnå kursens mål och leverera ett fungerande spelprojekt. Å andra sidan i ett relaterat arbete av andra TDDD96-kursstudenter rapporterade de att Scrum har varit svårt för dem att följa på grund av dess begränsningar som exempelvis det inte är tillåtet att göra ändringar i sprint-backlog under sprinten det vill säga att uppgifter läggs till sprint-backlog endast under sprintplanering.~\cite{overvakn73:online}


\subsection{Analys av resultaten}
Resultaten av undersökningen visar att Scrum var till stor nytta för teamet och kunden. Att följa Scrum gjorde det enklare för teammedlemmarna att ha koll på vad som ska göras genom att titta på Scrum-brädet i Trello-board. Dessutom var majoriteten av teamet nöjda med sitt arbete och tyckte att Scrum hjälpte dem att jobba effektivare. Å andra sidan var versionen av Scrum som följdes under projektets gång en mycket mindre version av ramverket Scrum. Man kan säga att de delar som valdes in i vår version av Scrum inkluderade bara de allra essentiella delarna av Scrum.

Dessutom visade resultaten att vissa Scrumartefakter är nödvändiga för ett framgångsrikt projekt. Anledning till att den har valts ut kan vara på grund av att väldigt få av teammedlemmarna har bra kunskap om ramverket. Därför ser vi i resultaten av enkäten att teamet tyckte att stå-upp möten kunde ha varit till nytta för teamet. 

Resultatet visar också  att två teammedlemmar tyckte att Scrum inte var speciellt givande och inte har gjort en skillnad. Detta kan vara på grund av att ramverket inte följdes helt d.v.s. som nämndes tidigare många delar valdes bort utan vidare undersökning. Därför blev vår version av Scrum mindre strukturerad. Dessutom blev sprinten mindre än vanligt på grund av dokumentskrivning, seminarier, presentationer men också på grund av helg och lov.
\subsection{Utvärdering}
Trots att Scrum har fungerat bra för det mesta har vi ändå stött på problem och svårigheter. Ett av dem problemen som vi har stött på är planering av aktiviteter. Anledning till detta var att vi hade väldigt korta sprints och väldigt mycket att göra både projektmässigt och skolmässigt. I vissa fall hade vi många aktiviteter från en förgående sprinten flyttade till den innevarande sprinten för att kompensera för tiden som togs för dokumentskrivning, seminarier och liknande. Ett annat problem som vi har stött på var när vi valde bort fler viktiga delar av Scrum som exempelvis stå-upp möten som visade sig var fördelaktig hos andra grupper. Anledning till att detta har skett kan vara brist på kunskap inom agil mjukvaruutveckling då få av oss har faktiskt haft chansen att applicera agil mjukvaruutveckling i praktiken. 

\subsubsection{Metod}
Metoden var baserad på observation, gruppdiskussioner och  ett frågeformulär därför finns det alltid ett rum för misstolkning. Detta gör det väldigt utmanande att dra generella slutsatser som kan gälla för en och varje grupp från liknande undersökningar. Trots att extrema åtgärder togs för att minimera misstolkningen när det gällde rundfrågor var det emellertid svårt att veta huruvida teammedlemmerna har förstått frågorna rätt.

En annan sak som kan ha påverkat resultaten av undersökningen är val av frågor både i gruppdiskussioner och frågeformuläret. Trots att en förstudie har utförts innan frågorna valdes var det ändå svårt att hitta rätt frågor och formulera dem på ett rätt sätt. En anledning till detta kan vara brist på kunskaper inom detta området samt brist på tidsbudget för att kunna utföra djupare studier inom området i och med att utvecklingsprocessen skedde parallellt med förstudien. 

Generellt, kan metoden som användes i denna undersökningen förbättras. För att förbättra datainsamling borde flera frågeformulär skickats till teammedlemmarna med olika frågor för att få tydligare bild på teamets upplevelser av Scrum. Dessutom borde djupare förstudier kring hur man väljer frågor för frågeformulären har genomförts i mån av tid. Detta hade lett till fler användbara exempel att presentera i resultaten. 

\subsubsection{Val av vetenskapliga artiklar}
De källor som författaren av denna rapport har valt att använda i denna undersökning har varit av akademiska källor i form akademiska artiklar, fallstudier samt kandidatrapporter. Valet gjordes för att hålla hög kvalitet på resultaten som fås från denna undersökning. Den kandidatrapporten som valdes som en källa för denna undersökning anses vara relevant av författaren eftersom rapporten lade fokus på relevanta frågor till denna undersökning. De akademiska artiklarna och fallstudierna som valdes som källor i denna rapport anses vara trovärdiga och är publicerade IEEE-hemsidan. Det är värt att nämna att alla källor som denna undersökningen använder sig har valts med källkritik i åtanke.