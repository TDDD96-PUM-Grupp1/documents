
\section{Diskussion}
\label{sec:Lieth_Wahid-discussion}
Scrum har varit till stor nytta till vårt projekt och underlättade arbetet delvis. Många fallstudier också bevisar att Scrum är fördelaktig för studentprojekten inom mjukvaruutveckling. En av dem fallstudien som stödjer detta är
en fallstudie som genomfördes av \textit{Federal University of São Carlos} som visade att Scrum ramverket var fördelaktigt för studenter och hjälpte att alla studenter i fallstudien att leverera ett komplett projekt vid kursavslutning. en annan fallstudie gemfördes av\textit{DanielWebster College} \cite{ASEEPEER95:online} som involverade en liknande studentprojektkurs. Reslusten också visade att Scrum kom till sort nytta för studenterna då alla studenter som följde Scrum ramverket har lyckats att uppnå kursens mål och leverera ett fungerande spelprojekt. Å andra sidan i ett relaterat arbeta av andra TDDD96-kursstudenter \cite{overvakn73:online} rapportera de att Scrum har varit svårt för dem att följa på grund av dess begränsningar som exempelvis den tillåter inte förändringar att ske i sprintbacklog under sprinten.


\subsection{Analys av resultaten}
Resultaten av undersökningen visar att Scrum var till stor nytta till teamet och kunden. Följning av Scrum gjorde det enklare för teammedlemmarna att ha koll på vad som ska göras genom att titta på Scrum-bräde i Trello-board. Dessutom var majoriteten av teamet nöjda med sitt arbete och tyckte att Scrum hjälpte dem att jobba effektivare. Å andra sidan Scrum metoden som följdes under projektets gång var en mycket minder version av Scrum ramverket. Man kan säga att de delar som valdes in i vår version av Scrum inkluderade bara dem allra essentiella delar av Scrum.

Dessutom visade resultaten att vissa Scrumartefakter är nödvändig för ett framgångsrikt projekt. Anledning till att den har valts ut kan vara på grund av att väldigt få av teammedlemmarna har bra kunskap om ramverket. Därför ser vi i resultaten av enkäten att teamet tyckte att stå-upp möten kunde ha varit till nytta till teamet. 

Resultatet också visar att två teammedlemmar tyckte att Scrum var inte givande och inte har gjort en skillnad. Detta kan vara på grund av att ramverket inte följdes helt d.v.s. som nämndes tidigare många delar valdes bort utan vidare undersökning. Därför blev vår version av Scrum mindre strukturerad. Dessutom sprinten blev mindre än vanligt p.g.a. dokumentskrivning, seminarier, presentationer men också på grund av helg och lov.
