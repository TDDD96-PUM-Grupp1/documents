\section{Slutsatser}
\label{sec:Lieth_Wahid-conclusion}
I detta avsnitt besvaras frågeställningarna baserade på de uppnådda resultaten som har fåtts från undersökningen.
\subsection{Frågeställningar}

\subsection*{\ref{LeyQ1} Vilken utvecklingsmetodik, vattenfall eller Scrum, lämpar sig bäst för studentprojekt där tidsbudget och resurser är begränsad?}
Scrum visade sig vara väldigt fördelaktig för projektet tack vara dess iterativa process och flexibilitet som tillåter ändringar att ske när det behövs. Teorin bakom detta ramverk är enkel att förstå och följa. Dessutom erbjuder metoden mycket flexibilitet och kan anpassas efter teamets behov. Eftersom att metoden följer en iterativ process innebär detta att ändringar kan ske när det behovs vilket gör den till ett bra arbetssätt för projektbaserade kurser in mjukvaruutveckling som ges vid högskolestudier. Anledningen till detta är att högskolestudenter får sina lärdomar kontinuerligt därför är det bra att ha ett utvecklingssätt som tillåter ändringar att ske.
	
Vattenfallsmodellen, å andra sidan, är en sekventiellmodell det vill säga att varje fas måste vara helt avklarad för att nästa kan påbörjas. Därför är det svår att göra ändringar när testningsfasen har uppnåtts. Detta innebär mindre flexibilitet och generellt svårt att slutföra utvecklingsprojektet och därmed blir det svårt att uppnå kursens lärandemål.
		
\subsection*{\ref{LeyQ2} Vilka delar av den valda utvecklingsmetodiken är mest relevanta för sådana projekt? }
Den versionen av Scrum som följdes under projektets gång var en mycket midre version av Scrum. Resultaten från undersökningen visar att teammedlemmarna är nöjda med de delar som inkluderades av Scrum och inte har saknat de delar som har valts bort som t.ex. Scrum Master. Dock visade resultaten också att teamet saknade stå-upp möten, som är en del av Scrumartefakter, och tyckte att det hade varit fördelaktigt till projektets framgång då det kunde ha hjälpt teamet till att jobba effektivare. Resultaten av undersökningen visar att de viktiga delar av Scrum är: sprintplanering, sprintutvärdering, Scrum-bräde som innehåller produkt-backlog och sprint-backlog, och Scrums stå-upp möten.