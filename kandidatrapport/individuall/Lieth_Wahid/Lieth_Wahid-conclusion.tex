\section{Slutsatser}
\label{sec:Lieth_Wahid-conclusion}
I detta avsnitt besvaras Frågeställningarna baserad på det uppnådda resultaten som har fåtts från undersökningen.
\subsection{Frågeställningar}

\subsection*{\ref{LeyQ1} Vilken utvecklingsmetodik, vattenfall eller Scrum, lämpar sig bäst för studentprojekt där tidsbudget och resurser är begränsad?}
Scrum visade sig vara väldigt fördelaktig för projektet tack vara dess iterativa process och flexibilitet som tillåter ändringar att ske när det behövs. Teorin bakom detta ramverk är enkelt att förstå och följa. Dessutom erbjuder metod mycket flexibilitet och kan anpassas efter teamets behov. Eftersom att metoden följer en iterativ process innebär detta att ändringar kan ske när det behovs vilket gör den ett bra arbetssätt för projektbaserade kurser in mjukvaruutveckling som ges vid högskolestudier. Anledningen till detta är att högskolestudenter får sina lärdomar kontinuerligt därför är det bra att ha ett utvecklingssätt som tillåter ändringar att ske.
	
Vattenfallsmodellen, å andra sidan, är en sekventiellmodell d.v.s. att varje fas måste vara avklarade för att nästa kan påbörjas. Därför är det svår att göra ändringar när testningsfasen har uppnåtts. Detta innebär mindre flexibilitet och generellt svårt att slutföra utvecklingsprojektet och därmed svårt att uppnå kursens lärandemålen.
		
\subsection*{\ref{LeyQ2} Vilka delar av den valda utvecklingsmetodiken är mest relevanta för sådana projekt? }
Scrum metoden som följdes under projektets gång var en mycket minder version av Scrum. Resultaten från undersökningen visar att teammedlemmarna är nöjda med de delar som inkluderades av Scrum och inte har saknat de delar som har valts bort som t.ex. Scrum Master. Dock visade resultaten också att teamet saknade stå-upp möten, som är en del av Scrumartefakter, och tyckte att hade varit fördelaktigt till projektets framgång då den hade kunnat bidra till att jobba effektivare. 