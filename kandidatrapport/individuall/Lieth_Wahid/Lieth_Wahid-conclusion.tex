\section{Slutsatser}
\label{sec:Lieth_Wahid-conclusion}
Scrum visade sig vara väldigt fördelaktig för projektet tack vara dess iterativa process och flexibilitet som tillåter ändringar att ske när det behövs. Ramverket är svårt att komma igång med sätta i praktiken. Dessutom har ramverket dess begränsningar till exempel aktiviteter får endast läggas till under sprintplanering. Kravet passare grupper som utför projekt där projektet är
det enda i fokus. Högskolestudenter har oftast två kurser som går parallellt vilket gör det svårare att hålla sig till Scrum artefakter varför blir det svårt att följa Scrum ramverket. I ett studentprojekt ingår oftast olika delar såsom seminarium, rapportskrivning och presentationer. Detta bidra till mindre tid för utveckling på grund av sprinten är oftast korta under sprintsen. Å andra sidan är vattenfall modellen väldigt oflexibla för studenter. Studenter oftast utvecklar sina kunskaper inom mjukvaruutveckling gradvis under projektets gång. Vattenfall bygger att på nästan allt som görs bör göras så perfekt som möjligt d.v.s att nya funktionaliteter kan inte läggas i efterhand utan att ändra kravspecifikationen och allt som kommer efter. Det gör det svårare för studenter att levererar ett helt projekt vid kursens-avslutning. Bortsett från det vattenfall modellen är lätt att komma igång med och följa samt fungerar bättre för mindre studentprojekt där projektkraven är sätta och tydliga att följa. Däremot kan man komma över Scrums svårigheter med genom att ha flera projektkurser där Scrum är en integrerande del av men i fallet av vattenfallmodellen är det inte lätt att göra ändringar när man befinner sig i testningsfasen. Därför är Scrum det bättre val för studentprojekten.  Därför en enklare version av Scrum som erbjuder både Scrum flexiblitet och vattenfalls enkelhet kan vara av stort nytta till högskolestudenter. En version inkluderar Scrumbräda som hjälper projektgruppen att ha kolla på vad är det som ska göras, vem gör vad m.m. samt sprintmöte där en utvärdering av förgående sprint skrivs och diskuteras, och stå-uppmöte som hjälper projektgruppen att hålla bra kommunikation mellan varandra, och ger en överskit på vad alla jobbar på.    
