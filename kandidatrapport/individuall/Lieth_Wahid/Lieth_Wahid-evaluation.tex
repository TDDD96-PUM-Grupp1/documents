\section{Utvärdering}
Trotts att Scrum har fungerat bra för det mesta har vi ändå stött på problem och svårigheter. Ett av dem problemen som vi har stött på är planering av aktiviteter. Anledning till detta var att vi hade väldigt korta sprints och väldigt mycket att gör både projektmässing och skolmässingt. I vissa fall hade vi många aktiviteter från en förgående sprinten flyttade till den innevarande sprinten för att kompensera för tiden som togs för dokumentskrivning, seminarier och liknande. Ett annat problem som vi har stött på var när vi valde bort fler vektiga delar av Scrum som exempelvis stå-upp möten som visade sig var fördelaktig hos andra grupper. Anledning till att detta har skett kan vara brist på kunskap inom agil mjukvaruutveckling då få av oss har faktistk haft chansen att applicera agil mjukvaruutveckling i praktiken. 

\subsection{Metod}
Metoden var baserad på observation, rundfrågor, och ett frågeformulär det finns alltid ett rum för mistolkning. Detta gör liknande undersökningar väldigt svår att dra en generell slutsats som kan gälla för en och varje grupp. Trots att exterm åtgärdar togs för att minimera mistolkning när det gällde rundfrågorna 