\section{Utvärdering}
Trotts att Scrum har fungerat bra för det mesta har vi ändå stött på problem och svårigheter. Ett av dem problem som vi har stött på är planering av aktiviteter. Anledning till detta var att vi hade väldigt korta sprints och väldigt mycket att gör både projektmässing och skolmässingt. I vissa fall hade vi många aktiviteter från förgående sprinten flyttade till nästkommande sprinten för att kompensera för tiden vi skrev dokument och liknande. Ett annat problem som vi har stött på var när vi valde bort fler vektiga delar av Scrum som exempelvis stå-upp möten som visade sig var fördelaktig hos andra grupper. Detta har skett på grund av brist på kunskap inom agil mjukvaruutveckling då få av oss har faktistk haft chansen att applicera agil mjukvaruutveckling i praktiken. 