\section{Utvärdering}
Trotts att Scrum har fungerat bra för det mesta har vi ändå stött på problem och svårigheter. Ett av dem problemen som vi har stött på är planering av aktiviteter. Anledning till detta var att vi hade väldigt korta sprints och väldigt mycket att gör både projektmässing och skolmässingt. I vissa fall hade vi många aktiviteter från en förgående sprinten flyttade till den innevarande sprinten för att kompensera för tiden som togs för dokumentskrivning, seminarier och liknande. Ett annat problem som vi har stött på var när vi valde bort fler vektiga delar av Scrum som exempelvis stå-upp möten som visade sig var fördelaktig hos andra grupper. Anledning till att detta har skett kan vara brist på kunskap inom agil mjukvaruutveckling då få av oss har faktistk haft chansen att applicera agil mjukvaruutveckling i praktiken. 

\subsection{Metod}
Metoden var baserad på observation, rundfrågor, och ett frågeformulär därför finns det alltid ett rum för misstolkning. Detta gör det väldigt utmanande att dra generella slutsatser som kan gälla för en och varje grupp från liknande undersökningar. Trots att extrema åtgärdar togs för att minimera misstolkning när det gällde rundfrågor var emellertid svårt att få teammedlemmarna att inte missuppfatta frågorna samt att inte misstolka svaren som har fåtts från dem. 

En annan sak som kan ha påverkat resultaten av undersökningen är val av frågor både i rundfrågorna och frågeformuläret. Trots att en förstudie har utförts innan frågorna valdes var det ändå svårt att hitta rätt frågor och formulera dem på ett rätt sätt. En anledning till detta kan vara brist på kunskaper inom detta området samt brist på tidsbudget för att kunna utföra djupare studier inom området i och med att utvecklingsprocessen skedde parallellt med förstudien. 
