\section{Bakgrund}
\label{sec:tim-background}
När det gäller datakomprimering så finns det andra studier som har gjorts som analyserat serialisering och deserialisering för olika typer av datastrukturer. I ''Performance Evaluation of Object Serialization Libraries in XML, JSON and Binary formats'' skriven av Kazuaki studerades olika dataformat, JSON, XML och binära format och hur dessa förhåller sig till varandra~\cite{serialization}. Detta gäller datastorlek och hastighet för serialisering och deserialisering. Utöver det så granskas även olika bibliotek för samma format, där alla körs i Java. Detta för att få en bredare syn på hur formatet påverkar exekveringstiden och inte hur väloptimerat biblioteket är. Resultatet från studien visar att binär data generellt sätt är snabbare och mer komprimerat i alla fall.
