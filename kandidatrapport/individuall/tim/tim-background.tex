\section{Bakgrund}
\label{sec:tim-background}
När det gäller datakomprimering så finns det andra studier som har gjorts som analyserat serialisering och deserialisering för olika typer av datastrukturer\cite{serialization}. Här analyseras till en början olika data format, JSON, XML och binära format och hur dessa förhåller sig till varandra. Detta gäller datastorlek, serialisering och deserialisering. Utöver det så granskas även olika bibliotek för samma format, där alla körs i Java. Detta för att få en bredare syn på hur formattet påverka exekveringstiden än att endast jämföra olika bibliotek, vilket då skulle ge en dålig bild av hur formattet påverka och mer om hur bra biblioteket är. Resultatet från studien visar att binärdata generellt sätt är snabbare i alla fall.


