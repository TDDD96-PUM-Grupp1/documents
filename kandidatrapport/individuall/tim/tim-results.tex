\section{Resultat}
\label{sec:tim-results}
Efter att ha följt metoderna från avsnitt \ref{sec:tim-method} beskrivs resultaten för de olika undersökningarna nedan. 

\subsection{Serialisering}
När endast ett objekt skickades kunde man observera att datan som skickades över event såg ut som i figur \ref{fig:tim-eventdata1} som illustrerar en hexdump från Wiresharks applikationslager. Här är den relevanta datan på rad \textit{0020} med start på \textit{\{}. Hexdumparna för RPC och event samt 1-5 objekt finns tillgängliga i appendix \ref{app:hexdumps}. Vad man kan se från resultatet är att deepstream inte gör någon form av komprimering av JSON-objekt för att minimera datan som skickas över nätverket. Dock så tar den bort alla former av blanksteg och tabbar, då dessa inte har någon påverkan på slutresultatet av ett JSON-Objekt. Från att kolla på datan ser vi att deepstream även skickar med lite extra data för att hålla koll på vad för anrop som görs. Det går även att se att antalet bytes som skickas för JSON-objektet är 74 bytes per objekt med 1 byte som kommatecken till att separera alla JSON-objekt. Detta går att läsa av i de olika hexdumparna från appendix \ref{app:hexdumps}. Denna anmärkningen används för resultatdelen av avsnitt \ref{subsec:tim-result-response}.

\subsection{Rundturstid}
\label{subsec:tim-result-response}
Resultatet av rundturstiden från metoden som beskrevs i avsnitt \ref{subsec:tim-method-response} presenteras i figur \ref{fig:tim-response-graph}. Från denna kan det ses att rundturstiden beror linjärt på hur mycket data som skickas. Vad som är intressant är att en RPC är mycket snabbare med att skicka data när det gäller att få ett svar tillbaka, samtidigt som att responstiden varierar mycket mindre. Det går dock inte att se någon speciell skillnad på komplexiteten hos båda då rundturstiderna ser ut att öka med samma hastighet. Notera att skalan är i millisekunder och att skillnaden mellan båda endast är ca 4 ms. I fallet då 990 JSON-objekt skickas det $74*990+989=74249$ bytes med data, vilket är ett extremfall som troligtvis aldrig händer i verkligheten.

\begin{figure}[t]
    \center
    \begin{tikzpicture}
        \begin{axis}[
            xlabel={Antal JSON-objekt},
            ylabel={Medeltid från 100 anrop (ms)},
            ytick={0,2,...,10},
            xtick={0,33,...,99},
            xticklabels={0,330,...,990},
            legend pos=north west,
            ymin=0,
            ymax=10,
            grid style=dashed,
            domain=0:100,
        ]
        \addplot[color=red,]
            table[x expr=\coordindex+1, y index=0]{individuall/tim/data/event_avg.dat};
            \addlegendentry{Event}
        \addplot[color=blue,]
            table[x expr=\coordindex+1, y index=0]{individuall/tim/data/rpc_avg.dat};
            \addlegendentry{RPC}
        \end{axis}
    \end{tikzpicture}
    \caption{Responstiden i jämförelse med antal objekt som skickas.}
    \label{fig:tim-response-graph}
\end{figure}
