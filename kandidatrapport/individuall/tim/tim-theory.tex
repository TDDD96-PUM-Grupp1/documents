\section{Teori}

\label{sec:tim-theory}
För att göra undersökningen behövs en viss grund av begrepp och terminologier fastställas för att underlätta läsandet.

\subsection{JSON}
JSON är ett dataformat för att lätt kunna hantera data, både för en dator och en användare.\cite{json} Formatet bygger översiktligt på fält med värden, objekt med flera fält och listor med objekt. Men listor kan även finnas i objekt och listor kan också innehålla fält. Formateringen kan se ut som i figur \ref{fig:tim-jsonformat} där en person äger tre bilar, vilket lagras som ett person objekt som innehåller ett objekt med olika bilar.

\lstset{language=Java}
\begin{figure}[h]
  \center
  \begin{minipage}[c]{5cm}
    \begin{lstlisting}
{
    "name": "John",
    "age": 30,
    "cars": {
        "car1": "Ford",
        "car2": "BMW",
        "car3": "Fiat"
    }
} 
    \end{lstlisting}
  \caption{JSON formatering}
  \label{fig:tim-jsonformat}
  \end{minipage}
\end{figure}

\subsection{Deepstream}
\label{subsec:tim-deepstream}
Deepstream är en Javascript-baserad tjänst med skalbarhet och datahantering i realtid i åtanke. Saker som utmärker detta gränssnitt är dess realtidsdatabaser, autentisering och att ingen backendutveckling behövs för att fungera. För att få en fungerande backend behöver man endast starta en deepstream server och sedan är allt igång. All integration med servern sker sedan på klientsidan av IoT systemet, med små saker som databas- och cachekonfiguration ifall nätverket kräver dessa. All datahantering sker då utan någon form av serverprogrammering. För att få denna modularitet hanteras all data av servern med hjälp av tre enkla koncept. Dessa koncept är records, events och RPC:s (Remote Procedure Call) och beskrivs mer ingående nedan. 

\subsubsection{Deepstream record}
\label{subsec:tim-ds-record}
Idén med records är att kunna lagra, uppdatera och ta bort data utan att den försvinner efter körning.\cite{ds:record} Detta samtidigt som att datan ska vara synkroniserad mellan alla uppkopplade enheter. För att uppnå detta använder deepstream en databasserver som kan konfigureras till att hantera Postgres, MongoDB, ElasticSearch eller RethinkDB. 

Då denna rapport inte kommnner hantera records så kommer ingen mer ingående förklaring ges, men finns att läsa på deras hemsida.\cite{ds-storingdata}

\subsubsection{Deepstream event}
\label{subsec:tim-ds-event}
Ett event kan ses som en direkt länk mellan flera publicerare och prenumeranter.\cite{ds:event} En publicerare är en användare som skickar data till deepstream server via en viss kanal som ges av en sträng (vanligtvis med format ''projekt/kanalnamn''). När deepstream servern mottagit datan skickas den vidare till alla användare som prenumererar på samma kanal. Här kan det finnas flertal publicerare och prenumeranter på samma kanal. Datan som skickas över en kanal kommer inte lagras på deepstream servern utan skickas direkt till alla prenumeranter, vilket leder till ett snabbt sätt att skicka data mellan klienter.

\subsubsection{Deepstream RPC}
\label{subsec:tim-ds-rpc}
Utifrån events och records behövs det något enkelt sätt att skicka samt få ett svar tillbaka från anrop. Detta kan vara ett enkelt fall då man kan implementera en addition RPC på en klient och anropa den från en annan för att få ett svar på beräkningen.\cite{ds:rpc} Detta är ett av de mest triviala exemplet, men all form av validering, beräkning eller liknande som man vill göra på en annan klient görs enklast med hjälp utav RPC:s. Man kan se det som att man implementerar en vanligt funktion som går att anropa över ett nätverk. Precis som i events så lagras inte datan som skickas på servern.

\subsection{Rundturstid}
När rundturstid nämns i denna delen av rapporten menas tiden det tar att skicka en förfrågan till en annan klient i nätverket och få tillbaka ett svar. Det vill säga tiden det tar att skicka ett meddelande till deepstream servern, få servern att skicka vidare till en annan klient som sedan skickar tillbaka ett svar genom servern. Det betyder \textbf{inte} den tiden det tar att få ett direkt svar från deepstream servern då detta inte ger ett relevant värde eftersom att klienter aldrig kommunicerar direkt med servern. I denna undersökningen körs både server och två klienter på samma dator, så rundturstiden kommer till stor del reflektera serialisering, deserialisering och eventuelll deepstream overhead på systemet.

\subsection{Wireshark}
Wireshark är ett verktyg för att analysera nätverkstrafik för att se över vad som skickas, över vilket protokoll och vem som skickar eller tar emot datan.\cite{wireshark:main} Detta ger användaren en bra förståelse för hur olika protokoll fungerar på en lägre nivå.

\subsubsection{Interface}
Utifrån det kan man ställa in vilket interface\cite[p.~364]{networking} man lyssnar på, detta inkluderar generellt Bluetooth, ethernet och loopback. I denna rapporten kommer loopback interface användas, detta interface är till för att skicka nätverksdata till och från samma dator. Det vill säga den data som inte skickas till en annan dator.
