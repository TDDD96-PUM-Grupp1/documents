\section{Metod}
\label{sec:tim-method}
För att uppnå ett relevant resultat kring frågeställningarna så har olika former av analyser genomförts på en lokal deepstream server och och två klienter. Där alla delarna körs på samma dator. Dessa undersökningar presenteras nedan.

\subsection{Serialisering}
\label{subsec:tim-method-serializing}
För att undersöka hur deepstream serialiserar data skickades olika mängder av JSON-objekt över nätverket samtidigt som Wireshark lyssnade på loopback interfacet. Detta ger då en bättre uppfattning om deepstream använder sig av några komprimeringsmetoder för att minska datan som skickas över nätverket.

JSON-objektet som skickades var en lista av objektet som beskrivs i figur \ref{fig:tim-jsonformat}. Antalet objekt i listan skickades inkrementellt från ett objekt till fem objekt. Det vill säga undersökningen kollar inte bara på datamängden för ett objekt utan även hur detta skalas upp till fler objekt. 

För att undersöka paketet så sätts Wiresharks interface till loopback och sedan skickas datan till deepstream via ett event anrop. Interface är loopback då undersökningen endast är intresserad av att analysera de paketen som skickas internt på datorn, eftersom den görs på en enda lokal dator. För att få tag i rätt paket att analysera användes ett filter\footnote{Filter för paketanalys: tcp.srcport == 60020 and !webSocket and (tcp and not tcp.len == 0)}. Detta filter filtrera bort paket som inte är relevant till undersökningen, såsom ACK meddelande och speciella portnummer. Datan som är intressant finns inuti applikationslagret av paketet, detta benämns ''Data'' i Wireshark.

\subsection{Rundturstid}
\label{subsec:tim-method-response}
När datamängden var analyserad var det tid för att analysera responstiderna givet olika parametrar. Dessa parametrar är datastorleken och RPC gentemot event. Till en början behövdes någon form av räknare som håller koll på hur lång tid det tar från att skicka data till att få tillbaka den. Detta uppnådes genom att använda en prestanda funktion \textit{performance.now()} från \textit{''performace-now''} npm-paketet.\cite{performance-now} Denna funktionen returnera tiden i mikrosekunder sedan starten av programmet. Genom att spara undan denna tiden innan ett deepstream anrop och sedan ta differensen mellan det sparade värdet och det nuvarande värdet när svaret har kommit tillbaka, fås tiden som det tog att skicka och ta emot ett svar. Detta svaret kallas då för rundturstiden och är vad denna rapporten jämför med de olika parametrarna. 

Hur denna datan skickades finns tillgängligt på GitHub\footnote{\url{https://github.com/Thraix/Deepstream-responsetime} \newline commit: 19a6de9f36b76aad85d42188deba26f1df291a14 } tillsammans med den data kommer tas upp i kapitel \ref{sec:tim-results}. Den översiktliga idén är att vid RPC så skickas ett anrop som besvaras direkt med ett responsobjekt. När det gäller event så skickas datan över ett event, som senare svarar med ett annat eventanrop tillbaka. Detta gjordes 100 gånger för varje lista med JSON-objekt. Dessa listor innehöll JSON-objektet som beskrivs i figur \ref{fig:tim-jsonformat} 0-99 gånger. Det vill säga första listan som skickades var tom, den andra hade ett objekt och den tredje had två och så vidare. Här skickades inte datan tillbaka till den första klienten, så svaret är alltid ett tomt JSON-objekt. 
