\section{Introduktion}
\label{sec:tim-introduction}
Under projektet fick vi tilldelat Deepstream som ett gränssnitt för att implementera spelet på Cybercoms backend. Men utöver att få det tilldelat finns det motivation att undersöka andra gränssnitt och hur de har löst de olika problemen som uppkommer vid nätverk.

\subsection{Syfte \& Mål}
\label{subsec:tim-aim}
Projektet som har utförts bygger en hel del på att nätverkslösningen som implementerats är snabb och skalbar för att uppnå maximal responsivitet för många användare. För att uppnå detta krävs en djupare förståelse för olika nätverksbibliotek och deras för- och nackdelar inom skalbarhet.

\subsection{Frågeställning}
\label{subsec:tim-research-questions}
För att göra denna undersökninen krävs några fundementala frågor som behöver besvaras, dessa presenteras nedan:

\begin{enumerate}
\item Hur påverka de olika biblioteken nätverket under större belastningar?

\item Vad utmärker de olika biblioteken och hur påverka detta skalbarheten inom dem?

\item Skulle slutprodukten fungerat bättre om den byggdes upp i ett annat bibliotek?

\end{enumerate}
\subsection{Avgränsingar}
\label{subsec:tim-delimitations}
Detta papret kommer endast att diskutera realtidslösningar från två bibiotek, Deepstream och Firebase. Med tanke på resurserna som finns tillgängligt kommer inte ett verkligt experiment att utföras vilket kan påverka slutresultatet. 


