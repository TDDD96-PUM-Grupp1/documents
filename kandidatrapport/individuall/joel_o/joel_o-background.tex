\section{Bakgrund}
\label{sec:joel_o-background}

Javascript har länge varit det främsta verktyget för att bygga interaktiva webbapplikationer. Med lanseringen av exekverings-miljön Node.js tog Javascript även plats utanför webbläsare som ett kvalificerat alternativ för server-programvara. Med Node.js lanserades även pakethanteraren npm, Node Package Manager. Både Node.js och npm har sedan 2010 vuxit och nått stor popularitet. \cite{node-timeline}

Javascript har idag nått en position som ett av de viktigaste programmeringsspråken. Undersökningar tyder på att Javascript i början av 2018 är det mest använda språket inom mjukvaruutveckling.\cite{githut}\cite{so-survey}

\subsection{Ekosystem för pakethantering}
Att förlita sig på färdiga paket med mjukvara har blivit en allt vanligare praxis. För många programmeringsspråk finns stora register av sådana paket. Förutom tidigare nämnda npm för Javascript bör bland annat Maven för Java och PyPi för Python nämnas. Paketen i dessa register beror på varandra och kopplas därmed ihop till ett stort ekosystem. Dessa kopplingar gör att en förändring i ett paket kan propagera genom ekosystemet och påverka andra paket. En konsekvens av detta är att utvecklare som förlitar sig på paket från dessa system behöver ha en viss medvetenhet för förändringar i ekosystemet.\cite{Henry2017}

Ett exempel på hur förändringar av ett paket kan påverka stora delar av npm-ekosystemet är en incident med paketet left-pad 2016. På grund av en konflikt gällande namnet på ett paket valde en utvecklare att radera alla sina publicerade paket från npm-registret. Då många andra projekt och paket direkt eller indirekt berodde på ett av utvecklarens paket, left-pad, slutade dessa fungera korrekt. Detta illustrerar hur till synes små förändringar kan få oväntade konsekvenser i ekosystemet. I detta fall var det beslut från en enskild utvecklare som resulterade i konsekvenser för stora delar av systemet. Historien visar också på styrkan i dessa öppna plattformar. Det tog nämligen endast 10 minuter innan en annan utvecklare publicerade ett funktionellt identiskt paket för att ersätta det raderade.\cite{npm-left-pad}

\subsection{Tidigare arbeten}
Tidigare arbeten som undersöker beroenden i Javascript har ofta fokuserat på frågeställningar om npm-paket och inte Javascript-project i allmänhet. Wittern et al. har utfört en undersökning av hur relationer mellan paket i npm har förändrats historiskt.\cite{Wittern:2016} Undersökningen visar på att beroenden till andra paket inom npm har ökat. En trend som noterats är att det existerar en mer och mer begränsad mängd av välanvända paket som växer allt mer i popularitet.

Kula et.al har undersökt en speciell klass av paket i npm som benämns micropackages.\cite{Kula2017} Detta definieras som paket innehållande endast en funktionsdefinition. Det visas att nästan hälften av npm-paket kan anses vara sådana micropackages. Jämfört med större paket innehåller micropackages färre beroenden till andra paket, men fler paket beror på dem.

S. Scott Henry har utfört fallstudier över uppdateringar av beroenden.\cite{Henry2017} Undersökningen visar några av de orsaker som gör att utvecklare väljer att uppdatera beroenden till nyare versioner av paket. Detta ger viss insikt i beroendens påverkan på utvecklingsprocessen.

Något gemensamt för många liknande arbeten är att beroenden modelleras som en riktad graf. Denna modell används även i detta arbete; se mer under \ref{subsec:joel_o-grafmodell}.

\subsection{Projekterfarenheter}
I det projekt som huvuddelen av denna rapport berör har Javascript och npm använts. Projektet har förlitat sig på flera färdiga paket både i slutprodukten och för att underlätta utvecklingen. Under projektet har val tagits om huruvida färdiga paket ska användas eller ej samt vilka paket som lämpar sig till olika uppgifter. Utvecklingsprocessen har på flera sätt påverkats av dessa val. Från dessa erfarenheter finns insikter om arbete med npm och beroenden.
