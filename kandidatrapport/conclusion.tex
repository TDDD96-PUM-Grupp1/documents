\chapter{Slutsatser}
\label{cha:slutsatser}

Baserat på de uppnådda resultaten kan frågeställningarna svaras på i stor utsträckning. Det genomförda projektet kan därmed erbjuda vissa intressanta insikter enligt syftet med denna rapport. I följande stycken presenteras de slutsatser som har uppnåtts och arbetets vidare lärdomar.

\section{Frågeställningar}

\subsection*{\ref{fs:fs_1} Hur kan ett realtidsspel som använder sig av Cybercoms backend implementeras så att man skapar värde för kunden?}

Ett system baserat runt Cybercoms backend har implementerats. Systemet består av ett spel med flera olika spellägen. Extra arbete har lagts på att försäkra sig om att spelet känns responsivt för användarna. Detta ger kunden värde då det demonstrerar effektiviteten i den existerande backenden. Responsiviteten har uppnåtts genom mycket testning och kalibrering av indatahantering. Med möjligheten att skapa nya spellägen finns stor potential till vidareutveckling av produkten, vilket ger kunden stort värde för framtida användning av systemet.

\subsection*{\ref{fs:fs_2} Vilka erfarenheter kan dokumenteras från programvaruprojektet som kan vara intressanta för framtida projekt?}

En organisatorisk erfarenhet som kan tas med från projektet är hur mjukvaruutveckling och dokumentskrivning kan balanseras. Att arbeta mer kontinuerligt och parallellt med dokument och kod i projekt har visat sig fördelaktigt. På den tekniska sidan finns flera erfarenheter av React-utveckling. En stor sådan är vikten av att organisera lager av komponenter. React har visat sig lätt introducera problem med rörig kod då inga verktyg för state-hantering används. Dessa tekniska erfarenheter av React har även utforskats vidare i bilaga \ref{individual:axel}.

\subsection*{\ref{fs:fs_3} Vilket stöd kan man få genom att skapa och följa upp en systemanatomi?}

En systemanatomi kan erbjuda viss hjälp med att ge en översiktlig bild över ett system. I en grupp kan den även bidra till en mer gemensam bild av vad som ska skapas och därmed förhindra missförstånd. Att ta fram systemanatomin har dock visat sig vara en svår process. I det utförda projektet upplevdes det svårt att få en gemensam bild över vad systemanatomin representerar. Det är troligt att detta har reducerat det stöd som gruppen har fått av modellen.

\subsection*{\ref{fs:fs_4} Hur kan kontinuerliga användardemonstrationer användas i utvecklingsfasen för att förbättra ett spels kvalitet?}

I det utförda projektet har spelet flera gånger demonstrerats för kunden. Dessa demonstrationer har använts som kontroll över att utvecklingen går i rätt riktning. De har även varit möjligheter för kunden att förtydliga eller justera sina mål med projektet. Idéer till olika lösningar har också kunnat diskuterats öppet.

\section{Måluppfyllelse}

Det genomförda projektet har till stor del lyckats skapa värde för kunden. Det slutgiltiga systemet demonstrerar flera kvalitetsaspekter hos kundens IoT-backend. Spelet erbjuder också goda möjligheter till vidareutveckling genom nya spellägen. Slutprodukten anses därför väsentligen ha uppfyllt de mål som projektet utgått från.

Denna rapport ger en god bild över den utvecklingsprocess som har följts och resultaten från utvecklingsarbetet. Genom erfarenheter och reflektioner från det utförda projektet har de presenterade frågeställningarna kunnat svarats på. Med avseende på projektets storlek anses dessa slutsatser vara av god relevans. För ytterligare generella aspekter av frågeställningarna skulle troligen flera olika projekt behöva studeras, men detta ligger utanför detta arbete.

\section{Viktiga insikter}

Från detta arbete finns flera intressanta insikter om både teknisk implementation och utvecklingsprocessen. Det utförda projektet presenterar en möjlig variant av agil utveckling som kan ge en värdefull slutprodukt. Detta visar att en mer avskalad variant av ett utvecklingsramverk som Scrum kan passa bra för projekt av mindre skala.

Många dokument har producerats i projektet och skapat stöd för gruppmedlemmarna. Dessa dokument har på olika sätt kommit till nytta innan, under och efter utvecklingsarbetet. Värdet hos olika sorters dokument är en nyttig insikt från det utförda projektet. Speciellt har det reflekterats över användning av systemanatomier. Denna typ av dokument, som till viss del skiljer sig från många andra systemmodeller, kan vara ett nyttigt verktyg att ta med sig till framtida projekt.

% This chapter contains a summarization of the purpose and the research
% questions. To what extent has the aim been achieved, and what are the
% answers to the research questions?
%
% The consequences for the target audience (and possibly for researchers
% and practitioners) must also be described. There should be a section
% on future work where ideas for continued work are described. If the
% conclusion chapter contains such a section, the ideas described
% therein must be concrete and well thought through.
